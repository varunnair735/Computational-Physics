
% Default to the notebook output style

    


% Inherit from the specified cell style.




    
\documentclass[11pt]{article}

    
    
    \usepackage[T1]{fontenc}
    % Nicer default font (+ math font) than Computer Modern for most use cases
    \usepackage{mathpazo}

    % Basic figure setup, for now with no caption control since it's done
    % automatically by Pandoc (which extracts ![](path) syntax from Markdown).
    \usepackage{graphicx}
    % We will generate all images so they have a width \maxwidth. This means
    % that they will get their normal width if they fit onto the page, but
    % are scaled down if they would overflow the margins.
    \makeatletter
    \def\maxwidth{\ifdim\Gin@nat@width>\linewidth\linewidth
    \else\Gin@nat@width\fi}
    \makeatother
    \let\Oldincludegraphics\includegraphics
    % Set max figure width to be 80% of text width, for now hardcoded.
    \renewcommand{\includegraphics}[1]{\Oldincludegraphics[width=.8\maxwidth]{#1}}
    % Ensure that by default, figures have no caption (until we provide a
    % proper Figure object with a Caption API and a way to capture that
    % in the conversion process - todo).
    \usepackage{caption}
    \DeclareCaptionLabelFormat{nolabel}{}
    \captionsetup{labelformat=nolabel}

    \usepackage{adjustbox} % Used to constrain images to a maximum size 
    \usepackage{xcolor} % Allow colors to be defined
    \usepackage{enumerate} % Needed for markdown enumerations to work
    \usepackage{geometry} % Used to adjust the document margins
    \usepackage{amsmath} % Equations
    \usepackage{amssymb} % Equations
    \usepackage{textcomp} % defines textquotesingle
    % Hack from http://tex.stackexchange.com/a/47451/13684:
    \AtBeginDocument{%
        \def\PYZsq{\textquotesingle}% Upright quotes in Pygmentized code
    }
    \usepackage{upquote} % Upright quotes for verbatim code
    \usepackage{eurosym} % defines \euro
    \usepackage[mathletters]{ucs} % Extended unicode (utf-8) support
    \usepackage[utf8x]{inputenc} % Allow utf-8 characters in the tex document
    \usepackage{fancyvrb} % verbatim replacement that allows latex
    \usepackage{grffile} % extends the file name processing of package graphics 
                         % to support a larger range 
    % The hyperref package gives us a pdf with properly built
    % internal navigation ('pdf bookmarks' for the table of contents,
    % internal cross-reference links, web links for URLs, etc.)
    \usepackage{hyperref}
    \usepackage{longtable} % longtable support required by pandoc >1.10
    \usepackage{booktabs}  % table support for pandoc > 1.12.2
    \usepackage[inline]{enumitem} % IRkernel/repr support (it uses the enumerate* environment)
    \usepackage[normalem]{ulem} % ulem is needed to support strikethroughs (\sout)
                                % normalem makes italics be italics, not underlines
    \usepackage{mathrsfs}
    

    
    
    % Colors for the hyperref package
    \definecolor{urlcolor}{rgb}{0,.145,.698}
    \definecolor{linkcolor}{rgb}{.71,0.21,0.01}
    \definecolor{citecolor}{rgb}{.12,.54,.11}

    % ANSI colors
    \definecolor{ansi-black}{HTML}{3E424D}
    \definecolor{ansi-black-intense}{HTML}{282C36}
    \definecolor{ansi-red}{HTML}{E75C58}
    \definecolor{ansi-red-intense}{HTML}{B22B31}
    \definecolor{ansi-green}{HTML}{00A250}
    \definecolor{ansi-green-intense}{HTML}{007427}
    \definecolor{ansi-yellow}{HTML}{DDB62B}
    \definecolor{ansi-yellow-intense}{HTML}{B27D12}
    \definecolor{ansi-blue}{HTML}{208FFB}
    \definecolor{ansi-blue-intense}{HTML}{0065CA}
    \definecolor{ansi-magenta}{HTML}{D160C4}
    \definecolor{ansi-magenta-intense}{HTML}{A03196}
    \definecolor{ansi-cyan}{HTML}{60C6C8}
    \definecolor{ansi-cyan-intense}{HTML}{258F8F}
    \definecolor{ansi-white}{HTML}{C5C1B4}
    \definecolor{ansi-white-intense}{HTML}{A1A6B2}
    \definecolor{ansi-default-inverse-fg}{HTML}{FFFFFF}
    \definecolor{ansi-default-inverse-bg}{HTML}{000000}

    % commands and environments needed by pandoc snippets
    % extracted from the output of `pandoc -s`
    \providecommand{\tightlist}{%
      \setlength{\itemsep}{0pt}\setlength{\parskip}{0pt}}
    \DefineVerbatimEnvironment{Highlighting}{Verbatim}{commandchars=\\\{\}}
    % Add ',fontsize=\small' for more characters per line
    \newenvironment{Shaded}{}{}
    \newcommand{\KeywordTok}[1]{\textcolor[rgb]{0.00,0.44,0.13}{\textbf{{#1}}}}
    \newcommand{\DataTypeTok}[1]{\textcolor[rgb]{0.56,0.13,0.00}{{#1}}}
    \newcommand{\DecValTok}[1]{\textcolor[rgb]{0.25,0.63,0.44}{{#1}}}
    \newcommand{\BaseNTok}[1]{\textcolor[rgb]{0.25,0.63,0.44}{{#1}}}
    \newcommand{\FloatTok}[1]{\textcolor[rgb]{0.25,0.63,0.44}{{#1}}}
    \newcommand{\CharTok}[1]{\textcolor[rgb]{0.25,0.44,0.63}{{#1}}}
    \newcommand{\StringTok}[1]{\textcolor[rgb]{0.25,0.44,0.63}{{#1}}}
    \newcommand{\CommentTok}[1]{\textcolor[rgb]{0.38,0.63,0.69}{\textit{{#1}}}}
    \newcommand{\OtherTok}[1]{\textcolor[rgb]{0.00,0.44,0.13}{{#1}}}
    \newcommand{\AlertTok}[1]{\textcolor[rgb]{1.00,0.00,0.00}{\textbf{{#1}}}}
    \newcommand{\FunctionTok}[1]{\textcolor[rgb]{0.02,0.16,0.49}{{#1}}}
    \newcommand{\RegionMarkerTok}[1]{{#1}}
    \newcommand{\ErrorTok}[1]{\textcolor[rgb]{1.00,0.00,0.00}{\textbf{{#1}}}}
    \newcommand{\NormalTok}[1]{{#1}}
    
    % Additional commands for more recent versions of Pandoc
    \newcommand{\ConstantTok}[1]{\textcolor[rgb]{0.53,0.00,0.00}{{#1}}}
    \newcommand{\SpecialCharTok}[1]{\textcolor[rgb]{0.25,0.44,0.63}{{#1}}}
    \newcommand{\VerbatimStringTok}[1]{\textcolor[rgb]{0.25,0.44,0.63}{{#1}}}
    \newcommand{\SpecialStringTok}[1]{\textcolor[rgb]{0.73,0.40,0.53}{{#1}}}
    \newcommand{\ImportTok}[1]{{#1}}
    \newcommand{\DocumentationTok}[1]{\textcolor[rgb]{0.73,0.13,0.13}{\textit{{#1}}}}
    \newcommand{\AnnotationTok}[1]{\textcolor[rgb]{0.38,0.63,0.69}{\textbf{\textit{{#1}}}}}
    \newcommand{\CommentVarTok}[1]{\textcolor[rgb]{0.38,0.63,0.69}{\textbf{\textit{{#1}}}}}
    \newcommand{\VariableTok}[1]{\textcolor[rgb]{0.10,0.09,0.49}{{#1}}}
    \newcommand{\ControlFlowTok}[1]{\textcolor[rgb]{0.00,0.44,0.13}{\textbf{{#1}}}}
    \newcommand{\OperatorTok}[1]{\textcolor[rgb]{0.40,0.40,0.40}{{#1}}}
    \newcommand{\BuiltInTok}[1]{{#1}}
    \newcommand{\ExtensionTok}[1]{{#1}}
    \newcommand{\PreprocessorTok}[1]{\textcolor[rgb]{0.74,0.48,0.00}{{#1}}}
    \newcommand{\AttributeTok}[1]{\textcolor[rgb]{0.49,0.56,0.16}{{#1}}}
    \newcommand{\InformationTok}[1]{\textcolor[rgb]{0.38,0.63,0.69}{\textbf{\textit{{#1}}}}}
    \newcommand{\WarningTok}[1]{\textcolor[rgb]{0.38,0.63,0.69}{\textbf{\textit{{#1}}}}}
    
    
    % Define a nice break command that doesn't care if a line doesn't already
    % exist.
    \def\br{\hspace*{\fill} \\* }
    % Math Jax compatibility definitions
    \def\gt{>}
    \def\lt{<}
    \let\Oldtex\TeX
    \let\Oldlatex\LaTeX
    \renewcommand{\TeX}{\textrm{\Oldtex}}
    \renewcommand{\LaTeX}{\textrm{\Oldlatex}}
    % Document parameters
    % Document title
    \title{Homework 7 \\ \vspace{10mm}
    {\large Varun Nair}}        
    
    
    
    

    % Pygments definitions
    
\makeatletter
\def\PY@reset{\let\PY@it=\relax \let\PY@bf=\relax%
    \let\PY@ul=\relax \let\PY@tc=\relax%
    \let\PY@bc=\relax \let\PY@ff=\relax}
\def\PY@tok#1{\csname PY@tok@#1\endcsname}
\def\PY@toks#1+{\ifx\relax#1\empty\else%
    \PY@tok{#1}\expandafter\PY@toks\fi}
\def\PY@do#1{\PY@bc{\PY@tc{\PY@ul{%
    \PY@it{\PY@bf{\PY@ff{#1}}}}}}}
\def\PY#1#2{\PY@reset\PY@toks#1+\relax+\PY@do{#2}}

\expandafter\def\csname PY@tok@w\endcsname{\def\PY@tc##1{\textcolor[rgb]{0.73,0.73,0.73}{##1}}}
\expandafter\def\csname PY@tok@c\endcsname{\let\PY@it=\textit\def\PY@tc##1{\textcolor[rgb]{0.25,0.50,0.50}{##1}}}
\expandafter\def\csname PY@tok@cp\endcsname{\def\PY@tc##1{\textcolor[rgb]{0.74,0.48,0.00}{##1}}}
\expandafter\def\csname PY@tok@k\endcsname{\let\PY@bf=\textbf\def\PY@tc##1{\textcolor[rgb]{0.00,0.50,0.00}{##1}}}
\expandafter\def\csname PY@tok@kp\endcsname{\def\PY@tc##1{\textcolor[rgb]{0.00,0.50,0.00}{##1}}}
\expandafter\def\csname PY@tok@kt\endcsname{\def\PY@tc##1{\textcolor[rgb]{0.69,0.00,0.25}{##1}}}
\expandafter\def\csname PY@tok@o\endcsname{\def\PY@tc##1{\textcolor[rgb]{0.40,0.40,0.40}{##1}}}
\expandafter\def\csname PY@tok@ow\endcsname{\let\PY@bf=\textbf\def\PY@tc##1{\textcolor[rgb]{0.67,0.13,1.00}{##1}}}
\expandafter\def\csname PY@tok@nb\endcsname{\def\PY@tc##1{\textcolor[rgb]{0.00,0.50,0.00}{##1}}}
\expandafter\def\csname PY@tok@nf\endcsname{\def\PY@tc##1{\textcolor[rgb]{0.00,0.00,1.00}{##1}}}
\expandafter\def\csname PY@tok@nc\endcsname{\let\PY@bf=\textbf\def\PY@tc##1{\textcolor[rgb]{0.00,0.00,1.00}{##1}}}
\expandafter\def\csname PY@tok@nn\endcsname{\let\PY@bf=\textbf\def\PY@tc##1{\textcolor[rgb]{0.00,0.00,1.00}{##1}}}
\expandafter\def\csname PY@tok@ne\endcsname{\let\PY@bf=\textbf\def\PY@tc##1{\textcolor[rgb]{0.82,0.25,0.23}{##1}}}
\expandafter\def\csname PY@tok@nv\endcsname{\def\PY@tc##1{\textcolor[rgb]{0.10,0.09,0.49}{##1}}}
\expandafter\def\csname PY@tok@no\endcsname{\def\PY@tc##1{\textcolor[rgb]{0.53,0.00,0.00}{##1}}}
\expandafter\def\csname PY@tok@nl\endcsname{\def\PY@tc##1{\textcolor[rgb]{0.63,0.63,0.00}{##1}}}
\expandafter\def\csname PY@tok@ni\endcsname{\let\PY@bf=\textbf\def\PY@tc##1{\textcolor[rgb]{0.60,0.60,0.60}{##1}}}
\expandafter\def\csname PY@tok@na\endcsname{\def\PY@tc##1{\textcolor[rgb]{0.49,0.56,0.16}{##1}}}
\expandafter\def\csname PY@tok@nt\endcsname{\let\PY@bf=\textbf\def\PY@tc##1{\textcolor[rgb]{0.00,0.50,0.00}{##1}}}
\expandafter\def\csname PY@tok@nd\endcsname{\def\PY@tc##1{\textcolor[rgb]{0.67,0.13,1.00}{##1}}}
\expandafter\def\csname PY@tok@s\endcsname{\def\PY@tc##1{\textcolor[rgb]{0.73,0.13,0.13}{##1}}}
\expandafter\def\csname PY@tok@sd\endcsname{\let\PY@it=\textit\def\PY@tc##1{\textcolor[rgb]{0.73,0.13,0.13}{##1}}}
\expandafter\def\csname PY@tok@si\endcsname{\let\PY@bf=\textbf\def\PY@tc##1{\textcolor[rgb]{0.73,0.40,0.53}{##1}}}
\expandafter\def\csname PY@tok@se\endcsname{\let\PY@bf=\textbf\def\PY@tc##1{\textcolor[rgb]{0.73,0.40,0.13}{##1}}}
\expandafter\def\csname PY@tok@sr\endcsname{\def\PY@tc##1{\textcolor[rgb]{0.73,0.40,0.53}{##1}}}
\expandafter\def\csname PY@tok@ss\endcsname{\def\PY@tc##1{\textcolor[rgb]{0.10,0.09,0.49}{##1}}}
\expandafter\def\csname PY@tok@sx\endcsname{\def\PY@tc##1{\textcolor[rgb]{0.00,0.50,0.00}{##1}}}
\expandafter\def\csname PY@tok@m\endcsname{\def\PY@tc##1{\textcolor[rgb]{0.40,0.40,0.40}{##1}}}
\expandafter\def\csname PY@tok@gh\endcsname{\let\PY@bf=\textbf\def\PY@tc##1{\textcolor[rgb]{0.00,0.00,0.50}{##1}}}
\expandafter\def\csname PY@tok@gu\endcsname{\let\PY@bf=\textbf\def\PY@tc##1{\textcolor[rgb]{0.50,0.00,0.50}{##1}}}
\expandafter\def\csname PY@tok@gd\endcsname{\def\PY@tc##1{\textcolor[rgb]{0.63,0.00,0.00}{##1}}}
\expandafter\def\csname PY@tok@gi\endcsname{\def\PY@tc##1{\textcolor[rgb]{0.00,0.63,0.00}{##1}}}
\expandafter\def\csname PY@tok@gr\endcsname{\def\PY@tc##1{\textcolor[rgb]{1.00,0.00,0.00}{##1}}}
\expandafter\def\csname PY@tok@ge\endcsname{\let\PY@it=\textit}
\expandafter\def\csname PY@tok@gs\endcsname{\let\PY@bf=\textbf}
\expandafter\def\csname PY@tok@gp\endcsname{\let\PY@bf=\textbf\def\PY@tc##1{\textcolor[rgb]{0.00,0.00,0.50}{##1}}}
\expandafter\def\csname PY@tok@go\endcsname{\def\PY@tc##1{\textcolor[rgb]{0.53,0.53,0.53}{##1}}}
\expandafter\def\csname PY@tok@gt\endcsname{\def\PY@tc##1{\textcolor[rgb]{0.00,0.27,0.87}{##1}}}
\expandafter\def\csname PY@tok@err\endcsname{\def\PY@bc##1{\setlength{\fboxsep}{0pt}\fcolorbox[rgb]{1.00,0.00,0.00}{1,1,1}{\strut ##1}}}
\expandafter\def\csname PY@tok@kc\endcsname{\let\PY@bf=\textbf\def\PY@tc##1{\textcolor[rgb]{0.00,0.50,0.00}{##1}}}
\expandafter\def\csname PY@tok@kd\endcsname{\let\PY@bf=\textbf\def\PY@tc##1{\textcolor[rgb]{0.00,0.50,0.00}{##1}}}
\expandafter\def\csname PY@tok@kn\endcsname{\let\PY@bf=\textbf\def\PY@tc##1{\textcolor[rgb]{0.00,0.50,0.00}{##1}}}
\expandafter\def\csname PY@tok@kr\endcsname{\let\PY@bf=\textbf\def\PY@tc##1{\textcolor[rgb]{0.00,0.50,0.00}{##1}}}
\expandafter\def\csname PY@tok@bp\endcsname{\def\PY@tc##1{\textcolor[rgb]{0.00,0.50,0.00}{##1}}}
\expandafter\def\csname PY@tok@fm\endcsname{\def\PY@tc##1{\textcolor[rgb]{0.00,0.00,1.00}{##1}}}
\expandafter\def\csname PY@tok@vc\endcsname{\def\PY@tc##1{\textcolor[rgb]{0.10,0.09,0.49}{##1}}}
\expandafter\def\csname PY@tok@vg\endcsname{\def\PY@tc##1{\textcolor[rgb]{0.10,0.09,0.49}{##1}}}
\expandafter\def\csname PY@tok@vi\endcsname{\def\PY@tc##1{\textcolor[rgb]{0.10,0.09,0.49}{##1}}}
\expandafter\def\csname PY@tok@vm\endcsname{\def\PY@tc##1{\textcolor[rgb]{0.10,0.09,0.49}{##1}}}
\expandafter\def\csname PY@tok@sa\endcsname{\def\PY@tc##1{\textcolor[rgb]{0.73,0.13,0.13}{##1}}}
\expandafter\def\csname PY@tok@sb\endcsname{\def\PY@tc##1{\textcolor[rgb]{0.73,0.13,0.13}{##1}}}
\expandafter\def\csname PY@tok@sc\endcsname{\def\PY@tc##1{\textcolor[rgb]{0.73,0.13,0.13}{##1}}}
\expandafter\def\csname PY@tok@dl\endcsname{\def\PY@tc##1{\textcolor[rgb]{0.73,0.13,0.13}{##1}}}
\expandafter\def\csname PY@tok@s2\endcsname{\def\PY@tc##1{\textcolor[rgb]{0.73,0.13,0.13}{##1}}}
\expandafter\def\csname PY@tok@sh\endcsname{\def\PY@tc##1{\textcolor[rgb]{0.73,0.13,0.13}{##1}}}
\expandafter\def\csname PY@tok@s1\endcsname{\def\PY@tc##1{\textcolor[rgb]{0.73,0.13,0.13}{##1}}}
\expandafter\def\csname PY@tok@mb\endcsname{\def\PY@tc##1{\textcolor[rgb]{0.40,0.40,0.40}{##1}}}
\expandafter\def\csname PY@tok@mf\endcsname{\def\PY@tc##1{\textcolor[rgb]{0.40,0.40,0.40}{##1}}}
\expandafter\def\csname PY@tok@mh\endcsname{\def\PY@tc##1{\textcolor[rgb]{0.40,0.40,0.40}{##1}}}
\expandafter\def\csname PY@tok@mi\endcsname{\def\PY@tc##1{\textcolor[rgb]{0.40,0.40,0.40}{##1}}}
\expandafter\def\csname PY@tok@il\endcsname{\def\PY@tc##1{\textcolor[rgb]{0.40,0.40,0.40}{##1}}}
\expandafter\def\csname PY@tok@mo\endcsname{\def\PY@tc##1{\textcolor[rgb]{0.40,0.40,0.40}{##1}}}
\expandafter\def\csname PY@tok@ch\endcsname{\let\PY@it=\textit\def\PY@tc##1{\textcolor[rgb]{0.25,0.50,0.50}{##1}}}
\expandafter\def\csname PY@tok@cm\endcsname{\let\PY@it=\textit\def\PY@tc##1{\textcolor[rgb]{0.25,0.50,0.50}{##1}}}
\expandafter\def\csname PY@tok@cpf\endcsname{\let\PY@it=\textit\def\PY@tc##1{\textcolor[rgb]{0.25,0.50,0.50}{##1}}}
\expandafter\def\csname PY@tok@c1\endcsname{\let\PY@it=\textit\def\PY@tc##1{\textcolor[rgb]{0.25,0.50,0.50}{##1}}}
\expandafter\def\csname PY@tok@cs\endcsname{\let\PY@it=\textit\def\PY@tc##1{\textcolor[rgb]{0.25,0.50,0.50}{##1}}}

\def\PYZbs{\char`\\}
\def\PYZus{\char`\_}
\def\PYZob{\char`\{}
\def\PYZcb{\char`\}}
\def\PYZca{\char`\^}
\def\PYZam{\char`\&}
\def\PYZlt{\char`\<}
\def\PYZgt{\char`\>}
\def\PYZsh{\char`\#}
\def\PYZpc{\char`\%}
\def\PYZdl{\char`\$}
\def\PYZhy{\char`\-}
\def\PYZsq{\char`\'}
\def\PYZdq{\char`\"}
\def\PYZti{\char`\~}
% for compatibility with earlier versions
\def\PYZat{@}
\def\PYZlb{[}
\def\PYZrb{]}
\makeatother


    % Exact colors from NB
    \definecolor{incolor}{rgb}{0.0, 0.0, 0.5}
    \definecolor{outcolor}{rgb}{0.545, 0.0, 0.0}



    
    % Prevent overflowing lines due to hard-to-break entities
    \sloppy 
    % Setup hyperref package
    \hypersetup{
      breaklinks=true,  % so long urls are correctly broken across lines
      colorlinks=true,
      urlcolor=urlcolor,
      linkcolor=linkcolor,
      citecolor=citecolor,
      }
    % Slightly bigger margins than the latex defaults
    
    \geometry{verbose,tmargin=1in,bmargin=1in,lmargin=1in,rmargin=1in}
    
    

    \begin{document}
    
    
    \maketitle
    
    

    
    \begin{Verbatim}[commandchars=\\\{\}]
{\color{incolor}In [{\color{incolor}21}]:} \PY{o}{\PYZpc{}}\PY{k}{precision} \PYZpc{}g
         \PY{o}{\PYZpc{}}\PY{k}{matplotlib} inline
         \PY{o}{\PYZpc{}}\PY{k}{config} InlineBackend.figure\PYZus{}format = \PYZsq{}retina\PYZsq{}
         
         \PY{k+kn}{from} \PY{n+nn}{math} \PY{k}{import} \PY{n}{sqrt}\PY{p}{,} \PY{n}{pi}\PY{p}{,} \PY{n}{sin}\PY{p}{,} \PY{n}{cos}\PY{p}{,} \PY{n}{floor}\PY{p}{,} \PY{n}{exp}
         \PY{k+kn}{from} \PY{n+nn}{cmath} \PY{k}{import} \PY{n}{exp} \PY{k}{as} \PY{n}{cexp}
         \PY{k+kn}{import} \PY{n+nn}{numpy} \PY{k}{as} \PY{n+nn}{np}
         \PY{k+kn}{from} \PY{n+nn}{numpy} \PY{k}{import} \PY{n}{linalg} \PY{k}{as} \PY{n}{LA}
         \PY{k+kn}{from} \PY{n+nn}{scipy} \PY{k}{import} \PY{n}{constants} \PY{k}{as} \PY{n}{con}
         \PY{k+kn}{import} \PY{n+nn}{matplotlib}\PY{n+nn}{.}\PY{n+nn}{pyplot} \PY{k}{as} \PY{n+nn}{plt}
         \PY{k+kn}{from} \PY{n+nn}{dcst} \PY{k}{import} \PY{n}{dst}\PY{p}{,}\PY{n}{idst}
\end{Verbatim}

    \section{CP 9.2}\label{cp-9.2}

The Jacobi method that the book described is quite slow. By their
measure, it took about 10 minutes to calculate the voltages at a 100x100
grid with fixed voltage edges. It offers a different solution that
speeds up the calculation in two ways: by "overrelaxing" and by
continuously updating the values. This is known as the Gauss-Seidel
method. Using the Laplace wave equation

\[\nabla^2 \phi = \frac{\partial^2\phi}{\partial x^2} + \frac{\partial^2\phi}{\partial y^2} = 0,\]

we can solve for the values of the electric potential on the interior of
a square region with side length \(L\) given the boundary conditions

\[\phi(x,L) = V \qquad \text{ and }\qquad \phi(x,0) = \phi(0,y) = \phi(L,y) = 0.\]

The Gauss-Seidel method iterates through and solves

\begin{equation}
\phi(x,y) \leftarrow \frac{1+\omega}{4} \bigr[\phi(x+a,y) + \phi(x-a,y) + \phi(x,y+a) + \phi(x,y-a)\bigr] - \omega\phi(x,y)
\end{equation}

to within a target accuracy \(\delta.\)

    \begin{Verbatim}[commandchars=\\\{\}]
{\color{incolor}In [{\color{incolor}22}]:} \PY{o}{\PYZpc{}\PYZpc{}}\PY{k}{time}
         
         M = 100         \PYZsh{}grid squares on a side
         V = 1.0         \PYZsh{}voltage at top wall
         target = 1e\PYZhy{}6   \PYZsh{}target accuracy
         w = 0.94 \PYZsh{}overrelaxation constant
         
         \PYZsh{}create arrays to hold potential values
         phi = np.zeros([M+1,M+1],float)
         phi[0,:] = V
         d1 = np.empty([M+1,M+1],float)
         
         d = 1.0
         while d\PYZgt{}target:
         
             \PYZsh{}calculate new values of the potential
             for i in range(M+1):
                 for j in range(M+1):
                     \PYZsh{}because no phiprime, calculates differences
                     d1[i,j] = phi[i,j]
                     
                     if i==0 or i==M or j==0 or j==M:
                         phi[i,j] = phi[i,j]
                     else:
                         phi[i,j] = (1+w)*(phi[i+1,j] + phi[i\PYZhy{}1,j] \PYZbs{}
                                          + phi[i,j+1] + phi[i,j\PYZhy{}1])/4\PYZbs{}
                                  \PYZhy{} w*phi[i,j]
                     d1[i,j] \PYZhy{}= phi[i,j]
         
             \PYZsh{} Calculate maximum difference from old values
             d = np.max(abs(d1))
         
         \PYZsh{}makes density plot
         fig1, ax1 = plt.subplots(1, 1, figsize = (5, 5))
         
         ax1.imshow(phi, cmap=\PYZsq{}gray\PYZsq{})
         ax1.set\PYZus{}title(\PYZdq{}Voltage in Two Dimensions\PYZdq{})
\end{Verbatim}

    \begin{Verbatim}[commandchars=\\\{\}]
CPU times: user 6.44 s, sys: 73.8 ms, total: 6.51 s
Wall time: 6.76 s

    \end{Verbatim}

    \begin{center}
    \adjustimage{max size={0.9\linewidth}{0.9\paperheight}}{output_2_1.png}
    \end{center}
    { \hspace*{\fill} \\}
    
    After \(\omega\) is increased beyond 0.94, the time it takes for the
program to run again begins to increase. However as promised, the 5.9s
it took to solve and plot the original partial differential equation is
far less than the original 10 minutes with the Jacobi method.

    \section{CP 9.3}\label{cp-9.3}

We can solve this exercise with the same method as above, simply
adjusting boundary conditions. The updated boundary conditions (in SI
Units) are

\[\phi(x,0.1) = \phi(x,0) = \phi(0,y) = \phi(0.1,y) = 0, \qquad \phi(0.02,0.02<y<0.08) = 1, \text{ and } \qquad \phi(0.08,0.02<y<0.08) = -1.\]

    \begin{Verbatim}[commandchars=\\\{\}]
{\color{incolor}In [{\color{incolor}23}]:} \PY{o}{\PYZpc{}\PYZpc{}}\PY{k}{time}
         
         M = 100         \PYZsh{}grid squares on a side
         V = 1.0         \PYZsh{}voltage at top wall
         target = 1e\PYZhy{}6   \PYZsh{}target accuracy
         w = 0.94 \PYZsh{}overrelaxation constant
         
         \PYZsh{}create arrays to hold potential values
         phi = np.zeros([M+1,M+1],float)
         phi[19:79,19] = V
         phi[19:79,79] = \PYZhy{}V
         d1 = np.empty([M+1,M+1],float)
         
         d = 1.0
         while d\PYZgt{}target:
         
             \PYZsh{}calculate new values of the potential
             for i in range(M+1):
                 for j in range(M+1):
                     \PYZsh{}because no phiprime, calculates differences
                     d1[i,j] = phi[i,j]
                     
                     if i==0 or i==M or j==0 or j==M:
                         phi[i,j] = phi[i,j]
                     elif j==19 and i \PYZgt{} 19 and i \PYZlt{} 79:
                         phi[i,j] = phi[i,j]
                     elif j==79 and i \PYZgt{} 19 and i \PYZlt{} 79:
                         phi[i,j] = phi[i,j]
                     else:
                         phi[i,j] = (1+w)*(phi[i+1,j] + phi[i\PYZhy{}1,j] \PYZbs{}
                                          + phi[i,j+1] + phi[i,j\PYZhy{}1])/4\PYZbs{}
                                  \PYZhy{} w*phi[i,j]
                     d1[i,j] \PYZhy{}= phi[i,j]
         
             \PYZsh{} Calculate maximum difference from old values
             d = np.max(abs(d1))
         
         \PYZsh{}makes density plot
         fig2, ax2 = plt.subplots(1, 1, figsize = (5,5))
         
         ax2.imshow(phi, cmap=\PYZsq{}gray\PYZsq{})
         ax2.set\PYZus{}title(\PYZdq{}Modeling a Capacitor\PYZdq{})
\end{Verbatim}

    \begin{Verbatim}[commandchars=\\\{\}]
CPU times: user 8.46 s, sys: 89.9 ms, total: 8.55 s
Wall time: 9.56 s

    \end{Verbatim}

    \begin{center}
    \adjustimage{max size={0.9\linewidth}{0.9\paperheight}}{output_5_1.png}
    \end{center}
    { \hspace*{\fill} \\}
    
    \section{CP 9.4 Thermal diffusion in the Earth's
crust}\label{cp-9.4-thermal-diffusion-in-the-earths-crust}

This problem has us solving a PDE initial value problem as opposed to a
boundary value problem, using the FTCS method. Specifically one where
the initial boundary condition is not constant. This problem looks at
the heat that diffuses into the Earth's crust as temperature varies with
the seasons according to

\[T_0(t) = A + B\sin {2\pi t\over\tau}\]

for \(\tau = 365\) days, \(A = 10^\text{o}C,\) and \(B = 12^\text{o}C.\)
This model assumes that temperature 20m below the Earth's surface is a
constant \(11^\text{o}C\) and that the thermal diffusivity constant
\(D = 0.1\) m\(^2\)day\(^{-1}\). Initially, the temperature everywhere
between the surface and a depth of 20 m is at temperature
\(10^\text{o}C.\)

    \begin{Verbatim}[commandchars=\\\{\}]
{\color{incolor}In [{\color{incolor}24}]:} \PY{o}{\PYZpc{}\PYZpc{}}\PY{k}{time}
         \PYZsh{} Constants 
         L = 20        \PYZsh{} Thickness of crust in meters
         D = 0.1       \PYZsh{} Thermal diffusivity
         N = 1000      \PYZsh{} Number of divisions in grid
         depth = np.linspace(0,L,N+1)
         a = L/N       \PYZsh{} Grid spacing
         h = 1e\PYZhy{}3      \PYZsh{} Time\PYZhy{}step
         epsilon = h/1000
         
         \PYZsh{}constant lower boundary temp in Celcius
         Tlo = 11.0
         \PYZsh{}starting intermediate temps
         Tmid = 10.0
         \PYZsh{}changing boundary temperature
             \PYZsh{}constants
         A = 10
         B = 12
         tau = 365
         def T0(t):
             return A + B*sin(2*pi*t/tau)
         
         t1 = 9.25*tau
         t2 = 9.50*tau
         t3 = 9.75*tau
         t4 = 10.0*tau
         tend = t4 + epsilon
         
         \PYZsh{} Create arrays
         T = np.empty(N+1,float)
         T[N] = Tlo
         T[1:N] = Tmid
         Tp = np.empty(N+1,float)
         Tp[0] = T0(0)
         Tp[N] = Tlo
         
         fig3, ax3 = plt.subplots(1, 1, figsize = (16, 4))
         
         t = 0.0
         c = h*D/(a**2)
         while t\PYZlt{}tend:
             
             T[0] = T0(t)
         
             \PYZsh{} Calculate the new values of T
             Tp[1:N] = T[1:N] + c*(T[0:N\PYZhy{}1] + T[2:N+1] \PYZhy{} 2*T[1:N])
             
             T,Tp = Tp,T
             t += h
         
             \PYZsh{} Make plots at the given times
             if abs(t\PYZhy{}t1) \PYZlt{} epsilon:
                 ax3.plot(depth,T,label=\PYZsq{}Summer\PYZsq{})
             if abs(t\PYZhy{}t2) \PYZlt{} epsilon:
                 ax3.plot(depth,T,label=\PYZsq{}Fall\PYZsq{})
             if abs(t\PYZhy{}t3) \PYZlt{} epsilon:
                 ax3.plot(depth,T,label=\PYZsq{}Winter\PYZsq{})
             if abs(t\PYZhy{}t4) \PYZlt{} epsilon:
                 ax3.plot(depth,T,label=\PYZsq{}Spring\PYZsq{})
         
         ax3.set\PYZus{}xlabel(\PYZdq{}Depth (m)\PYZdq{})
         ax3.set\PYZus{}ylabel(\PYZdq{}Temperature (\PYZdl{}\PYZca{}oC\PYZdl{})\PYZdq{})
         ax3.set\PYZus{}title(\PYZdq{}Temperature vs. Depth\PYZdq{})
         ax3.legend()
         plt.show()
\end{Verbatim}

    \begin{center}
    \adjustimage{max size={0.9\linewidth}{0.9\paperheight}}{output_7_0.png}
    \end{center}
    { \hspace*{\fill} \\}
    
    \begin{Verbatim}[commandchars=\\\{\}]
CPU times: user 35.7 s, sys: 104 ms, total: 35.8 s
Wall time: 36.2 s

    \end{Verbatim}

    We can see that the temperature near the surface of the Earth's crust is
the most sensitive to seasonal temperature swings. As the depth
approaches 2 m, the temperature of the crust is unaffected at this level
of detail. Also the seasons in the key were labeled assuming this is in
the Northern Hemisphere.

    \section{CP 9.5 FTCS solution of the wave
equation}\label{cp-9.5-ftcs-solution-of-the-wave-equation}

This problem looks at the behavior of a string after it's plucked and
how the wave propagates along it, holding both ends fixed. Despite
knowing that the FTCS method produces unstable solutions for anything
longer than \emph{short} time intervals, we'll solve this system using
the method for the times \(t =\) 2 ms, 50 ms, and 100 ms.

To model this system, we start with the wave equation

\[\frac{\partial^2\phi}{\partial t^2} = v^2 \frac{\partial^2\phi}{\partial x^2}.\]

Then we split this into 2 first order differential equations for each
grid point, so we can write the two relations

\[\frac{\text{d}\phi}{\text{d}t} = \psi(x,t) \qquad \frac{\text{d}\psi}{\text{d}t} = \frac{v^2}{a^2}\bigr[\phi(x+a,t) + \phi(x-a,t) - 2\phi(x,t)\bigr].\]

Then we can write the program to solve the equations

\begin{equation}
\phi(x,t+h) = \phi(x,t) + h\psi(x,t)
\end{equation}

\begin{equation*}
\psi(x,t+h) = \psi(x,t) + h\frac{v^2}{a^2}\bigr[\phi(x+a,t) + \phi(x-a,t) - 2\phi(x,t)\bigr].
\end{equation*}

Now for this specific problem, we define
\[\psi(x) = C {x(L-x)\over L^2} \exp \biggl[ -{(x-d)^2\over2\sigma^2} \biggr].\]

The initial condition for this problem is \[\phi(x,0)=0.\]

    \begin{Verbatim}[commandchars=\\\{\}]
{\color{incolor}In [{\color{incolor}25}]:} \PY{o}{\PYZpc{}\PYZpc{}}\PY{k}{time}
         \PYZsh{} Constants 
         L = 1         \PYZsh{}length of string (m)
         d = 0.1       \PYZsh{}point of contact (m)
         v = 100       \PYZsh{}wave speed (ms\PYZca{}\PYZhy{}1)
         N = 100       \PYZsh{} Number of divisions in grid
         length = np.linspace(0,L,N+1)
         a = L/N       \PYZsh{} Grid spacing
         h = 1e\PYZhy{}6      \PYZsh{} Time\PYZhy{}step
         
         epsilon = h/1000
         
         C = 1 \PYZsh{}(ms\PYZca{}\PYZhy{}1)
         s = 0.3 \PYZsh{}sigma in the exercise (m)
         
         def wave(x):
             p1 = C*(x*(L\PYZhy{}x)) / L**2
             p2 = exp(\PYZhy{}(x\PYZhy{}d)**2 / (2*s**2))
             return p1*p2
         
         t1 = 0.002
         t2 = 0.050
         t3 = 0.100
         tend = t3 + epsilon
         
         \PYZsh{} Create arrays
         phi = np.empty(N+1,float)
         phi[:] = 0.0
         phip = np.empty(N+1,float)
         phip[:] = 0.0
         
         psi = np.empty(N+1,float)
         psi[:] = 0.0
         psip = np.empty(N+1,float)
         psip[:] = 0.0
         
         fig4, ax4 = plt.subplots(3, 1, figsize = (16, 12))
         
         \PYZsh{}setting wave velocity
         for i in range(len(psi)):
             psi[i] = wave(i*L/100)
         
         t = 0.0
         c = (h*v**2)/(a*a)
         while t\PYZlt{}tend:
             
             \PYZsh{}hold ends fixed
             phi[0] = 0.0
             phi[N] = 0.0
             
             \PYZsh{}solving differential equations
             phip[1:N] = phi[1:N] + h*psi[1:N]
             
             psip[1:N] = psi[1:N]\PYZbs{}
                       + c*(phi[0:N\PYZhy{}1] + phi[2:N+1] \PYZhy{} 2*phi[1:N])
             
             phi,phip = phip,phi
             psi,psip = psip,psi
             t += h
         
             \PYZsh{}plots displacement
             if abs(t\PYZhy{}t1) \PYZlt{} epsilon:
                 ax4[0].plot(length,phi,label=\PYZdq{}t = 2 ms\PYZdq{})
             if abs(t\PYZhy{}t2) \PYZlt{} epsilon:
                 ax4[1].plot(length,phi,label=\PYZdq{}t = 50 ms\PYZdq{})
             if abs(t\PYZhy{}t3) \PYZlt{} epsilon:
                 ax4[2].plot(length,phi,label=\PYZdq{}t = 100 ms\PYZdq{})
         
         ax4[2].set\PYZus{}xlabel(\PYZdq{}Length (m)\PYZdq{})
         ax4[0].set\PYZus{}ylabel(\PYZdq{}Displacement (m)\PYZdq{})
         ax4[1].set\PYZus{}ylabel(\PYZdq{}Displacement (m)\PYZdq{})
         ax4[2].set\PYZus{}ylabel(\PYZdq{}Displacement (m)\PYZdq{})
         ax4[0].set\PYZus{}title(\PYZdq{}String Displacement at Different Times\PYZdq{})
         ax4[0].legend()
         ax4[1].legend()
         ax4[2].legend()
         plt.show()
\end{Verbatim}

    \begin{center}
    \adjustimage{max size={0.9\linewidth}{0.9\paperheight}}{output_10_0.png}
    \end{center}
    { \hspace*{\fill} \\}
    
    \begin{Verbatim}[commandchars=\\\{\}]
CPU times: user 1.91 s, sys: 46.2 ms, total: 1.95 s
Wall time: 1.94 s

    \end{Verbatim}

    \section{CP 9.9}\label{cp-9.9}

Here we attempt to solve the time-dependent Schrodinger equation
\[-{\hbar^2\over2M} {\partial^2\psi\over\partial x^2} = \text{i}\hbar {\partial\psi\over\partial t}\]

for a particle in an infinite well of width \(L.\) One known
unnormalized solution is

\[\psi_k(x,t) = \sin \biggl( {\pi k x\over L} \biggr)\,\text{e}^{\text{i} Et/\hbar}, \qquad \text{where } \qquad E = {\pi^2\hbar^2k^2\over2ML^2}.\]

We can take a linear combination of these to express the wavefunction as

\begin{equation}
\psi(x_n,t) = {1\over N}
              \sum_{k=1}^{N-1} b_k \sin \biggl( {\pi k n\over N} \biggr)\>
              \exp \biggl( \text{i}{\pi^2\hbar k^2\over2ML^2} t \biggr).
              \end{equation}

for all \(x_n = \frac{nL}{N}\) along the length of the the well.
Furthermore, the coefficients \(b_k\) are complex and can be expanded to
separate the real and imaginary components such that
\(b_k = \alpha_k + \text{i}\eta_k.\) These arrays of values
(\(\alpha_k\) and \(\eta_k\)) can be calculated using discrete sine
transforms. Once these arrays are found, the real portion of the
wavefunction (the part we're interested in) can be found by the relation

\begin{equation}
\text{Re}\ \psi(x_n,t) = {1\over N} \sum_{k=1}^{N-1}
            \biggl[ \alpha_k \cos \biggl( {\pi^2\hbar k^2\over2ML^2} t \biggr)
            - \eta_k \sin \biggl( {\pi^2\hbar k^2\over2ML^2} t \biggr) \biggr]
            \sin \biggl( {\pi k n\over N} \biggr).
            \end{equation}

    \begin{Verbatim}[commandchars=\\\{\}]
{\color{incolor}In [{\color{incolor}26}]:} \PY{c+c1}{\PYZsh{} Constants }
         \PY{n}{L} \PY{o}{=} \PY{l+m+mf}{1e\PYZhy{}8}      \PY{c+c1}{\PYZsh{}width of box (m)}
         \PY{n}{M} \PY{o}{=} \PY{n}{con}\PY{o}{.}\PY{n}{m\PYZus{}e}   \PY{c+c1}{\PYZsh{}electron mass (m)}
         \PY{n}{x0} \PY{o}{=} \PY{n}{L}\PY{o}{/}\PY{l+m+mi}{2}      \PY{c+c1}{\PYZsh{}wave speed (ms\PYZca{}\PYZhy{}1)}
         \PY{n}{N} \PY{o}{=} \PY{l+m+mi}{1000}      \PY{c+c1}{\PYZsh{}number of divisions in grid}
         \PY{n}{kappa} \PY{o}{=} \PY{l+m+mf}{5e10}  \PY{c+c1}{\PYZsh{}(m\PYZca{}\PYZhy{}1)}
         \PY{n}{s} \PY{o}{=} \PY{l+m+mf}{1e\PYZhy{}10}     \PY{c+c1}{\PYZsh{}sigma (m)}
         \PY{n}{hbar} \PY{o}{=} \PY{n}{con}\PY{o}{.}\PY{n}{hbar}  \PY{c+c1}{\PYZsh{} Time\PYZhy{}step}
         \PY{n}{x} \PY{o}{=} \PY{n}{np}\PY{o}{.}\PY{n}{linspace}\PY{p}{(}\PY{l+m+mi}{0}\PY{p}{,}\PY{n}{L}\PY{p}{,}\PY{n}{N}\PY{o}{+}\PY{l+m+mi}{1}\PY{p}{)}
         
         \PY{k}{def} \PY{n+nf}{E}\PY{p}{(}\PY{n}{k}\PY{p}{)}\PY{p}{:}
             \PY{l+s+sd}{\PYZdq{}\PYZdq{}\PYZdq{}Returns energy of particle based on k\PYZdq{}\PYZdq{}\PYZdq{}}
             \PY{n}{num} \PY{o}{=} \PY{n}{pi}\PY{o}{*}\PY{o}{*}\PY{l+m+mi}{2} \PY{o}{*} \PY{n}{hbar}\PY{o}{*}\PY{o}{*}\PY{l+m+mi}{2} \PY{o}{*} \PY{n}{k}\PY{o}{*}\PY{o}{*}\PY{l+m+mi}{2}
             \PY{n}{den} \PY{o}{=} \PY{l+m+mi}{2} \PY{o}{*} \PY{n}{M} \PY{o}{*} \PY{n}{L}\PY{o}{*}\PY{o}{*}\PY{l+m+mi}{2}
             \PY{k}{return} \PY{n}{num} \PY{o}{/} \PY{n}{den}
         
         \PY{c+c1}{\PYZsh{}initial condition wavefunction}
         \PY{k}{def} \PY{n+nf}{wave0}\PY{p}{(}\PY{n}{x}\PY{p}{)}\PY{p}{:}
             \PY{k}{return} \PY{n}{np}\PY{o}{.}\PY{n}{exp}\PY{p}{(}\PY{o}{\PYZhy{}}\PY{p}{(}\PY{n}{x}\PY{o}{\PYZhy{}}\PY{n}{x0}\PY{p}{)}\PY{o}{*}\PY{o}{*}\PY{l+m+mi}{2}\PY{o}{/}\PY{p}{(}\PY{l+m+mi}{2}\PY{o}{*}\PY{n}{s}\PY{o}{*}\PY{o}{*}\PY{l+m+mi}{2}\PY{p}{)} \PY{o}{+} \PY{l+m+mi}{1}\PY{n}{j}\PY{o}{*}\PY{n}{kappa}\PY{o}{*}\PY{n}{x}\PY{p}{)}
         
         \PY{n}{psi0} \PY{o}{=} \PY{n}{wave0}\PY{p}{(}\PY{n}{x}\PY{p}{)}
         
         \PY{n}{alpha} \PY{o}{=} \PY{n}{dst}\PY{p}{(}\PY{n}{np}\PY{o}{.}\PY{n}{real}\PY{p}{(}\PY{n}{psi0}\PY{p}{)}\PY{p}{)}
         \PY{n}{eta} \PY{o}{=} \PY{n}{dst}\PY{p}{(}\PY{n}{np}\PY{o}{.}\PY{n}{imag}\PY{p}{(}\PY{n}{psi0}\PY{p}{)}\PY{p}{)}
         
         \PY{n}{t0} \PY{o}{=} \PY{l+m+mf}{0.0}
         \PY{n}{t1} \PY{o}{=} \PY{l+m+mf}{1e\PYZhy{}16}
         \PY{n}{t2} \PY{o}{=} \PY{l+m+mf}{2e\PYZhy{}16}
         \PY{n}{t3} \PY{o}{=} \PY{l+m+mf}{4e\PYZhy{}16}
         
         \PY{k}{def} \PY{n+nf}{psi}\PY{p}{(}\PY{n}{t}\PY{p}{)}\PY{p}{:}
             \PY{l+s+sd}{\PYZdq{}\PYZdq{}\PYZdq{}Finds the real component of the wavefunction}
         \PY{l+s+sd}{    according to the above equation at time t\PYZdq{}\PYZdq{}\PYZdq{}}  
             \PY{n}{y} \PY{o}{=} \PY{n}{np}\PY{o}{.}\PY{n}{zeros\PYZus{}like}\PY{p}{(}\PY{n}{x}\PY{p}{)}
             \PY{k}{for} \PY{n}{k} \PY{o+ow}{in} \PY{n+nb}{range}\PY{p}{(}\PY{n}{N}\PY{o}{+}\PY{l+m+mi}{1}\PY{p}{)}\PY{p}{:}
                 \PY{n}{y}\PY{p}{[}\PY{n}{k}\PY{p}{]} \PY{o}{=} \PY{n}{alpha}\PY{p}{[}\PY{n}{k}\PY{p}{]} \PY{o}{*} \PY{n}{cos}\PY{p}{(}\PY{n}{E}\PY{p}{(}\PY{n}{k}\PY{p}{)}\PY{o}{*}\PY{n}{t}\PY{o}{/}\PY{n}{hbar}\PY{p}{)}\PYZbs{}
                      \PY{o}{\PYZhy{}} \PY{n}{eta}\PY{p}{[}\PY{n}{k}\PY{p}{]}   \PY{o}{*} \PY{n}{sin}\PY{p}{(}\PY{n}{E}\PY{p}{(}\PY{n}{k}\PY{p}{)}\PY{o}{*}\PY{n}{t}\PY{o}{/}\PY{n}{hbar}\PY{p}{)}
             
             \PY{k}{return} \PY{n}{idst}\PY{p}{(}\PY{n}{y}\PY{p}{)} \PY{c+c1}{\PYZsh{}/ N}
         
         \PY{n}{fig5}\PY{p}{,} \PY{n}{ax5} \PY{o}{=} \PY{n}{plt}\PY{o}{.}\PY{n}{subplots}\PY{p}{(}\PY{l+m+mi}{1}\PY{p}{,} \PY{l+m+mi}{1}\PY{p}{,} \PY{n}{figsize} \PY{o}{=} \PY{p}{(}\PY{l+m+mi}{12}\PY{p}{,} \PY{l+m+mi}{6}\PY{p}{)}\PY{p}{)}
         
         \PY{n}{psi0} \PY{o}{=} \PY{n}{psi}\PY{p}{(}\PY{n}{t0}\PY{p}{)}
         \PY{n}{psi1} \PY{o}{=} \PY{n}{psi}\PY{p}{(}\PY{n}{t1}\PY{p}{)}
         \PY{n}{psi2} \PY{o}{=} \PY{n}{psi}\PY{p}{(}\PY{n}{t2}\PY{p}{)}
         \PY{n}{psi3} \PY{o}{=} \PY{n}{psi}\PY{p}{(}\PY{n}{t3}\PY{p}{)}
         \PY{n}{psi4} \PY{o}{=} \PY{n}{psi}\PY{p}{(}\PY{n}{t4}\PY{p}{)}
         
         \PY{n}{ax5}\PY{o}{.}\PY{n}{plot}\PY{p}{(}\PY{n}{x}\PY{p}{,} \PY{n}{psi0}\PY{p}{,}\PY{n}{label}\PY{o}{=}\PY{l+s+s1}{\PYZsq{}}\PY{l+s+s1}{t = }\PY{l+s+si}{\PYZob{}:1.0f\PYZcb{}}\PY{l+s+s1}{ s}\PY{l+s+s1}{\PYZsq{}}\PY{o}{.}\PY{n}{format}\PY{p}{(}\PY{n}{t0}\PY{p}{)}\PY{p}{)}
         \PY{n}{ax5}\PY{o}{.}\PY{n}{plot}\PY{p}{(}\PY{n}{x}\PY{p}{,} \PY{n}{psi1}\PY{p}{,}\PY{n}{label}\PY{o}{=}\PY{l+s+s1}{\PYZsq{}}\PY{l+s+s1}{t = }\PY{l+s+si}{\PYZob{}:4.0e\PYZcb{}}\PY{l+s+s1}{ s}\PY{l+s+s1}{\PYZsq{}}\PY{o}{.}\PY{n}{format}\PY{p}{(}\PY{n}{t1}\PY{p}{)}\PY{p}{)}
         \PY{n}{ax5}\PY{o}{.}\PY{n}{plot}\PY{p}{(}\PY{n}{x}\PY{p}{,} \PY{n}{psi2}\PY{p}{,}\PY{n}{label}\PY{o}{=}\PY{l+s+s1}{\PYZsq{}}\PY{l+s+s1}{t = }\PY{l+s+si}{\PYZob{}:4.0e\PYZcb{}}\PY{l+s+s1}{ s}\PY{l+s+s1}{\PYZsq{}}\PY{o}{.}\PY{n}{format}\PY{p}{(}\PY{n}{t2}\PY{p}{)}\PY{p}{)}
         \PY{n}{ax5}\PY{o}{.}\PY{n}{plot}\PY{p}{(}\PY{n}{x}\PY{p}{,} \PY{n}{psi3}\PY{p}{,}\PY{n}{label}\PY{o}{=}\PY{l+s+s1}{\PYZsq{}}\PY{l+s+s1}{t = }\PY{l+s+si}{\PYZob{}:4.0e\PYZcb{}}\PY{l+s+s1}{ s}\PY{l+s+s1}{\PYZsq{}}\PY{o}{.}\PY{n}{format}\PY{p}{(}\PY{n}{t3}\PY{p}{)}\PY{p}{)}
         \PY{n}{ax5}\PY{o}{.}\PY{n}{set\PYZus{}title}\PY{p}{(}\PY{l+s+s2}{\PYZdq{}}\PY{l+s+s2}{Wavefunction over Time}\PY{l+s+s2}{\PYZdq{}}\PY{p}{)}
         \PY{n}{ax5}\PY{o}{.}\PY{n}{legend}\PY{p}{(}\PY{p}{)}
         
         \PY{n}{plt}\PY{o}{.}\PY{n}{show}\PY{p}{(}\PY{p}{)}
\end{Verbatim}

    \begin{center}
    \adjustimage{max size={0.9\linewidth}{0.9\paperheight}}{output_12_0.png}
    \end{center}
    { \hspace*{\fill} \\}
    
    From the graphs, we can see that over time the wavefunction disperses
and initially travels leftward. It will do this until it hits the
boundary of the infinite well at which point, its uncertainty will
reduce and then it will reverse direction and travel rightwards.

    \begin{Verbatim}[commandchars=\\\{\}]
{\color{incolor}In [{\color{incolor} }]:} 
\end{Verbatim}


    % Add a bibliography block to the postdoc
    
    
    
    \end{document}

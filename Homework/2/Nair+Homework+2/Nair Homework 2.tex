
% Default to the notebook output style

    


% Inherit from the specified cell style.




    
\documentclass[11pt]{article}

    
    
    \usepackage[T1]{fontenc}
    % Nicer default font (+ math font) than Computer Modern for most use cases
    \usepackage{mathpazo}

    % Basic figure setup, for now with no caption control since it's done
    % automatically by Pandoc (which extracts ![](path) syntax from Markdown).
    \usepackage{graphicx}
    % We will generate all images so they have a width \maxwidth. This means
    % that they will get their normal width if they fit onto the page, but
    % are scaled down if they would overflow the margins.
    \makeatletter
    \def\maxwidth{\ifdim\Gin@nat@width>\linewidth\linewidth
    \else\Gin@nat@width\fi}
    \makeatother
    \let\Oldincludegraphics\includegraphics
    % Set max figure width to be 80% of text width, for now hardcoded.
    \renewcommand{\includegraphics}[1]{\Oldincludegraphics[width=.8\maxwidth]{#1}}
    % Ensure that by default, figures have no caption (until we provide a
    % proper Figure object with a Caption API and a way to capture that
    % in the conversion process - todo).
    \usepackage{caption}
    \DeclareCaptionLabelFormat{nolabel}{}
    \captionsetup{labelformat=nolabel}

    \usepackage{adjustbox} % Used to constrain images to a maximum size 
    \usepackage{xcolor} % Allow colors to be defined
    \usepackage{enumerate} % Needed for markdown enumerations to work
    \usepackage{geometry} % Used to adjust the document margins
    \usepackage{amsmath} % Equations
    \usepackage{amssymb} % Equations
    \usepackage{textcomp} % defines textquotesingle
    % Hack from http://tex.stackexchange.com/a/47451/13684:
    \AtBeginDocument{%
        \def\PYZsq{\textquotesingle}% Upright quotes in Pygmentized code
    }
    \usepackage{upquote} % Upright quotes for verbatim code
    \usepackage{eurosym} % defines \euro
    \usepackage[mathletters]{ucs} % Extended unicode (utf-8) support
    \usepackage[utf8x]{inputenc} % Allow utf-8 characters in the tex document
    \usepackage{fancyvrb} % verbatim replacement that allows latex
    \usepackage{grffile} % extends the file name processing of package graphics 
                         % to support a larger range 
    % The hyperref package gives us a pdf with properly built
    % internal navigation ('pdf bookmarks' for the table of contents,
    % internal cross-reference links, web links for URLs, etc.)
    \usepackage{hyperref}
    \usepackage{longtable} % longtable support required by pandoc >1.10
    \usepackage{booktabs}  % table support for pandoc > 1.12.2
    \usepackage[inline]{enumitem} % IRkernel/repr support (it uses the enumerate* environment)
    \usepackage[normalem]{ulem} % ulem is needed to support strikethroughs (\sout)
                                % normalem makes italics be italics, not underlines
    \usepackage{mathrsfs}
    

    
    
    % Colors for the hyperref package
    \definecolor{urlcolor}{rgb}{0,.145,.698}
    \definecolor{linkcolor}{rgb}{.71,0.21,0.01}
    \definecolor{citecolor}{rgb}{.12,.54,.11}

    % ANSI colors
    \definecolor{ansi-black}{HTML}{3E424D}
    \definecolor{ansi-black-intense}{HTML}{282C36}
    \definecolor{ansi-red}{HTML}{E75C58}
    \definecolor{ansi-red-intense}{HTML}{B22B31}
    \definecolor{ansi-green}{HTML}{00A250}
    \definecolor{ansi-green-intense}{HTML}{007427}
    \definecolor{ansi-yellow}{HTML}{DDB62B}
    \definecolor{ansi-yellow-intense}{HTML}{B27D12}
    \definecolor{ansi-blue}{HTML}{208FFB}
    \definecolor{ansi-blue-intense}{HTML}{0065CA}
    \definecolor{ansi-magenta}{HTML}{D160C4}
    \definecolor{ansi-magenta-intense}{HTML}{A03196}
    \definecolor{ansi-cyan}{HTML}{60C6C8}
    \definecolor{ansi-cyan-intense}{HTML}{258F8F}
    \definecolor{ansi-white}{HTML}{C5C1B4}
    \definecolor{ansi-white-intense}{HTML}{A1A6B2}
    \definecolor{ansi-default-inverse-fg}{HTML}{FFFFFF}
    \definecolor{ansi-default-inverse-bg}{HTML}{000000}

    % commands and environments needed by pandoc snippets
    % extracted from the output of `pandoc -s`
    \providecommand{\tightlist}{%
      \setlength{\itemsep}{0pt}\setlength{\parskip}{0pt}}
    \DefineVerbatimEnvironment{Highlighting}{Verbatim}{commandchars=\\\{\}}
    % Add ',fontsize=\small' for more characters per line
    \newenvironment{Shaded}{}{}
    \newcommand{\KeywordTok}[1]{\textcolor[rgb]{0.00,0.44,0.13}{\textbf{{#1}}}}
    \newcommand{\DataTypeTok}[1]{\textcolor[rgb]{0.56,0.13,0.00}{{#1}}}
    \newcommand{\DecValTok}[1]{\textcolor[rgb]{0.25,0.63,0.44}{{#1}}}
    \newcommand{\BaseNTok}[1]{\textcolor[rgb]{0.25,0.63,0.44}{{#1}}}
    \newcommand{\FloatTok}[1]{\textcolor[rgb]{0.25,0.63,0.44}{{#1}}}
    \newcommand{\CharTok}[1]{\textcolor[rgb]{0.25,0.44,0.63}{{#1}}}
    \newcommand{\StringTok}[1]{\textcolor[rgb]{0.25,0.44,0.63}{{#1}}}
    \newcommand{\CommentTok}[1]{\textcolor[rgb]{0.38,0.63,0.69}{\textit{{#1}}}}
    \newcommand{\OtherTok}[1]{\textcolor[rgb]{0.00,0.44,0.13}{{#1}}}
    \newcommand{\AlertTok}[1]{\textcolor[rgb]{1.00,0.00,0.00}{\textbf{{#1}}}}
    \newcommand{\FunctionTok}[1]{\textcolor[rgb]{0.02,0.16,0.49}{{#1}}}
    \newcommand{\RegionMarkerTok}[1]{{#1}}
    \newcommand{\ErrorTok}[1]{\textcolor[rgb]{1.00,0.00,0.00}{\textbf{{#1}}}}
    \newcommand{\NormalTok}[1]{{#1}}
    
    % Additional commands for more recent versions of Pandoc
    \newcommand{\ConstantTok}[1]{\textcolor[rgb]{0.53,0.00,0.00}{{#1}}}
    \newcommand{\SpecialCharTok}[1]{\textcolor[rgb]{0.25,0.44,0.63}{{#1}}}
    \newcommand{\VerbatimStringTok}[1]{\textcolor[rgb]{0.25,0.44,0.63}{{#1}}}
    \newcommand{\SpecialStringTok}[1]{\textcolor[rgb]{0.73,0.40,0.53}{{#1}}}
    \newcommand{\ImportTok}[1]{{#1}}
    \newcommand{\DocumentationTok}[1]{\textcolor[rgb]{0.73,0.13,0.13}{\textit{{#1}}}}
    \newcommand{\AnnotationTok}[1]{\textcolor[rgb]{0.38,0.63,0.69}{\textbf{\textit{{#1}}}}}
    \newcommand{\CommentVarTok}[1]{\textcolor[rgb]{0.38,0.63,0.69}{\textbf{\textit{{#1}}}}}
    \newcommand{\VariableTok}[1]{\textcolor[rgb]{0.10,0.09,0.49}{{#1}}}
    \newcommand{\ControlFlowTok}[1]{\textcolor[rgb]{0.00,0.44,0.13}{\textbf{{#1}}}}
    \newcommand{\OperatorTok}[1]{\textcolor[rgb]{0.40,0.40,0.40}{{#1}}}
    \newcommand{\BuiltInTok}[1]{{#1}}
    \newcommand{\ExtensionTok}[1]{{#1}}
    \newcommand{\PreprocessorTok}[1]{\textcolor[rgb]{0.74,0.48,0.00}{{#1}}}
    \newcommand{\AttributeTok}[1]{\textcolor[rgb]{0.49,0.56,0.16}{{#1}}}
    \newcommand{\InformationTok}[1]{\textcolor[rgb]{0.38,0.63,0.69}{\textbf{\textit{{#1}}}}}
    \newcommand{\WarningTok}[1]{\textcolor[rgb]{0.38,0.63,0.69}{\textbf{\textit{{#1}}}}}
    
    
    % Define a nice break command that doesn't care if a line doesn't already
    % exist.
    \def\br{\hspace*{\fill} \\* }
    % Math Jax compatibility definitions
    \def\gt{>}
    \def\lt{<}
    \let\Oldtex\TeX
    \let\Oldlatex\LaTeX
    \renewcommand{\TeX}{\textrm{\Oldtex}}
    \renewcommand{\LaTeX}{\textrm{\Oldlatex}}
    % Document parameters
    % Document title
    \title{Homework 2 \\ \vspace{10mm}
    {\large Varun Nair}}
    
    
    
    
    

    % Pygments definitions
    
\makeatletter
\def\PY@reset{\let\PY@it=\relax \let\PY@bf=\relax%
    \let\PY@ul=\relax \let\PY@tc=\relax%
    \let\PY@bc=\relax \let\PY@ff=\relax}
\def\PY@tok#1{\csname PY@tok@#1\endcsname}
\def\PY@toks#1+{\ifx\relax#1\empty\else%
    \PY@tok{#1}\expandafter\PY@toks\fi}
\def\PY@do#1{\PY@bc{\PY@tc{\PY@ul{%
    \PY@it{\PY@bf{\PY@ff{#1}}}}}}}
\def\PY#1#2{\PY@reset\PY@toks#1+\relax+\PY@do{#2}}

\expandafter\def\csname PY@tok@w\endcsname{\def\PY@tc##1{\textcolor[rgb]{0.73,0.73,0.73}{##1}}}
\expandafter\def\csname PY@tok@c\endcsname{\let\PY@it=\textit\def\PY@tc##1{\textcolor[rgb]{0.25,0.50,0.50}{##1}}}
\expandafter\def\csname PY@tok@cp\endcsname{\def\PY@tc##1{\textcolor[rgb]{0.74,0.48,0.00}{##1}}}
\expandafter\def\csname PY@tok@k\endcsname{\let\PY@bf=\textbf\def\PY@tc##1{\textcolor[rgb]{0.00,0.50,0.00}{##1}}}
\expandafter\def\csname PY@tok@kp\endcsname{\def\PY@tc##1{\textcolor[rgb]{0.00,0.50,0.00}{##1}}}
\expandafter\def\csname PY@tok@kt\endcsname{\def\PY@tc##1{\textcolor[rgb]{0.69,0.00,0.25}{##1}}}
\expandafter\def\csname PY@tok@o\endcsname{\def\PY@tc##1{\textcolor[rgb]{0.40,0.40,0.40}{##1}}}
\expandafter\def\csname PY@tok@ow\endcsname{\let\PY@bf=\textbf\def\PY@tc##1{\textcolor[rgb]{0.67,0.13,1.00}{##1}}}
\expandafter\def\csname PY@tok@nb\endcsname{\def\PY@tc##1{\textcolor[rgb]{0.00,0.50,0.00}{##1}}}
\expandafter\def\csname PY@tok@nf\endcsname{\def\PY@tc##1{\textcolor[rgb]{0.00,0.00,1.00}{##1}}}
\expandafter\def\csname PY@tok@nc\endcsname{\let\PY@bf=\textbf\def\PY@tc##1{\textcolor[rgb]{0.00,0.00,1.00}{##1}}}
\expandafter\def\csname PY@tok@nn\endcsname{\let\PY@bf=\textbf\def\PY@tc##1{\textcolor[rgb]{0.00,0.00,1.00}{##1}}}
\expandafter\def\csname PY@tok@ne\endcsname{\let\PY@bf=\textbf\def\PY@tc##1{\textcolor[rgb]{0.82,0.25,0.23}{##1}}}
\expandafter\def\csname PY@tok@nv\endcsname{\def\PY@tc##1{\textcolor[rgb]{0.10,0.09,0.49}{##1}}}
\expandafter\def\csname PY@tok@no\endcsname{\def\PY@tc##1{\textcolor[rgb]{0.53,0.00,0.00}{##1}}}
\expandafter\def\csname PY@tok@nl\endcsname{\def\PY@tc##1{\textcolor[rgb]{0.63,0.63,0.00}{##1}}}
\expandafter\def\csname PY@tok@ni\endcsname{\let\PY@bf=\textbf\def\PY@tc##1{\textcolor[rgb]{0.60,0.60,0.60}{##1}}}
\expandafter\def\csname PY@tok@na\endcsname{\def\PY@tc##1{\textcolor[rgb]{0.49,0.56,0.16}{##1}}}
\expandafter\def\csname PY@tok@nt\endcsname{\let\PY@bf=\textbf\def\PY@tc##1{\textcolor[rgb]{0.00,0.50,0.00}{##1}}}
\expandafter\def\csname PY@tok@nd\endcsname{\def\PY@tc##1{\textcolor[rgb]{0.67,0.13,1.00}{##1}}}
\expandafter\def\csname PY@tok@s\endcsname{\def\PY@tc##1{\textcolor[rgb]{0.73,0.13,0.13}{##1}}}
\expandafter\def\csname PY@tok@sd\endcsname{\let\PY@it=\textit\def\PY@tc##1{\textcolor[rgb]{0.73,0.13,0.13}{##1}}}
\expandafter\def\csname PY@tok@si\endcsname{\let\PY@bf=\textbf\def\PY@tc##1{\textcolor[rgb]{0.73,0.40,0.53}{##1}}}
\expandafter\def\csname PY@tok@se\endcsname{\let\PY@bf=\textbf\def\PY@tc##1{\textcolor[rgb]{0.73,0.40,0.13}{##1}}}
\expandafter\def\csname PY@tok@sr\endcsname{\def\PY@tc##1{\textcolor[rgb]{0.73,0.40,0.53}{##1}}}
\expandafter\def\csname PY@tok@ss\endcsname{\def\PY@tc##1{\textcolor[rgb]{0.10,0.09,0.49}{##1}}}
\expandafter\def\csname PY@tok@sx\endcsname{\def\PY@tc##1{\textcolor[rgb]{0.00,0.50,0.00}{##1}}}
\expandafter\def\csname PY@tok@m\endcsname{\def\PY@tc##1{\textcolor[rgb]{0.40,0.40,0.40}{##1}}}
\expandafter\def\csname PY@tok@gh\endcsname{\let\PY@bf=\textbf\def\PY@tc##1{\textcolor[rgb]{0.00,0.00,0.50}{##1}}}
\expandafter\def\csname PY@tok@gu\endcsname{\let\PY@bf=\textbf\def\PY@tc##1{\textcolor[rgb]{0.50,0.00,0.50}{##1}}}
\expandafter\def\csname PY@tok@gd\endcsname{\def\PY@tc##1{\textcolor[rgb]{0.63,0.00,0.00}{##1}}}
\expandafter\def\csname PY@tok@gi\endcsname{\def\PY@tc##1{\textcolor[rgb]{0.00,0.63,0.00}{##1}}}
\expandafter\def\csname PY@tok@gr\endcsname{\def\PY@tc##1{\textcolor[rgb]{1.00,0.00,0.00}{##1}}}
\expandafter\def\csname PY@tok@ge\endcsname{\let\PY@it=\textit}
\expandafter\def\csname PY@tok@gs\endcsname{\let\PY@bf=\textbf}
\expandafter\def\csname PY@tok@gp\endcsname{\let\PY@bf=\textbf\def\PY@tc##1{\textcolor[rgb]{0.00,0.00,0.50}{##1}}}
\expandafter\def\csname PY@tok@go\endcsname{\def\PY@tc##1{\textcolor[rgb]{0.53,0.53,0.53}{##1}}}
\expandafter\def\csname PY@tok@gt\endcsname{\def\PY@tc##1{\textcolor[rgb]{0.00,0.27,0.87}{##1}}}
\expandafter\def\csname PY@tok@err\endcsname{\def\PY@bc##1{\setlength{\fboxsep}{0pt}\fcolorbox[rgb]{1.00,0.00,0.00}{1,1,1}{\strut ##1}}}
\expandafter\def\csname PY@tok@kc\endcsname{\let\PY@bf=\textbf\def\PY@tc##1{\textcolor[rgb]{0.00,0.50,0.00}{##1}}}
\expandafter\def\csname PY@tok@kd\endcsname{\let\PY@bf=\textbf\def\PY@tc##1{\textcolor[rgb]{0.00,0.50,0.00}{##1}}}
\expandafter\def\csname PY@tok@kn\endcsname{\let\PY@bf=\textbf\def\PY@tc##1{\textcolor[rgb]{0.00,0.50,0.00}{##1}}}
\expandafter\def\csname PY@tok@kr\endcsname{\let\PY@bf=\textbf\def\PY@tc##1{\textcolor[rgb]{0.00,0.50,0.00}{##1}}}
\expandafter\def\csname PY@tok@bp\endcsname{\def\PY@tc##1{\textcolor[rgb]{0.00,0.50,0.00}{##1}}}
\expandafter\def\csname PY@tok@fm\endcsname{\def\PY@tc##1{\textcolor[rgb]{0.00,0.00,1.00}{##1}}}
\expandafter\def\csname PY@tok@vc\endcsname{\def\PY@tc##1{\textcolor[rgb]{0.10,0.09,0.49}{##1}}}
\expandafter\def\csname PY@tok@vg\endcsname{\def\PY@tc##1{\textcolor[rgb]{0.10,0.09,0.49}{##1}}}
\expandafter\def\csname PY@tok@vi\endcsname{\def\PY@tc##1{\textcolor[rgb]{0.10,0.09,0.49}{##1}}}
\expandafter\def\csname PY@tok@vm\endcsname{\def\PY@tc##1{\textcolor[rgb]{0.10,0.09,0.49}{##1}}}
\expandafter\def\csname PY@tok@sa\endcsname{\def\PY@tc##1{\textcolor[rgb]{0.73,0.13,0.13}{##1}}}
\expandafter\def\csname PY@tok@sb\endcsname{\def\PY@tc##1{\textcolor[rgb]{0.73,0.13,0.13}{##1}}}
\expandafter\def\csname PY@tok@sc\endcsname{\def\PY@tc##1{\textcolor[rgb]{0.73,0.13,0.13}{##1}}}
\expandafter\def\csname PY@tok@dl\endcsname{\def\PY@tc##1{\textcolor[rgb]{0.73,0.13,0.13}{##1}}}
\expandafter\def\csname PY@tok@s2\endcsname{\def\PY@tc##1{\textcolor[rgb]{0.73,0.13,0.13}{##1}}}
\expandafter\def\csname PY@tok@sh\endcsname{\def\PY@tc##1{\textcolor[rgb]{0.73,0.13,0.13}{##1}}}
\expandafter\def\csname PY@tok@s1\endcsname{\def\PY@tc##1{\textcolor[rgb]{0.73,0.13,0.13}{##1}}}
\expandafter\def\csname PY@tok@mb\endcsname{\def\PY@tc##1{\textcolor[rgb]{0.40,0.40,0.40}{##1}}}
\expandafter\def\csname PY@tok@mf\endcsname{\def\PY@tc##1{\textcolor[rgb]{0.40,0.40,0.40}{##1}}}
\expandafter\def\csname PY@tok@mh\endcsname{\def\PY@tc##1{\textcolor[rgb]{0.40,0.40,0.40}{##1}}}
\expandafter\def\csname PY@tok@mi\endcsname{\def\PY@tc##1{\textcolor[rgb]{0.40,0.40,0.40}{##1}}}
\expandafter\def\csname PY@tok@il\endcsname{\def\PY@tc##1{\textcolor[rgb]{0.40,0.40,0.40}{##1}}}
\expandafter\def\csname PY@tok@mo\endcsname{\def\PY@tc##1{\textcolor[rgb]{0.40,0.40,0.40}{##1}}}
\expandafter\def\csname PY@tok@ch\endcsname{\let\PY@it=\textit\def\PY@tc##1{\textcolor[rgb]{0.25,0.50,0.50}{##1}}}
\expandafter\def\csname PY@tok@cm\endcsname{\let\PY@it=\textit\def\PY@tc##1{\textcolor[rgb]{0.25,0.50,0.50}{##1}}}
\expandafter\def\csname PY@tok@cpf\endcsname{\let\PY@it=\textit\def\PY@tc##1{\textcolor[rgb]{0.25,0.50,0.50}{##1}}}
\expandafter\def\csname PY@tok@c1\endcsname{\let\PY@it=\textit\def\PY@tc##1{\textcolor[rgb]{0.25,0.50,0.50}{##1}}}
\expandafter\def\csname PY@tok@cs\endcsname{\let\PY@it=\textit\def\PY@tc##1{\textcolor[rgb]{0.25,0.50,0.50}{##1}}}

\def\PYZbs{\char`\\}
\def\PYZus{\char`\_}
\def\PYZob{\char`\{}
\def\PYZcb{\char`\}}
\def\PYZca{\char`\^}
\def\PYZam{\char`\&}
\def\PYZlt{\char`\<}
\def\PYZgt{\char`\>}
\def\PYZsh{\char`\#}
\def\PYZpc{\char`\%}
\def\PYZdl{\char`\$}
\def\PYZhy{\char`\-}
\def\PYZsq{\char`\'}
\def\PYZdq{\char`\"}
\def\PYZti{\char`\~}
% for compatibility with earlier versions
\def\PYZat{@}
\def\PYZlb{[}
\def\PYZrb{]}
\makeatother


    % Exact colors from NB
    \definecolor{incolor}{rgb}{0.0, 0.0, 0.5}
    \definecolor{outcolor}{rgb}{0.545, 0.0, 0.0}



    
    % Prevent overflowing lines due to hard-to-break entities
    \sloppy 
    % Setup hyperref package
    \hypersetup{
      breaklinks=true,  % so long urls are correctly broken across lines
      colorlinks=true,
      urlcolor=urlcolor,
      linkcolor=linkcolor,
      citecolor=citecolor,
      }
    % Slightly bigger margins than the latex defaults
    
    \geometry{verbose,tmargin=1in,bmargin=1in,lmargin=1in,rmargin=1in}
    
    

    \begin{document}
    
    
    \maketitle
    
    

    
    \begin{Verbatim}[commandchars=\\\{\}]
{\color{incolor}In [{\color{incolor}166}]:} \PY{o}{\PYZpc{}}\PY{k}{precision} \PYZpc{}g
          \PY{o}{\PYZpc{}}\PY{k}{matplotlib} inline
          \PY{o}{\PYZpc{}}\PY{k}{config} InlineBackend.figure\PYZus{}format = \PYZsq{}retina\PYZsq{}
\end{Verbatim}

    \begin{Verbatim}[commandchars=\\\{\}]
{\color{incolor}In [{\color{incolor}167}]:} \PY{k+kn}{from} \PY{n+nn}{math} \PY{k}{import} \PY{n}{sqrt}\PY{p}{,} \PY{n}{pi}\PY{p}{,} \PY{n}{sin}\PY{p}{,} \PY{n}{cos}\PY{p}{,} \PY{n}{exp}
          \PY{c+c1}{\PYZsh{}from cmath import sqrt as csqrt}
          \PY{k+kn}{import} \PY{n+nn}{numpy} \PY{k}{as} \PY{n+nn}{np}
          \PY{k+kn}{from} \PY{n+nn}{scipy} \PY{k}{import} \PY{n}{constants} \PY{k}{as} \PY{n}{C}
          \PY{k+kn}{import} \PY{n+nn}{matplotlib}\PY{n+nn}{.}\PY{n+nn}{pyplot} \PY{k}{as} \PY{n+nn}{plt}
          
          \PY{k+kn}{from} \PY{n+nn}{IPython}\PY{n+nn}{.}\PY{n+nn}{display} \PY{k}{import} \PY{n}{set\PYZus{}matplotlib\PYZus{}formats}
          \PY{n}{set\PYZus{}matplotlib\PYZus{}formats}\PY{p}{(}\PY{l+s+s1}{\PYZsq{}}\PY{l+s+s1}{png}\PY{l+s+s1}{\PYZsq{}}\PY{p}{,} \PY{l+s+s1}{\PYZsq{}}\PY{l+s+s1}{pdf}\PY{l+s+s1}{\PYZsq{}}\PY{p}{)}
\end{Verbatim}

    \section{CP 3.1 Plotting experimental
data}\label{cp-3.1-plotting-experimental-data}

This problem looks at the monthly sunspot data since January 1979 which
is identified as the 0th month in the data.

    \begin{Verbatim}[commandchars=\\\{\}]
{\color{incolor}In [{\color{incolor}168}]:} \PY{c+c1}{\PYZsh{}this program creates a graph from a set of data}
          
          \PY{k}{def} \PY{n+nf}{graph\PYZus{}sunspots}\PY{p}{(}\PY{n}{file\PYZus{}name}\PY{p}{)}\PY{p}{:}
              \PY{n}{sunspots} \PY{o}{=} \PY{n}{np}\PY{o}{.}\PY{n}{loadtxt}\PY{p}{(}\PY{n}{file\PYZus{}name}\PY{p}{,} \PY{n+nb}{float}\PY{p}{)}
          
              \PY{n}{fig}\PY{p}{,} \PY{n}{ax} \PY{o}{=} \PY{n}{plt}\PY{o}{.}\PY{n}{subplots}\PY{p}{(}\PY{l+m+mi}{1}\PY{p}{,} \PY{l+m+mi}{1}\PY{p}{,} \PY{n}{figsize} \PY{o}{=} \PY{p}{(}\PY{l+m+mi}{10}\PY{p}{,} \PY{l+m+mi}{5}\PY{p}{)}\PY{p}{)}
              \PY{n}{x} \PY{o}{=} \PY{n}{sunspots}\PY{p}{[}\PY{p}{:}\PY{p}{,}\PY{l+m+mi}{0}\PY{p}{]}
              \PY{n}{y} \PY{o}{=} \PY{n}{sunspots}\PY{p}{[}\PY{p}{:}\PY{p}{,}\PY{l+m+mi}{1}\PY{p}{]}
              \PY{n}{plt}\PY{o}{.}\PY{n}{plot}\PY{p}{(}\PY{n}{x}\PY{p}{,}\PY{n}{y}\PY{p}{)}
              \PY{n}{plt}\PY{o}{.}\PY{n}{show}\PY{p}{(}\PY{p}{)}
              
          \PY{n}{graph\PYZus{}sunspots}\PY{p}{(}\PY{l+s+s2}{\PYZdq{}}\PY{l+s+s2}{sunspots.txt}\PY{l+s+s2}{\PYZdq{}}\PY{p}{)}
\end{Verbatim}

    \begin{center}
    \adjustimage{max size={0.9\linewidth}{0.9\paperheight}}{output_3_0.pdf}
    \end{center}
    { \hspace*{\fill} \\}
    
    \begin{Verbatim}[commandchars=\\\{\}]
{\color{incolor}In [{\color{incolor}169}]:} \PY{c+c1}{\PYZsh{}this program is modified to only graph the first 1000 data points}
          \PY{k}{def} \PY{n+nf}{graph\PYZus{}sunspots}\PY{p}{(}\PY{n}{file\PYZus{}name}\PY{p}{)}\PY{p}{:}
              \PY{n}{sunspots} \PY{o}{=} \PY{n}{np}\PY{o}{.}\PY{n}{loadtxt}\PY{p}{(}\PY{n}{file\PYZus{}name}\PY{p}{,} \PY{n+nb}{float}\PY{p}{)}
          
              \PY{n}{fig}\PY{p}{,} \PY{n}{ax} \PY{o}{=} \PY{n}{plt}\PY{o}{.}\PY{n}{subplots}\PY{p}{(}\PY{l+m+mi}{1}\PY{p}{,} \PY{l+m+mi}{1}\PY{p}{,} \PY{n}{figsize} \PY{o}{=} \PY{p}{(}\PY{l+m+mi}{8}\PY{p}{,} \PY{l+m+mi}{4}\PY{p}{)}\PY{p}{)}
              \PY{n}{x} \PY{o}{=} \PY{p}{[}\PY{p}{]}
              \PY{n}{y} \PY{o}{=} \PY{p}{[}\PY{p}{]}
              \PY{k}{for} \PY{n}{i} \PY{o+ow}{in} \PY{n+nb}{range}\PY{p}{(}\PY{l+m+mi}{1000}\PY{p}{)}\PY{p}{:}
                  \PY{n}{x}\PY{o}{.}\PY{n}{append}\PY{p}{(}\PY{n}{sunspots}\PY{p}{[}\PY{n}{i}\PY{p}{,}\PY{l+m+mi}{0}\PY{p}{]}\PY{p}{)} 
                  \PY{n}{y}\PY{o}{.}\PY{n}{append}\PY{p}{(}\PY{n}{sunspots}\PY{p}{[}\PY{n}{i}\PY{p}{,}\PY{l+m+mi}{1}\PY{p}{]}\PY{p}{)}
              \PY{n}{plt}\PY{o}{.}\PY{n}{plot}\PY{p}{(}\PY{n}{x}\PY{p}{,}\PY{n}{y}\PY{p}{)}
              \PY{n}{plt}\PY{o}{.}\PY{n}{show}\PY{p}{(}\PY{p}{)} 
              
          \PY{n}{graph\PYZus{}sunspots}\PY{p}{(}\PY{l+s+s2}{\PYZdq{}}\PY{l+s+s2}{sunspots.txt}\PY{l+s+s2}{\PYZdq{}}\PY{p}{)}
\end{Verbatim}

    \begin{center}
    \adjustimage{max size={0.9\linewidth}{0.9\paperheight}}{output_4_0.pdf}
    \end{center}
    { \hspace*{\fill} \\}
    
    A running average of the form
\(\, Y_k = {1\over2r} \sum_{m=-r}^r y_{k+m}\,\) can be overlaid on this
plot of the first 1000 data points.

    \begin{Verbatim}[commandchars=\\\{\}]
{\color{incolor}In [{\color{incolor}170}]:} \PY{k}{def} \PY{n+nf}{graph\PYZus{}sunspots}\PY{p}{(}\PY{n}{file\PYZus{}name}\PY{p}{,} \PY{n}{r}\PY{p}{)}\PY{p}{:}
              \PY{l+s+sd}{\PYZsq{}\PYZsq{}\PYZsq{}graphs first 1000 data points and their running average with bandwidth 2r\PYZsq{}\PYZsq{}\PYZsq{}}
              \PY{n}{sunspots} \PY{o}{=} \PY{n}{np}\PY{o}{.}\PY{n}{loadtxt}\PY{p}{(}\PY{n}{file\PYZus{}name}\PY{p}{,} \PY{n+nb}{float}\PY{p}{)}
          
              \PY{n}{fig}\PY{p}{,} \PY{n}{ax} \PY{o}{=} \PY{n}{plt}\PY{o}{.}\PY{n}{subplots}\PY{p}{(}\PY{l+m+mi}{1}\PY{p}{,} \PY{l+m+mi}{1}\PY{p}{,} \PY{n}{figsize} \PY{o}{=} \PY{p}{(}\PY{l+m+mi}{12}\PY{p}{,} \PY{l+m+mi}{6}\PY{p}{)}\PY{p}{)}
              \PY{n}{x} \PY{o}{=} \PY{p}{[}\PY{p}{]}
              \PY{n}{y} \PY{o}{=} \PY{p}{[}\PY{p}{]}
              \PY{k}{for} \PY{n}{i} \PY{o+ow}{in} \PY{n+nb}{range}\PY{p}{(}\PY{l+m+mi}{1000}\PY{p}{)}\PY{p}{:}
                  \PY{n}{x}\PY{o}{.}\PY{n}{append}\PY{p}{(}\PY{n}{sunspots}\PY{p}{[}\PY{n}{i}\PY{p}{,}\PY{l+m+mi}{0}\PY{p}{]}\PY{p}{)} 
                  \PY{n}{y}\PY{o}{.}\PY{n}{append}\PY{p}{(}\PY{n}{sunspots}\PY{p}{[}\PY{n}{i}\PY{p}{,}\PY{l+m+mi}{1}\PY{p}{]}\PY{p}{)}
              
              \PY{n}{averages} \PY{o}{=} \PY{p}{[}\PY{p}{]}
              \PY{k}{for} \PY{n}{k} \PY{o+ow}{in} \PY{n+nb}{range}\PY{p}{(}\PY{l+m+mi}{1000}\PY{p}{)}\PY{p}{:}
                  \PY{n+nb}{sum} \PY{o}{=} \PY{l+m+mi}{0}
                  \PY{k}{for} \PY{n}{m} \PY{o+ow}{in} \PY{n+nb}{range}\PY{p}{(}\PY{o}{\PYZhy{}}\PY{n}{r}\PY{p}{,} \PY{n}{r}\PY{p}{)}\PY{p}{:}
                      \PY{k}{if} \PY{n}{k} \PY{o}{+} \PY{n}{m} \PY{o}{\PYZlt{}} \PY{l+m+mi}{0}\PY{p}{:}
                          \PY{n+nb}{sum} \PY{o}{+}\PY{o}{=} \PY{l+m+mi}{0}
                      \PY{k}{elif} \PY{n}{k} \PY{o}{+} \PY{n}{m} \PY{o}{\PYZgt{}} \PY{n+nb}{len}\PY{p}{(}\PY{n}{y}\PY{p}{)}\PY{o}{\PYZhy{}}\PY{l+m+mi}{1}\PY{p}{:}
                          \PY{n+nb}{sum} \PY{o}{+}\PY{o}{=} \PY{l+m+mi}{0}
                      \PY{k}{else}\PY{p}{:}
                          \PY{n+nb}{sum} \PY{o}{+}\PY{o}{=} \PY{n}{y}\PY{p}{[}\PY{n}{k}\PY{o}{+}\PY{n}{m}\PY{p}{]}
                  
                  \PY{n}{averages}\PY{o}{.}\PY{n}{append}\PY{p}{(}\PY{n+nb}{sum} \PY{o}{/} \PY{p}{(}\PY{l+m+mi}{2}\PY{o}{*}\PY{n}{r}\PY{p}{)}\PY{p}{)}
          
              \PY{c+c1}{\PYZsh{}first 1000 months of sunspot data        }
              \PY{n}{plt}\PY{o}{.}\PY{n}{plot}\PY{p}{(}\PY{n}{x}\PY{p}{,} \PY{n}{y}\PY{p}{,} \PY{n}{color}\PY{o}{=}\PY{l+s+s1}{\PYZsq{}}\PY{l+s+s1}{blue}\PY{l+s+s1}{\PYZsq{}}\PY{p}{)}
              
              \PY{c+c1}{\PYZsh{}first 1000 of running average data}
              \PY{n}{plt}\PY{o}{.}\PY{n}{plot}\PY{p}{(}\PY{n}{x}\PY{p}{,} \PY{n}{averages}\PY{p}{,} \PY{l+s+s1}{\PYZsq{}}\PY{l+s+s1}{k\PYZhy{}}\PY{l+s+s1}{\PYZsq{}}\PY{p}{,} \PY{n}{color} \PY{o}{=} \PY{l+s+s1}{\PYZsq{}}\PY{l+s+s1}{red}\PY{l+s+s1}{\PYZsq{}}\PY{p}{)}
              
              \PY{n}{plt}\PY{o}{.}\PY{n}{title}\PY{p}{(}\PY{l+s+s2}{\PYZdq{}}\PY{l+s+s2}{Raw Data (blue) and the Running Average (red)}\PY{l+s+s2}{\PYZdq{}}\PY{p}{)}
              \PY{n}{plt}\PY{o}{.}\PY{n}{show}\PY{p}{(}\PY{p}{)}
              
          \PY{n}{graph\PYZus{}sunspots}\PY{p}{(}\PY{l+s+s2}{\PYZdq{}}\PY{l+s+s2}{sunspots.txt}\PY{l+s+s2}{\PYZdq{}}\PY{p}{,} \PY{l+m+mi}{5}\PY{p}{)}
\end{Verbatim}

    \begin{center}
    \adjustimage{max size={0.9\linewidth}{0.9\paperheight}}{output_6_0.pdf}
    \end{center}
    { \hspace*{\fill} \\}
    
    So we can see that once the running average is plotted on the same set
of axes, the number of sunspots over time still oscillated but has less
extreme swings. The running average creates a smoother dataset for
plotting.

    \section{CP 3.2 Curve plotting}\label{cp-3.2-curve-plotting}

Let's take a look at different polar functions graphed on a Cartesian
plane. The following functions (deltoid, Galilean spiral, and Fey) have
different intervals over which they're evaluated.

    \begin{Verbatim}[commandchars=\\\{\}]
{\color{incolor}In [{\color{incolor}171}]:} \PY{c+c1}{\PYZsh{}plots deltoid function as blue line from 0 to 2pi by 0.01}
          
          \PY{n}{x} \PY{o}{=} \PY{p}{[}\PY{p}{]}
          \PY{n}{y} \PY{o}{=} \PY{p}{[}\PY{p}{]}
          \PY{k}{for} \PY{n}{theta} \PY{o+ow}{in} \PY{n}{np}\PY{o}{.}\PY{n}{arange}\PY{p}{(}\PY{l+m+mi}{0}\PY{p}{,} \PY{l+m+mi}{2}\PY{o}{*}\PY{n}{pi}\PY{p}{,} \PY{o}{.}\PY{l+m+mi}{01}\PY{p}{)}\PY{p}{:}
              \PY{n}{x}\PY{o}{.}\PY{n}{append}\PY{p}{(}\PY{l+m+mi}{2} \PY{o}{*} \PY{n}{cos}\PY{p}{(}\PY{n}{theta}\PY{p}{)} \PY{o}{+} \PY{n}{cos}\PY{p}{(}\PY{l+m+mi}{2}\PY{o}{*}\PY{n}{theta}\PY{p}{)}\PY{p}{)}
              \PY{n}{y}\PY{o}{.}\PY{n}{append}\PY{p}{(}\PY{l+m+mi}{2} \PY{o}{*} \PY{n}{sin}\PY{p}{(}\PY{n}{theta}\PY{p}{)} \PY{o}{\PYZhy{}} \PY{n}{sin}\PY{p}{(}\PY{l+m+mi}{2}\PY{o}{*}\PY{n}{theta}\PY{p}{)}\PY{p}{)}
          
          \PY{n}{fig}\PY{p}{,} \PY{n}{ax} \PY{o}{=} \PY{n}{plt}\PY{o}{.}\PY{n}{subplots}\PY{p}{(}\PY{l+m+mi}{1}\PY{p}{,} \PY{l+m+mi}{1}\PY{p}{,} \PY{n}{figsize} \PY{o}{=} \PY{p}{(}\PY{l+m+mi}{4}\PY{p}{,} \PY{l+m+mi}{4}\PY{p}{)}\PY{p}{)}
          
          \PY{n}{plt}\PY{o}{.}\PY{n}{plot}\PY{p}{(}\PY{n}{x}\PY{p}{,} \PY{n}{y}\PY{p}{,} \PY{l+s+s1}{\PYZsq{}}\PY{l+s+s1}{k\PYZhy{}}\PY{l+s+s1}{\PYZsq{}}\PY{p}{,}\PY{n}{color}\PY{o}{=}\PY{l+s+s1}{\PYZsq{}}\PY{l+s+s1}{blue}\PY{l+s+s1}{\PYZsq{}}\PY{p}{)}
          \PY{c+c1}{\PYZsh{}plt.xlim(\PYZhy{}3.5, 3.5)}
          \PY{c+c1}{\PYZsh{}plt.ylim(\PYZhy{}3.5, 3.5)}
          \PY{n}{plt}\PY{o}{.}\PY{n}{show}\PY{p}{(}\PY{p}{)}
\end{Verbatim}

    \begin{center}
    \adjustimage{max size={0.9\linewidth}{0.9\paperheight}}{output_9_0.pdf}
    \end{center}
    { \hspace*{\fill} \\}
    
    \begin{Verbatim}[commandchars=\\\{\}]
{\color{incolor}In [{\color{incolor}172}]:} \PY{c+c1}{\PYZsh{}plots Galilean spiral as red line from 0 to 10pi by 0.01}
          
          \PY{n}{x} \PY{o}{=} \PY{p}{[}\PY{p}{]}
          \PY{n}{y} \PY{o}{=} \PY{p}{[}\PY{p}{]}
          \PY{k}{for} \PY{n}{theta} \PY{o+ow}{in} \PY{n}{np}\PY{o}{.}\PY{n}{arange}\PY{p}{(}\PY{l+m+mi}{0}\PY{p}{,} \PY{l+m+mi}{10}\PY{o}{*}\PY{n}{pi}\PY{p}{,} \PY{o}{.}\PY{l+m+mi}{01}\PY{p}{)}\PY{p}{:}
              \PY{n}{r} \PY{o}{=} \PY{n}{theta}\PY{o}{*}\PY{o}{*}\PY{l+m+mi}{2}
              \PY{n}{x}\PY{o}{.}\PY{n}{append}\PY{p}{(}\PY{n}{r} \PY{o}{*} \PY{n}{cos}\PY{p}{(}\PY{n}{theta}\PY{p}{)}\PY{p}{)}
              \PY{n}{y}\PY{o}{.}\PY{n}{append}\PY{p}{(}\PY{n}{r} \PY{o}{*} \PY{n}{sin}\PY{p}{(}\PY{n}{theta}\PY{p}{)}\PY{p}{)}
          
          \PY{n}{fig}\PY{p}{,} \PY{n}{ax} \PY{o}{=} \PY{n}{plt}\PY{o}{.}\PY{n}{subplots}\PY{p}{(}\PY{l+m+mi}{1}\PY{p}{,} \PY{l+m+mi}{1}\PY{p}{,} \PY{n}{figsize} \PY{o}{=} \PY{p}{(}\PY{l+m+mi}{5}\PY{p}{,} \PY{l+m+mi}{5}\PY{p}{)}\PY{p}{)}
          
          \PY{n}{plt}\PY{o}{.}\PY{n}{plot}\PY{p}{(}\PY{n}{x}\PY{p}{,} \PY{n}{y}\PY{p}{,} \PY{l+s+s1}{\PYZsq{}}\PY{l+s+s1}{k\PYZhy{}}\PY{l+s+s1}{\PYZsq{}}\PY{p}{,}\PY{n}{color} \PY{o}{=} \PY{l+s+s1}{\PYZsq{}}\PY{l+s+s1}{red}\PY{l+s+s1}{\PYZsq{}}\PY{p}{)}
          \PY{c+c1}{\PYZsh{}plt.xlim(\PYZhy{}3.5, 3.5)}
          \PY{c+c1}{\PYZsh{}plt.ylim(\PYZhy{}3.5, 3.5)}
          \PY{n}{plt}\PY{o}{.}\PY{n}{show}\PY{p}{(}\PY{p}{)}
\end{Verbatim}

    \begin{center}
    \adjustimage{max size={0.9\linewidth}{0.9\paperheight}}{output_10_0.pdf}
    \end{center}
    { \hspace*{\fill} \\}
    
    \begin{Verbatim}[commandchars=\\\{\}]
{\color{incolor}In [{\color{incolor}173}]:} \PY{c+c1}{\PYZsh{}plots Fey\PYZsq{}s function from 0 to 24pi by 0.01}
          
          \PY{n}{x} \PY{o}{=} \PY{p}{[}\PY{p}{]}
          \PY{n}{y} \PY{o}{=} \PY{p}{[}\PY{p}{]}
          \PY{k}{for} \PY{n}{theta} \PY{o+ow}{in} \PY{n}{np}\PY{o}{.}\PY{n}{arange}\PY{p}{(}\PY{l+m+mi}{0}\PY{p}{,} \PY{l+m+mi}{24}\PY{o}{*}\PY{n}{pi}\PY{p}{,} \PY{o}{.}\PY{l+m+mi}{01}\PY{p}{)}\PY{p}{:}
              \PY{n}{r} \PY{o}{=} \PY{n}{exp}\PY{p}{(}\PY{n}{cos}\PY{p}{(}\PY{n}{theta}\PY{p}{)}\PY{p}{)} \PY{o}{\PYZhy{}} \PY{l+m+mi}{2} \PY{o}{*} \PY{n}{cos}\PY{p}{(}\PY{l+m+mi}{4}\PY{o}{*}\PY{n}{theta}\PY{p}{)} \PY{o}{+} \PY{p}{(}\PY{n}{sin}\PY{p}{(}\PY{n}{theta}\PY{o}{/}\PY{l+m+mi}{12}\PY{p}{)}\PY{p}{)}\PY{o}{*}\PY{o}{*}\PY{l+m+mi}{5}
              \PY{n}{x}\PY{o}{.}\PY{n}{append}\PY{p}{(}\PY{n}{r} \PY{o}{*} \PY{n}{cos}\PY{p}{(}\PY{n}{theta}\PY{p}{)}\PY{p}{)}
              \PY{n}{y}\PY{o}{.}\PY{n}{append}\PY{p}{(}\PY{n}{r} \PY{o}{*} \PY{n}{sin}\PY{p}{(}\PY{n}{theta}\PY{p}{)}\PY{p}{)}
          
          \PY{n}{fig}\PY{p}{,} \PY{n}{ax} \PY{o}{=} \PY{n}{plt}\PY{o}{.}\PY{n}{subplots}\PY{p}{(}\PY{l+m+mi}{1}\PY{p}{,} \PY{l+m+mi}{1}\PY{p}{,} \PY{n}{figsize} \PY{o}{=} \PY{p}{(}\PY{l+m+mi}{8}\PY{p}{,} \PY{l+m+mi}{8}\PY{p}{)}\PY{p}{)}
          
          \PY{n}{plt}\PY{o}{.}\PY{n}{plot}\PY{p}{(}\PY{n}{x}\PY{p}{,} \PY{n}{y}\PY{p}{,} \PY{l+s+s1}{\PYZsq{}}\PY{l+s+s1}{k\PYZhy{}}\PY{l+s+s1}{\PYZsq{}}\PY{p}{)}
          \PY{n}{plt}\PY{o}{.}\PY{n}{show}\PY{p}{(}\PY{p}{)}
\end{Verbatim}

    \begin{center}
    \adjustimage{max size={0.9\linewidth}{0.9\paperheight}}{output_11_0.pdf}
    \end{center}
    { \hspace*{\fill} \\}
    
    Thus, it's clearly very simple to plot parametric polar equations in
Cartesian coordinates using matplotlib. All that needs to be done is a
conversion from \(r\) and \(\theta\) \(\to\) \(x\) and \(y\).

    \section{CP 3.8 Least-squares fitting and the photoelectric
effect}\label{cp-3.8-least-squares-fitting-and-the-photoelectric-effect}

By minimizing the sum of squares of the residuals (the distances from
the fit line to the observed data points), a line of best fit can be
drawn that will go through the mean of the data and provide the best
linear prediction for its relationship.

Some of the relations necessary for fitting are given by

\(E_x = {1\over N} \sum_{i=1}^N x_i,\qquad\)
\(E_y = {1\over N} \sum_{i=1}^N y_i,\qquad\)
\(E_{xx} = {1\over N} \sum_{i=1}^N x_i^2,\qquad\) \&
\(E_{xy} = {1\over N} \sum_{i=1}^N x_iy_i.\)

Once we have these expressions, the slope and the intercept of the line
can be found with the following:

\(m = {E_{xy}-E_x E_y\over E_{xx} - E_x^2},\qquad\)
\(c = {E_{xx}E_y-E_x E_{xy}\over E_{xx} - E_x^2}\)

The relationship between Voltage required to stop a released electron
and the frequency of the light releasing it is linear. The slope
coefficient calculated from data can be compared to \(h \over e\) to
experimentally determine a value for Planck's constant.

    \begin{Verbatim}[commandchars=\\\{\}]
{\color{incolor}In [{\color{incolor}174}]:} \PY{k}{def} \PY{n+nf}{photoelectric}\PY{p}{(}\PY{n}{filename}\PY{p}{)}\PY{p}{:}
              \PY{l+s+sd}{\PYZsq{}\PYZsq{}\PYZsq{}by specifying a filename when calling this function,}
          \PY{l+s+sd}{        you can graph the raw data from a file.}
          \PY{l+s+sd}{        It also calculates line of best fit, and}
          \PY{l+s+sd}{        calculates an experimental value of Planks constant\PYZsq{}\PYZsq{}\PYZsq{}}
              
              \PY{n}{data} \PY{o}{=} \PY{n}{np}\PY{o}{.}\PY{n}{loadtxt}\PY{p}{(}\PY{n}{filename}\PY{p}{,} \PY{n+nb}{float}\PY{p}{)} \PY{c+c1}{\PYZsh{}loads data from drive}
              
              \PY{n}{fig}\PY{p}{,} \PY{n}{ax} \PY{o}{=} \PY{n}{plt}\PY{o}{.}\PY{n}{subplots}\PY{p}{(}\PY{l+m+mi}{1}\PY{p}{,} \PY{l+m+mi}{1}\PY{p}{,} \PY{n}{figsize} \PY{o}{=} \PY{p}{(}\PY{l+m+mi}{10}\PY{p}{,} \PY{l+m+mi}{5}\PY{p}{)}\PY{p}{)} \PY{c+c1}{\PYZsh{}initializes plot}
              \PY{n}{x} \PY{o}{=} \PY{n}{data}\PY{p}{[}\PY{p}{:}\PY{p}{,}\PY{l+m+mi}{0}\PY{p}{]} \PY{c+c1}{\PYZsh{}sets x data}
              \PY{n}{y} \PY{o}{=} \PY{n}{data}\PY{p}{[}\PY{p}{:}\PY{p}{,}\PY{l+m+mi}{1}\PY{p}{]} \PY{c+c1}{\PYZsh{}sets y data}
              
              \PY{n}{ex}\PY{p}{,} \PY{n}{ey}\PY{p}{,} \PY{n}{exx}\PY{p}{,} \PY{n}{exy} \PY{o}{=} \PY{l+m+mi}{0}\PY{p}{,} \PY{l+m+mi}{0}\PY{p}{,} \PY{l+m+mi}{0}\PY{p}{,} \PY{l+m+mi}{0} \PY{c+c1}{\PYZsh{}initializes variables used for running sum}
              \PY{k}{for} \PY{n}{i} \PY{o+ow}{in} \PY{n+nb}{range}\PY{p}{(}\PY{n}{np}\PY{o}{.}\PY{n}{shape}\PY{p}{(}\PY{n}{data}\PY{p}{)}\PY{p}{[}\PY{l+m+mi}{0}\PY{p}{]}\PY{p}{)}\PY{p}{:} \PY{c+c1}{\PYZsh{}this for loop calculates the running sum}
                  \PY{n}{ex} \PY{o}{+}\PY{o}{=} \PY{n}{x}\PY{p}{[}\PY{n}{i}\PY{p}{]}
                  \PY{n}{ey} \PY{o}{+}\PY{o}{=} \PY{n}{y}\PY{p}{[}\PY{n}{i}\PY{p}{]}
                  \PY{n}{exx} \PY{o}{+}\PY{o}{=} \PY{p}{(}\PY{n}{x}\PY{p}{[}\PY{n}{i}\PY{p}{]}\PY{p}{)}\PY{o}{*}\PY{o}{*}\PY{l+m+mi}{2}
                  \PY{n}{exy} \PY{o}{+}\PY{o}{=} \PY{n}{x}\PY{p}{[}\PY{n}{i}\PY{p}{]} \PY{o}{*} \PY{n}{y}\PY{p}{[}\PY{n}{i}\PY{p}{]}
              
              \PY{c+c1}{\PYZsh{}divides by the number of elements summed over}
              \PY{n}{Ex} \PY{o}{=} \PY{n}{ex} \PY{o}{/} \PY{n}{np}\PY{o}{.}\PY{n}{shape}\PY{p}{(}\PY{n}{data}\PY{p}{)}\PY{p}{[}\PY{l+m+mi}{0}\PY{p}{]}
              \PY{n}{Ey} \PY{o}{=} \PY{n}{ey} \PY{o}{/} \PY{n}{np}\PY{o}{.}\PY{n}{shape}\PY{p}{(}\PY{n}{data}\PY{p}{)}\PY{p}{[}\PY{l+m+mi}{0}\PY{p}{]}
              \PY{n}{Exx} \PY{o}{=} \PY{n}{exx} \PY{o}{/} \PY{n}{np}\PY{o}{.}\PY{n}{shape}\PY{p}{(}\PY{n}{data}\PY{p}{)}\PY{p}{[}\PY{l+m+mi}{0}\PY{p}{]}
              \PY{n}{Exy} \PY{o}{=} \PY{n}{exy} \PY{o}{/} \PY{n}{np}\PY{o}{.}\PY{n}{shape}\PY{p}{(}\PY{n}{data}\PY{p}{)}\PY{p}{[}\PY{l+m+mi}{0}\PY{p}{]}
              
              \PY{c+c1}{\PYZsh{}print(Ex, Ey, Exx, Exy)}
              \PY{n}{m} \PY{o}{=} \PY{p}{(}\PY{n}{Exy} \PY{o}{\PYZhy{}} \PY{n}{Ex}\PY{o}{*}\PY{n}{Ey}\PY{p}{)} \PY{o}{/} \PY{p}{(}\PY{n}{Exx} \PY{o}{\PYZhy{}} \PY{n}{Ex}\PY{o}{*}\PY{o}{*}\PY{l+m+mi}{2}\PY{p}{)}
              \PY{n}{c} \PY{o}{=} \PY{p}{(}\PY{n}{Exx}\PY{o}{*}\PY{n}{Ey} \PY{o}{\PYZhy{}} \PY{n}{Ex}\PY{o}{*}\PY{n}{Exy}\PY{p}{)} \PY{o}{/} \PY{p}{(}\PY{n}{Exx} \PY{o}{\PYZhy{}} \PY{n}{Ex}\PY{o}{*}\PY{o}{*}\PY{l+m+mi}{2}\PY{p}{)}
              \PY{n+nb}{print}\PY{p}{(}\PY{l+s+s2}{\PYZdq{}}\PY{l+s+s2}{The best fit line has slope m = }\PY{l+s+si}{\PYZob{}:4.2e\PYZcb{}}\PY{l+s+s2}{ and y intercept c = }\PY{l+s+si}{\PYZob{}:4.2f\PYZcb{}}\PY{l+s+s2}{.}\PY{l+s+s2}{\PYZdq{}}\PYZbs{}
                    \PY{o}{.}\PY{n}{format}\PY{p}{(}\PY{n}{m}\PY{p}{,} \PY{n}{c}\PY{p}{)}\PY{p}{)}
              
              \PY{c+c1}{\PYZsh{}part c, this creates predicted values based on the estimated parameters}
              \PY{n}{xhat} \PY{o}{=} \PY{n}{np}\PY{o}{.}\PY{n}{zeros}\PY{p}{(}\PY{n}{np}\PY{o}{.}\PY{n}{shape}\PY{p}{(}\PY{n}{data}\PY{p}{)}\PY{p}{[}\PY{l+m+mi}{0}\PY{p}{]}\PY{p}{)}
              \PY{k}{for} \PY{n}{i} \PY{o+ow}{in} \PY{n+nb}{range}\PY{p}{(}\PY{n}{np}\PY{o}{.}\PY{n}{shape}\PY{p}{(}\PY{n}{xhat}\PY{p}{)}\PY{p}{[}\PY{l+m+mi}{0}\PY{p}{]}\PY{p}{)}\PY{p}{:}
                  \PY{n}{xhat}\PY{p}{[}\PY{n}{i}\PY{p}{]} \PY{o}{=} \PY{n}{data}\PY{p}{[}\PY{n}{i}\PY{p}{,}\PY{l+m+mi}{0}\PY{p}{]} \PY{o}{*} \PY{n}{m} \PY{o}{+} \PY{n}{c}
              
              \PY{n}{plt}\PY{o}{.}\PY{n}{plot}\PY{p}{(}\PY{n}{x}\PY{p}{,} \PY{n}{y}\PY{p}{,} \PY{l+s+s1}{\PYZsq{}}\PY{l+s+s1}{k.}\PY{l+s+s1}{\PYZsq{}}\PY{p}{)}
              \PY{n}{plt}\PY{o}{.}\PY{n}{plot}\PY{p}{(}\PY{n}{x}\PY{p}{,} \PY{n}{xhat}\PY{p}{,} \PY{l+s+s1}{\PYZsq{}}\PY{l+s+s1}{k\PYZhy{}}\PY{l+s+s1}{\PYZsq{}}\PY{p}{,} \PY{n}{color}\PY{o}{=}\PY{l+s+s1}{\PYZsq{}}\PY{l+s+s1}{red}\PY{l+s+s1}{\PYZsq{}}\PY{p}{)}
              \PY{n}{plt}\PY{o}{.}\PY{n}{xlabel}\PY{p}{(}\PY{l+s+s2}{\PYZdq{}}\PY{l+s+s2}{Frequency (v)}\PY{l+s+s2}{\PYZdq{}}\PY{p}{)}
              \PY{n}{plt}\PY{o}{.}\PY{n}{ylabel}\PY{p}{(}\PY{l+s+s2}{\PYZdq{}}\PY{l+s+s2}{Voltage (V)}\PY{l+s+s2}{\PYZdq{}}\PY{p}{)}
              \PY{n}{plt}\PY{o}{.}\PY{n}{title}\PY{p}{(}\PY{l+s+s2}{\PYZdq{}}\PY{l+s+s2}{The photoelectric effect}\PY{l+s+s2}{\PYZdq{}}\PY{p}{)}
              \PY{n}{plt}\PY{o}{.}\PY{n}{show}\PY{p}{(}\PY{p}{)}
              
              \PY{n}{h} \PY{o}{=} \PY{n}{C}\PY{o}{.}\PY{n}{e} \PY{o}{*} \PY{n}{m}
              \PY{n}{error} \PY{o}{=} \PY{l+m+mi}{100}\PY{o}{*} \PY{n+nb}{abs}\PY{p}{(}\PY{n}{h} \PY{o}{\PYZhy{}} \PY{n}{C}\PY{o}{.}\PY{n}{h}\PY{p}{)} \PY{o}{/} \PY{n}{C}\PY{o}{.}\PY{n}{h}
              
              \PY{n+nb}{print}\PY{p}{(}\PY{l+s+s2}{\PYZdq{}}\PY{l+s+s2}{The experimental value of Planck}\PY{l+s+s2}{\PYZsq{}}\PY{l+s+s2}{s constant is h = }\PY{l+s+si}{\PYZob{}:4.2e\PYZcb{}}\PY{l+s+s2}{.}\PY{l+s+s2}{\PYZdq{}}\PYZbs{}
                    \PY{o}{.}\PY{n}{format}\PY{p}{(}\PY{n}{h}\PY{p}{)}\PY{p}{)}
              
              \PY{n+nb}{print}\PY{p}{(}\PY{l+s+s2}{\PYZdq{}}\PY{l+s+s2}{This is }\PY{l+s+si}{\PYZob{}:4.2f\PYZcb{}}\PY{l+s+s2}{\PYZpc{}}\PY{l+s+s2}{ away from the actual value of 6.626e\PYZhy{}34}\PY{l+s+s2}{\PYZdq{}}\PYZbs{}
                    \PY{o}{.}\PY{n}{format}\PY{p}{(}\PY{n}{error}\PY{p}{)}\PY{p}{)}
\end{Verbatim}

    \begin{Verbatim}[commandchars=\\\{\}]
{\color{incolor}In [{\color{incolor}175}]:} \PY{n}{photoelectric}\PY{p}{(}\PY{l+s+s2}{\PYZdq{}}\PY{l+s+s2}{millikan.txt}\PY{l+s+s2}{\PYZdq{}}\PY{p}{)}
\end{Verbatim}

    \begin{Verbatim}[commandchars=\\\{\}]
The best fit line has slope m = 4.09e-15 and y intercept c = -1.73.

    \end{Verbatim}

    \begin{center}
    \adjustimage{max size={0.9\linewidth}{0.9\paperheight}}{output_15_1.pdf}
    \end{center}
    { \hspace*{\fill} \\}
    
    \begin{Verbatim}[commandchars=\\\{\}]
The experimental value of Planck's constant is h = 6.55e-34.
This is 1.15\% away from the actual value of 6.626e-34

    \end{Verbatim}

    \section{CP 4.4 Calculating
integrals}\label{cp-4.4-calculating-integrals}

This problem looks at a rudimentary method of integration for a simple
plot of which we know the functional form is

\[y = \sqrt{1-x^2}.\]

By subdividing this region into N partitions and calculating a Riemann
sum as N becomes large, an approximation to the integral can be found.
It's exact value is \(\tfrac12 \pi = 1.57079632679 \ldots\)

\[I = \int_{-1}^1 \sqrt{1-x^2} dx \ \sim \ \lim_{N\to\infty} \sum_{k=1}^N hy_k\,, \textrm{ for  } y_k = \sqrt{1 - x_k^2}\qquad\mbox{and}\qquad
x_k = -1 + hk.\]

As stated the value of the integral for \(N = 100\) did not fair very
well, however it ran in a total of \(164 \mu s.\) The best value that
can be obtained while still running for under a second is for
\(N = 10000\)

    \begin{Verbatim}[commandchars=\\\{\}]
{\color{incolor}In [{\color{incolor}176}]:} \PY{k}{def} \PY{n+nf}{Riemann\PYZus{}semicircle}\PY{p}{(}\PY{n}{N}\PY{p}{)}\PY{p}{:}
              \PY{l+s+sd}{\PYZsq{}\PYZsq{}\PYZsq{}takes the number of partitions as an argument\PYZsq{}\PYZsq{}\PYZsq{}}
              
              \PY{n}{h} \PY{o}{=} \PY{l+m+mi}{2} \PY{o}{/} \PY{n}{N}
              
              \PY{n+nb}{sum} \PY{o}{=} \PY{l+m+mi}{0}
              \PY{k}{for} \PY{n}{k} \PY{o+ow}{in} \PY{n+nb}{range}\PY{p}{(}\PY{l+m+mi}{1}\PY{p}{,} \PY{n}{N}\PY{o}{+}\PY{l+m+mi}{1}\PY{p}{)}\PY{p}{:}
                  \PY{n}{x} \PY{o}{=} \PY{o}{\PYZhy{}}\PY{l+m+mi}{1} \PY{o}{+} \PY{n}{h}\PY{o}{*}\PY{n}{k}
                  \PY{n}{y} \PY{o}{=} \PY{n}{sqrt}\PY{p}{(}\PY{l+m+mi}{1} \PY{o}{\PYZhy{}} \PY{n}{x}\PY{o}{*}\PY{o}{*}\PY{l+m+mi}{2}\PY{p}{)}
                  
                  \PY{n+nb}{sum} \PY{o}{+}\PY{o}{=} \PY{n}{h} \PY{o}{*} \PY{n}{y}
                  
              \PY{k}{return} \PY{n+nb}{sum}
\end{Verbatim}

    \begin{Verbatim}[commandchars=\\\{\}]
{\color{incolor}In [{\color{incolor}177}]:} \PY{o}{\PYZpc{}\PYZpc{}}\PY{k}{time}
          print(\PYZdq{}For 100 partitions, this method of integration results in I = \PYZob{}:4.4f\PYZcb{}.\PYZdq{}\PYZbs{}
                .format(Riemann\PYZus{}semicircle(100)))
\end{Verbatim}

    \begin{Verbatim}[commandchars=\\\{\}]
For 100 partitions, this method of integration results in I = 1.5691.
CPU times: user 812 µs, sys: 783 µs, total: 1.59 ms
Wall time: 932 µs

    \end{Verbatim}

    \begin{Verbatim}[commandchars=\\\{\}]
{\color{incolor}In [{\color{incolor}178}]:} \PY{o}{\PYZpc{}\PYZpc{}}\PY{k}{time}
          Riemann\PYZus{}semicircle(100000)
          
          \PYZsh{}after N = 100000, the value reported is just 1.5708 as you keep increasing it
          \PYZsh{}further accuracy beyond that isn\PYZsq{}t obtained
\end{Verbatim}

    \begin{Verbatim}[commandchars=\\\{\}]
CPU times: user 46.8 ms, sys: 2.07 ms, total: 48.8 ms
Wall time: 74.1 ms

    \end{Verbatim}

\begin{Verbatim}[commandchars=\\\{\}]
{\color{outcolor}Out[{\color{outcolor}178}]:} 1.5708
\end{Verbatim}
            
    \section{CP 5.1 Velocity integration with the trapezoid
rule}\label{cp-5.1-velocity-integration-with-the-trapezoid-rule}

The theory behind the trapezoid rule is to approximate the local slope
of a function, so that the area of a thin slice can be calculated with
precision. The concept behind it is to match the first derivative of the
function in question.

The analytial form of the trapezoid rule is

\[I(a,b) = h \ \left( \frac{1}{2}(f(a) + f(b)) + \sum_k^{N-1} f(a+ kh) \right)\]

    \begin{Verbatim}[commandchars=\\\{\}]
{\color{incolor}In [{\color{incolor}179}]:} \PY{l+s+sd}{\PYZsq{}\PYZsq{}\PYZsq{}This cells defines a function based on the fact that the data}
          \PY{l+s+sd}{    has a fixed granularity. That is, it uses the existing jumps}
          \PY{l+s+sd}{    and calculates the area of the trapezoid between each}
          \PY{l+s+sd}{    observation\PYZsq{}\PYZsq{}\PYZsq{}}
          \PY{k}{def} \PY{n+nf}{trap}\PY{p}{(}\PY{n}{filename}\PY{p}{)}\PY{p}{:}
              \PY{l+s+sd}{\PYZsq{}\PYZsq{}\PYZsq{}directed to a file of data, this function will }
          \PY{l+s+sd}{        integrate the area under the curve mapped out\PYZsq{}\PYZsq{}\PYZsq{}}
          
              \PY{n}{data} \PY{o}{=} \PY{n}{np}\PY{o}{.}\PY{n}{loadtxt}\PY{p}{(}\PY{n}{filename}\PY{p}{,} \PY{n+nb}{float}\PY{p}{)}
              \PY{n}{x} \PY{o}{=} \PY{n}{data}\PY{p}{[}\PY{p}{:}\PY{p}{,}\PY{l+m+mi}{0}\PY{p}{]}
              \PY{n}{y} \PY{o}{=} \PY{n}{data}\PY{p}{[}\PY{p}{:}\PY{p}{,}\PY{l+m+mi}{1}\PY{p}{]}
              
              \PY{c+c1}{\PYZsh{}width of slices is 1, so area is average of bases}
              \PY{n}{dist} \PY{o}{=} \PY{n}{np}\PY{o}{.}\PY{n}{zeros}\PY{p}{(}\PY{n}{np}\PY{o}{.}\PY{n}{shape}\PY{p}{(}\PY{n}{data}\PY{p}{)}\PY{p}{[}\PY{l+m+mi}{0}\PY{p}{]} \PY{o}{\PYZhy{}} \PY{l+m+mi}{1}\PY{p}{)}
              \PY{n}{run} \PY{o}{=} \PY{l+m+mi}{0} \PY{c+c1}{\PYZsh{}the total distance traveled}
              \PY{k}{for} \PY{n}{i} \PY{o+ow}{in} \PY{n+nb}{range}\PY{p}{(}\PY{n}{np}\PY{o}{.}\PY{n}{shape}\PY{p}{(}\PY{n}{dist}\PY{p}{)}\PY{p}{[}\PY{l+m+mi}{0}\PY{p}{]}\PY{p}{)}\PY{p}{:}
                  \PY{n}{slice\PYZus{}area} \PY{o}{=} \PY{l+m+mf}{0.5} \PY{o}{*} \PY{p}{(}\PY{n}{y}\PY{p}{[}\PY{n}{i}\PY{p}{]} \PY{o}{+} \PY{n}{y}\PY{p}{[}\PY{n}{i}\PY{o}{\PYZhy{}}\PY{l+m+mi}{1}\PY{p}{]}\PY{p}{)}
                  \PY{n}{run} \PY{o}{+}\PY{o}{=} \PY{n}{slice\PYZus{}area}
                  \PY{n}{dist}\PY{p}{[}\PY{n}{i}\PY{p}{]} \PY{o}{=} \PY{n}{run}
              
              \PY{n}{fig}\PY{p}{,} \PY{n}{ax} \PY{o}{=} \PY{n}{plt}\PY{o}{.}\PY{n}{subplots}\PY{p}{(}\PY{l+m+mi}{1}\PY{p}{,} \PY{l+m+mi}{1}\PY{p}{,} \PY{n}{figsize} \PY{o}{=} \PY{p}{(}\PY{l+m+mi}{8}\PY{p}{,} \PY{l+m+mi}{8}\PY{p}{)}\PY{p}{)} \PY{c+c1}{\PYZsh{}initializes plot}
              \PY{n}{plt}\PY{o}{.}\PY{n}{plot}\PY{p}{(}\PY{n}{x}\PY{p}{,} \PY{n}{y}\PY{p}{,} \PY{l+s+s1}{\PYZsq{}}\PY{l+s+s1}{k\PYZhy{}}\PY{l+s+s1}{\PYZsq{}}\PY{p}{,} \PY{n}{color}\PY{o}{=}\PY{l+s+s1}{\PYZsq{}}\PY{l+s+s1}{blue}\PY{l+s+s1}{\PYZsq{}}\PY{p}{)} \PY{c+c1}{\PYZsh{}graphs velocity data}
              \PY{n}{plt}\PY{o}{.}\PY{n}{plot}\PY{p}{(}\PY{n}{x}\PY{p}{[}\PY{l+m+mi}{1}\PY{p}{:}\PY{p}{]}\PY{p}{,} \PY{n}{dist}\PY{p}{,} \PY{l+s+s1}{\PYZsq{}}\PY{l+s+s1}{k\PYZhy{}}\PY{l+s+s1}{\PYZsq{}}\PY{p}{,} \PY{n}{color}\PY{o}{=}\PY{l+s+s1}{\PYZsq{}}\PY{l+s+s1}{red}\PY{l+s+s1}{\PYZsq{}}\PY{p}{)} \PY{c+c1}{\PYZsh{}graphs cumulative distance}
              
              \PY{c+c1}{\PYZsh{}plot aesthetics}
              \PY{n}{plt}\PY{o}{.}\PY{n}{title}\PY{p}{(}\PY{l+s+s1}{\PYZsq{}}\PY{l+s+s1}{Velocity (blue) and Displacement (red) via Trapezoid Rule}\PY{l+s+s1}{\PYZsq{}}\PY{p}{)}
              \PY{n}{plt}\PY{o}{.}\PY{n}{xlabel}\PY{p}{(}\PY{l+s+s1}{\PYZsq{}}\PY{l+s+s1}{time}\PY{l+s+s1}{\PYZsq{}}\PY{p}{)}
              \PY{n}{plt}\PY{o}{.}\PY{n}{show}\PY{p}{(}\PY{p}{)}
          
          \PY{n}{trap}\PY{p}{(}\PY{l+s+s2}{\PYZdq{}}\PY{l+s+s2}{velocities.txt}\PY{l+s+s2}{\PYZdq{}}\PY{p}{)}
\end{Verbatim}

    \begin{center}
    \adjustimage{max size={0.9\linewidth}{0.9\paperheight}}{output_21_0.pdf}
    \end{center}
    { \hspace*{\fill} \\}
    
    \begin{Verbatim}[commandchars=\\\{\}]
{\color{incolor}In [{\color{incolor}180}]:} \PY{l+s+sd}{\PYZsq{}\PYZsq{}\PYZsq{}This cells defines a function in a more general way,}
          \PY{l+s+sd}{    in which you can set the number of slices to calculate}
          \PY{l+s+sd}{    partial areas for. However, because there are only}
          \PY{l+s+sd}{    100 data data points, you cannot have more slices}
          \PY{l+s+sd}{    than this.\PYZsq{}\PYZsq{}\PYZsq{}}
          \PY{k}{def} \PY{n+nf}{trap}\PY{p}{(}\PY{n}{filename}\PY{p}{,} \PY{n}{N}\PY{p}{)}\PY{p}{:}
              \PY{l+s+sd}{\PYZsq{}\PYZsq{}\PYZsq{}directed to a file of data, this function will }
          \PY{l+s+sd}{        integrate the area under the curve mapped out\PYZsq{}\PYZsq{}\PYZsq{}}
          
              \PY{n}{data} \PY{o}{=} \PY{n}{np}\PY{o}{.}\PY{n}{loadtxt}\PY{p}{(}\PY{n}{filename}\PY{p}{,} \PY{n+nb}{float}\PY{p}{)}
              \PY{n}{x} \PY{o}{=} \PY{n}{data}\PY{p}{[}\PY{p}{:}\PY{p}{,}\PY{l+m+mi}{0}\PY{p}{]}
              \PY{n}{y} \PY{o}{=} \PY{n}{data}\PY{p}{[}\PY{p}{:}\PY{p}{,}\PY{l+m+mi}{1}\PY{p}{]}
              
              \PY{n}{a} \PY{o}{=} \PY{n}{x}\PY{p}{[}\PY{l+m+mi}{0}\PY{p}{]}
              \PY{n}{b} \PY{o}{=} \PY{n}{y}\PY{p}{[}\PY{o}{\PYZhy{}}\PY{l+m+mi}{1}\PY{p}{]}
              \PY{n}{h} \PY{o}{=} \PY{p}{(}\PY{n}{b}\PY{o}{\PYZhy{}}\PY{n}{a}\PY{p}{)} \PY{o}{/} \PY{n}{N}
              
              \PY{c+c1}{\PYZsh{}width of slices is 1, so area is average of bases}
              \PY{n}{dist} \PY{o}{=} \PY{n}{np}\PY{o}{.}\PY{n}{zeros}\PY{p}{(}\PY{n}{np}\PY{o}{.}\PY{n}{shape}\PY{p}{(}\PY{n}{data}\PY{p}{)}\PY{p}{[}\PY{l+m+mi}{0}\PY{p}{]} \PY{o}{\PYZhy{}} \PY{l+m+mi}{1}\PY{p}{)}
              \PY{n}{run} \PY{o}{=} \PY{l+m+mi}{0} \PY{c+c1}{\PYZsh{}the total distance traveled}
              \PY{n}{run} \PY{o}{+}\PY{o}{=} \PY{l+m+mf}{0.5} \PY{o}{*} \PY{n}{y}\PY{p}{[}\PY{l+m+mi}{0}\PY{p}{]}
              \PY{n}{run} \PY{o}{+}\PY{o}{=} \PY{l+m+mf}{0.5} \PY{o}{*} \PY{n}{y}\PY{p}{[}\PY{o}{\PYZhy{}}\PY{l+m+mi}{1}\PY{p}{]}
              \PY{k}{for} \PY{n}{i} \PY{o+ow}{in} \PY{n+nb}{range}\PY{p}{(}\PY{n}{N}\PY{p}{)}\PY{p}{:}
                  \PY{n}{slice\PYZus{}area} \PY{o}{=} \PY{l+m+mf}{0.5} \PY{o}{*} \PY{p}{(}\PY{n}{y}\PY{p}{[}\PY{n}{i}\PY{p}{]} \PY{o}{+} \PY{n}{y}\PY{p}{[}\PY{n}{i}\PY{o}{\PYZhy{}}\PY{l+m+mi}{1}\PY{p}{]}\PY{p}{)} \PY{o}{*} \PY{n}{h}
                  \PY{n}{run} \PY{o}{+}\PY{o}{=} \PY{n}{slice\PYZus{}area}
                  \PY{n}{dist}\PY{p}{[}\PY{n}{i}\PY{p}{]} \PY{o}{=} \PY{n}{run}
              
              \PY{n}{fig}\PY{p}{,} \PY{n}{ax} \PY{o}{=} \PY{n}{plt}\PY{o}{.}\PY{n}{subplots}\PY{p}{(}\PY{l+m+mi}{1}\PY{p}{,} \PY{l+m+mi}{1}\PY{p}{,} \PY{n}{figsize} \PY{o}{=} \PY{p}{(}\PY{l+m+mi}{4}\PY{p}{,}\PY{l+m+mi}{4}\PY{p}{)}\PY{p}{)} \PY{c+c1}{\PYZsh{}initializes plot}
              \PY{n}{plt}\PY{o}{.}\PY{n}{plot}\PY{p}{(}\PY{n}{x}\PY{p}{,} \PY{n}{y}\PY{p}{,} \PY{l+s+s1}{\PYZsq{}}\PY{l+s+s1}{k\PYZhy{}}\PY{l+s+s1}{\PYZsq{}}\PY{p}{,} \PY{n}{color}\PY{o}{=}\PY{l+s+s1}{\PYZsq{}}\PY{l+s+s1}{blue}\PY{l+s+s1}{\PYZsq{}}\PY{p}{)} \PY{c+c1}{\PYZsh{}graphs velocity data}
              \PY{n}{plt}\PY{o}{.}\PY{n}{plot}\PY{p}{(}\PY{n}{x}\PY{p}{[}\PY{l+m+mi}{1}\PY{p}{:}\PY{p}{]}\PY{p}{,} \PY{n}{dist}\PY{p}{,} \PY{l+s+s1}{\PYZsq{}}\PY{l+s+s1}{k\PYZhy{}}\PY{l+s+s1}{\PYZsq{}}\PY{p}{,} \PY{n}{color}\PY{o}{=}\PY{l+s+s1}{\PYZsq{}}\PY{l+s+s1}{red}\PY{l+s+s1}{\PYZsq{}}\PY{p}{)} \PY{c+c1}{\PYZsh{}graphs cumulative distance}
              
              \PY{c+c1}{\PYZsh{}plot aesthetics}
              \PY{n}{plt}\PY{o}{.}\PY{n}{title}\PY{p}{(}\PY{l+s+s1}{\PYZsq{}}\PY{l+s+s1}{Velocity (blue) and Displacement (red) via Trapezoid Rule}\PY{l+s+s1}{\PYZsq{}}\PY{p}{)}
              \PY{n}{plt}\PY{o}{.}\PY{n}{xlabel}\PY{p}{(}\PY{l+s+s1}{\PYZsq{}}\PY{l+s+s1}{time}\PY{l+s+s1}{\PYZsq{}}\PY{p}{)}
              \PY{n}{plt}\PY{o}{.}\PY{n}{show}\PY{p}{(}\PY{p}{)}
          
          \PY{n}{trap}\PY{p}{(}\PY{l+s+s2}{\PYZdq{}}\PY{l+s+s2}{velocities.txt}\PY{l+s+s2}{\PYZdq{}}\PY{p}{,} \PY{l+m+mi}{50}\PY{p}{)}
\end{Verbatim}

    \begin{center}
    \adjustimage{max size={0.9\linewidth}{0.9\paperheight}}{output_22_0.pdf}
    \end{center}
    { \hspace*{\fill} \\}
    
    \section{CP 5.2 Integration with Simpson's
rule}\label{cp-5.2-integration-with-simpsons-rule}

This problem looks at evaluating the following integral with Simpson's
rule

\[\int_0^2 x^4 -2x+1 \ \textrm{d}x.\]

The correct value of this integral can be found easily because we know
the functional form explicitly.

\[\int_0^2 x^4 -2x+1 \ \textrm{d}x = \tfrac15 x^5 - x^2 + x\biggr\rvert_0^2 = 4.4.\]

The simplified form of Simpson's rule is the integral

\[ I = \frac{1}{3} \ h \ \left( f(a) + f(b) + 4 \sum_{k = \textrm{odd}} f(a+kh) + 2 \sum_{k = \textrm{even}} f(a+kh)  \right).\]

    \begin{Verbatim}[commandchars=\\\{\}]
{\color{incolor}In [{\color{incolor}181}]:} \PY{k}{def} \PY{n+nf}{f}\PY{p}{(}\PY{n}{x}\PY{p}{)}\PY{p}{:}
              \PY{k}{return} \PY{n}{x}\PY{o}{*}\PY{o}{*}\PY{l+m+mi}{4} \PY{o}{\PYZhy{}} \PY{l+m+mi}{2}\PY{o}{*}\PY{n}{x} \PY{o}{+} \PY{l+m+mi}{1}
          
          \PY{k}{def} \PY{n+nf}{simps}\PY{p}{(}\PY{n}{a}\PY{p}{,} \PY{n}{b}\PY{p}{,} \PY{n}{N}\PY{p}{)}\PY{p}{:}
              \PY{l+s+sd}{\PYZdq{}\PYZdq{}\PYZdq{}integrates a function using Simpson\PYZsq{}s Rule}
          \PY{l+s+sd}{        from a to b with N slices\PYZdq{}\PYZdq{}\PYZdq{}}
              \PY{c+c1}{\PYZsh{}\PYZpc{}\PYZpc{}time}
              
              \PY{n}{h} \PY{o}{=} \PY{p}{(}\PY{n}{b}\PY{o}{\PYZhy{}}\PY{n}{a}\PY{p}{)} \PY{o}{/} \PY{n}{N}
              
              \PY{n}{sum1} \PY{o}{=} \PY{l+m+mi}{0}
              \PY{n}{sum2} \PY{o}{=} \PY{l+m+mi}{0}
              
              \PY{k}{for} \PY{n}{k} \PY{o+ow}{in} \PY{n+nb}{range}\PY{p}{(}\PY{n}{N}\PY{p}{)}\PY{p}{:}
                  \PY{k}{if} \PY{n}{k} \PY{o}{\PYZpc{}} \PY{l+m+mi}{2} \PY{o}{==} \PY{l+m+mi}{1}\PY{p}{:}
                      \PY{n}{sum1} \PY{o}{+}\PY{o}{=} \PY{n}{f}\PY{p}{(}\PY{n}{a} \PY{o}{+} \PY{n}{k}\PY{o}{*}\PY{n}{h}\PY{p}{)}
                  \PY{k}{elif} \PY{n}{k}\PY{o}{\PYZpc{}} \PY{l+m+mi}{2} \PY{o}{==} \PY{l+m+mi}{0}\PY{p}{:}
                      \PY{n}{sum2} \PY{o}{+}\PY{o}{=} \PY{n}{f}\PY{p}{(}\PY{n}{a} \PY{o}{+} \PY{n}{k}\PY{o}{*}\PY{n}{h}\PY{p}{)}
              \PY{n}{I} \PY{o}{=} \PY{p}{(}\PY{l+m+mi}{1}\PY{o}{/}\PY{l+m+mi}{3}\PY{p}{)}\PY{o}{*}\PY{n}{h} \PY{o}{*} \PY{p}{(}\PY{n}{f}\PY{p}{(}\PY{n}{a}\PY{p}{)} \PY{o}{+} \PY{n}{f}\PY{p}{(}\PY{n}{b}\PY{p}{)} \PY{o}{+} \PY{l+m+mi}{4}\PY{o}{*}\PY{n}{sum1} \PY{o}{+} \PY{l+m+mi}{2}\PY{o}{*}\PY{n}{sum2}\PY{p}{)}
              \PY{n}{error} \PY{o}{=} \PY{l+m+mi}{100} \PY{o}{*} \PY{p}{(}\PY{n}{I} \PY{o}{\PYZhy{}} \PY{l+m+mf}{4.4}\PY{p}{)} \PY{o}{/} \PY{l+m+mf}{4.4} \PY{c+c1}{\PYZsh{}percent error}
              
              \PY{k}{return} \PY{n}{I}\PY{p}{,} \PY{n}{error}
\end{Verbatim}

    \begin{Verbatim}[commandchars=\\\{\}]
{\color{incolor}In [{\color{incolor}182}]:} \PY{n+nb}{print}\PY{p}{(}\PY{l+s+s2}{\PYZdq{}}\PY{l+s+s2}{For N = 10, the integral and fractional error are (}\PY{l+s+si}{\PYZob{}:4.4f\PYZcb{}}\PY{l+s+s2}{, }\PY{l+s+si}{\PYZob{}:4.3f\PYZcb{}}\PY{l+s+s2}{\PYZpc{}}\PY{l+s+s2}{).}\PY{l+s+s2}{\PYZdq{}}\PYZbs{}
                \PY{o}{.}\PY{n}{format}\PY{p}{(}\PY{n}{simps}\PY{p}{(}\PY{l+m+mi}{0}\PY{p}{,} \PY{l+m+mi}{2}\PY{p}{,} \PY{l+m+mi}{10}\PY{p}{)}\PY{p}{[}\PY{l+m+mi}{0}\PY{p}{]}\PY{p}{,}\PY{n}{simps}\PY{p}{(}\PY{l+m+mi}{0}\PY{p}{,} \PY{l+m+mi}{2}\PY{p}{,} \PY{l+m+mi}{10}\PY{p}{)}\PY{p}{[}\PY{l+m+mi}{1}\PY{p}{]}\PY{p}{)}\PY{p}{)}
          \PY{n+nb}{print}\PY{p}{(}\PY{l+s+s2}{\PYZdq{}}\PY{l+s+s2}{For N = 100, the integral and fractional error are (}\PY{l+s+si}{\PYZob{}:4.4f\PYZcb{}}\PY{l+s+s2}{, }\PY{l+s+si}{\PYZob{}:4.3f\PYZcb{}}\PY{l+s+s2}{\PYZpc{}}\PY{l+s+s2}{).}\PY{l+s+s2}{\PYZdq{}}\PYZbs{}
                \PY{o}{.}\PY{n}{format}\PY{p}{(}\PY{n}{simps}\PY{p}{(}\PY{l+m+mi}{0}\PY{p}{,} \PY{l+m+mi}{2}\PY{p}{,} \PY{l+m+mi}{100}\PY{p}{)}\PY{p}{[}\PY{l+m+mi}{0}\PY{p}{]}\PY{p}{,}\PY{n}{simps}\PY{p}{(}\PY{l+m+mi}{0}\PY{p}{,} \PY{l+m+mi}{2}\PY{p}{,} \PY{l+m+mi}{100}\PY{p}{)}\PY{p}{[}\PY{l+m+mi}{1}\PY{p}{]}\PY{p}{)}\PY{p}{)}
          \PY{n+nb}{print}\PY{p}{(}\PY{l+s+s2}{\PYZdq{}}\PY{l+s+s2}{For N = 1000, the integral and fractional error are (}\PY{l+s+si}{\PYZob{}:4.4f\PYZcb{}}\PY{l+s+s2}{, }\PY{l+s+si}{\PYZob{}:4.3f\PYZcb{}}\PY{l+s+s2}{\PYZpc{}}\PY{l+s+s2}{).}\PY{l+s+s2}{\PYZdq{}}\PYZbs{}
                \PY{o}{.}\PY{n}{format}\PY{p}{(}\PY{n}{simps}\PY{p}{(}\PY{l+m+mi}{0}\PY{p}{,} \PY{l+m+mi}{2}\PY{p}{,} \PY{l+m+mi}{1000}\PY{p}{)}\PY{p}{[}\PY{l+m+mi}{0}\PY{p}{]}\PY{p}{,}\PY{n}{simps}\PY{p}{(}\PY{l+m+mi}{0}\PY{p}{,} \PY{l+m+mi}{2}\PY{p}{,} \PY{l+m+mi}{1000}\PY{p}{)}\PY{p}{[}\PY{l+m+mi}{1}\PY{p}{]}\PY{p}{)}\PY{p}{)}
          \PY{n+nb}{print}\PY{p}{(}\PY{l+s+s2}{\PYZdq{}}\PY{l+s+s2}{For N = 10000, the integral and fractional error are (}\PY{l+s+si}{\PYZob{}:4.4f\PYZcb{}}\PY{l+s+s2}{, }\PY{l+s+si}{\PYZob{}:4.3f\PYZcb{}}\PY{l+s+s2}{\PYZpc{}}\PY{l+s+s2}{).}\PY{l+s+s2}{\PYZdq{}}\PYZbs{}
                \PY{o}{.}\PY{n}{format}\PY{p}{(}\PY{n}{simps}\PY{p}{(}\PY{l+m+mi}{0}\PY{p}{,} \PY{l+m+mi}{2}\PY{p}{,} \PY{l+m+mi}{10000}\PY{p}{)}\PY{p}{[}\PY{l+m+mi}{0}\PY{p}{]}\PY{p}{,}\PY{n}{simps}\PY{p}{(}\PY{l+m+mi}{0}\PY{p}{,} \PY{l+m+mi}{2}\PY{p}{,} \PY{l+m+mi}{10000}\PY{p}{)}\PY{p}{[}\PY{l+m+mi}{1}\PY{p}{]}\PY{p}{)}\PY{p}{)}
\end{Verbatim}

    \begin{Verbatim}[commandchars=\\\{\}]
For N = 10, the integral and fractional error are (4.5338, 3.040\%).
For N = 100, the integral and fractional error are (4.4133, 0.303\%).
For N = 1000, the integral and fractional error are (4.4013, 0.030\%).
For N = 10000, the integral and fractional error are (4.4001, 0.003\%).

    \end{Verbatim}

    If only using 10 slices, the fractional error using Simpson's Rule is
3.04\%. Adding 10 times more slices each time results in fractional
errors almost exactly 10 times less on each iteration. The results are
far superior to the trapezoid rule for equivalent amounts of slices.
However, these cannot be directly compared because the slice limit for
5.1's use of the trapezoid rule had a floor to the slice width. Because
there was a limited number of data points, we could only calculate
trapezoid areas for the data we had. However in this problem, we were
given a function and were able to slice the plot as finely as we liked.

    \begin{Verbatim}[commandchars=\\\{\}]
{\color{incolor}In [{\color{incolor} }]:} 
\end{Verbatim}


    % Add a bibliography block to the postdoc
    
    
    
    \end{document}

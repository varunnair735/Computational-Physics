
% Default to the notebook output style

    


% Inherit from the specified cell style.




    
\documentclass[11pt]{article}

    
    
    \usepackage[T1]{fontenc}
    % Nicer default font (+ math font) than Computer Modern for most use cases
    \usepackage{mathpazo}

    % Basic figure setup, for now with no caption control since it's done
    % automatically by Pandoc (which extracts ![](path) syntax from Markdown).
    \usepackage{graphicx}
    % We will generate all images so they have a width \maxwidth. This means
    % that they will get their normal width if they fit onto the page, but
    % are scaled down if they would overflow the margins.
    \makeatletter
    \def\maxwidth{\ifdim\Gin@nat@width>\linewidth\linewidth
    \else\Gin@nat@width\fi}
    \makeatother
    \let\Oldincludegraphics\includegraphics
    % Set max figure width to be 80% of text width, for now hardcoded.
    \renewcommand{\includegraphics}[1]{\Oldincludegraphics[width=.8\maxwidth]{#1}}
    % Ensure that by default, figures have no caption (until we provide a
    % proper Figure object with a Caption API and a way to capture that
    % in the conversion process - todo).
    \usepackage{caption}
    \DeclareCaptionLabelFormat{nolabel}{}
    \captionsetup{labelformat=nolabel}

    \usepackage{adjustbox} % Used to constrain images to a maximum size 
    \usepackage{xcolor} % Allow colors to be defined
    \usepackage{enumerate} % Needed for markdown enumerations to work
    \usepackage{geometry} % Used to adjust the document margins
    \usepackage{amsmath} % Equations
    \usepackage{amssymb} % Equations
    \usepackage{textcomp} % defines textquotesingle
    % Hack from http://tex.stackexchange.com/a/47451/13684:
    \AtBeginDocument{%
        \def\PYZsq{\textquotesingle}% Upright quotes in Pygmentized code
    }
    \usepackage{upquote} % Upright quotes for verbatim code
    \usepackage{eurosym} % defines \euro
    \usepackage[mathletters]{ucs} % Extended unicode (utf-8) support
    \usepackage[utf8x]{inputenc} % Allow utf-8 characters in the tex document
    \usepackage{fancyvrb} % verbatim replacement that allows latex
    \usepackage{grffile} % extends the file name processing of package graphics 
                         % to support a larger range 
    % The hyperref package gives us a pdf with properly built
    % internal navigation ('pdf bookmarks' for the table of contents,
    % internal cross-reference links, web links for URLs, etc.)
    \usepackage{hyperref}
    \usepackage{longtable} % longtable support required by pandoc >1.10
    \usepackage{booktabs}  % table support for pandoc > 1.12.2
    \usepackage[inline]{enumitem} % IRkernel/repr support (it uses the enumerate* environment)
    \usepackage[normalem]{ulem} % ulem is needed to support strikethroughs (\sout)
                                % normalem makes italics be italics, not underlines
    \usepackage{mathrsfs}
    

    
    
    % Colors for the hyperref package
    \definecolor{urlcolor}{rgb}{0,.145,.698}
    \definecolor{linkcolor}{rgb}{.71,0.21,0.01}
    \definecolor{citecolor}{rgb}{.12,.54,.11}

    % ANSI colors
    \definecolor{ansi-black}{HTML}{3E424D}
    \definecolor{ansi-black-intense}{HTML}{282C36}
    \definecolor{ansi-red}{HTML}{E75C58}
    \definecolor{ansi-red-intense}{HTML}{B22B31}
    \definecolor{ansi-green}{HTML}{00A250}
    \definecolor{ansi-green-intense}{HTML}{007427}
    \definecolor{ansi-yellow}{HTML}{DDB62B}
    \definecolor{ansi-yellow-intense}{HTML}{B27D12}
    \definecolor{ansi-blue}{HTML}{208FFB}
    \definecolor{ansi-blue-intense}{HTML}{0065CA}
    \definecolor{ansi-magenta}{HTML}{D160C4}
    \definecolor{ansi-magenta-intense}{HTML}{A03196}
    \definecolor{ansi-cyan}{HTML}{60C6C8}
    \definecolor{ansi-cyan-intense}{HTML}{258F8F}
    \definecolor{ansi-white}{HTML}{C5C1B4}
    \definecolor{ansi-white-intense}{HTML}{A1A6B2}
    \definecolor{ansi-default-inverse-fg}{HTML}{FFFFFF}
    \definecolor{ansi-default-inverse-bg}{HTML}{000000}

    % commands and environments needed by pandoc snippets
    % extracted from the output of `pandoc -s`
    \providecommand{\tightlist}{%
      \setlength{\itemsep}{0pt}\setlength{\parskip}{0pt}}
    \DefineVerbatimEnvironment{Highlighting}{Verbatim}{commandchars=\\\{\}}
    % Add ',fontsize=\small' for more characters per line
    \newenvironment{Shaded}{}{}
    \newcommand{\KeywordTok}[1]{\textcolor[rgb]{0.00,0.44,0.13}{\textbf{{#1}}}}
    \newcommand{\DataTypeTok}[1]{\textcolor[rgb]{0.56,0.13,0.00}{{#1}}}
    \newcommand{\DecValTok}[1]{\textcolor[rgb]{0.25,0.63,0.44}{{#1}}}
    \newcommand{\BaseNTok}[1]{\textcolor[rgb]{0.25,0.63,0.44}{{#1}}}
    \newcommand{\FloatTok}[1]{\textcolor[rgb]{0.25,0.63,0.44}{{#1}}}
    \newcommand{\CharTok}[1]{\textcolor[rgb]{0.25,0.44,0.63}{{#1}}}
    \newcommand{\StringTok}[1]{\textcolor[rgb]{0.25,0.44,0.63}{{#1}}}
    \newcommand{\CommentTok}[1]{\textcolor[rgb]{0.38,0.63,0.69}{\textit{{#1}}}}
    \newcommand{\OtherTok}[1]{\textcolor[rgb]{0.00,0.44,0.13}{{#1}}}
    \newcommand{\AlertTok}[1]{\textcolor[rgb]{1.00,0.00,0.00}{\textbf{{#1}}}}
    \newcommand{\FunctionTok}[1]{\textcolor[rgb]{0.02,0.16,0.49}{{#1}}}
    \newcommand{\RegionMarkerTok}[1]{{#1}}
    \newcommand{\ErrorTok}[1]{\textcolor[rgb]{1.00,0.00,0.00}{\textbf{{#1}}}}
    \newcommand{\NormalTok}[1]{{#1}}
    
    % Additional commands for more recent versions of Pandoc
    \newcommand{\ConstantTok}[1]{\textcolor[rgb]{0.53,0.00,0.00}{{#1}}}
    \newcommand{\SpecialCharTok}[1]{\textcolor[rgb]{0.25,0.44,0.63}{{#1}}}
    \newcommand{\VerbatimStringTok}[1]{\textcolor[rgb]{0.25,0.44,0.63}{{#1}}}
    \newcommand{\SpecialStringTok}[1]{\textcolor[rgb]{0.73,0.40,0.53}{{#1}}}
    \newcommand{\ImportTok}[1]{{#1}}
    \newcommand{\DocumentationTok}[1]{\textcolor[rgb]{0.73,0.13,0.13}{\textit{{#1}}}}
    \newcommand{\AnnotationTok}[1]{\textcolor[rgb]{0.38,0.63,0.69}{\textbf{\textit{{#1}}}}}
    \newcommand{\CommentVarTok}[1]{\textcolor[rgb]{0.38,0.63,0.69}{\textbf{\textit{{#1}}}}}
    \newcommand{\VariableTok}[1]{\textcolor[rgb]{0.10,0.09,0.49}{{#1}}}
    \newcommand{\ControlFlowTok}[1]{\textcolor[rgb]{0.00,0.44,0.13}{\textbf{{#1}}}}
    \newcommand{\OperatorTok}[1]{\textcolor[rgb]{0.40,0.40,0.40}{{#1}}}
    \newcommand{\BuiltInTok}[1]{{#1}}
    \newcommand{\ExtensionTok}[1]{{#1}}
    \newcommand{\PreprocessorTok}[1]{\textcolor[rgb]{0.74,0.48,0.00}{{#1}}}
    \newcommand{\AttributeTok}[1]{\textcolor[rgb]{0.49,0.56,0.16}{{#1}}}
    \newcommand{\InformationTok}[1]{\textcolor[rgb]{0.38,0.63,0.69}{\textbf{\textit{{#1}}}}}
    \newcommand{\WarningTok}[1]{\textcolor[rgb]{0.38,0.63,0.69}{\textbf{\textit{{#1}}}}}
    
    
    % Define a nice break command that doesn't care if a line doesn't already
    % exist.
    \def\br{\hspace*{\fill} \\* }
    % Math Jax compatibility definitions
    \def\gt{>}
    \def\lt{<}
    \let\Oldtex\TeX
    \let\Oldlatex\LaTeX
    \renewcommand{\TeX}{\textrm{\Oldtex}}
    \renewcommand{\LaTeX}{\textrm{\Oldlatex}}
    % Document parameters
    % Document title
    \title{Homework 3 \\ \vspace{10mm}
    {\large Varun Nair}}
    
    
    
    
    

    % Pygments definitions
    
\makeatletter
\def\PY@reset{\let\PY@it=\relax \let\PY@bf=\relax%
    \let\PY@ul=\relax \let\PY@tc=\relax%
    \let\PY@bc=\relax \let\PY@ff=\relax}
\def\PY@tok#1{\csname PY@tok@#1\endcsname}
\def\PY@toks#1+{\ifx\relax#1\empty\else%
    \PY@tok{#1}\expandafter\PY@toks\fi}
\def\PY@do#1{\PY@bc{\PY@tc{\PY@ul{%
    \PY@it{\PY@bf{\PY@ff{#1}}}}}}}
\def\PY#1#2{\PY@reset\PY@toks#1+\relax+\PY@do{#2}}

\expandafter\def\csname PY@tok@w\endcsname{\def\PY@tc##1{\textcolor[rgb]{0.73,0.73,0.73}{##1}}}
\expandafter\def\csname PY@tok@c\endcsname{\let\PY@it=\textit\def\PY@tc##1{\textcolor[rgb]{0.25,0.50,0.50}{##1}}}
\expandafter\def\csname PY@tok@cp\endcsname{\def\PY@tc##1{\textcolor[rgb]{0.74,0.48,0.00}{##1}}}
\expandafter\def\csname PY@tok@k\endcsname{\let\PY@bf=\textbf\def\PY@tc##1{\textcolor[rgb]{0.00,0.50,0.00}{##1}}}
\expandafter\def\csname PY@tok@kp\endcsname{\def\PY@tc##1{\textcolor[rgb]{0.00,0.50,0.00}{##1}}}
\expandafter\def\csname PY@tok@kt\endcsname{\def\PY@tc##1{\textcolor[rgb]{0.69,0.00,0.25}{##1}}}
\expandafter\def\csname PY@tok@o\endcsname{\def\PY@tc##1{\textcolor[rgb]{0.40,0.40,0.40}{##1}}}
\expandafter\def\csname PY@tok@ow\endcsname{\let\PY@bf=\textbf\def\PY@tc##1{\textcolor[rgb]{0.67,0.13,1.00}{##1}}}
\expandafter\def\csname PY@tok@nb\endcsname{\def\PY@tc##1{\textcolor[rgb]{0.00,0.50,0.00}{##1}}}
\expandafter\def\csname PY@tok@nf\endcsname{\def\PY@tc##1{\textcolor[rgb]{0.00,0.00,1.00}{##1}}}
\expandafter\def\csname PY@tok@nc\endcsname{\let\PY@bf=\textbf\def\PY@tc##1{\textcolor[rgb]{0.00,0.00,1.00}{##1}}}
\expandafter\def\csname PY@tok@nn\endcsname{\let\PY@bf=\textbf\def\PY@tc##1{\textcolor[rgb]{0.00,0.00,1.00}{##1}}}
\expandafter\def\csname PY@tok@ne\endcsname{\let\PY@bf=\textbf\def\PY@tc##1{\textcolor[rgb]{0.82,0.25,0.23}{##1}}}
\expandafter\def\csname PY@tok@nv\endcsname{\def\PY@tc##1{\textcolor[rgb]{0.10,0.09,0.49}{##1}}}
\expandafter\def\csname PY@tok@no\endcsname{\def\PY@tc##1{\textcolor[rgb]{0.53,0.00,0.00}{##1}}}
\expandafter\def\csname PY@tok@nl\endcsname{\def\PY@tc##1{\textcolor[rgb]{0.63,0.63,0.00}{##1}}}
\expandafter\def\csname PY@tok@ni\endcsname{\let\PY@bf=\textbf\def\PY@tc##1{\textcolor[rgb]{0.60,0.60,0.60}{##1}}}
\expandafter\def\csname PY@tok@na\endcsname{\def\PY@tc##1{\textcolor[rgb]{0.49,0.56,0.16}{##1}}}
\expandafter\def\csname PY@tok@nt\endcsname{\let\PY@bf=\textbf\def\PY@tc##1{\textcolor[rgb]{0.00,0.50,0.00}{##1}}}
\expandafter\def\csname PY@tok@nd\endcsname{\def\PY@tc##1{\textcolor[rgb]{0.67,0.13,1.00}{##1}}}
\expandafter\def\csname PY@tok@s\endcsname{\def\PY@tc##1{\textcolor[rgb]{0.73,0.13,0.13}{##1}}}
\expandafter\def\csname PY@tok@sd\endcsname{\let\PY@it=\textit\def\PY@tc##1{\textcolor[rgb]{0.73,0.13,0.13}{##1}}}
\expandafter\def\csname PY@tok@si\endcsname{\let\PY@bf=\textbf\def\PY@tc##1{\textcolor[rgb]{0.73,0.40,0.53}{##1}}}
\expandafter\def\csname PY@tok@se\endcsname{\let\PY@bf=\textbf\def\PY@tc##1{\textcolor[rgb]{0.73,0.40,0.13}{##1}}}
\expandafter\def\csname PY@tok@sr\endcsname{\def\PY@tc##1{\textcolor[rgb]{0.73,0.40,0.53}{##1}}}
\expandafter\def\csname PY@tok@ss\endcsname{\def\PY@tc##1{\textcolor[rgb]{0.10,0.09,0.49}{##1}}}
\expandafter\def\csname PY@tok@sx\endcsname{\def\PY@tc##1{\textcolor[rgb]{0.00,0.50,0.00}{##1}}}
\expandafter\def\csname PY@tok@m\endcsname{\def\PY@tc##1{\textcolor[rgb]{0.40,0.40,0.40}{##1}}}
\expandafter\def\csname PY@tok@gh\endcsname{\let\PY@bf=\textbf\def\PY@tc##1{\textcolor[rgb]{0.00,0.00,0.50}{##1}}}
\expandafter\def\csname PY@tok@gu\endcsname{\let\PY@bf=\textbf\def\PY@tc##1{\textcolor[rgb]{0.50,0.00,0.50}{##1}}}
\expandafter\def\csname PY@tok@gd\endcsname{\def\PY@tc##1{\textcolor[rgb]{0.63,0.00,0.00}{##1}}}
\expandafter\def\csname PY@tok@gi\endcsname{\def\PY@tc##1{\textcolor[rgb]{0.00,0.63,0.00}{##1}}}
\expandafter\def\csname PY@tok@gr\endcsname{\def\PY@tc##1{\textcolor[rgb]{1.00,0.00,0.00}{##1}}}
\expandafter\def\csname PY@tok@ge\endcsname{\let\PY@it=\textit}
\expandafter\def\csname PY@tok@gs\endcsname{\let\PY@bf=\textbf}
\expandafter\def\csname PY@tok@gp\endcsname{\let\PY@bf=\textbf\def\PY@tc##1{\textcolor[rgb]{0.00,0.00,0.50}{##1}}}
\expandafter\def\csname PY@tok@go\endcsname{\def\PY@tc##1{\textcolor[rgb]{0.53,0.53,0.53}{##1}}}
\expandafter\def\csname PY@tok@gt\endcsname{\def\PY@tc##1{\textcolor[rgb]{0.00,0.27,0.87}{##1}}}
\expandafter\def\csname PY@tok@err\endcsname{\def\PY@bc##1{\setlength{\fboxsep}{0pt}\fcolorbox[rgb]{1.00,0.00,0.00}{1,1,1}{\strut ##1}}}
\expandafter\def\csname PY@tok@kc\endcsname{\let\PY@bf=\textbf\def\PY@tc##1{\textcolor[rgb]{0.00,0.50,0.00}{##1}}}
\expandafter\def\csname PY@tok@kd\endcsname{\let\PY@bf=\textbf\def\PY@tc##1{\textcolor[rgb]{0.00,0.50,0.00}{##1}}}
\expandafter\def\csname PY@tok@kn\endcsname{\let\PY@bf=\textbf\def\PY@tc##1{\textcolor[rgb]{0.00,0.50,0.00}{##1}}}
\expandafter\def\csname PY@tok@kr\endcsname{\let\PY@bf=\textbf\def\PY@tc##1{\textcolor[rgb]{0.00,0.50,0.00}{##1}}}
\expandafter\def\csname PY@tok@bp\endcsname{\def\PY@tc##1{\textcolor[rgb]{0.00,0.50,0.00}{##1}}}
\expandafter\def\csname PY@tok@fm\endcsname{\def\PY@tc##1{\textcolor[rgb]{0.00,0.00,1.00}{##1}}}
\expandafter\def\csname PY@tok@vc\endcsname{\def\PY@tc##1{\textcolor[rgb]{0.10,0.09,0.49}{##1}}}
\expandafter\def\csname PY@tok@vg\endcsname{\def\PY@tc##1{\textcolor[rgb]{0.10,0.09,0.49}{##1}}}
\expandafter\def\csname PY@tok@vi\endcsname{\def\PY@tc##1{\textcolor[rgb]{0.10,0.09,0.49}{##1}}}
\expandafter\def\csname PY@tok@vm\endcsname{\def\PY@tc##1{\textcolor[rgb]{0.10,0.09,0.49}{##1}}}
\expandafter\def\csname PY@tok@sa\endcsname{\def\PY@tc##1{\textcolor[rgb]{0.73,0.13,0.13}{##1}}}
\expandafter\def\csname PY@tok@sb\endcsname{\def\PY@tc##1{\textcolor[rgb]{0.73,0.13,0.13}{##1}}}
\expandafter\def\csname PY@tok@sc\endcsname{\def\PY@tc##1{\textcolor[rgb]{0.73,0.13,0.13}{##1}}}
\expandafter\def\csname PY@tok@dl\endcsname{\def\PY@tc##1{\textcolor[rgb]{0.73,0.13,0.13}{##1}}}
\expandafter\def\csname PY@tok@s2\endcsname{\def\PY@tc##1{\textcolor[rgb]{0.73,0.13,0.13}{##1}}}
\expandafter\def\csname PY@tok@sh\endcsname{\def\PY@tc##1{\textcolor[rgb]{0.73,0.13,0.13}{##1}}}
\expandafter\def\csname PY@tok@s1\endcsname{\def\PY@tc##1{\textcolor[rgb]{0.73,0.13,0.13}{##1}}}
\expandafter\def\csname PY@tok@mb\endcsname{\def\PY@tc##1{\textcolor[rgb]{0.40,0.40,0.40}{##1}}}
\expandafter\def\csname PY@tok@mf\endcsname{\def\PY@tc##1{\textcolor[rgb]{0.40,0.40,0.40}{##1}}}
\expandafter\def\csname PY@tok@mh\endcsname{\def\PY@tc##1{\textcolor[rgb]{0.40,0.40,0.40}{##1}}}
\expandafter\def\csname PY@tok@mi\endcsname{\def\PY@tc##1{\textcolor[rgb]{0.40,0.40,0.40}{##1}}}
\expandafter\def\csname PY@tok@il\endcsname{\def\PY@tc##1{\textcolor[rgb]{0.40,0.40,0.40}{##1}}}
\expandafter\def\csname PY@tok@mo\endcsname{\def\PY@tc##1{\textcolor[rgb]{0.40,0.40,0.40}{##1}}}
\expandafter\def\csname PY@tok@ch\endcsname{\let\PY@it=\textit\def\PY@tc##1{\textcolor[rgb]{0.25,0.50,0.50}{##1}}}
\expandafter\def\csname PY@tok@cm\endcsname{\let\PY@it=\textit\def\PY@tc##1{\textcolor[rgb]{0.25,0.50,0.50}{##1}}}
\expandafter\def\csname PY@tok@cpf\endcsname{\let\PY@it=\textit\def\PY@tc##1{\textcolor[rgb]{0.25,0.50,0.50}{##1}}}
\expandafter\def\csname PY@tok@c1\endcsname{\let\PY@it=\textit\def\PY@tc##1{\textcolor[rgb]{0.25,0.50,0.50}{##1}}}
\expandafter\def\csname PY@tok@cs\endcsname{\let\PY@it=\textit\def\PY@tc##1{\textcolor[rgb]{0.25,0.50,0.50}{##1}}}

\def\PYZbs{\char`\\}
\def\PYZus{\char`\_}
\def\PYZob{\char`\{}
\def\PYZcb{\char`\}}
\def\PYZca{\char`\^}
\def\PYZam{\char`\&}
\def\PYZlt{\char`\<}
\def\PYZgt{\char`\>}
\def\PYZsh{\char`\#}
\def\PYZpc{\char`\%}
\def\PYZdl{\char`\$}
\def\PYZhy{\char`\-}
\def\PYZsq{\char`\'}
\def\PYZdq{\char`\"}
\def\PYZti{\char`\~}
% for compatibility with earlier versions
\def\PYZat{@}
\def\PYZlb{[}
\def\PYZrb{]}
\makeatother


    % Exact colors from NB
    \definecolor{incolor}{rgb}{0.0, 0.0, 0.5}
    \definecolor{outcolor}{rgb}{0.545, 0.0, 0.0}



    
    % Prevent overflowing lines due to hard-to-break entities
    \sloppy 
    % Setup hyperref package
    \hypersetup{
      breaklinks=true,  % so long urls are correctly broken across lines
      colorlinks=true,
      urlcolor=urlcolor,
      linkcolor=linkcolor,
      citecolor=citecolor,
      }
    % Slightly bigger margins than the latex defaults
    
    \geometry{verbose,tmargin=1in,bmargin=1in,lmargin=1in,rmargin=1in}
    
    

    \begin{document}
    
    
    \maketitle
    
    

    
    \begin{Verbatim}[commandchars=\\\{\}]
{\color{incolor}In [{\color{incolor}1}]:} \PY{o}{\PYZpc{}}\PY{k}{precision} \PYZpc{}g
        \PY{o}{\PYZpc{}}\PY{k}{matplotlib} inline
        \PY{o}{\PYZpc{}}\PY{k}{config} InlineBackend.figure\PYZus{}format = \PYZsq{}retina\PYZsq{}
\end{Verbatim}

    \begin{Verbatim}[commandchars=\\\{\}]
{\color{incolor}In [{\color{incolor}2}]:} \PY{k+kn}{from} \PY{n+nn}{math} \PY{k}{import} \PY{n}{sqrt}\PY{p}{,} \PY{n}{pi}\PY{p}{,} \PY{n}{sin}\PY{p}{,} \PY{n}{cos}\PY{p}{,} \PY{n}{exp}\PY{p}{,} \PY{n}{inf}\PY{p}{,} \PY{n}{factorial}\PY{p}{,} \PY{n}{tan}
        \PY{k+kn}{from} \PY{n+nn}{cmath} \PY{k}{import} \PY{n}{exp} \PY{k}{as} \PY{n}{cexp}
        \PY{k+kn}{import} \PY{n+nn}{numpy} \PY{k}{as} \PY{n+nn}{np}
        \PY{k+kn}{from} \PY{n+nn}{scipy} \PY{k}{import} \PY{n}{constants} \PY{k}{as} \PY{n}{C}
        \PY{k+kn}{from} \PY{n+nn}{scipy} \PY{k}{import} \PY{n}{integrate}
        \PY{k+kn}{import} \PY{n+nn}{matplotlib}\PY{n+nn}{.}\PY{n+nn}{pyplot} \PY{k}{as} \PY{n+nn}{plt}
        
        \PY{c+c1}{\PYZsh{}from IPython.display import set\PYZus{}matplotlib\PYZus{}formats}
        \PY{c+c1}{\PYZsh{}set\PYZus{}matplotlib\PYZus{}formats(\PYZsq{}png\PYZsq{}, \PYZsq{}pdf\PYZsq{})}
\end{Verbatim}

    \section{CP 5.3 Gaussian error
function}\label{cp-5.3-gaussian-error-function}

The function \(E(x) = \int_0^x \ e^{-t^2} \ dt\) must be solved
numerically. In this problem, I'll employ the trapezoidal rule

\[I(a,b) = h \ \left( \frac{1}{2}(f(a) + f(b)) + \sum_k^{N-1} f(a+ kh) \right)\]

to find the values for \(E(x)\) given \(x \in [0,3]\) with increments of
0.1.

    \begin{Verbatim}[commandchars=\\\{\}]
{\color{incolor}In [{\color{incolor}3}]:} \PY{c+c1}{\PYZsh{}defining integrand function}
        \PY{k}{def} \PY{n+nf}{f}\PY{p}{(}\PY{n}{t}\PY{p}{)}\PY{p}{:}
            \PY{k}{return} \PY{n}{exp}\PY{p}{(}\PY{o}{\PYZhy{}}\PY{p}{(}\PY{n}{t}\PY{o}{*}\PY{o}{*}\PY{l+m+mi}{2}\PY{p}{)}\PY{p}{)}
        
        \PY{c+c1}{\PYZsh{}defines a trapezoidal rule based on the one listed above}
        \PY{k}{def} \PY{n+nf}{E}\PY{p}{(}\PY{n}{x}\PY{p}{,} \PY{n}{N}\PY{p}{)}\PY{p}{:}
            \PY{l+s+sd}{\PYZsq{}\PYZsq{}\PYZsq{}x is the upper limit for the integral}
        \PY{l+s+sd}{        N is the number of slices used for integration\PYZsq{}\PYZsq{}\PYZsq{}}
            
            \PY{n}{h} \PY{o}{=} \PY{n}{x} \PY{o}{/} \PY{n}{N}
            
            \PY{n}{s1} \PY{o}{=} \PY{n}{f}\PY{p}{(}\PY{l+m+mi}{0}\PY{p}{)}
            \PY{n}{s2} \PY{o}{=} \PY{n}{f}\PY{p}{(}\PY{n}{x}\PY{p}{)}
            \PY{n}{s3} \PY{o}{=} \PY{l+m+mi}{0}
            \PY{k}{for} \PY{n}{k} \PY{o+ow}{in} \PY{n+nb}{range}\PY{p}{(}\PY{n}{N}\PY{p}{)}\PY{p}{:}
                \PY{n}{s3} \PY{o}{+}\PY{o}{=} \PY{n}{f}\PY{p}{(}\PY{n}{k}\PY{o}{*}\PY{n}{h}\PY{p}{)}
            
            \PY{k}{return} \PY{p}{(}\PY{p}{(}\PY{l+m+mf}{0.5} \PY{o}{*} \PY{p}{(}\PY{n}{s1} \PY{o}{+} \PY{n}{s2}\PY{p}{)}\PY{p}{)} \PY{o}{+} \PY{n}{s3}\PY{p}{)} \PY{o}{*} \PY{n}{h}
            
        \PY{c+c1}{\PYZsh{}calculating E(x) for 0\PYZhy{}3 by 0.1}
        \PY{n}{sols} \PY{o}{=} \PY{n}{np}\PY{o}{.}\PY{n}{zeros}\PY{p}{(}\PY{p}{[}\PY{l+m+mi}{30}\PY{p}{,}\PY{l+m+mi}{2}\PY{p}{]}\PY{p}{,} \PY{n}{dtype}\PY{o}{=}\PY{n+nb}{float}\PY{p}{)}
        \PY{n}{sols}\PY{p}{[}\PY{p}{:}\PY{p}{,}\PY{l+m+mi}{0}\PY{p}{]} \PY{o}{=} \PY{n}{np}\PY{o}{.}\PY{n}{linspace}\PY{p}{(}\PY{l+m+mi}{0}\PY{p}{,}\PY{l+m+mi}{3}\PY{p}{,}\PY{l+m+mi}{30}\PY{p}{)}
        
        \PY{k}{for} \PY{n}{i} \PY{o+ow}{in} \PY{n+nb}{range}\PY{p}{(}\PY{n}{np}\PY{o}{.}\PY{n}{shape}\PY{p}{(}\PY{n}{sols}\PY{p}{)}\PY{p}{[}\PY{l+m+mi}{0}\PY{p}{]}\PY{p}{)}\PY{p}{:}
            \PY{n}{sols}\PY{p}{[}\PY{n}{i}\PY{p}{,}\PY{l+m+mi}{1}\PY{p}{]} \PY{o}{=} \PY{n}{E}\PY{p}{(}\PY{n}{sols}\PY{p}{[}\PY{n}{i}\PY{p}{,}\PY{l+m+mi}{0}\PY{p}{]}\PY{p}{,} \PY{l+m+mi}{100000}\PY{p}{)}
\end{Verbatim}

    \begin{Verbatim}[commandchars=\\\{\}]
{\color{incolor}In [{\color{incolor}4}]:} \PY{c+c1}{\PYZsh{}created a new cell for plotting, so}
        \PY{c+c1}{\PYZsh{}for loop doesn\PYZsq{}t run every time}
        
        \PY{n}{fig}\PY{p}{,} \PY{n}{ax} \PY{o}{=} \PY{n}{plt}\PY{o}{.}\PY{n}{subplots}\PY{p}{(}\PY{l+m+mi}{1}\PY{p}{,} \PY{l+m+mi}{1}\PY{p}{,} \PY{n}{figsize} \PY{o}{=} \PY{p}{(}\PY{l+m+mi}{5}\PY{p}{,} \PY{l+m+mi}{5}\PY{p}{)}\PY{p}{)}
        \PY{n}{plt}\PY{o}{.}\PY{n}{plot}\PY{p}{(}\PY{n}{sols}\PY{p}{[}\PY{p}{:}\PY{p}{,}\PY{l+m+mi}{0}\PY{p}{]}\PY{p}{,}\PY{n}{sols}\PY{p}{[}\PY{p}{:}\PY{p}{,}\PY{l+m+mi}{1}\PY{p}{]}\PY{p}{,} \PY{l+s+s1}{\PYZsq{}}\PY{l+s+s1}{k\PYZhy{}}\PY{l+s+s1}{\PYZsq{}}\PY{p}{,} \PY{n}{color}\PY{o}{=}\PY{l+s+s1}{\PYZsq{}}\PY{l+s+s1}{blue}\PY{l+s+s1}{\PYZsq{}}\PY{p}{)}
        \PY{n}{plt}\PY{o}{.}\PY{n}{title}\PY{p}{(}\PY{l+s+s2}{\PYZdq{}}\PY{l+s+s2}{Plotting the Gaussian Error Function}\PY{l+s+s2}{\PYZdq{}}\PY{p}{)}
        \PY{n}{plt}\PY{o}{.}\PY{n}{xlabel}\PY{p}{(}\PY{l+s+s2}{\PYZdq{}}\PY{l+s+s2}{x}\PY{l+s+s2}{\PYZdq{}}\PY{p}{)}
        \PY{n}{plt}\PY{o}{.}\PY{n}{ylabel}\PY{p}{(}\PY{l+s+s2}{\PYZdq{}}\PY{l+s+s2}{E(x)}\PY{l+s+s2}{\PYZdq{}}\PY{p}{)}
        \PY{n}{plt}\PY{o}{.}\PY{n}{show}\PY{p}{(}\PY{p}{)}
\end{Verbatim}

    \begin{center}
    \adjustimage{max size={0.9\linewidth}{0.9\paperheight}}{output_4_0.png}
    \end{center}
    { \hspace*{\fill} \\}
    
    \section{CP 5.4 The diffraction limit of a
telescope}\label{cp-5.4-the-diffraction-limit-of-a-telescope}

The circular diffraction pattern for a star in a telescope is given by

\[I(r) = \biggl( {J_1(kr)\over kr} \biggr)^2.\]

This value makes use of Bessel functions \(J_m(x).\) The Bessel
functions are given by

\[J_m(x) = {1\over\pi} \int_0^\pi \cos(m\theta - x\sin\theta) \ d\theta.\]

We can evaluate this integral numerically and thus find the value of any
given Bessel function via Simpson's rule. The simplified form of
Simpson's rule is

\[ I = \frac{1}{3} \ h \ \left( f(a) + f(b) + 4 \sum_{k = \textrm{odd}} f(a+kh) + 2 \sum_{k = \textrm{even}} f(a+kh)  \right).\]

    \begin{Verbatim}[commandchars=\\\{\}]
{\color{incolor}In [{\color{incolor}5}]:} \PY{k}{def} \PY{n+nf}{J}\PY{p}{(}\PY{n}{m}\PY{p}{,}\PY{n}{x}\PY{p}{)}\PY{p}{:}
            \PY{l+s+sd}{\PYZsq{}\PYZsq{}\PYZsq{}Evaluates the Bessel function for order m and x}
        \PY{l+s+sd}{        with N = 1000 slices for Simpsons Rule\PYZsq{}\PYZsq{}\PYZsq{}}
        
            \PY{c+c1}{\PYZsh{}defines integrand}
            \PY{k}{def} \PY{n+nf}{f}\PY{p}{(}\PY{n}{m}\PY{p}{,} \PY{n}{x}\PY{p}{,} \PY{n}{theta}\PY{p}{)}\PY{p}{:}
                \PY{k}{return} \PY{n}{cos}\PY{p}{(}\PY{n}{m} \PY{o}{*} \PY{n}{theta} \PY{o}{\PYZhy{}} \PY{n}{x} \PY{o}{*} \PY{n}{sin}\PY{p}{(}\PY{n}{theta}\PY{p}{)}\PY{p}{)}
            
            \PY{n}{N} \PY{o}{=} \PY{l+m+mi}{1000}
        
            \PY{c+c1}{\PYZsh{}always this b/c bounds and slices are set}
            \PY{n}{h} \PY{o}{=} \PY{n}{pi} \PY{o}{/} \PY{n}{N}
            
            \PY{n}{s1} \PY{o}{=} \PY{n}{f}\PY{p}{(}\PY{n}{m}\PY{p}{,} \PY{n}{x}\PY{p}{,} \PY{l+m+mi}{0}\PY{p}{)}
            \PY{n}{s2} \PY{o}{=} \PY{n}{f}\PY{p}{(}\PY{n}{m}\PY{p}{,} \PY{n}{x}\PY{p}{,} \PY{n}{pi}\PY{p}{)}
            \PY{n}{s3} \PY{o}{=} \PY{l+m+mi}{0}
            \PY{n}{s4} \PY{o}{=} \PY{l+m+mi}{0}
            
            \PY{k}{for} \PY{n}{k} \PY{o+ow}{in} \PY{n+nb}{range}\PY{p}{(}\PY{l+m+mi}{1}\PY{p}{,}\PY{n}{N}\PY{p}{,}\PY{l+m+mi}{2}\PY{p}{)}\PY{p}{:}
                \PY{n}{s3} \PY{o}{+}\PY{o}{=} \PY{n}{f}\PY{p}{(}\PY{n}{m}\PY{p}{,} \PY{n}{x}\PY{p}{,} \PY{n}{k}\PY{o}{*}\PY{n}{h}\PY{p}{)}
            \PY{k}{for} \PY{n}{k} \PY{o+ow}{in} \PY{n+nb}{range}\PY{p}{(}\PY{l+m+mi}{2}\PY{p}{,}\PY{n}{N}\PY{p}{,}\PY{l+m+mi}{2}\PY{p}{)}\PY{p}{:}
                \PY{n}{s4} \PY{o}{+}\PY{o}{=} \PY{n}{f}\PY{p}{(}\PY{n}{m}\PY{p}{,} \PY{n}{x}\PY{p}{,} \PY{n}{k}\PY{o}{*}\PY{n}{h}\PY{p}{)}
            
            \PY{n}{s} \PY{o}{=} \PY{n}{s1} \PY{o}{+} \PY{n}{s2} \PY{o}{+} \PY{l+m+mi}{4}\PY{o}{*}\PY{n}{s3} \PY{o}{+} \PY{l+m+mi}{2}\PY{o}{*}\PY{n}{s4}
            \PY{n}{I} \PY{o}{=} \PY{p}{(}\PY{l+m+mi}{1}\PY{o}{/}\PY{l+m+mi}{3}\PY{p}{)} \PY{o}{*} \PY{n}{h} \PY{o}{*} \PY{n}{s}
            
            \PY{k}{return} \PY{n}{I} \PY{o}{/} \PY{n}{pi}
        
        
        \PY{n}{small} \PY{o}{=} \PY{l+m+mf}{0.000001}
        \PY{n}{J}\PY{p}{(}\PY{l+m+mi}{1}\PY{p}{,} \PY{n}{small}\PY{p}{)} \PY{o}{/} \PY{n}{small}
\end{Verbatim}

\begin{Verbatim}[commandchars=\\\{\}]
{\color{outcolor}Out[{\color{outcolor}5}]:} 0.5
\end{Verbatim}
            
    \begin{Verbatim}[commandchars=\\\{\}]
{\color{incolor}In [{\color{incolor}6}]:} \PY{n}{x} \PY{o}{=} \PY{n}{np}\PY{o}{.}\PY{n}{linspace}\PY{p}{(}\PY{l+m+mi}{0}\PY{p}{,} \PY{l+m+mi}{20}\PY{p}{,} \PY{l+m+mi}{200}\PY{p}{)}
        \PY{c+c1}{\PYZsh{}initializes plot}
        \PY{n}{fig}\PY{p}{,} \PY{n}{ax} \PY{o}{=} \PY{n}{plt}\PY{o}{.}\PY{n}{subplots}\PY{p}{(}\PY{l+m+mi}{1}\PY{p}{,} \PY{l+m+mi}{1}\PY{p}{,} \PY{n}{figsize} \PY{o}{=} \PY{p}{(}\PY{l+m+mi}{10}\PY{p}{,} \PY{l+m+mi}{5}\PY{p}{)}\PY{p}{)}
        
        \PY{n}{j0} \PY{o}{=} \PY{n}{np}\PY{o}{.}\PY{n}{empty}\PY{p}{(}\PY{l+m+mi}{200}\PY{p}{)}
        \PY{n}{j1} \PY{o}{=} \PY{n}{np}\PY{o}{.}\PY{n}{empty}\PY{p}{(}\PY{l+m+mi}{200}\PY{p}{)}
        \PY{n}{j2} \PY{o}{=} \PY{n}{np}\PY{o}{.}\PY{n}{empty}\PY{p}{(}\PY{l+m+mi}{200}\PY{p}{)}
        
        \PY{k}{for} \PY{n}{i} \PY{o+ow}{in} \PY{n+nb}{range}\PY{p}{(}\PY{l+m+mi}{200}\PY{p}{)}\PY{p}{:}
            \PY{n}{j0}\PY{p}{[}\PY{n}{i}\PY{p}{]} \PY{o}{=} \PY{n}{J}\PY{p}{(}\PY{l+m+mi}{0}\PY{p}{,}\PY{n}{x}\PY{p}{[}\PY{n}{i}\PY{p}{]}\PY{p}{)}
            \PY{n}{j1}\PY{p}{[}\PY{n}{i}\PY{p}{]} \PY{o}{=} \PY{n}{J}\PY{p}{(}\PY{l+m+mi}{1}\PY{p}{,}\PY{n}{x}\PY{p}{[}\PY{n}{i}\PY{p}{]}\PY{p}{)}
            \PY{n}{j2}\PY{p}{[}\PY{n}{i}\PY{p}{]} \PY{o}{=} \PY{n}{J}\PY{p}{(}\PY{l+m+mi}{2}\PY{p}{,}\PY{n}{x}\PY{p}{[}\PY{n}{i}\PY{p}{]}\PY{p}{)}
            
        \PY{n}{plt}\PY{o}{.}\PY{n}{plot}\PY{p}{(}\PY{n}{x}\PY{p}{,} \PY{n}{j0}\PY{p}{,} \PY{l+s+s1}{\PYZsq{}}\PY{l+s+s1}{c\PYZhy{}}\PY{l+s+s1}{\PYZsq{}}\PY{p}{)}
        \PY{n}{plt}\PY{o}{.}\PY{n}{plot}\PY{p}{(}\PY{n}{x}\PY{p}{,} \PY{n}{j1}\PY{p}{,} \PY{l+s+s1}{\PYZsq{}}\PY{l+s+s1}{k\PYZhy{}}\PY{l+s+s1}{\PYZsq{}}\PY{p}{)}
        \PY{n}{plt}\PY{o}{.}\PY{n}{plot}\PY{p}{(}\PY{n}{x}\PY{p}{,} \PY{n}{j2}\PY{p}{,} \PY{l+s+s1}{\PYZsq{}}\PY{l+s+s1}{m\PYZhy{}}\PY{l+s+s1}{\PYZsq{}}\PY{p}{)}
        
        \PY{n}{plt}\PY{o}{.}\PY{n}{title}\PY{p}{(}\PY{l+s+s2}{\PYZdq{}}\PY{l+s+s2}{Plotting Bessel Functions}\PY{l+s+s2}{\PYZdq{}}\PY{p}{)}
        \PY{n}{plt}\PY{o}{.}\PY{n}{xlabel}\PY{p}{(}\PY{l+s+s2}{\PYZdq{}}\PY{l+s+s2}{x}\PY{l+s+s2}{\PYZdq{}}\PY{p}{)}
        \PY{n}{plt}\PY{o}{.}\PY{n}{ylabel}\PY{p}{(}\PY{l+s+s2}{\PYZdq{}}\PY{l+s+s2}{J(m,x)}\PY{l+s+s2}{\PYZdq{}}\PY{p}{)}
        \PY{n}{plt}\PY{o}{.}\PY{n}{show}\PY{p}{(}\PY{p}{)}
\end{Verbatim}

    \begin{center}
    \adjustimage{max size={0.9\linewidth}{0.9\paperheight}}{output_7_0.png}
    \end{center}
    { \hspace*{\fill} \\}
    
    \begin{Verbatim}[commandchars=\\\{\}]
{\color{incolor}In [{\color{incolor}7}]:} \PY{o}{\PYZpc{}\PYZpc{}}\PY{k}{time}
        def diffraction():
            \PYZdq{}\PYZdq{}\PYZdq{}plots the diffraction pattern for a point
                source of light through small circular
                aperture\PYZdq{}\PYZdq{}\PYZdq{}
            
            l = 500e\PYZhy{}9 \PYZsh{}wavelength (m)
        
            m = 1 \PYZsh{}order of the bessel function used for intensity
            k = 2 * pi / l
            
            side = 2e\PYZhy{}6 \PYZsh{}side length of the square in m
            points = 500 \PYZsh{}number of grid points along each side
            spacing = side / points \PYZsh{}spacing of points in nm
            
            I = np.empty([points,points],float)
            \PYZsh{}calculate the values in the array
            for i in range(points):
                y = spacing * i
                for j in range(points):
                    x = spacing * j
                    r = sqrt((x\PYZhy{}side/2)**2 + (y\PYZhy{}side/2)**2)
                    if k* r != 0:
                        I[i,j] = (J(m, k*r) / (k*r))**2
                    
            fig, ax = plt.subplots(1, 1, figsize = (10, 10))
            
            \PYZsh{}increases readability of plot
            ax.set\PYZus{}title(\PYZdq{}Diffraction Pattern of a Point Source\PYZdq{})
            ax.set\PYZus{}xlabel(\PYZdq{}x (\PYZdl{}\PYZbs{}mu m\PYZdl{})\PYZdq{})
            ax.set\PYZus{}ylabel(\PYZdq{}y (\PYZdl{}\PYZbs{}mu m\PYZdl{})\PYZdq{})
            ax.set\PYZus{}xticks([\PYZhy{}1, \PYZhy{}0.5, 0, 0.5, 1])
            ax.set\PYZus{}yticks([\PYZhy{}1, \PYZhy{}0.5, 0, 0.5, 1])
            
            ax.imshow(I,origin=\PYZdq{}lower\PYZdq{},extent=\PYZbs{}
                      [\PYZhy{}1,1,\PYZhy{}1,1],\PYZbs{}
                      cmap=\PYZdq{}hot\PYZdq{}, vmax=0.01)
        
        diffraction()
\end{Verbatim}

    \begin{Verbatim}[commandchars=\\\{\}]
CPU times: user 1min 29s, sys: 636 ms, total: 1min 30s
Wall time: 1min 32s

    \end{Verbatim}

    \begin{center}
    \adjustimage{max size={0.9\linewidth}{0.9\paperheight}}{output_8_1.png}
    \end{center}
    { \hspace*{\fill} \\}
    
    \section{CP 5.7 Romberg integration}\label{cp-5.7-romberg-integration}

This problem will evaluate the integral
\[I = \int_0^1 \sin^2 \sqrt{100x} \ dx\]

in two different ways. First, through adaptive trapezoidal integration
and second, through Romberg integration. The basic trapezoidal
integration method is given by

\[I(a,b) = h \ \left( \frac{1}{2}(f(a) + f(b)) + \sum_k^{N-1} f(a+ kh) \right).\]

The adaptive method is given by

\[I_i = \tfrac12 I_{i-1} + h_i \sum_{k \text{ odd}} f(a+kh_i).\]

Thus, because the adaptive method uses the result of the previou
iteration, it takes marginally more computing power to do this. The
error term for this can be written as

\[\epsilon_i = \tfrac13 (I_i - I_{i-1}).\]

    \begin{Verbatim}[commandchars=\\\{\}]
{\color{incolor}In [{\color{incolor}43}]:} \PY{n}{epsilon} \PY{o}{=} \PY{l+m+mf}{1e\PYZhy{}6}
         
         \PY{k}{def} \PY{n+nf}{f}\PY{p}{(}\PY{n}{x}\PY{p}{)}\PY{p}{:}
             \PY{k}{return} \PY{n}{sin}\PY{p}{(}\PY{n}{sqrt}\PY{p}{(}\PY{l+m+mi}{100}\PY{o}{*}\PY{n}{x}\PY{p}{)}\PY{p}{)}\PY{o}{*}\PY{o}{*}\PY{l+m+mi}{2}
\end{Verbatim}

    \begin{Verbatim}[commandchars=\\\{\}]
{\color{incolor}In [{\color{incolor}44}]:} \PY{o}{\PYZpc{}\PYZpc{}}\PY{k}{time}
         
         def adaptive\PYZus{}trap():
             \PYZsq{}\PYZsq{}\PYZsq{}performs adaptive trapezoidal integration,
                 starting with N slices and stopping after
                 set error has been reached\PYZsq{}\PYZsq{}\PYZsq{}
             
             a = 0
             b = 1
             N = 1
             error = 1 \PYZsh{}initialization value to start while loop
             
             h = (b\PYZhy{}a) / N
             I1 = h * 0.5 * (f(a) + f(b))
                 
             while error \PYZgt{} epsilon: \PYZsh{}condition to keep running integration
                 h = (b\PYZhy{}a) / N
                 I2 = 0.5 * I1
                             
                 for k in range(1,N,2):
                     I2 += h * f(a + k*h)
                     
                 error = abs((I2 \PYZhy{} I1)/3)
                     
                 print(\PYZdq{}For N = \PYZob{}:4\PYZcb{}, I = \PYZob{}:4.6f\PYZcb{} and the error is \PYZob{}:4.3e\PYZcb{} \PYZbs{}n\PYZdq{}\PYZbs{}
                           .format(N, I2, error))
                 
                 I1 = I2
                 N *= 2
             
             return I2
                                 
         adaptive\PYZus{}trap()
\end{Verbatim}

    \begin{Verbatim}[commandchars=\\\{\}]
For N =    1, I = 0.073990 and the error is 2.466e-02 

For N =    2, I = 0.288237 and the error is 7.142e-02 

For N =    4, I = 0.493785 and the error is 6.852e-02 

For N =    8, I = 0.393749 and the error is 3.335e-02 

For N =   16, I = 0.425479 and the error is 1.058e-02 

For N =   32, I = 0.446102 and the error is 6.874e-03 

For N =   64, I = 0.452757 and the error is 2.218e-03 

For N =  128, I = 0.454770 and the error is 6.712e-04 

For N =  256, I = 0.455422 and the error is 2.173e-04 

For N =  512, I = 0.455658 and the error is 7.848e-05 

For N = 1024, I = 0.455753 and the error is 3.167e-05 

For N = 2048, I = 0.455795 and the error is 1.394e-05 

For N = 4096, I = 0.455814 and the error is 6.495e-06 

For N = 8192, I = 0.455823 and the error is 3.129e-06 

For N = 16384, I = 0.455828 and the error is 1.535e-06 

For N = 32768, I = 0.455830 and the error is 7.601e-07 

CPU times: user 18.3 ms, sys: 3.11 ms, total: 21.4 ms
Wall time: 19.6 ms

    \end{Verbatim}

    So, on the 16th iteration of the adaptive trapezoidal integration, we
reached the desired error set out by the problem.

However, Romberg integration will likely give us this result faster. The
idea behind Romberg integration is to cancel out higher and higher order
error terms by using the values we've already calculated from
trapezoidal integration.

The value of the integral is given by

\[I = R_{i, m+1} + O(h_i^{2m+2})\]

with errors given by

\[c_mh_i^{2m} = \frac{1}{4^m -1} (R_{i,m} - R_{i-1, m}) + O(h_i^{2m+2}).\]

The actual terms in the 'series' can be calculated from the preceding
terms, which is what makes this such an efficient method of integration.

\[R_{i,m+1} = R_{i,m} + \frac{1}{4^m -1} (R_{i,m} - R_{i-1, m})\]

    \begin{Verbatim}[commandchars=\\\{\}]
{\color{incolor}In [{\color{incolor}45}]:} \PY{k}{def} \PY{n+nf}{simple\PYZus{}trap}\PY{p}{(}\PY{n}{N}\PY{p}{)}\PY{p}{:}
                 \PY{l+s+sd}{\PYZsq{}\PYZsq{}\PYZsq{}defines a simple trapezoidal integration}
         \PY{l+s+sd}{            for use in the Romberg integration\PYZsq{}\PYZsq{}\PYZsq{}}
                 \PY{n}{a} \PY{o}{=} \PY{l+m+mi}{0}
                 \PY{n}{b} \PY{o}{=} \PY{l+m+mi}{1}
                 
                 \PY{n}{h} \PY{o}{=} \PY{p}{(}\PY{n}{b} \PY{o}{\PYZhy{}} \PY{n}{a}\PY{p}{)} \PY{o}{/} \PY{n}{N}
                 
                 \PY{n}{I} \PY{o}{=} \PY{n}{f}\PY{p}{(}\PY{n}{a}\PY{p}{)} \PY{o}{*} \PY{l+m+mf}{0.5}
                 \PY{n}{I} \PY{o}{+}\PY{o}{=} \PY{n}{f}\PY{p}{(}\PY{n}{b}\PY{p}{)} \PY{o}{*} \PY{l+m+mf}{0.5}
                 \PY{k}{for} \PY{n}{k} \PY{o+ow}{in} \PY{n+nb}{range}\PY{p}{(}\PY{n}{N}\PY{p}{)}\PY{p}{:}
                     \PY{n}{I} \PY{o}{+}\PY{o}{=} \PY{n}{f}\PY{p}{(}\PY{n}{a} \PY{o}{+} \PY{n}{k}\PY{o}{*}\PY{n}{h}\PY{p}{)}
                     
                 \PY{n}{I} \PY{o}{*}\PY{o}{=} \PY{n}{h}
                 \PY{k}{return} \PY{n}{I}
\end{Verbatim}

    \begin{Verbatim}[commandchars=\\\{\}]
{\color{incolor}In [{\color{incolor}66}]:} \PY{o}{\PYZpc{}\PYZpc{}}\PY{k}{time}
         \PYZsh{}romberg integration
         
         def romberg(m):
             \PYZsq{}\PYZsq{}\PYZsq{}Romberg integrates a function up to m rows\PYZsq{}\PYZsq{}\PYZsq{}
             
             a = 0
             b = 1
                 
             R = np.zeros((m, m))
             for i in range(0, m):
                 R[i, 0] = simple\PYZus{}trap(2**i)
                 \PYZsh{}R[i,0] = adaptive\PYZus{}trap()
         
                 for m in range(0, i):
                     R[i, m+1] = R[i,m] + ((R[i,m] \PYZhy{} R[i\PYZhy{}1, m])/(4**(m+1) \PYZhy{} 1))
         
                     error = (R[i,m] \PYZhy{} R[i \PYZhy{} 1, m]) / (4**m \PYZhy{} 1)
         
                     if error \PYZlt{} epsilon:
                         break
         
                 print(R[i, 0:i+1])
                 
                         
         
         romberg(7)
\end{Verbatim}

    \begin{Verbatim}[commandchars=\\\{\}]
[ 0.14797948]
[ 0.32523191  0.38431605]
[ 0.51228285  0.57463317  0.58732097]
[ 0.40299745  0.36656898  0.          0.        ]
[ 0.43010337  0.43913868  0.44397666  0.4510239   0.45279263]
[ 0.44841467  0.45451843  0.45554375  0.45572735  0.4557458   0.45574868]
[ 0.45391293  0.45574569  0.4558275   0.45583201  0.45583242  0.4558325   0.        ]
CPU times: user 6.89 ms, sys: 2.49 ms, total: 9.38 ms
Wall time: 7.89 ms

    \end{Verbatim}

    \begin{Verbatim}[commandchars=\\\{\}]
/Users/Varun/anaconda/lib/python3.6/site-packages/ipykernel\_launcher.py:17: RuntimeWarning: divide by zero encountered in double\_scalars

    \end{Verbatim}

    As can be seen from the results of Romberg integration, a precise and
accurate numerical value is reached for the integral in less time (about
a quarter) than with the adaptive trapezoidal method. We see that the
our desired error level is reached in the 7th row

    \begin{Verbatim}[commandchars=\\\{\}]
{\color{incolor}In [{\color{incolor}12}]:} \PY{o}{\PYZpc{}\PYZpc{}}\PY{k}{time}
         
         \PYZsh{}testing romberg integration
         integrate.romberg(f, 0, 1, show=1)
\end{Verbatim}

    \begin{Verbatim}[commandchars=\\\{\}]
Romberg integration of <function vectorize1.<locals>.vfunc at 0x1176069d8> from [0, 1]

 Steps  StepSize   Results
     1  1.000000  0.147979 
     2  0.500000  0.325232  0.384316 
     4  0.250000  0.512283  0.574633  0.587321 
     8  0.125000  0.402997  0.366569  0.352698  0.348974 
    16  0.062500  0.430103  0.439139  0.443977  0.445426  0.445804 
    32  0.031250  0.448415  0.454518  0.455544  0.455727  0.455768  0.455777 
    64  0.015625  0.453913  0.455746  0.455828  0.455832  0.455832  0.455832  0.455832 
   128  0.007812  0.455349  0.455827  0.455832  0.455833  0.455833  0.455833  0.455833  0.455833 
   256  0.003906  0.455711  0.455832  0.455833  0.455833  0.455833  0.455833  0.455833  0.455833  0.455833 

The final result is 0.455832532309 after 257 function evaluations.
CPU times: user 8.9 ms, sys: 4.52 ms, total: 13.4 ms
Wall time: 11.9 ms

    \end{Verbatim}

\begin{Verbatim}[commandchars=\\\{\}]
{\color{outcolor}Out[{\color{outcolor}12}]:} 0.455833
\end{Verbatim}
            
    \subsubsection{Gaussian Quadrature
Problems}\label{gaussian-quadrature-problems}

For the following questions, the method of Gaussian Quadrature will be
used to estimate the value of integrals. Gaussian Quadrature attempts to
pick points with varied spacing along a curve in the most efficient way,
so as to best calculate the area under the curve. In practice this gives
a sum that is very close to the actual value of an integral.

\[\sum_i w_i\ f(x_i) \approx \int_a^b f(x)\ dx\]

    \begin{Verbatim}[commandchars=\\\{\}]
{\color{incolor}In [{\color{incolor}13}]:} \PY{l+s+sd}{\PYZdq{}\PYZdq{}\PYZdq{}Gaussian Quadrature Functions as defined by Mark Newman}
         
         \PY{l+s+sd}{    used for the following problems in the homework.}
         \PY{l+s+sd}{    }
         \PY{l+s+sd}{    I edited the numpy functions, adding the prefix \PYZsq{}np.\PYZsq{}\PYZsq{}\PYZdq{}\PYZdq{}\PYZdq{}}
         
         \PY{c+c1}{\PYZsh{} x,w = gaussxw(N) returns integration points x and integration}
         \PY{c+c1}{\PYZsh{}           weights w such that sum\PYZus{}i w[i]*f(x[i]) is the Nth\PYZhy{}order}
         \PY{c+c1}{\PYZsh{}           Gaussian approximation to the integral int\PYZus{}\PYZob{}\PYZhy{}1\PYZcb{}\PYZca{}1 f(x) dx}
         \PY{c+c1}{\PYZsh{}}
         \PY{c+c1}{\PYZsh{} x,w = gaussxwab(N,a,b) returns integration points and weights}
         \PY{c+c1}{\PYZsh{}           mapped to the interval [a,b], so that sum\PYZus{}i w[i]*f(x[i])}
         \PY{c+c1}{\PYZsh{}           is the Nth\PYZhy{}order Gaussian approximation to the integral}
         \PY{c+c1}{\PYZsh{}           int\PYZus{}a\PYZca{}b f(x) dx}
         
         \PY{k}{def} \PY{n+nf}{gaussxw}\PY{p}{(}\PY{n}{N}\PY{p}{)}\PY{p}{:}
         
             \PY{c+c1}{\PYZsh{} Initial approximation to roots of the Legendre polynomial}
             \PY{n}{a} \PY{o}{=} \PY{n}{np}\PY{o}{.}\PY{n}{linspace}\PY{p}{(}\PY{l+m+mi}{3}\PY{p}{,}\PY{l+m+mi}{4}\PY{o}{*}\PY{n}{N}\PY{o}{\PYZhy{}}\PY{l+m+mi}{1}\PY{p}{,}\PY{n}{N}\PY{p}{)}\PY{o}{/}\PY{p}{(}\PY{l+m+mi}{4}\PY{o}{*}\PY{n}{N}\PY{o}{+}\PY{l+m+mi}{2}\PY{p}{)}
             \PY{n}{x} \PY{o}{=} \PY{n}{np}\PY{o}{.}\PY{n}{cos}\PY{p}{(}\PY{n}{np}\PY{o}{.}\PY{n}{pi}\PY{o}{*}\PY{n}{a}\PY{o}{+}\PY{l+m+mi}{1}\PY{o}{/}\PY{p}{(}\PY{l+m+mi}{8}\PY{o}{*}\PY{n}{N}\PY{o}{*}\PY{n}{N}\PY{o}{*}\PY{n}{np}\PY{o}{.}\PY{n}{tan}\PY{p}{(}\PY{n}{a}\PY{p}{)}\PY{p}{)}\PY{p}{)}
         
             \PY{c+c1}{\PYZsh{} Find roots using Newton\PYZsq{}s method}
             \PY{n}{epsilon} \PY{o}{=} \PY{l+m+mf}{1e\PYZhy{}15}
             \PY{n}{delta} \PY{o}{=} \PY{l+m+mf}{1.0}
             \PY{k}{while} \PY{n}{delta}\PY{o}{\PYZgt{}}\PY{n}{epsilon}\PY{p}{:}
                 \PY{n}{p0} \PY{o}{=} \PY{n}{np}\PY{o}{.}\PY{n}{ones}\PY{p}{(}\PY{n}{N}\PY{p}{,}\PY{n+nb}{float}\PY{p}{)}
                 \PY{n}{p1} \PY{o}{=} \PY{n}{np}\PY{o}{.}\PY{n}{copy}\PY{p}{(}\PY{n}{x}\PY{p}{)}
                 \PY{k}{for} \PY{n}{k} \PY{o+ow}{in} \PY{n+nb}{range}\PY{p}{(}\PY{l+m+mi}{1}\PY{p}{,}\PY{n}{N}\PY{p}{)}\PY{p}{:}
                     \PY{n}{p0}\PY{p}{,}\PY{n}{p1} \PY{o}{=} \PY{n}{p1}\PY{p}{,}\PY{p}{(}\PY{p}{(}\PY{l+m+mi}{2}\PY{o}{*}\PY{n}{k}\PY{o}{+}\PY{l+m+mi}{1}\PY{p}{)}\PY{o}{*}\PY{n}{x}\PY{o}{*}\PY{n}{p1}\PY{o}{\PYZhy{}}\PY{n}{k}\PY{o}{*}\PY{n}{p0}\PY{p}{)}\PY{o}{/}\PY{p}{(}\PY{n}{k}\PY{o}{+}\PY{l+m+mi}{1}\PY{p}{)}
                 \PY{n}{dp} \PY{o}{=} \PY{p}{(}\PY{n}{N}\PY{o}{+}\PY{l+m+mi}{1}\PY{p}{)}\PY{o}{*}\PY{p}{(}\PY{n}{p0}\PY{o}{\PYZhy{}}\PY{n}{x}\PY{o}{*}\PY{n}{p1}\PY{p}{)}\PY{o}{/}\PY{p}{(}\PY{l+m+mi}{1}\PY{o}{\PYZhy{}}\PY{n}{x}\PY{o}{*}\PY{n}{x}\PY{p}{)}
                 \PY{n}{dx} \PY{o}{=} \PY{n}{p1}\PY{o}{/}\PY{n}{dp}
                 \PY{n}{x} \PY{o}{\PYZhy{}}\PY{o}{=} \PY{n}{dx}
                 \PY{n}{delta} \PY{o}{=} \PY{n+nb}{max}\PY{p}{(}\PY{n+nb}{abs}\PY{p}{(}\PY{n}{dx}\PY{p}{)}\PY{p}{)}
         
             \PY{c+c1}{\PYZsh{} Calculate the weights}
             \PY{n}{w} \PY{o}{=} \PY{l+m+mi}{2}\PY{o}{*}\PY{p}{(}\PY{n}{N}\PY{o}{+}\PY{l+m+mi}{1}\PY{p}{)}\PY{o}{*}\PY{p}{(}\PY{n}{N}\PY{o}{+}\PY{l+m+mi}{1}\PY{p}{)}\PY{o}{/}\PY{p}{(}\PY{n}{N}\PY{o}{*}\PY{n}{N}\PY{o}{*}\PY{p}{(}\PY{l+m+mi}{1}\PY{o}{\PYZhy{}}\PY{n}{x}\PY{o}{*}\PY{n}{x}\PY{p}{)}\PY{o}{*}\PY{n}{dp}\PY{o}{*}\PY{n}{dp}\PY{p}{)}
         
             \PY{k}{return} \PY{n}{x}\PY{p}{,}\PY{n}{w}
         
         \PY{k}{def} \PY{n+nf}{gaussxwab}\PY{p}{(}\PY{n}{N}\PY{p}{,}\PY{n}{a}\PY{p}{,}\PY{n}{b}\PY{p}{)}\PY{p}{:}
             \PY{n}{x}\PY{p}{,}\PY{n}{w} \PY{o}{=} \PY{n}{gaussxw}\PY{p}{(}\PY{n}{N}\PY{p}{)}
             \PY{k}{return} \PY{l+m+mf}{0.5}\PY{o}{*}\PY{p}{(}\PY{n}{b}\PY{o}{\PYZhy{}}\PY{n}{a}\PY{p}{)}\PY{o}{*}\PY{n}{x}\PY{o}{+}\PY{l+m+mf}{0.5}\PY{o}{*}\PY{p}{(}\PY{n}{b}\PY{o}{+}\PY{n}{a}\PY{p}{)}\PY{p}{,}\PY{l+m+mf}{0.5}\PY{o}{*}\PY{p}{(}\PY{n}{b}\PY{o}{\PYZhy{}}\PY{n}{a}\PY{p}{)}\PY{o}{*}\PY{n}{w}
\end{Verbatim}

    \section{CP 5.9 Heat apacity using Gaussian
quadrature}\label{cp-5.9-heat-apacity-using-gaussian-quadrature}

Debye gives the relationship between heat capacity of solids and
temperature as

\[C_V (T) = 9V\rho k_B \biggl( {T\over\theta_D} \biggr)^3 \int_0^{\theta_D/T}{x^4 \ e^x\over(\ e^x-1)^2}\ d x.\]

Thus we can numerically evaluate this integral in order to find the heat
capacity of a given solid at a given temperature. I'll look at the case
specific to Aluminum.

    \begin{Verbatim}[commandchars=\\\{\}]
{\color{incolor}In [{\color{incolor}14}]:} \PY{c+c1}{\PYZsh{}definining constants for this problem in SI units}
         \PY{n}{V} \PY{o}{=} \PY{l+m+mf}{0.001} \PY{c+c1}{\PYZsh{}volume (m\PYZca{}3)}
         \PY{n}{rho} \PY{o}{=} \PY{l+m+mf}{6.022e28} \PY{c+c1}{\PYZsh{}number density (m\PYZca{}\PYZhy{}3)}
         \PY{n}{thetaD} \PY{o}{=} \PY{l+m+mf}{428.0} \PY{c+c1}{\PYZsh{}Debye temp (K)}
         \PY{n}{kB} \PY{o}{=} \PY{n}{C}\PY{o}{.}\PY{n}{k} \PY{c+c1}{\PYZsh{}Boltzmann\PYZsq{}s constant (J/K)}
         
         \PY{c+c1}{\PYZsh{}defining function}
         \PY{k}{def} \PY{n+nf}{g}\PY{p}{(}\PY{n}{x}\PY{p}{)}\PY{p}{:}
             \PY{n}{num} \PY{o}{=} \PY{p}{(}\PY{n}{x}\PY{o}{*}\PY{o}{*}\PY{l+m+mi}{4}\PY{p}{)}\PY{o}{*}\PY{n}{exp}\PY{p}{(}\PY{n}{x}\PY{p}{)}
             \PY{n}{den} \PY{o}{=} \PY{p}{(}\PY{n}{exp}\PY{p}{(}\PY{n}{x}\PY{p}{)} \PY{o}{\PYZhy{}} \PY{l+m+mi}{1}\PY{p}{)}\PY{o}{*}\PY{o}{*}\PY{l+m+mi}{2}
             \PY{k}{return} \PY{n}{num} \PY{o}{/} \PY{n}{den}
\end{Verbatim}

    \begin{Verbatim}[commandchars=\\\{\}]
{\color{incolor}In [{\color{incolor}15}]:} \PY{k}{def} \PY{n+nf}{cv}\PY{p}{(}\PY{n}{T}\PY{p}{)}\PY{p}{:}
             \PY{l+s+sd}{\PYZdq{}\PYZdq{}\PYZdq{}This will calculate the heat capacity of a solid}
         \PY{l+s+sd}{        for a given temperature (K)\PYZdq{}\PYZdq{}\PYZdq{}}
             
             \PY{c+c1}{\PYZsh{}part of function outside of integral}
             \PY{n}{out} \PY{o}{=} \PY{l+m+mi}{9} \PY{o}{*} \PY{n}{V} \PY{o}{*} \PY{n}{rho} \PY{o}{*} \PY{n}{kB} \PY{o}{*} \PY{p}{(}\PY{p}{(}\PY{n}{T}\PY{o}{/}\PY{n}{thetaD}\PY{p}{)}\PY{o}{*}\PY{o}{*}\PY{l+m+mi}{3}\PY{p}{)}
             
             \PY{n}{x}\PY{p}{,}\PY{n}{w} \PY{o}{=} \PY{n}{gaussxwab}\PY{p}{(}\PY{l+m+mi}{50}\PY{p}{,} \PY{l+m+mi}{0}\PY{p}{,} \PY{n}{thetaD}\PY{o}{/}\PY{n}{T}\PY{p}{)}
             
             \PY{n}{I} \PY{o}{=} \PY{l+m+mi}{0}
             \PY{k}{for} \PY{n}{i} \PY{o+ow}{in} \PY{n+nb}{range}\PY{p}{(}\PY{n+nb}{len}\PY{p}{(}\PY{n}{x}\PY{p}{)}\PY{p}{)}\PY{p}{:}
                 \PY{n}{I} \PY{o}{+}\PY{o}{=} \PY{n}{w}\PY{p}{[}\PY{n}{i}\PY{p}{]}\PY{o}{*}\PY{n}{g}\PY{p}{(}\PY{n}{x}\PY{p}{[}\PY{n}{i}\PY{p}{]}\PY{p}{)}
                 
             \PY{k}{return} \PY{n}{out}\PY{o}{*}\PY{n}{I}
\end{Verbatim}

    \begin{Verbatim}[commandchars=\\\{\}]
{\color{incolor}In [{\color{incolor}16}]:} \PY{c+c1}{\PYZsh{}plots heat capacity from 5 to 500 K}
         
         \PY{n}{T} \PY{o}{=} \PY{n}{np}\PY{o}{.}\PY{n}{linspace}\PY{p}{(}\PY{l+m+mi}{5}\PY{p}{,} \PY{l+m+mi}{500}\PY{p}{,} \PY{l+m+mi}{5000}\PY{p}{)}
         \PY{n}{CV} \PY{o}{=} \PY{n}{np}\PY{o}{.}\PY{n}{zeros}\PY{p}{(}\PY{l+m+mi}{5000}\PY{p}{)}
         \PY{k}{for} \PY{n}{i} \PY{o+ow}{in} \PY{n+nb}{range}\PY{p}{(}\PY{l+m+mi}{5000}\PY{p}{)}\PY{p}{:}
             \PY{n}{CV}\PY{p}{[}\PY{n}{i}\PY{p}{]} \PY{o}{=} \PY{n}{cv}\PY{p}{(}\PY{n}{T}\PY{p}{[}\PY{n}{i}\PY{p}{]}\PY{p}{)}
         
         \PY{n}{fig}\PY{p}{,} \PY{n}{ax} \PY{o}{=} \PY{n}{plt}\PY{o}{.}\PY{n}{subplots}\PY{p}{(}\PY{l+m+mi}{1}\PY{p}{,} \PY{l+m+mi}{1}\PY{p}{,} \PY{n}{figsize} \PY{o}{=} \PY{p}{(}\PY{l+m+mi}{10}\PY{p}{,} \PY{l+m+mi}{7}\PY{p}{)}\PY{p}{)}
         
         \PY{n}{plt}\PY{o}{.}\PY{n}{plot}\PY{p}{(}\PY{n}{T}\PY{p}{,} \PY{n}{CV}\PY{p}{,} \PY{l+s+s1}{\PYZsq{}}\PY{l+s+s1}{k\PYZhy{}}\PY{l+s+s1}{\PYZsq{}}\PY{p}{,} \PY{n}{color}\PY{o}{=}\PY{l+s+s1}{\PYZsq{}}\PY{l+s+s1}{blue}\PY{l+s+s1}{\PYZsq{}}\PY{p}{)}
         
         \PY{n}{plt}\PY{o}{.}\PY{n}{title}\PY{p}{(}\PY{l+s+s2}{\PYZdq{}}\PY{l+s+s2}{Plotting Heat Capacity vs Temperature}\PY{l+s+s2}{\PYZdq{}}\PY{p}{)}
         \PY{n}{plt}\PY{o}{.}\PY{n}{xlabel}\PY{p}{(}\PY{l+s+s2}{\PYZdq{}}\PY{l+s+s2}{Temperature (K)}\PY{l+s+s2}{\PYZdq{}}\PY{p}{)}
         \PY{n}{plt}\PY{o}{.}\PY{n}{ylabel}\PY{p}{(}\PY{l+s+s2}{\PYZdq{}}\PY{l+s+s2}{Heat Capacity (J/K)}\PY{l+s+s2}{\PYZdq{}}\PY{p}{)}
         \PY{n}{plt}\PY{o}{.}\PY{n}{show}\PY{p}{(}\PY{p}{)}
\end{Verbatim}

    \begin{center}
    \adjustimage{max size={0.9\linewidth}{0.9\paperheight}}{output_22_0.png}
    \end{center}
    { \hspace*{\fill} \\}
    
    \section{CP 5.10 Period of an anharmonic oscillator with Gaussian
quadrature}\label{cp-5.10-period-of-an-anharmonic-oscillator-with-gaussian-quadrature}

Anharmonic oscillators are any oscillators not of the form
\(V(x) \propto x^2.\) So by treating the total energy as constant given
mass \(m\) and position \(x\), we have

\[E = \tfrac12 m \biggr(\frac{dx}{dt}\biggr)^2 + V(x)\]

which is a nonlinear differential equation in the variables \(x\) and
\(t\). We can then derive a relation for the period of the oscillator
for a particle starting at rest a distance \(a\) from the center.

In this scenario, the particle's total energy \(E = V(a)\), so we can
rewrite the entire equation as

\[V(a) = \tfrac12 m \biggr(\frac{dx}{dt}\biggr)^2 + V(x).\]

We can use algebra to rearrange it to

\[dt = \sqrt{\frac{m}{2}} \frac{dx}{\sqrt{V(a)-V(x)}}.\] This can be
integrated on either side to find the time it takes the particle to
travel any distance. For this, because we know it travels a quarter of a
period in a distance \(a\) (the distance to the center of the
potential), we'll use these as bounds for the integration.

\[\int_0^{\tfrac14T}dt = \tfrac14T = \int_0^a \sqrt{\frac{m}{2}} \frac{dx}{\sqrt{V(a)-V(x)}}\]

Finally, we are left with the expression for the period of the particle
that we were looking for:

\[T = \sqrt{8m} \int_0^a \frac{dx}{\sqrt{V(a)-V(x)}}.\]

    \begin{Verbatim}[commandchars=\\\{\}]
{\color{incolor}In [{\color{incolor}17}]:} \PY{c+c1}{\PYZsh{}definining constants for this problem in SI units}
         \PY{n}{m} \PY{o}{=} \PY{l+m+mi}{1} \PY{c+c1}{\PYZsh{}mass (kg)}
         
         \PY{c+c1}{\PYZsh{}defining function}
         \PY{k}{def} \PY{n+nf}{h}\PY{p}{(}\PY{n}{a}\PY{p}{,} \PY{n}{x}\PY{p}{)}\PY{p}{:}
             \PY{k}{return} \PY{l+m+mi}{1} \PY{o}{/} \PY{p}{(}\PY{n}{sqrt}\PY{p}{(}\PY{n}{a}\PY{o}{*}\PY{o}{*}\PY{l+m+mi}{4} \PY{o}{\PYZhy{}} \PY{n}{x}\PY{o}{*}\PY{o}{*}\PY{l+m+mi}{4}\PY{p}{)}\PY{p}{)}
\end{Verbatim}

    \begin{Verbatim}[commandchars=\\\{\}]
{\color{incolor}In [{\color{incolor}18}]:} \PY{k}{def} \PY{n+nf}{period}\PY{p}{(}\PY{n}{a}\PY{p}{,} \PY{n}{N}\PY{p}{)}\PY{p}{:}
             \PY{l+s+sd}{\PYZdq{}\PYZdq{}\PYZdq{}This will calculate the period of an anharmonic}
         \PY{l+s+sd}{        oscillator\PYZsq{}s period given the amplitude\PYZdq{}\PYZdq{}\PYZdq{}}
             
             \PY{c+c1}{\PYZsh{}part of function outside of integral}
             \PY{n}{p1} \PY{o}{=} \PY{n}{sqrt}\PY{p}{(}\PY{l+m+mi}{8}\PY{o}{*}\PY{n}{m}\PY{p}{)}
             
             \PY{n}{x}\PY{p}{,}\PY{n}{w} \PY{o}{=} \PY{n}{gaussxwab}\PY{p}{(}\PY{n}{N}\PY{p}{,} \PY{l+m+mi}{0}\PY{p}{,} \PY{n}{a}\PY{p}{)}
             
             \PY{n}{p2} \PY{o}{=} \PY{l+m+mi}{0}
             \PY{k}{for} \PY{n}{i} \PY{o+ow}{in} \PY{n+nb}{range}\PY{p}{(}\PY{n}{N}\PY{p}{)}\PY{p}{:}
                 \PY{n}{p2} \PY{o}{+}\PY{o}{=} \PY{n}{w}\PY{p}{[}\PY{n}{i}\PY{p}{]}\PY{o}{*}\PY{n}{h}\PY{p}{(}\PY{n}{a}\PY{p}{,} \PY{n}{x}\PY{p}{[}\PY{n}{i}\PY{p}{]}\PY{p}{)}
                 
             \PY{k}{return} \PY{n}{p1}\PY{o}{*}\PY{n}{p2}
\end{Verbatim}

    \begin{Verbatim}[commandchars=\\\{\}]
{\color{incolor}In [{\color{incolor}19}]:} \PY{c+c1}{\PYZsh{}plots period as amplitude ranges from a=0 to a=2}
         
         \PY{n}{a} \PY{o}{=} \PY{n}{np}\PY{o}{.}\PY{n}{linspace}\PY{p}{(}\PY{l+m+mf}{1e\PYZhy{}1}\PY{p}{,} \PY{l+m+mi}{2}\PY{p}{,} \PY{l+m+mi}{100}\PY{p}{)}
         \PY{n}{T} \PY{o}{=} \PY{n}{np}\PY{o}{.}\PY{n}{zeros}\PY{p}{(}\PY{l+m+mi}{100}\PY{p}{)}
         \PY{k}{for} \PY{n}{i} \PY{o+ow}{in} \PY{n+nb}{range}\PY{p}{(}\PY{l+m+mi}{100}\PY{p}{)}\PY{p}{:}
             \PY{n}{T}\PY{p}{[}\PY{n}{i}\PY{p}{]} \PY{o}{=} \PY{n}{period}\PY{p}{(}\PY{n}{a}\PY{p}{[}\PY{n}{i}\PY{p}{]}\PY{p}{,} \PY{l+m+mi}{20}\PY{p}{)}
         
         \PY{n}{fig}\PY{p}{,} \PY{n}{ax} \PY{o}{=} \PY{n}{plt}\PY{o}{.}\PY{n}{subplots}\PY{p}{(}\PY{l+m+mi}{1}\PY{p}{,} \PY{l+m+mi}{1}\PY{p}{,} \PY{n}{figsize} \PY{o}{=} \PY{p}{(}\PY{l+m+mi}{10}\PY{p}{,} \PY{l+m+mi}{7}\PY{p}{)}\PY{p}{)}
         
         \PY{n}{plt}\PY{o}{.}\PY{n}{plot}\PY{p}{(}\PY{n}{a}\PY{p}{,} \PY{n}{T}\PY{p}{,} \PY{l+s+s1}{\PYZsq{}}\PY{l+s+s1}{k\PYZhy{}}\PY{l+s+s1}{\PYZsq{}}\PY{p}{,} \PY{n}{color}\PY{o}{=}\PY{l+s+s1}{\PYZsq{}}\PY{l+s+s1}{blue}\PY{l+s+s1}{\PYZsq{}}\PY{p}{)}
         
         \PY{n}{plt}\PY{o}{.}\PY{n}{title}\PY{p}{(}\PY{l+s+s2}{\PYZdq{}}\PY{l+s+s2}{Plotting Period vs. Amplitude}\PY{l+s+s2}{\PYZdq{}}\PY{p}{)}
         \PY{n}{plt}\PY{o}{.}\PY{n}{xlabel}\PY{p}{(}\PY{l+s+s2}{\PYZdq{}}\PY{l+s+s2}{Amplitude (m)}\PY{l+s+s2}{\PYZdq{}}\PY{p}{)}
         \PY{n}{plt}\PY{o}{.}\PY{n}{ylabel}\PY{p}{(}\PY{l+s+s2}{\PYZdq{}}\PY{l+s+s2}{Period (s)}\PY{l+s+s2}{\PYZdq{}}\PY{p}{)}
         \PY{n}{plt}\PY{o}{.}\PY{n}{show}\PY{p}{(}\PY{p}{)}
\end{Verbatim}

    \begin{center}
    \adjustimage{max size={0.9\linewidth}{0.9\paperheight}}{output_26_0.png}
    \end{center}
    { \hspace*{\fill} \\}
    
    The oscillator's increased speed, and thus decreasing period as a result
of increasing amplitude is explained by the fact that the potential is
quartic. Because this problem was done under the assumptions of energy
conservation, the (quadratic in velocity) kinetic energy
\[K = \tfrac12mv^2 = \tfrac12m\biggr(\frac{dx}{dt}\biggr)^2\] must make
up for any lost potential energy. So as the potential drops off as a
quartic function while it approaches zero, the speed must increase in
greater proportion because it is only quadratic.

This same trade-off is responsible for the period going to infinity as
amplitude goes to zero. Because the maximum potential energy is
incredibly small in this scenario, at \(x=0\) the speed must also be
incredibly small (and this is where it is at its maximum). Therefore,
the period will diverge as amplitude goes to zero.

    \section{CP 5.11 Diffraction around edges with Gaussian
quadrature}\label{cp-5.11-diffraction-around-edges-with-gaussian-quadrature}

When a plane wave diffracts aroud an edge, the intensity at a point
\((x,z)\) is given by

\[I = \frac{I_0}{8} \Bigl( \bigl[ 2C(u) + 1 \bigr]^2 + \bigl[ 2S(u) + 1 \bigr]^2 \Bigr)\]

where
\[\qquad u = x \sqrt{2\over\lambda z}\,, \qquad C(u) = \int_0^u \cos \tfrac12\pi t^2 \ dt, \qquad S(u) = \int_0^u \sin \tfrac12\pi t^2 \ dt.\]

So, given this, a function can be defined to calculate the intensity
\(I\) of a sound wave relative to its pre diffracted intensity \(I_0.\)

    \begin{Verbatim}[commandchars=\\\{\}]
{\color{incolor}In [{\color{incolor}20}]:} \PY{c+c1}{\PYZsh{}definining constants for this problem in SI units}
         \PY{n}{l} \PY{o}{=} \PY{l+m+mi}{1} \PY{c+c1}{\PYZsh{}wavelength (m)}
         \PY{n}{N} \PY{o}{=} \PY{l+m+mi}{50} \PY{c+c1}{\PYZsh{}sampling points for Gaussian quadrature}
         
         \PY{c+c1}{\PYZsh{}defining position function}
         \PY{k}{def} \PY{n+nf}{u}\PY{p}{(}\PY{n}{x}\PY{p}{,}\PY{n}{z}\PY{p}{)}\PY{p}{:}
             \PY{k}{return} \PY{n}{x}\PY{o}{*}\PY{n}{sqrt}\PY{p}{(}\PY{l+m+mi}{2} \PY{o}{/} \PY{p}{(}\PY{n}{l}\PY{o}{*}\PY{n}{z}\PY{p}{)}\PY{p}{)}
         
         \PY{c+c1}{\PYZsh{}Main functions are defined below}
         
         \PY{k}{def} \PY{n+nf}{Cu}\PY{p}{(}\PY{n}{u}\PY{p}{)}\PY{p}{:}
             \PY{k}{def} \PY{n+nf}{c}\PY{p}{(}\PY{n}{t}\PY{p}{)}\PY{p}{:}
                 \PY{k}{return} \PY{n}{cos}\PY{p}{(}\PY{l+m+mf}{0.5}\PY{o}{*}\PY{n}{pi}\PY{o}{*}\PY{p}{(}\PY{n}{t}\PY{o}{*}\PY{o}{*}\PY{l+m+mi}{2}\PY{p}{)}\PY{p}{)}
             
             \PY{n}{x}\PY{p}{,}\PY{n}{w} \PY{o}{=} \PY{n}{gaussxwab}\PY{p}{(}\PY{n}{N}\PY{p}{,} \PY{l+m+mi}{0}\PY{p}{,} \PY{n}{u}\PY{p}{)}
             
             \PY{n}{pp1} \PY{o}{=} \PY{l+m+mi}{0}
             \PY{k}{for} \PY{n}{i} \PY{o+ow}{in} \PY{n+nb}{range}\PY{p}{(}\PY{n}{N}\PY{p}{)}\PY{p}{:}
                 \PY{n}{pp1} \PY{o}{+}\PY{o}{=} \PY{n}{w}\PY{p}{[}\PY{n}{i}\PY{p}{]}\PY{o}{*}\PY{n}{c}\PY{p}{(}\PY{n}{x}\PY{p}{[}\PY{n}{i}\PY{p}{]}\PY{p}{)}
                 
             \PY{k}{return} \PY{n}{pp1}
         
         \PY{k}{def} \PY{n+nf}{Su}\PY{p}{(}\PY{n}{u}\PY{p}{)}\PY{p}{:}
             \PY{k}{def} \PY{n+nf}{s}\PY{p}{(}\PY{n}{t}\PY{p}{)}\PY{p}{:}
                 \PY{k}{return} \PY{n}{sin}\PY{p}{(}\PY{l+m+mf}{0.5}\PY{o}{*}\PY{n}{pi}\PY{o}{*}\PY{p}{(}\PY{n}{t}\PY{o}{*}\PY{o}{*}\PY{l+m+mi}{2}\PY{p}{)}\PY{p}{)}
             
             \PY{n}{x}\PY{p}{,}\PY{n}{w} \PY{o}{=} \PY{n}{gaussxwab}\PY{p}{(}\PY{n}{N}\PY{p}{,} \PY{l+m+mi}{0}\PY{p}{,} \PY{n}{u}\PY{p}{)}
             
             \PY{n}{pp1} \PY{o}{=} \PY{l+m+mi}{0}
             \PY{k}{for} \PY{n}{i} \PY{o+ow}{in} \PY{n+nb}{range}\PY{p}{(}\PY{n}{N}\PY{p}{)}\PY{p}{:}
                 \PY{n}{pp1} \PY{o}{+}\PY{o}{=} \PY{n}{w}\PY{p}{[}\PY{n}{i}\PY{p}{]}\PY{o}{*}\PY{n}{s}\PY{p}{(}\PY{n}{x}\PY{p}{[}\PY{n}{i}\PY{p}{]}\PY{p}{)}
                 
             \PY{k}{return} \PY{n}{pp1}
         
         \PY{k}{def} \PY{n+nf}{ratio}\PY{p}{(}\PY{n}{u}\PY{p}{)}\PY{p}{:}
             \PY{l+s+sd}{\PYZdq{}\PYZdq{}\PYZdq{}Takes position as an argument and calculates the fraction}
         \PY{l+s+sd}{        of original intensity of a diffracted wave after edge\PYZdq{}\PYZdq{}\PYZdq{}}
             
             \PY{n}{II0} \PY{o}{=} \PY{p}{(}\PY{p}{(}\PY{l+m+mi}{2}\PY{o}{*}\PY{n}{Cu}\PY{p}{(}\PY{n}{u}\PY{p}{)} \PY{o}{+} \PY{l+m+mi}{1}\PY{p}{)}\PY{o}{*}\PY{o}{*}\PY{l+m+mi}{2} \PY{o}{+} \PY{p}{(}\PY{l+m+mi}{2}\PY{o}{*}\PY{n}{Su}\PY{p}{(}\PY{n}{u}\PY{p}{)} \PY{o}{+} \PY{l+m+mi}{1}\PY{p}{)}\PY{o}{*}\PY{o}{*}\PY{l+m+mi}{2}\PY{p}{)} \PY{o}{/} \PY{l+m+mi}{8}
             \PY{k}{return} \PY{n}{II0}
\end{Verbatim}

    \begin{Verbatim}[commandchars=\\\{\}]
{\color{incolor}In [{\color{incolor}21}]:} \PY{c+c1}{\PYZsh{}plots diffraction pattern}
         
         \PY{n}{density} \PY{o}{=} \PY{l+m+mi}{100}
         \PY{n}{z} \PY{o}{=} \PY{l+m+mi}{3} \PY{c+c1}{\PYZsh{}distance past edge (m)}
         
         \PY{n}{x} \PY{o}{=} \PY{n}{np}\PY{o}{.}\PY{n}{linspace}\PY{p}{(}\PY{o}{\PYZhy{}}\PY{l+m+mi}{5}\PY{p}{,}\PY{l+m+mi}{5}\PY{p}{,}\PY{n}{density}\PY{p}{)}
         \PY{n}{position} \PY{o}{=} \PY{n}{np}\PY{o}{.}\PY{n}{zeros}\PY{p}{(}\PY{n}{density}\PY{p}{)}
         \PY{n}{intensity} \PY{o}{=} \PY{n}{np}\PY{o}{.}\PY{n}{zeros}\PY{p}{(}\PY{n}{density}\PY{p}{)}
         
         \PY{k}{for} \PY{n}{i} \PY{o+ow}{in} \PY{n+nb}{range}\PY{p}{(}\PY{n}{density}\PY{p}{)}\PY{p}{:}
             \PY{n}{position}\PY{p}{[}\PY{n}{i}\PY{p}{]} \PY{o}{=} \PY{n}{u}\PY{p}{(}\PY{n}{x}\PY{p}{[}\PY{n}{i}\PY{p}{]}\PY{p}{,}\PY{n}{z}\PY{p}{)}
             \PY{n}{intensity}\PY{p}{[}\PY{n}{i}\PY{p}{]} \PY{o}{=} \PY{n}{ratio}\PY{p}{(}\PY{n}{position}\PY{p}{[}\PY{n}{i}\PY{p}{]}\PY{p}{)}
             
         \PY{n}{fig}\PY{p}{,} \PY{n}{ax} \PY{o}{=} \PY{n}{plt}\PY{o}{.}\PY{n}{subplots}\PY{p}{(}\PY{l+m+mi}{1}\PY{p}{,} \PY{l+m+mi}{1}\PY{p}{,} \PY{n}{figsize} \PY{o}{=} \PY{p}{(}\PY{l+m+mi}{7}\PY{p}{,} \PY{l+m+mi}{7}\PY{p}{)}\PY{p}{)}
         
         \PY{n}{plt}\PY{o}{.}\PY{n}{plot}\PY{p}{(}\PY{n}{x}\PY{p}{,} \PY{n}{intensity}\PY{p}{,} \PY{l+s+s1}{\PYZsq{}}\PY{l+s+s1}{k\PYZhy{}}\PY{l+s+s1}{\PYZsq{}}\PY{p}{)}
         
         \PY{n}{plt}\PY{o}{.}\PY{n}{title}\PY{p}{(}\PY{l+s+s2}{\PYZdq{}}\PY{l+s+s2}{Intensity of Diffracted Plane Waves}\PY{l+s+s2}{\PYZdq{}}\PY{p}{)}
         \PY{n}{plt}\PY{o}{.}\PY{n}{xlabel}\PY{p}{(}\PY{l+s+s2}{\PYZdq{}}\PY{l+s+s2}{Perpendicular Distance from Edge (m)}\PY{l+s+s2}{\PYZdq{}}\PY{p}{)}
         \PY{n}{plt}\PY{o}{.}\PY{n}{ylabel}\PY{p}{(}\PY{l+s+s2}{\PYZdq{}}\PY{l+s+s2}{Fraction of Original Intensity}\PY{l+s+s2}{\PYZdq{}}\PY{p}{)}
         \PY{n}{plt}\PY{o}{.}\PY{n}{show}\PY{p}{(}\PY{p}{)}
\end{Verbatim}

    \begin{center}
    \adjustimage{max size={0.9\linewidth}{0.9\paperheight}}{output_30_0.png}
    \end{center}
    { \hspace*{\fill} \\}
    
    As per the problem, there is noticeable variation in the intensity of
sound diffracted around the edge, from almost nonexistent if you're well
'covered' by the screen to greater than the original intensity a little
under 2 m up when 3 m away.

    \section{CP 5.12 The Stefan-Boltzmann constant with Gaussian
quadrature}\label{cp-5.12-the-stefan-boltzmann-constant-with-gaussian-quadrature}

In some time interval \(d\omega\) a black body electromagnetically
radiates thermal energy \(I(\omega) d\omega,\) where

\[I(\omega) = {\hbar\over4\pi^2c^2}\,{\omega^3\over(e^{\hbar\omega/k_BT}-1)}.\]

Therefore the total energy per unit area radiated by that same black
body is

\[W = {k_B^4 T^4\over4\pi^2c^2\hbar^3} \int_0^\infty {x^3\over e^x-1}\ dx.\]

    \begin{Verbatim}[commandchars=\\\{\}]
{\color{incolor}In [{\color{incolor}22}]:} \PY{c+c1}{\PYZsh{}defining constants}
         \PY{n}{kB} \PY{o}{=} \PY{n}{C}\PY{o}{.}\PY{n}{k} \PY{c+c1}{\PYZsh{}Boltzmann\PYZsq{}s constant (J/K)}
         \PY{n}{hbar} \PY{o}{=} \PY{n}{C}\PY{o}{.}\PY{n}{hbar} \PY{c+c1}{\PYZsh{}reduced Planck\PYZsq{}s constant (kg m\PYZca{}2 s\PYZca{}\PYZhy{}1)}
         \PY{n}{c} \PY{o}{=} \PY{n}{C}\PY{o}{.}\PY{n}{c} \PY{c+c1}{\PYZsh{}speed of light (m/s)}
         
         \PY{c+c1}{\PYZsh{}defining the integrand}
         \PY{k}{def} \PY{n+nf}{intgrnd}\PY{p}{(}\PY{n}{z}\PY{p}{)}\PY{p}{:}
             \PY{n}{num} \PY{o}{=} \PY{p}{(}\PY{n}{z}\PY{o}{/}\PY{p}{(}\PY{l+m+mi}{1}\PY{o}{\PYZhy{}}\PY{n}{z}\PY{p}{)}\PY{p}{)}\PY{o}{*}\PY{o}{*}\PY{l+m+mi}{3} \PY{c+c1}{\PYZsh{}numerator of f(x)}
             \PY{n}{den} \PY{o}{=} \PY{p}{(}\PY{n}{exp}\PY{p}{(}\PY{n}{z}\PY{o}{/}\PY{p}{(}\PY{l+m+mi}{1}\PY{o}{\PYZhy{}}\PY{n}{z}\PY{p}{)}\PY{p}{)} \PY{o}{\PYZhy{}} \PY{l+m+mi}{1}\PY{p}{)} \PY{c+c1}{\PYZsh{}denominator of f(x)}
             \PY{k}{return} \PY{p}{(}\PY{n}{num} \PY{o}{/} \PY{n}{den}\PY{p}{)} \PY{o}{*} \PY{p}{(}\PY{l+m+mi}{1}\PY{o}{/} \PY{p}{(}\PY{p}{(}\PY{l+m+mi}{1}\PY{o}{\PYZhy{}}\PY{n}{z}\PY{p}{)}\PY{o}{*}\PY{o}{*}\PY{l+m+mi}{2}\PY{p}{)}\PY{p}{)} \PY{c+c1}{\PYZsh{}chain rule}
         
         
         \PY{k}{def} \PY{n+nf}{radiated\PYZus{}energy}\PY{p}{(}\PY{n}{T}\PY{p}{)}\PY{p}{:}
             \PY{l+s+sd}{\PYZdq{}\PYZdq{}\PYZdq{}Gives the total energy per unit area radiated}
         \PY{l+s+sd}{        by a black body.\PYZdq{}\PYZdq{}\PYZdq{}}
             
             \PY{n}{N} \PY{o}{=} \PY{l+m+mi}{50}
             \PY{n}{num} \PY{o}{=} \PY{p}{(}\PY{n}{kB}\PY{o}{*}\PY{o}{*}\PY{l+m+mi}{4}\PY{p}{)} \PY{o}{*} \PY{n}{T}\PY{o}{*}\PY{o}{*}\PY{l+m+mi}{4}
             \PY{n}{den} \PY{o}{=} \PY{l+m+mi}{4} \PY{o}{*} \PY{p}{(}\PY{n}{pi}\PY{o}{*}\PY{n}{c}\PY{p}{)}\PY{o}{*}\PY{o}{*}\PY{l+m+mi}{2} \PY{o}{*} \PY{n}{hbar}\PY{o}{*}\PY{o}{*}\PY{l+m+mi}{3}
             \PY{n}{out} \PY{o}{=} \PY{n}{num} \PY{o}{/} \PY{n}{den} \PY{c+c1}{\PYZsh{}constants outside of integral}
             
             \PY{n}{x}\PY{p}{,}\PY{n}{w} \PY{o}{=} \PY{n}{gaussxwab}\PY{p}{(}\PY{n}{N}\PY{p}{,} \PY{l+m+mf}{0.0001}\PY{p}{,} \PY{l+m+mf}{0.999}\PY{p}{)}
             
             \PY{n}{I} \PY{o}{=} \PY{l+m+mi}{0}
             \PY{k}{for} \PY{n}{i} \PY{o+ow}{in} \PY{n+nb}{range}\PY{p}{(}\PY{n}{N}\PY{p}{)}\PY{p}{:}
                 \PY{n}{I} \PY{o}{+}\PY{o}{=} \PY{n}{w}\PY{p}{[}\PY{n}{i}\PY{p}{]}\PY{o}{*}\PY{n}{intgrnd}\PY{p}{(}\PY{n}{x}\PY{p}{[}\PY{n}{i}\PY{p}{]}\PY{p}{)}
                 
             \PY{k}{return} \PY{n}{out}\PY{o}{*}\PY{n}{I}
\end{Verbatim}

    To evaluate this integral I used Gaussian quadrature over the range
\((0,1).\) To change the range, and calculate this integral from zero to
infinity, a substitution was required to change it to a finite integral.
In general,

\[\int_0^\infty f(x)\ dx = \int_0^1 \frac{1}{(1-z)^2}\ f\biggr(\frac{z}{1-z}\biggr)\ dz.\]

My answer should be accurate the same order as all other integrals
calculated through Gaussian quadrature, i.e., about 50 points would
correspond to the limit of the computer's precision.

    \begin{Verbatim}[commandchars=\\\{\}]
{\color{incolor}In [{\color{incolor}23}]:} \PY{c+c1}{\PYZsh{}calcualtes Stefan\PYZhy{}Boltzmann constant and fractional error}
         
         \PY{k}{def} \PY{n+nf}{find\PYZus{}sigma}\PY{p}{(}\PY{n}{T}\PY{p}{)}\PY{p}{:}
             \PY{l+s+sd}{\PYZdq{}\PYZdq{}\PYZdq{}This function ends up always taking the same value}
         \PY{l+s+sd}{        because the T\PYZca{}4 from the energy and the Stefan\PYZhy{}Boltzmann}
         \PY{l+s+sd}{        relation cancel. It also finds the error on calculating}
         \PY{l+s+sd}{        sigma\PYZdq{}\PYZdq{}\PYZdq{}}
             \PY{n}{sigma} \PY{o}{=} \PY{n}{radiated\PYZus{}energy}\PY{p}{(}\PY{n}{T}\PY{p}{)} \PY{o}{/} \PY{n}{T}\PY{o}{*}\PY{o}{*}\PY{l+m+mi}{4}
             \PY{n}{error} \PY{o}{=} \PY{p}{(}\PY{n+nb}{abs}\PY{p}{(}\PY{n}{sigma} \PY{o}{\PYZhy{}} \PY{n}{C}\PY{o}{.}\PY{n}{sigma}\PY{p}{)}\PY{p}{)} \PY{o}{/} \PY{n}{C}\PY{o}{.}\PY{n}{sigma} \PY{o}{*} \PY{l+m+mi}{100} \PY{c+c1}{\PYZsh{}in percent}
             
             \PY{k}{return} \PY{n}{sigma}\PY{p}{,} \PY{n}{error}
         
         
         \PY{n+nb}{print}\PY{p}{(}\PY{l+s+s2}{\PYZdq{}}\PY{l+s+s2}{Stefan\PYZhy{}Boltzmann constant is }\PY{l+s+si}{\PYZob{}:2.3e\PYZcb{}}\PY{l+s+s2}{ and error is }\PY{l+s+si}{\PYZob{}:2.2e\PYZcb{}}\PY{l+s+s2}{\PYZpc{}}\PY{l+s+s2}{.}\PY{l+s+s2}{\PYZdq{}}\PYZbs{}
              \PY{o}{.}\PY{n}{format}\PY{p}{(}\PY{n}{find\PYZus{}sigma}\PY{p}{(}\PY{l+m+mi}{1000}\PY{p}{)}\PY{p}{[}\PY{l+m+mi}{0}\PY{p}{]}\PY{p}{,} \PY{n}{find\PYZus{}sigma}\PY{p}{(}\PY{l+m+mi}{1000}\PY{p}{)}\PY{p}{[}\PY{l+m+mi}{1}\PY{p}{]}\PY{p}{)}\PY{p}{)}
\end{Verbatim}

    \begin{Verbatim}[commandchars=\\\{\}]
Stefan-Boltzmann constant is 5.670e-08 and error is 3.28e-06\%.

    \end{Verbatim}

    \section{CP 5.13 Quantum uncertainty in the harmonic oscillator with
Gaussian
quadrature}\label{cp-5.13-quantum-uncertainty-in-the-harmonic-oscillator-with-gaussian-quadrature}

In a one-dimensional quantum harmonic oscillator, a spinless point
particle's wavefunction is

\[\psi_n(x) = {1\over\sqrt{2^n n!\sqrt{\pi}}}\, e^{-x^2/2}\,H_n(x)\]

if that particle is in the \(n\)th energy level. (Assuming use of units
that make constants 1.)

The Hermite polynomials \(H_n(x)\) are given by

\[H_{n+1}(x) = 2xH_n(x) - 2nH_{n-1}(x),\]

where \(H_0(x) = 1\) and \(H_1(x) = 2x.\)

    \begin{Verbatim}[commandchars=\\\{\}]
{\color{incolor}In [{\color{incolor}71}]:} \PY{k}{def} \PY{n+nf}{Hermite}\PY{p}{(}\PY{n}{n}\PY{p}{,}\PY{n}{x}\PY{p}{)}\PY{p}{:}
             \PY{l+s+sd}{\PYZdq{}\PYZdq{}\PYZdq{}Gives nth Hermite polynomial for given n and x\PYZdq{}\PYZdq{}\PYZdq{}}
             
             \PY{k}{if} \PY{n}{n} \PY{o}{==} \PY{l+m+mi}{0}\PY{p}{:}
                 \PY{k}{return} \PY{l+m+mi}{1}
             \PY{k}{elif} \PY{n}{n} \PY{o}{==} \PY{l+m+mi}{1}\PY{p}{:}
                 \PY{k}{return} \PY{l+m+mi}{2}\PY{o}{*}\PY{n}{x}
             \PY{k}{else}\PY{p}{:}
                 \PY{k}{return} \PY{l+m+mi}{2}\PY{o}{*}\PY{n}{x}\PY{o}{*}\PY{n}{Hermite}\PY{p}{(}\PY{n}{n}\PY{o}{\PYZhy{}}\PY{l+m+mi}{1}\PY{p}{,}\PY{n}{x}\PY{p}{)} \PY{o}{\PYZhy{}} \PY{l+m+mi}{2}\PY{o}{*}\PY{p}{(}\PY{n}{n}\PY{o}{\PYZhy{}}\PY{l+m+mi}{1}\PY{p}{)}\PY{o}{*}\PY{n}{Hermite}\PY{p}{(}\PY{n}{n}\PY{o}{\PYZhy{}}\PY{l+m+mi}{2}\PY{p}{,}\PY{n}{x}\PY{p}{)}
             
         
         \PY{k}{def} \PY{n+nf}{psi}\PY{p}{(}\PY{n}{n}\PY{p}{,}\PY{n}{x}\PY{p}{)}\PY{p}{:}
             \PY{l+s+sd}{\PYZdq{}\PYZdq{}\PYZdq{}Gives the wavefunction for a particle in the nth energy}
         \PY{l+s+sd}{        level over position x\PYZdq{}\PYZdq{}\PYZdq{}}
             
             \PY{n}{frac} \PY{o}{=} \PY{l+m+mi}{1} \PY{o}{/} \PY{p}{(}\PY{n}{sqrt}\PY{p}{(}\PY{p}{(}\PY{l+m+mi}{2}\PY{o}{*}\PY{o}{*}\PY{n}{n}\PY{p}{)}\PY{o}{*}\PY{n}{factorial}\PY{p}{(}\PY{n}{n}\PY{p}{)}\PY{o}{*}\PY{n}{sqrt}\PY{p}{(}\PY{n}{pi}\PY{p}{)}\PY{p}{)}\PY{p}{)}
             \PY{n}{expo} \PY{o}{=} \PY{n}{exp}\PY{p}{(}\PY{p}{(}\PY{o}{\PYZhy{}}\PY{n}{x}\PY{o}{*}\PY{o}{*}\PY{l+m+mi}{2}\PY{p}{)}\PY{o}{/}\PY{l+m+mi}{2}\PY{p}{)}
             
             \PY{k}{return} \PY{n}{frac} \PY{o}{*} \PY{n}{expo} \PY{o}{*} \PY{n}{Hermite}\PY{p}{(}\PY{n}{n}\PY{p}{,}\PY{n}{x}\PY{p}{)}
\end{Verbatim}

    \begin{Verbatim}[commandchars=\\\{\}]
{\color{incolor}In [{\color{incolor}73}]:} \PY{c+c1}{\PYZsh{}plots on same set of axes}
         
         \PY{n}{x} \PY{o}{=} \PY{n}{np}\PY{o}{.}\PY{n}{linspace}\PY{p}{(}\PY{o}{\PYZhy{}}\PY{l+m+mi}{4}\PY{p}{,}\PY{l+m+mi}{4}\PY{p}{,}\PY{l+m+mi}{400}\PY{p}{)}
         
         \PY{n}{fig}\PY{p}{,} \PY{n}{ax} \PY{o}{=} \PY{n}{plt}\PY{o}{.}\PY{n}{subplots}\PY{p}{(}\PY{l+m+mi}{1}\PY{p}{,} \PY{l+m+mi}{1}\PY{p}{,} \PY{n}{figsize} \PY{o}{=} \PY{p}{(}\PY{l+m+mi}{10}\PY{p}{,} \PY{l+m+mi}{6}\PY{p}{)}\PY{p}{)}
         \PY{c+c1}{\PYZsh{}jet= plt.get\PYZus{}cmap(\PYZsq{}jet\PYZsq{})}
         \PY{c+c1}{\PYZsh{}colors = iter(jet(np.linspace(0,1,4)))}
         
         \PY{n}{psi0} \PY{o}{=} \PY{n}{np}\PY{o}{.}\PY{n}{zeros}\PY{p}{(}\PY{l+m+mi}{400}\PY{p}{)}
         \PY{n}{psi1} \PY{o}{=} \PY{n}{np}\PY{o}{.}\PY{n}{zeros}\PY{p}{(}\PY{l+m+mi}{400}\PY{p}{)}
         \PY{n}{psi2} \PY{o}{=} \PY{n}{np}\PY{o}{.}\PY{n}{zeros}\PY{p}{(}\PY{l+m+mi}{400}\PY{p}{)}
         \PY{n}{psi3} \PY{o}{=} \PY{n}{np}\PY{o}{.}\PY{n}{zeros}\PY{p}{(}\PY{l+m+mi}{400}\PY{p}{)}
         
         \PY{k}{for} \PY{n}{i} \PY{o+ow}{in} \PY{n+nb}{range}\PY{p}{(}\PY{l+m+mi}{400}\PY{p}{)}\PY{p}{:} 
             \PY{n}{psi0}\PY{p}{[}\PY{n}{i}\PY{p}{]} \PY{o}{=} \PY{n}{psi}\PY{p}{(}\PY{l+m+mi}{0}\PY{p}{,}\PY{n}{x}\PY{p}{[}\PY{n}{i}\PY{p}{]}\PY{p}{)}
             \PY{n}{psi1}\PY{p}{[}\PY{n}{i}\PY{p}{]} \PY{o}{=} \PY{n}{psi}\PY{p}{(}\PY{l+m+mi}{1}\PY{p}{,}\PY{n}{x}\PY{p}{[}\PY{n}{i}\PY{p}{]}\PY{p}{)}
             \PY{n}{psi2}\PY{p}{[}\PY{n}{i}\PY{p}{]} \PY{o}{=} \PY{n}{psi}\PY{p}{(}\PY{l+m+mi}{2}\PY{p}{,}\PY{n}{x}\PY{p}{[}\PY{n}{i}\PY{p}{]}\PY{p}{)}
             \PY{n}{psi3}\PY{p}{[}\PY{n}{i}\PY{p}{]} \PY{o}{=} \PY{n}{psi}\PY{p}{(}\PY{l+m+mi}{3}\PY{p}{,}\PY{n}{x}\PY{p}{[}\PY{n}{i}\PY{p}{]}\PY{p}{)}
             
         \PY{n}{plt}\PY{o}{.}\PY{n}{plot}\PY{p}{(}\PY{n}{x}\PY{p}{,} \PY{n}{psi0}\PY{p}{,} \PY{l+s+s1}{\PYZsq{}}\PY{l+s+s1}{k\PYZhy{}}\PY{l+s+s1}{\PYZsq{}}\PY{p}{)}
         \PY{n}{plt}\PY{o}{.}\PY{n}{plot}\PY{p}{(}\PY{n}{x}\PY{p}{,} \PY{n}{psi1}\PY{p}{,} \PY{l+s+s1}{\PYZsq{}}\PY{l+s+s1}{b\PYZhy{}}\PY{l+s+s1}{\PYZsq{}}\PY{p}{)}
         \PY{n}{plt}\PY{o}{.}\PY{n}{plot}\PY{p}{(}\PY{n}{x}\PY{p}{,} \PY{n}{psi2}\PY{p}{,} \PY{l+s+s1}{\PYZsq{}}\PY{l+s+s1}{r\PYZhy{}}\PY{l+s+s1}{\PYZsq{}}\PY{p}{)}
         \PY{n}{plt}\PY{o}{.}\PY{n}{plot}\PY{p}{(}\PY{n}{x}\PY{p}{,} \PY{n}{psi3}\PY{p}{,} \PY{l+s+s1}{\PYZsq{}}\PY{l+s+s1}{g\PYZhy{}}\PY{l+s+s1}{\PYZsq{}}\PY{p}{)}
             
         \PY{n}{plt}\PY{o}{.}\PY{n}{title}\PY{p}{(}\PY{l+s+s2}{\PYZdq{}}\PY{l+s+s2}{Wavefunctions in a 1\PYZhy{}D Harmonic Oscillator}\PY{l+s+s2}{\PYZdq{}}\PY{p}{)}
         \PY{n}{plt}\PY{o}{.}\PY{n}{xlabel}\PY{p}{(}\PY{l+s+s2}{\PYZdq{}}\PY{l+s+s2}{Position}\PY{l+s+s2}{\PYZdq{}}\PY{p}{)}
         \PY{n}{plt}\PY{o}{.}\PY{n}{ylabel}\PY{p}{(}\PY{l+s+s2}{\PYZdq{}}\PY{l+s+s2}{Probability Density}\PY{l+s+s2}{\PYZdq{}}\PY{p}{)}
         \PY{n}{plt}\PY{o}{.}\PY{n}{show}\PY{p}{(}\PY{p}{)}
\end{Verbatim}

    \begin{center}
    \adjustimage{max size={0.9\linewidth}{0.9\paperheight}}{output_38_0.png}
    \end{center}
    { \hspace*{\fill} \\}
    
    \begin{Verbatim}[commandchars=\\\{\}]
{\color{incolor}In [{\color{incolor}79}]:} \PY{c+c1}{\PYZsh{}plot for 30th energy level from x=\PYZhy{}10 to x=10}
         \PY{n}{count} \PY{o}{=} \PY{l+m+mi}{500}
         
         \PY{n}{x2} \PY{o}{=} \PY{n}{np}\PY{o}{.}\PY{n}{linspace}\PY{p}{(}\PY{o}{\PYZhy{}}\PY{l+m+mi}{10}\PY{p}{,}\PY{l+m+mi}{10}\PY{p}{,}\PY{n}{count}\PY{p}{)}
         
         \PY{n}{fig}\PY{p}{,} \PY{n}{ax} \PY{o}{=} \PY{n}{plt}\PY{o}{.}\PY{n}{subplots}\PY{p}{(}\PY{l+m+mi}{1}\PY{p}{,} \PY{l+m+mi}{1}\PY{p}{,} \PY{n}{figsize} \PY{o}{=} \PY{p}{(}\PY{l+m+mi}{7}\PY{p}{,} \PY{l+m+mi}{7}\PY{p}{)}\PY{p}{)}
         
         \PY{n}{psi30} \PY{o}{=} \PY{n}{np}\PY{o}{.}\PY{n}{zeros}\PY{p}{(}\PY{n}{count}\PY{p}{)}
         \PY{k}{for} \PY{n}{i} \PY{o+ow}{in} \PY{n+nb}{range}\PY{p}{(}\PY{n}{count}\PY{p}{)}\PY{p}{:}
             \PY{n}{psi30}\PY{p}{[}\PY{n}{i}\PY{p}{]} \PY{o}{=} \PY{n}{psi}\PY{p}{(}\PY{l+m+mi}{30}\PY{p}{,}\PY{n}{x2}\PY{p}{[}\PY{n}{i}\PY{p}{]}\PY{p}{)}
         
         \PY{n}{plt}\PY{o}{.}\PY{n}{plot}\PY{p}{(}\PY{n}{x2}\PY{p}{,}\PY{n}{psi30}\PY{p}{,} \PY{l+s+s1}{\PYZsq{}}\PY{l+s+s1}{\PYZhy{}}\PY{l+s+s1}{\PYZsq{}}\PY{p}{)}
             
         \PY{n}{plt}\PY{o}{.}\PY{n}{title}\PY{p}{(}\PY{l+s+s2}{\PYZdq{}}\PY{l+s+s2}{Wavefunction for n=30 in a 1\PYZhy{}D Harmonic Oscillator}\PY{l+s+s2}{\PYZdq{}}\PY{p}{)}
         \PY{n}{plt}\PY{o}{.}\PY{n}{xlabel}\PY{p}{(}\PY{l+s+s2}{\PYZdq{}}\PY{l+s+s2}{Position}\PY{l+s+s2}{\PYZdq{}}\PY{p}{)}
         \PY{n}{plt}\PY{o}{.}\PY{n}{ylabel}\PY{p}{(}\PY{l+s+s2}{\PYZdq{}}\PY{l+s+s2}{Probability Density}\PY{l+s+s2}{\PYZdq{}}\PY{p}{)}
         \PY{n}{plt}\PY{o}{.}\PY{n}{show}\PY{p}{(}\PY{p}{)}
\end{Verbatim}

    \begin{center}
    \adjustimage{max size={0.9\linewidth}{0.9\paperheight}}{output_39_0.png}
    \end{center}
    { \hspace*{\fill} \\}
    
    A different method of substitution can be used to evaluate the integral
\[\langle{x^2}\rangle = \int_{-\infty}^\infty x^2 |\psi_n(x)|^2 \ d x.\]

In this case, we make the substitutions
\(x = \tan{z} \text{ and } dx = \frac{dz}{\cos^2{z}}.\) This will give
us the equivalency
\[\int_\infty^\infty f(x)\ dx = \int_{-\frac{\pi}{2}}^{\frac{\pi}{2}} \frac{f(\tan{z})}{\cos^2{z}}\ dz.\]

So, we can make these substitutions in the wavefunctions to find the RMS
position.

The wavefunctions in \(x\) are
\[\psi_n(x) = {1\over\sqrt{2^n n!\sqrt{\pi}}}\, e^{-x^2/2}\,H_n(x).\]

We should be evaluating the integral

\[\int_{-\frac{\pi}{2}}^{\frac{\pi}{2}} \frac{\tan^2{z}}{\cos^2{z}} \biggr|\ \psi_n(\tan{z}) \biggr|^2 \ dz\]

For \(n=5,\) we should find that the root-mean-square position
\(\sqrt{\langle x^2 \rangle} = 2.3.\)

    \begin{Verbatim}[commandchars=\\\{\}]
{\color{incolor}In [{\color{incolor}77}]:} \PY{c+c1}{\PYZsh{}defines integrand in change of variables}
         \PY{k}{def} \PY{n+nf}{change}\PY{p}{(}\PY{n}{z}\PY{p}{)}\PY{p}{:}
             \PY{n}{n} \PY{o}{=} \PY{l+m+mi}{5}
             \PY{n}{front} \PY{o}{=} \PY{p}{(}\PY{p}{(}\PY{n}{tan}\PY{p}{(}\PY{n}{z}\PY{p}{)}\PY{p}{)}\PY{o}{*}\PY{o}{*}\PY{l+m+mi}{2} \PY{o}{/} \PY{p}{(}\PY{n}{cos}\PY{p}{(}\PY{n}{z}\PY{p}{)}\PY{p}{)}\PY{o}{*}\PY{o}{*}\PY{l+m+mi}{2}\PY{p}{)}
             \PY{k}{return} \PY{n}{front} \PY{o}{*} \PY{p}{(}\PY{n+nb}{abs}\PY{p}{(}\PY{n}{psi}\PY{p}{(}\PY{n}{n}\PY{p}{,} \PY{n}{tan}\PY{p}{(}\PY{n}{z}\PY{p}{)}\PY{p}{)}\PY{p}{)}\PY{p}{)}\PY{o}{*}\PY{o}{*}\PY{l+m+mi}{2}
         
         \PY{k}{def} \PY{n+nf}{RMS}\PY{p}{(}\PY{p}{)}\PY{p}{:}
             \PY{n}{a} \PY{o}{=} \PY{o}{\PYZhy{}}\PY{n}{pi}\PY{o}{/}\PY{l+m+mi}{2}
             \PY{n}{b} \PY{o}{=} \PY{n}{pi}\PY{o}{/}\PY{l+m+mi}{2}
             \PY{n}{N} \PY{o}{=} \PY{l+m+mi}{100}
             
             \PY{n}{x}\PY{p}{,}\PY{n}{w} \PY{o}{=} \PY{n}{gaussxwab}\PY{p}{(}\PY{n}{N}\PY{p}{,} \PY{n}{a}\PY{p}{,} \PY{n}{b}\PY{p}{)}
             
             \PY{n}{s} \PY{o}{=} \PY{l+m+mi}{0}
             \PY{k}{for} \PY{n}{i} \PY{o+ow}{in} \PY{n+nb}{range}\PY{p}{(}\PY{n}{N}\PY{p}{)}\PY{p}{:}
                 \PY{n}{s} \PY{o}{+}\PY{o}{=} \PY{n}{w}\PY{p}{[}\PY{n}{i}\PY{p}{]}\PY{o}{*}\PY{n}{change}\PY{p}{(}\PY{n}{x}\PY{p}{[}\PY{n}{i}\PY{p}{]}\PY{p}{)}
                 
             \PY{k}{return} \PY{n}{sqrt}\PY{p}{(}\PY{n}{s}\PY{p}{)}
         
         \PY{n}{error} \PY{o}{=} \PY{l+m+mi}{100} \PY{o}{*} \PY{p}{(}\PY{n}{RMS}\PY{p}{(}\PY{p}{)} \PY{o}{\PYZhy{}} \PY{l+m+mf}{2.3}\PY{p}{)}\PY{o}{/}\PY{l+m+mf}{2.3}
         
         \PY{n+nb}{print}\PY{p}{(}\PY{l+s+s2}{\PYZdq{}}\PY{l+s+s2}{The RMS = }\PY{l+s+si}{\PYZob{}:2.3f\PYZcb{}}\PY{l+s+s2}{, which has an error of }\PY{l+s+si}{\PYZob{}:2.2f\PYZcb{}}\PY{l+s+s2}{\PYZpc{}}\PY{l+s+s2}{\PYZdq{}}\PYZbs{}
              \PY{o}{.}\PY{n}{format}\PY{p}{(}\PY{n}{RMS}\PY{p}{(}\PY{p}{)}\PY{p}{,} \PY{n}{error}\PY{p}{)}\PY{p}{)}
\end{Verbatim}

    \begin{Verbatim}[commandchars=\\\{\}]
The RMS = 2.345, which has an error of 1.97\%

    \end{Verbatim}

    This value of the RMS \(\sqrt{\langle x^2 \rangle}\) is in the
neighborhood of the known value.

    \section{CP 5.19 Diffraction
gratings}\label{cp-5.19-diffraction-gratings}

Light diffracted through a grating then put through a lens has intensity
given by the function

\[I(x) = \biggl| \int_{-w/2}^{w/2} \sqrt{q(u)}\ e^{2\pi i xu/\lambda f} \ d u\ \biggr|^2\]
for grating of width \(w,\) light of wavelength \(\lambda,\) lens of
focal length \(f,\) and distance from the central axis \(u.\)

For a grating with the transmission function
\(q(u) = \sin^2{\alpha u},\) the separation of slits
\(d = \frac{\pi}{\alpha}.\)

    \begin{Verbatim}[commandchars=\\\{\}]
{\color{incolor}In [{\color{incolor}31}]:} \PY{c+c1}{\PYZsh{}constants}
         \PY{n}{alpha} \PY{o}{=} \PY{n}{pi} \PY{o}{/} \PY{l+m+mf}{20e\PYZhy{}6} \PY{c+c1}{\PYZsh{}(m)}
         \PY{n}{l} \PY{o}{=} \PY{l+m+mf}{500e\PYZhy{}9} \PY{c+c1}{\PYZsh{}(m)}
         \PY{n}{f} \PY{o}{=} \PY{l+m+mi}{1} \PY{c+c1}{\PYZsh{}(m)}
         \PY{n}{w} \PY{o}{=} \PY{l+m+mi}{10} \PY{o}{*} \PY{p}{(}\PY{n}{pi}\PY{o}{/}\PY{n}{alpha}\PY{p}{)}
         
         \PY{n}{slow} \PY{o}{=} \PY{o}{\PYZhy{}}\PY{l+m+mf}{0.05} \PY{c+c1}{\PYZsh{}screen lower bound (m)}
         \PY{n}{supp} \PY{o}{=} \PY{l+m+mf}{0.05} \PY{c+c1}{\PYZsh{}screen upper bound (m)}
         
         \PY{k}{def} \PY{n+nf}{q}\PY{p}{(}\PY{n}{u}\PY{p}{)}\PY{p}{:}
             \PY{k}{return} \PY{p}{(}\PY{n}{sin}\PY{p}{(}\PY{n}{alpha}\PY{o}{*}\PY{n}{u}\PY{p}{)}\PY{p}{)}\PY{o}{*}\PY{o}{*}\PY{l+m+mi}{2}
\end{Verbatim}

    \begin{Verbatim}[commandchars=\\\{\}]
{\color{incolor}In [{\color{incolor}32}]:} \PY{c+c1}{\PYZsh{}defines integrand}
         \PY{k}{def} \PY{n+nf}{intgrd}\PY{p}{(}\PY{n}{u}\PY{p}{,}\PY{n}{x}\PY{p}{)}\PY{p}{:}
             \PY{l+s+sd}{\PYZdq{}\PYZdq{}\PYZdq{}Gives value of the integrand for given u and x\PYZdq{}\PYZdq{}\PYZdq{}}
             \PY{k}{return} \PY{n}{sqrt}\PY{p}{(}\PY{n}{q}\PY{p}{(}\PY{n}{u}\PY{p}{)}\PY{p}{)} \PY{o}{*} \PY{p}{(}\PY{n}{cexp}\PY{p}{(}\PY{l+m+mi}{2}\PY{o}{*}\PY{n}{pi}\PY{o}{*}\PY{n}{x}\PY{o}{*}\PY{n}{u}\PY{o}{*}\PY{l+m+mi}{1}\PY{n}{j}\PY{o}{/}\PY{p}{(}\PY{n}{l}\PY{o}{*}\PY{n}{f}\PY{p}{)}\PY{p}{)}\PY{p}{)}
         
         
         \PY{k}{def} \PY{n+nf}{intensity}\PY{p}{(}\PY{n}{x}\PY{p}{)}\PY{p}{:}
             \PY{l+s+sd}{\PYZdq{}\PYZdq{}\PYZdq{}Uses Simpson\PYZsq{}s rule to integrate the intensity function\PYZdq{}\PYZdq{}\PYZdq{}}
             \PY{n}{a} \PY{o}{=} \PY{o}{\PYZhy{}}\PY{n}{w}\PY{o}{/}\PY{l+m+mi}{2}
             \PY{n}{b} \PY{o}{=} \PY{n}{w}\PY{o}{/}\PY{l+m+mi}{2}
             \PY{n}{N} \PY{o}{=} \PY{l+m+mi}{50}
             \PY{n}{h} \PY{o}{=} \PY{p}{(}\PY{n}{b}\PY{o}{\PYZhy{}}\PY{n}{a}\PY{p}{)} \PY{o}{/} \PY{n}{N}
             
             \PY{n}{sum1} \PY{o}{=} \PY{l+m+mi}{0}
             \PY{n}{sum2} \PY{o}{=} \PY{l+m+mi}{0}
             \PY{k}{for} \PY{n}{k} \PY{o+ow}{in} \PY{n+nb}{range}\PY{p}{(}\PY{n}{N}\PY{p}{)}\PY{p}{:}
                 \PY{k}{if} \PY{n}{k} \PY{o}{\PYZpc{}} \PY{l+m+mi}{2} \PY{o}{==} \PY{l+m+mi}{1}\PY{p}{:}
                     \PY{n}{sum1} \PY{o}{+}\PY{o}{=} \PY{n}{intgrd}\PY{p}{(}\PY{n}{a}\PY{o}{+}\PY{n}{k}\PY{o}{*}\PY{n}{h}\PY{p}{,} \PY{n}{x}\PY{p}{)}
                 \PY{k}{elif} \PY{n}{k}\PY{o}{\PYZpc{}}\PY{k}{1} == 0:
                     \PY{n}{sum2} \PY{o}{+}\PY{o}{=} \PY{n}{intgrd}\PY{p}{(}\PY{n}{a}\PY{o}{+}\PY{n}{k}\PY{o}{*}\PY{n}{h}\PY{p}{,} \PY{n}{x}\PY{p}{)}
             
             \PY{n}{intensity} \PY{o}{=} \PY{n}{h}\PY{o}{*}\PY{p}{(}\PY{n}{intgrd}\PY{p}{(}\PY{n}{a}\PY{p}{,}\PY{n}{x}\PY{p}{)} \PY{o}{+} \PY{n}{intgrd}\PY{p}{(}\PY{n}{b}\PY{p}{,}\PY{n}{x}\PY{p}{)} \PY{o}{+} \PY{l+m+mi}{4}\PY{o}{*}\PY{n}{sum1} \PY{o}{+} \PY{l+m+mi}{2}\PY{o}{*}\PY{n}{sum2}\PY{p}{)} \PY{o}{/} \PY{l+m+mi}{3}
             
             \PY{n}{I} \PY{o}{=} \PY{p}{(}\PY{n+nb}{abs}\PY{p}{(}\PY{n}{intensity}\PY{p}{)}\PY{p}{)}\PY{o}{*}\PY{o}{*}\PY{l+m+mi}{2}
                     
             \PY{k}{return} \PY{n}{I}
\end{Verbatim}

    \begin{Verbatim}[commandchars=\\\{\}]
{\color{incolor}In [{\color{incolor}33}]:} \PY{c+c1}{\PYZsh{}plotting intensity profile to find ideal vmax}
         
         \PY{n}{points} \PY{o}{=} \PY{l+m+mi}{1000}
         \PY{n}{xx} \PY{o}{=} \PY{n}{np}\PY{o}{.}\PY{n}{linspace}\PY{p}{(}\PY{n}{slow}\PY{p}{,} \PY{n}{supp}\PY{p}{,} \PY{n}{points}\PY{p}{)}
         \PY{n}{density} \PY{o}{=} \PY{n}{np}\PY{o}{.}\PY{n}{empty}\PY{p}{(}\PY{n}{points}\PY{p}{)}
         
         \PY{k}{for} \PY{n}{i} \PY{o+ow}{in} \PY{n+nb}{range}\PY{p}{(}\PY{n}{points}\PY{p}{)}\PY{p}{:}
             \PY{n}{density}\PY{p}{[}\PY{n}{i}\PY{p}{]} \PY{o}{=} \PY{n}{intensity}\PY{p}{(}\PY{n}{xx}\PY{p}{[}\PY{n}{i}\PY{p}{]}\PY{p}{)}
             
         \PY{n}{fig}\PY{p}{,} \PY{n}{ax} \PY{o}{=} \PY{n}{plt}\PY{o}{.}\PY{n}{subplots}\PY{p}{(}\PY{l+m+mi}{1}\PY{p}{,} \PY{l+m+mi}{1}\PY{p}{,} \PY{n}{figsize} \PY{o}{=} \PY{p}{(}\PY{l+m+mi}{3}\PY{p}{,} \PY{l+m+mi}{2}\PY{p}{)}\PY{p}{)}
         
         \PY{n}{plt}\PY{o}{.}\PY{n}{plot}\PY{p}{(}\PY{n}{xx}\PY{p}{,} \PY{n}{density}\PY{p}{)}
         \PY{n}{plt}\PY{o}{.}\PY{n}{show}\PY{p}{(}\PY{p}{)}
\end{Verbatim}

    \begin{center}
    \adjustimage{max size={0.9\linewidth}{0.9\paperheight}}{output_46_0.png}
    \end{center}
    { \hspace*{\fill} \\}
    
    \begin{Verbatim}[commandchars=\\\{\}]
{\color{incolor}In [{\color{incolor}34}]:} \PY{c+c1}{\PYZsh{}code creating the screen}
         \PY{n}{screen} \PY{o}{=} \PY{l+m+mf}{0.1} \PY{c+c1}{\PYZsh{}(m)}
         \PY{n}{points} \PY{o}{=} \PY{l+m+mi}{1000} \PY{c+c1}{\PYZsh{}number of grid points on long side}
         \PY{n}{spacing} \PY{o}{=} \PY{n}{screen} \PY{o}{/} \PY{n}{points} \PY{c+c1}{\PYZsh{}spacing of points in m}
         \PY{n}{center} \PY{o}{=} \PY{n}{screen}\PY{o}{/}\PY{l+m+mi}{2}
         
         \PY{n}{pattern} \PY{o}{=} \PY{n}{np}\PY{o}{.}\PY{n}{zeros}\PY{p}{(}\PY{p}{[}\PY{l+m+mi}{400}\PY{p}{,}\PY{n}{points}\PY{p}{]}\PY{p}{,}\PY{n+nb}{float}\PY{p}{)}
         \PY{c+c1}{\PYZsh{}calculate the values in the array}
         
         \PY{k}{for} \PY{n}{i} \PY{o+ow}{in} \PY{n+nb}{range}\PY{p}{(}\PY{n}{points}\PY{p}{)}\PY{p}{:}
             \PY{n}{x} \PY{o}{=} \PY{n}{spacing} \PY{o}{*} \PY{n}{i}
             \PY{n}{pattern}\PY{p}{[}\PY{p}{:}\PY{p}{,}\PY{n}{i}\PY{p}{]} \PY{o}{=} \PY{n}{intensity}\PY{p}{(}\PY{n}{x}\PY{o}{\PYZhy{}}\PY{n}{center}\PY{p}{)}
         
         \PY{n}{fig}\PY{p}{,} \PY{n}{ax} \PY{o}{=} \PY{n}{plt}\PY{o}{.}\PY{n}{subplots}\PY{p}{(}\PY{l+m+mi}{1}\PY{p}{,} \PY{l+m+mi}{1}\PY{p}{,} \PY{n}{figsize} \PY{o}{=} \PY{p}{(}\PY{l+m+mi}{12}\PY{p}{,}\PY{l+m+mi}{3}\PY{p}{)}\PY{p}{)}
         
         \PY{c+c1}{\PYZsh{}increases readability of plot}
         \PY{n}{ax}\PY{o}{.}\PY{n}{set\PYZus{}title}\PY{p}{(}\PY{l+s+s2}{\PYZdq{}}\PY{l+s+s2}{Diffraction Pattern on Screen from Simple Grating}\PY{l+s+s2}{\PYZdq{}}\PY{p}{)}
         \PY{n}{ax}\PY{o}{.}\PY{n}{set\PYZus{}xlabel}\PY{p}{(}\PY{l+s+s2}{\PYZdq{}}\PY{l+s+s2}{x (\PYZdl{}m\PYZdl{})}\PY{l+s+s2}{\PYZdq{}}\PY{p}{)}
         \PY{n}{ax}\PY{o}{.}\PY{n}{set\PYZus{}xticks}\PY{p}{(}\PY{p}{[}\PY{o}{\PYZhy{}}\PY{l+m+mf}{0.05}\PY{p}{,} \PY{o}{\PYZhy{}}\PY{l+m+mf}{0.04}\PY{p}{,} \PY{o}{\PYZhy{}}\PY{l+m+mf}{0.03}\PY{p}{,} \PY{o}{\PYZhy{}}\PY{l+m+mf}{0.02}\PY{p}{,} \PY{o}{\PYZhy{}}\PY{l+m+mf}{0.01}\PY{p}{,} \PY{l+m+mi}{0}\PY{p}{,}\PYZbs{}
                        \PY{l+m+mf}{0.01}\PY{p}{,} \PY{l+m+mf}{0.02}\PY{p}{,} \PY{l+m+mf}{0.03}\PY{p}{,} \PY{l+m+mf}{0.04}\PY{p}{,} \PY{l+m+mf}{0.05}\PY{p}{]}\PY{p}{)}
         \PY{n}{ax}\PY{o}{.}\PY{n}{set\PYZus{}yticks}\PY{p}{(}\PY{p}{[}\PY{p}{]}\PY{p}{)}
         
         \PY{c+c1}{\PYZsh{}vmax deduced from plot above}
         \PY{n}{ax}\PY{o}{.}\PY{n}{imshow}\PY{p}{(}\PY{n}{pattern}\PY{p}{,}\PY{n}{origin}\PY{o}{=}\PY{l+s+s2}{\PYZdq{}}\PY{l+s+s2}{lower}\PY{l+s+s2}{\PYZdq{}}\PY{p}{,}\PYZbs{}
                   \PY{n}{cmap}\PY{o}{=}\PY{l+s+s2}{\PYZdq{}}\PY{l+s+s2}{gray}\PY{l+s+s2}{\PYZdq{}}\PY{p}{,} \PY{n}{vmax}\PY{o}{=}\PY{l+m+mf}{0.2e\PYZhy{}8}\PY{p}{,}\PYZbs{}
                   \PY{n}{extent}\PY{o}{=}\PY{p}{[}\PY{o}{\PYZhy{}}\PY{l+m+mf}{0.05}\PY{p}{,}\PY{l+m+mf}{0.05}\PY{p}{,} \PY{l+m+mi}{0}\PY{p}{,}\PY{o}{.}\PY{l+m+mi}{025}\PY{p}{]}\PY{p}{)}
\end{Verbatim}

\begin{Verbatim}[commandchars=\\\{\}]
{\color{outcolor}Out[{\color{outcolor}34}]:} <matplotlib.image.AxesImage at 0x11744bcc0>
\end{Verbatim}
            
    \begin{center}
    \adjustimage{max size={0.9\linewidth}{0.9\paperheight}}{output_47_1.png}
    \end{center}
    { \hspace*{\fill} \\}
    
    The next few cells redefine the transmission function \(q(u)\) and
create a density plot for this physical phenomenon. The code for
plotting is the same, except the value for vmax on the density plot will
likely be different and now determined from a new intensity profile.

    \begin{Verbatim}[commandchars=\\\{\}]
{\color{incolor}In [{\color{incolor}35}]:} \PY{c+c1}{\PYZsh{}new cell for defining things}
         \PY{n}{beta} \PY{o}{=} \PY{l+m+mf}{0.5} \PY{o}{*} \PY{n}{alpha}
         
         \PY{k}{def} \PY{n+nf}{q}\PY{p}{(}\PY{n}{u}\PY{p}{)}\PY{p}{:}
             \PY{k}{return} \PY{p}{(}\PY{p}{(}\PY{n}{sin}\PY{p}{(}\PY{n}{alpha}\PY{o}{*}\PY{n}{u}\PY{p}{)}\PY{p}{)}\PY{o}{*}\PY{o}{*}\PY{l+m+mi}{2}\PY{p}{)} \PY{o}{*} \PY{p}{(}\PY{p}{(}\PY{n}{sin}\PY{p}{(}\PY{n}{beta}\PY{o}{*}\PY{n}{u}\PY{p}{)}\PY{p}{)}\PY{o}{*}\PY{o}{*}\PY{l+m+mi}{2}\PY{p}{)}
\end{Verbatim}

    \begin{Verbatim}[commandchars=\\\{\}]
{\color{incolor}In [{\color{incolor}36}]:} \PY{c+c1}{\PYZsh{}plotting intensity profile to find ideal vmax}
         
         \PY{n}{points} \PY{o}{=} \PY{l+m+mi}{1000}
         \PY{n}{xx} \PY{o}{=} \PY{n}{np}\PY{o}{.}\PY{n}{linspace}\PY{p}{(}\PY{n}{slow}\PY{p}{,} \PY{n}{supp}\PY{p}{,} \PY{n}{points}\PY{p}{)}
         \PY{n}{density} \PY{o}{=} \PY{n}{np}\PY{o}{.}\PY{n}{empty}\PY{p}{(}\PY{n}{points}\PY{p}{)}
         
         \PY{k}{for} \PY{n}{i} \PY{o+ow}{in} \PY{n+nb}{range}\PY{p}{(}\PY{n}{points}\PY{p}{)}\PY{p}{:}
             \PY{n}{density}\PY{p}{[}\PY{n}{i}\PY{p}{]} \PY{o}{=} \PY{n}{intensity}\PY{p}{(}\PY{n}{xx}\PY{p}{[}\PY{n}{i}\PY{p}{]}\PY{p}{)}
             
         \PY{n}{fig}\PY{p}{,} \PY{n}{ax} \PY{o}{=} \PY{n}{plt}\PY{o}{.}\PY{n}{subplots}\PY{p}{(}\PY{l+m+mi}{1}\PY{p}{,} \PY{l+m+mi}{1}\PY{p}{,} \PY{n}{figsize} \PY{o}{=} \PY{p}{(}\PY{l+m+mi}{3}\PY{p}{,} \PY{l+m+mi}{2}\PY{p}{)}\PY{p}{)}
         
         \PY{n}{plt}\PY{o}{.}\PY{n}{plot}\PY{p}{(}\PY{n}{xx}\PY{p}{,} \PY{n}{density}\PY{p}{)}
         \PY{n}{plt}\PY{o}{.}\PY{n}{show}\PY{p}{(}\PY{p}{)}
\end{Verbatim}

    \begin{center}
    \adjustimage{max size={0.9\linewidth}{0.9\paperheight}}{output_50_0.png}
    \end{center}
    { \hspace*{\fill} \\}
    
    \begin{Verbatim}[commandchars=\\\{\}]
{\color{incolor}In [{\color{incolor}37}]:} \PY{c+c1}{\PYZsh{}code creating the screen}
         \PY{n}{screen} \PY{o}{=} \PY{l+m+mf}{0.1} \PY{c+c1}{\PYZsh{}(m)}
         \PY{n}{points} \PY{o}{=} \PY{l+m+mi}{1000} \PY{c+c1}{\PYZsh{}number of grid points on long side}
         \PY{n}{spacing} \PY{o}{=} \PY{n}{screen} \PY{o}{/} \PY{n}{points} \PY{c+c1}{\PYZsh{}spacing of points in m}
         \PY{n}{center} \PY{o}{=} \PY{n}{screen}\PY{o}{/}\PY{l+m+mi}{2}
         
         \PY{n}{pattern} \PY{o}{=} \PY{n}{np}\PY{o}{.}\PY{n}{zeros}\PY{p}{(}\PY{p}{[}\PY{l+m+mi}{400}\PY{p}{,}\PY{n}{points}\PY{p}{]}\PY{p}{,}\PY{n+nb}{float}\PY{p}{)}
         \PY{c+c1}{\PYZsh{}calculate the values in the array}
         
         \PY{k}{for} \PY{n}{i} \PY{o+ow}{in} \PY{n+nb}{range}\PY{p}{(}\PY{n}{points}\PY{p}{)}\PY{p}{:}
             \PY{n}{x} \PY{o}{=} \PY{n}{spacing} \PY{o}{*} \PY{n}{i}
             \PY{n}{pattern}\PY{p}{[}\PY{p}{:}\PY{p}{,}\PY{n}{i}\PY{p}{]} \PY{o}{=} \PY{n}{intensity}\PY{p}{(}\PY{n}{x}\PY{o}{\PYZhy{}}\PY{n}{center}\PY{p}{)}
         
         \PY{n}{fig}\PY{p}{,} \PY{n}{ax} \PY{o}{=} \PY{n}{plt}\PY{o}{.}\PY{n}{subplots}\PY{p}{(}\PY{l+m+mi}{1}\PY{p}{,} \PY{l+m+mi}{1}\PY{p}{,} \PY{n}{figsize} \PY{o}{=} \PY{p}{(}\PY{l+m+mi}{12}\PY{p}{,}\PY{l+m+mi}{3}\PY{p}{)}\PY{p}{)}
         
         \PY{c+c1}{\PYZsh{}increases readability of plot}
         \PY{n}{ax}\PY{o}{.}\PY{n}{set\PYZus{}title}\PY{p}{(}\PY{l+s+s2}{\PYZdq{}}\PY{l+s+s2}{Diffraction Pattern on Screen for More Complex Grating}\PY{l+s+s2}{\PYZdq{}}\PY{p}{)}
         \PY{n}{ax}\PY{o}{.}\PY{n}{set\PYZus{}xlabel}\PY{p}{(}\PY{l+s+s2}{\PYZdq{}}\PY{l+s+s2}{x (\PYZdl{}m\PYZdl{})}\PY{l+s+s2}{\PYZdq{}}\PY{p}{)}
         \PY{n}{ax}\PY{o}{.}\PY{n}{set\PYZus{}xticks}\PY{p}{(}\PY{p}{[}\PY{o}{\PYZhy{}}\PY{l+m+mf}{0.05}\PY{p}{,} \PY{o}{\PYZhy{}}\PY{l+m+mf}{0.04}\PY{p}{,} \PY{o}{\PYZhy{}}\PY{l+m+mf}{0.03}\PY{p}{,} \PY{o}{\PYZhy{}}\PY{l+m+mf}{0.02}\PY{p}{,} \PY{o}{\PYZhy{}}\PY{l+m+mf}{0.01}\PY{p}{,} \PY{l+m+mi}{0}\PY{p}{,}\PYZbs{}
                        \PY{l+m+mf}{0.01}\PY{p}{,} \PY{l+m+mf}{0.02}\PY{p}{,} \PY{l+m+mf}{0.03}\PY{p}{,} \PY{l+m+mf}{0.04}\PY{p}{,} \PY{l+m+mf}{0.05}\PY{p}{]}\PY{p}{)}
         \PY{n}{ax}\PY{o}{.}\PY{n}{set\PYZus{}yticks}\PY{p}{(}\PY{p}{[}\PY{p}{]}\PY{p}{)}
         
         \PY{c+c1}{\PYZsh{}vmax deduced from plot above}
         \PY{n}{ax}\PY{o}{.}\PY{n}{imshow}\PY{p}{(}\PY{n}{pattern}\PY{p}{,}\PY{n}{origin}\PY{o}{=}\PY{l+s+s2}{\PYZdq{}}\PY{l+s+s2}{lower}\PY{l+s+s2}{\PYZdq{}}\PY{p}{,}\PYZbs{}
                   \PY{n}{cmap}\PY{o}{=}\PY{l+s+s2}{\PYZdq{}}\PY{l+s+s2}{gray}\PY{l+s+s2}{\PYZdq{}}\PY{p}{,}\PY{n}{vmax}\PY{o}{=}\PY{l+m+mf}{0.8e\PYZhy{}9}\PY{p}{,}\PYZbs{}
                   \PY{n}{extent}\PY{o}{=}\PY{p}{[}\PY{o}{\PYZhy{}}\PY{l+m+mf}{0.05}\PY{p}{,}\PY{l+m+mf}{0.05}\PY{p}{,} \PY{l+m+mi}{0}\PY{p}{,}\PY{o}{.}\PY{l+m+mi}{025}\PY{p}{]}\PY{p}{)}
\end{Verbatim}

\begin{Verbatim}[commandchars=\\\{\}]
{\color{outcolor}Out[{\color{outcolor}37}]:} <matplotlib.image.AxesImage at 0x116fbce48>
\end{Verbatim}
            
    \begin{center}
    \adjustimage{max size={0.9\linewidth}{0.9\paperheight}}{output_51_1.png}
    \end{center}
    { \hspace*{\fill} \\}
    
    The next few cells redefine the transmission function \(q(u)\) again and
create a density plot for the two "square" slits scenario. The
transmission function for this system should be defined piecewise for
the 4 regions that exist. The regions can be combined into ones of
transmission and ones without.

\begin{enumerate}
\def\labelenumi{\arabic{enumi}.}
\tightlist
\item
  Beyond the far edges of the slits
\item
  In front of the 10 \(\mu m\) slit
\item
  Behind the 60 \(\mu m\) gap
\item
  In front of the 20 \(\mu m\) slit
\end{enumerate}

The code for plotting is the same, except the value for vmax on the
density plot will likely be different and now determined from a new
intensity profile.

    \begin{Verbatim}[commandchars=\\\{\}]
{\color{incolor}In [{\color{incolor}38}]:} \PY{k}{def} \PY{n+nf}{q}\PY{p}{(}\PY{n}{u}\PY{p}{)}\PY{p}{:}
             \PY{c+c1}{\PYZsh{}in front of slits}
             \PY{k}{if} \PY{p}{(}\PY{n}{u} \PY{o}{\PYZgt{}} \PY{l+m+mi}{0} \PY{o+ow}{and} \PY{n}{u} \PY{o}{\PYZlt{}} \PY{l+m+mf}{10e\PYZhy{}6}\PY{p}{)} \PY{o+ow}{or} \PY{p}{(}\PY{n}{u} \PY{o}{\PYZgt{}} \PY{l+m+mf}{70e\PYZhy{}6} \PY{o+ow}{and} \PY{n}{u} \PY{o}{\PYZlt{}} \PY{l+m+mf}{90e\PYZhy{}6}\PY{p}{)}\PY{p}{:}
                 \PY{k}{return} \PY{l+m+mi}{1}
             
             \PY{c+c1}{\PYZsh{}outside}
             \PY{k}{else}\PY{p}{:}
                 \PY{k}{return} \PY{l+m+mi}{0}
\end{Verbatim}

    \begin{Verbatim}[commandchars=\\\{\}]
{\color{incolor}In [{\color{incolor}39}]:} \PY{c+c1}{\PYZsh{}plotting intensity profile to find ideal vmax}
         
         \PY{n}{points} \PY{o}{=} \PY{l+m+mi}{1000}
         \PY{n}{xx} \PY{o}{=} \PY{n}{np}\PY{o}{.}\PY{n}{linspace}\PY{p}{(}\PY{n}{slow}\PY{p}{,} \PY{n}{supp}\PY{p}{,} \PY{n}{points}\PY{p}{)}
         \PY{n}{density} \PY{o}{=} \PY{n}{np}\PY{o}{.}\PY{n}{empty}\PY{p}{(}\PY{n}{points}\PY{p}{)}
         
         \PY{k}{for} \PY{n}{i} \PY{o+ow}{in} \PY{n+nb}{range}\PY{p}{(}\PY{n}{points}\PY{p}{)}\PY{p}{:}
             \PY{n}{density}\PY{p}{[}\PY{n}{i}\PY{p}{]} \PY{o}{=} \PY{n}{intensity}\PY{p}{(}\PY{n}{xx}\PY{p}{[}\PY{n}{i}\PY{p}{]}\PY{p}{)}
             
         \PY{n}{fig}\PY{p}{,} \PY{n}{ax} \PY{o}{=} \PY{n}{plt}\PY{o}{.}\PY{n}{subplots}\PY{p}{(}\PY{l+m+mi}{1}\PY{p}{,} \PY{l+m+mi}{1}\PY{p}{,} \PY{n}{figsize} \PY{o}{=} \PY{p}{(}\PY{l+m+mi}{3}\PY{p}{,} \PY{l+m+mi}{2}\PY{p}{)}\PY{p}{)}
         
         \PY{n}{plt}\PY{o}{.}\PY{n}{plot}\PY{p}{(}\PY{n}{xx}\PY{p}{,} \PY{n}{density}\PY{p}{)}
         \PY{n}{plt}\PY{o}{.}\PY{n}{show}\PY{p}{(}\PY{p}{)}
\end{Verbatim}

    \begin{center}
    \adjustimage{max size={0.9\linewidth}{0.9\paperheight}}{output_54_0.png}
    \end{center}
    { \hspace*{\fill} \\}
    
    \begin{Verbatim}[commandchars=\\\{\}]
{\color{incolor}In [{\color{incolor}40}]:} \PY{c+c1}{\PYZsh{}code creating the screen}
         \PY{n}{screen} \PY{o}{=} \PY{l+m+mf}{0.1} \PY{c+c1}{\PYZsh{}(m)}
         \PY{n}{points} \PY{o}{=} \PY{l+m+mi}{1000} \PY{c+c1}{\PYZsh{}number of grid points on long side}
         \PY{n}{spacing} \PY{o}{=} \PY{n}{screen} \PY{o}{/} \PY{n}{points} \PY{c+c1}{\PYZsh{}spacing of points in m}
         \PY{n}{center} \PY{o}{=} \PY{n}{screen}\PY{o}{/}\PY{l+m+mi}{2}
         
         \PY{n}{pattern} \PY{o}{=} \PY{n}{np}\PY{o}{.}\PY{n}{zeros}\PY{p}{(}\PY{p}{[}\PY{l+m+mi}{400}\PY{p}{,}\PY{n}{points}\PY{p}{]}\PY{p}{,}\PY{n+nb}{float}\PY{p}{)}
         \PY{c+c1}{\PYZsh{}calculate the values in the array}
         
         \PY{k}{for} \PY{n}{i} \PY{o+ow}{in} \PY{n+nb}{range}\PY{p}{(}\PY{n}{points}\PY{p}{)}\PY{p}{:}
             \PY{n}{x} \PY{o}{=} \PY{n}{spacing} \PY{o}{*} \PY{n}{i}
             \PY{n}{pattern}\PY{p}{[}\PY{p}{:}\PY{p}{,}\PY{n}{i}\PY{p}{]} \PY{o}{=} \PY{n}{intensity}\PY{p}{(}\PY{n}{x}\PY{o}{\PYZhy{}}\PY{n}{center}\PY{p}{)}
         
         \PY{n}{fig}\PY{p}{,} \PY{n}{ax} \PY{o}{=} \PY{n}{plt}\PY{o}{.}\PY{n}{subplots}\PY{p}{(}\PY{l+m+mi}{1}\PY{p}{,} \PY{l+m+mi}{1}\PY{p}{,} \PY{n}{figsize} \PY{o}{=} \PY{p}{(}\PY{l+m+mi}{12}\PY{p}{,}\PY{l+m+mi}{3}\PY{p}{)}\PY{p}{)}
         
         \PY{c+c1}{\PYZsh{}increases readability of plot}
         \PY{n}{ax}\PY{o}{.}\PY{n}{set\PYZus{}title}\PY{p}{(}\PY{l+s+s2}{\PYZdq{}}\PY{l+s+s2}{Diffraction Pattern on Screen from Two Square Slits}\PY{l+s+s2}{\PYZdq{}}\PY{p}{)}
         \PY{n}{ax}\PY{o}{.}\PY{n}{set\PYZus{}xlabel}\PY{p}{(}\PY{l+s+s2}{\PYZdq{}}\PY{l+s+s2}{x (\PYZdl{}m\PYZdl{})}\PY{l+s+s2}{\PYZdq{}}\PY{p}{)}
         \PY{n}{ax}\PY{o}{.}\PY{n}{set\PYZus{}xticks}\PY{p}{(}\PY{p}{[}\PY{o}{\PYZhy{}}\PY{l+m+mf}{0.05}\PY{p}{,} \PY{o}{\PYZhy{}}\PY{l+m+mf}{0.04}\PY{p}{,} \PY{o}{\PYZhy{}}\PY{l+m+mf}{0.03}\PY{p}{,} \PY{o}{\PYZhy{}}\PY{l+m+mf}{0.02}\PY{p}{,} \PY{o}{\PYZhy{}}\PY{l+m+mf}{0.01}\PY{p}{,} \PY{l+m+mi}{0}\PY{p}{,}\PYZbs{}
                        \PY{l+m+mf}{0.01}\PY{p}{,} \PY{l+m+mf}{0.02}\PY{p}{,} \PY{l+m+mf}{0.03}\PY{p}{,} \PY{l+m+mf}{0.04}\PY{p}{,} \PY{l+m+mf}{0.05}\PY{p}{]}\PY{p}{)}
         \PY{n}{ax}\PY{o}{.}\PY{n}{set\PYZus{}yticks}\PY{p}{(}\PY{p}{[}\PY{p}{]}\PY{p}{)}
         
         \PY{c+c1}{\PYZsh{}vmax deduced from plot above}
         \PY{n}{ax}\PY{o}{.}\PY{n}{imshow}\PY{p}{(}\PY{n}{pattern}\PY{p}{,}\PY{n}{origin}\PY{o}{=}\PY{l+s+s2}{\PYZdq{}}\PY{l+s+s2}{lower}\PY{l+s+s2}{\PYZdq{}}\PY{p}{,}\PYZbs{}
                   \PY{n}{cmap}\PY{o}{=}\PY{l+s+s2}{\PYZdq{}}\PY{l+s+s2}{gray}\PY{l+s+s2}{\PYZdq{}}\PY{p}{,}\PY{n}{vmax}\PY{o}{=}\PY{l+m+mf}{0.7e\PYZhy{}9}\PY{p}{,}\PYZbs{}
                   \PY{n}{extent}\PY{o}{=}\PY{p}{[}\PY{o}{\PYZhy{}}\PY{l+m+mf}{0.05}\PY{p}{,}\PY{l+m+mf}{0.05}\PY{p}{,} \PY{l+m+mi}{0}\PY{p}{,}\PY{o}{.}\PY{l+m+mi}{025}\PY{p}{]}\PY{p}{)}
\end{Verbatim}

\begin{Verbatim}[commandchars=\\\{\}]
{\color{outcolor}Out[{\color{outcolor}40}]:} <matplotlib.image.AxesImage at 0x116f5e550>
\end{Verbatim}
            
    \begin{center}
    \adjustimage{max size={0.9\linewidth}{0.9\paperheight}}{output_55_1.png}
    \end{center}
    { \hspace*{\fill} \\}
    
    \begin{Verbatim}[commandchars=\\\{\}]
{\color{incolor}In [{\color{incolor} }]:} 
\end{Verbatim}


    % Add a bibliography block to the postdoc
    
    
    
    \end{document}


% Default to the notebook output style

    


% Inherit from the specified cell style.




    
\documentclass[11pt]{article}

    
    
    \usepackage[T1]{fontenc}
    % Nicer default font (+ math font) than Computer Modern for most use cases
    \usepackage{mathpazo}

    % Basic figure setup, for now with no caption control since it's done
    % automatically by Pandoc (which extracts ![](path) syntax from Markdown).
    \usepackage{graphicx}
    % We will generate all images so they have a width \maxwidth. This means
    % that they will get their normal width if they fit onto the page, but
    % are scaled down if they would overflow the margins.
    \makeatletter
    \def\maxwidth{\ifdim\Gin@nat@width>\linewidth\linewidth
    \else\Gin@nat@width\fi}
    \makeatother
    \let\Oldincludegraphics\includegraphics
    % Set max figure width to be 80% of text width, for now hardcoded.
    \renewcommand{\includegraphics}[1]{\Oldincludegraphics[width=.8\maxwidth]{#1}}
    % Ensure that by default, figures have no caption (until we provide a
    % proper Figure object with a Caption API and a way to capture that
    % in the conversion process - todo).
    \usepackage{caption}
    \DeclareCaptionLabelFormat{nolabel}{}
    \captionsetup{labelformat=nolabel}

    \usepackage{adjustbox} % Used to constrain images to a maximum size 
    \usepackage{xcolor} % Allow colors to be defined
    \usepackage{enumerate} % Needed for markdown enumerations to work
    \usepackage{geometry} % Used to adjust the document margins
    \usepackage{amsmath} % Equations
    \usepackage{amssymb} % Equations
    \usepackage{textcomp} % defines textquotesingle
    % Hack from http://tex.stackexchange.com/a/47451/13684:
    \AtBeginDocument{%
        \def\PYZsq{\textquotesingle}% Upright quotes in Pygmentized code
    }
    \usepackage{upquote} % Upright quotes for verbatim code
    \usepackage{eurosym} % defines \euro
    \usepackage[mathletters]{ucs} % Extended unicode (utf-8) support
    \usepackage[utf8x]{inputenc} % Allow utf-8 characters in the tex document
    \usepackage{fancyvrb} % verbatim replacement that allows latex
    \usepackage{grffile} % extends the file name processing of package graphics 
                         % to support a larger range 
    % The hyperref package gives us a pdf with properly built
    % internal navigation ('pdf bookmarks' for the table of contents,
    % internal cross-reference links, web links for URLs, etc.)
    \usepackage{hyperref}
    \usepackage{longtable} % longtable support required by pandoc >1.10
    \usepackage{booktabs}  % table support for pandoc > 1.12.2
    \usepackage[inline]{enumitem} % IRkernel/repr support (it uses the enumerate* environment)
    \usepackage[normalem]{ulem} % ulem is needed to support strikethroughs (\sout)
                                % normalem makes italics be italics, not underlines
    \usepackage{mathrsfs}
    

    
    
    % Colors for the hyperref package
    \definecolor{urlcolor}{rgb}{0,.145,.698}
    \definecolor{linkcolor}{rgb}{.71,0.21,0.01}
    \definecolor{citecolor}{rgb}{.12,.54,.11}

    % ANSI colors
    \definecolor{ansi-black}{HTML}{3E424D}
    \definecolor{ansi-black-intense}{HTML}{282C36}
    \definecolor{ansi-red}{HTML}{E75C58}
    \definecolor{ansi-red-intense}{HTML}{B22B31}
    \definecolor{ansi-green}{HTML}{00A250}
    \definecolor{ansi-green-intense}{HTML}{007427}
    \definecolor{ansi-yellow}{HTML}{DDB62B}
    \definecolor{ansi-yellow-intense}{HTML}{B27D12}
    \definecolor{ansi-blue}{HTML}{208FFB}
    \definecolor{ansi-blue-intense}{HTML}{0065CA}
    \definecolor{ansi-magenta}{HTML}{D160C4}
    \definecolor{ansi-magenta-intense}{HTML}{A03196}
    \definecolor{ansi-cyan}{HTML}{60C6C8}
    \definecolor{ansi-cyan-intense}{HTML}{258F8F}
    \definecolor{ansi-white}{HTML}{C5C1B4}
    \definecolor{ansi-white-intense}{HTML}{A1A6B2}
    \definecolor{ansi-default-inverse-fg}{HTML}{FFFFFF}
    \definecolor{ansi-default-inverse-bg}{HTML}{000000}

    % commands and environments needed by pandoc snippets
    % extracted from the output of `pandoc -s`
    \providecommand{\tightlist}{%
      \setlength{\itemsep}{0pt}\setlength{\parskip}{0pt}}
    \DefineVerbatimEnvironment{Highlighting}{Verbatim}{commandchars=\\\{\}}
    % Add ',fontsize=\small' for more characters per line
    \newenvironment{Shaded}{}{}
    \newcommand{\KeywordTok}[1]{\textcolor[rgb]{0.00,0.44,0.13}{\textbf{{#1}}}}
    \newcommand{\DataTypeTok}[1]{\textcolor[rgb]{0.56,0.13,0.00}{{#1}}}
    \newcommand{\DecValTok}[1]{\textcolor[rgb]{0.25,0.63,0.44}{{#1}}}
    \newcommand{\BaseNTok}[1]{\textcolor[rgb]{0.25,0.63,0.44}{{#1}}}
    \newcommand{\FloatTok}[1]{\textcolor[rgb]{0.25,0.63,0.44}{{#1}}}
    \newcommand{\CharTok}[1]{\textcolor[rgb]{0.25,0.44,0.63}{{#1}}}
    \newcommand{\StringTok}[1]{\textcolor[rgb]{0.25,0.44,0.63}{{#1}}}
    \newcommand{\CommentTok}[1]{\textcolor[rgb]{0.38,0.63,0.69}{\textit{{#1}}}}
    \newcommand{\OtherTok}[1]{\textcolor[rgb]{0.00,0.44,0.13}{{#1}}}
    \newcommand{\AlertTok}[1]{\textcolor[rgb]{1.00,0.00,0.00}{\textbf{{#1}}}}
    \newcommand{\FunctionTok}[1]{\textcolor[rgb]{0.02,0.16,0.49}{{#1}}}
    \newcommand{\RegionMarkerTok}[1]{{#1}}
    \newcommand{\ErrorTok}[1]{\textcolor[rgb]{1.00,0.00,0.00}{\textbf{{#1}}}}
    \newcommand{\NormalTok}[1]{{#1}}
    
    % Additional commands for more recent versions of Pandoc
    \newcommand{\ConstantTok}[1]{\textcolor[rgb]{0.53,0.00,0.00}{{#1}}}
    \newcommand{\SpecialCharTok}[1]{\textcolor[rgb]{0.25,0.44,0.63}{{#1}}}
    \newcommand{\VerbatimStringTok}[1]{\textcolor[rgb]{0.25,0.44,0.63}{{#1}}}
    \newcommand{\SpecialStringTok}[1]{\textcolor[rgb]{0.73,0.40,0.53}{{#1}}}
    \newcommand{\ImportTok}[1]{{#1}}
    \newcommand{\DocumentationTok}[1]{\textcolor[rgb]{0.73,0.13,0.13}{\textit{{#1}}}}
    \newcommand{\AnnotationTok}[1]{\textcolor[rgb]{0.38,0.63,0.69}{\textbf{\textit{{#1}}}}}
    \newcommand{\CommentVarTok}[1]{\textcolor[rgb]{0.38,0.63,0.69}{\textbf{\textit{{#1}}}}}
    \newcommand{\VariableTok}[1]{\textcolor[rgb]{0.10,0.09,0.49}{{#1}}}
    \newcommand{\ControlFlowTok}[1]{\textcolor[rgb]{0.00,0.44,0.13}{\textbf{{#1}}}}
    \newcommand{\OperatorTok}[1]{\textcolor[rgb]{0.40,0.40,0.40}{{#1}}}
    \newcommand{\BuiltInTok}[1]{{#1}}
    \newcommand{\ExtensionTok}[1]{{#1}}
    \newcommand{\PreprocessorTok}[1]{\textcolor[rgb]{0.74,0.48,0.00}{{#1}}}
    \newcommand{\AttributeTok}[1]{\textcolor[rgb]{0.49,0.56,0.16}{{#1}}}
    \newcommand{\InformationTok}[1]{\textcolor[rgb]{0.38,0.63,0.69}{\textbf{\textit{{#1}}}}}
    \newcommand{\WarningTok}[1]{\textcolor[rgb]{0.38,0.63,0.69}{\textbf{\textit{{#1}}}}}
    
    
    % Define a nice break command that doesn't care if a line doesn't already
    % exist.
    \def\br{\hspace*{\fill} \\* }
    % Math Jax compatibility definitions
    \def\gt{>}
    \def\lt{<}
    \let\Oldtex\TeX
    \let\Oldlatex\LaTeX
    \renewcommand{\TeX}{\textrm{\Oldtex}}
    \renewcommand{\LaTeX}{\textrm{\Oldlatex}}
    % Document parameters
    % Document title
    \title{Nair Homework 4}
    
    
    
    
    

    % Pygments definitions
    
\makeatletter
\def\PY@reset{\let\PY@it=\relax \let\PY@bf=\relax%
    \let\PY@ul=\relax \let\PY@tc=\relax%
    \let\PY@bc=\relax \let\PY@ff=\relax}
\def\PY@tok#1{\csname PY@tok@#1\endcsname}
\def\PY@toks#1+{\ifx\relax#1\empty\else%
    \PY@tok{#1}\expandafter\PY@toks\fi}
\def\PY@do#1{\PY@bc{\PY@tc{\PY@ul{%
    \PY@it{\PY@bf{\PY@ff{#1}}}}}}}
\def\PY#1#2{\PY@reset\PY@toks#1+\relax+\PY@do{#2}}

\expandafter\def\csname PY@tok@w\endcsname{\def\PY@tc##1{\textcolor[rgb]{0.73,0.73,0.73}{##1}}}
\expandafter\def\csname PY@tok@c\endcsname{\let\PY@it=\textit\def\PY@tc##1{\textcolor[rgb]{0.25,0.50,0.50}{##1}}}
\expandafter\def\csname PY@tok@cp\endcsname{\def\PY@tc##1{\textcolor[rgb]{0.74,0.48,0.00}{##1}}}
\expandafter\def\csname PY@tok@k\endcsname{\let\PY@bf=\textbf\def\PY@tc##1{\textcolor[rgb]{0.00,0.50,0.00}{##1}}}
\expandafter\def\csname PY@tok@kp\endcsname{\def\PY@tc##1{\textcolor[rgb]{0.00,0.50,0.00}{##1}}}
\expandafter\def\csname PY@tok@kt\endcsname{\def\PY@tc##1{\textcolor[rgb]{0.69,0.00,0.25}{##1}}}
\expandafter\def\csname PY@tok@o\endcsname{\def\PY@tc##1{\textcolor[rgb]{0.40,0.40,0.40}{##1}}}
\expandafter\def\csname PY@tok@ow\endcsname{\let\PY@bf=\textbf\def\PY@tc##1{\textcolor[rgb]{0.67,0.13,1.00}{##1}}}
\expandafter\def\csname PY@tok@nb\endcsname{\def\PY@tc##1{\textcolor[rgb]{0.00,0.50,0.00}{##1}}}
\expandafter\def\csname PY@tok@nf\endcsname{\def\PY@tc##1{\textcolor[rgb]{0.00,0.00,1.00}{##1}}}
\expandafter\def\csname PY@tok@nc\endcsname{\let\PY@bf=\textbf\def\PY@tc##1{\textcolor[rgb]{0.00,0.00,1.00}{##1}}}
\expandafter\def\csname PY@tok@nn\endcsname{\let\PY@bf=\textbf\def\PY@tc##1{\textcolor[rgb]{0.00,0.00,1.00}{##1}}}
\expandafter\def\csname PY@tok@ne\endcsname{\let\PY@bf=\textbf\def\PY@tc##1{\textcolor[rgb]{0.82,0.25,0.23}{##1}}}
\expandafter\def\csname PY@tok@nv\endcsname{\def\PY@tc##1{\textcolor[rgb]{0.10,0.09,0.49}{##1}}}
\expandafter\def\csname PY@tok@no\endcsname{\def\PY@tc##1{\textcolor[rgb]{0.53,0.00,0.00}{##1}}}
\expandafter\def\csname PY@tok@nl\endcsname{\def\PY@tc##1{\textcolor[rgb]{0.63,0.63,0.00}{##1}}}
\expandafter\def\csname PY@tok@ni\endcsname{\let\PY@bf=\textbf\def\PY@tc##1{\textcolor[rgb]{0.60,0.60,0.60}{##1}}}
\expandafter\def\csname PY@tok@na\endcsname{\def\PY@tc##1{\textcolor[rgb]{0.49,0.56,0.16}{##1}}}
\expandafter\def\csname PY@tok@nt\endcsname{\let\PY@bf=\textbf\def\PY@tc##1{\textcolor[rgb]{0.00,0.50,0.00}{##1}}}
\expandafter\def\csname PY@tok@nd\endcsname{\def\PY@tc##1{\textcolor[rgb]{0.67,0.13,1.00}{##1}}}
\expandafter\def\csname PY@tok@s\endcsname{\def\PY@tc##1{\textcolor[rgb]{0.73,0.13,0.13}{##1}}}
\expandafter\def\csname PY@tok@sd\endcsname{\let\PY@it=\textit\def\PY@tc##1{\textcolor[rgb]{0.73,0.13,0.13}{##1}}}
\expandafter\def\csname PY@tok@si\endcsname{\let\PY@bf=\textbf\def\PY@tc##1{\textcolor[rgb]{0.73,0.40,0.53}{##1}}}
\expandafter\def\csname PY@tok@se\endcsname{\let\PY@bf=\textbf\def\PY@tc##1{\textcolor[rgb]{0.73,0.40,0.13}{##1}}}
\expandafter\def\csname PY@tok@sr\endcsname{\def\PY@tc##1{\textcolor[rgb]{0.73,0.40,0.53}{##1}}}
\expandafter\def\csname PY@tok@ss\endcsname{\def\PY@tc##1{\textcolor[rgb]{0.10,0.09,0.49}{##1}}}
\expandafter\def\csname PY@tok@sx\endcsname{\def\PY@tc##1{\textcolor[rgb]{0.00,0.50,0.00}{##1}}}
\expandafter\def\csname PY@tok@m\endcsname{\def\PY@tc##1{\textcolor[rgb]{0.40,0.40,0.40}{##1}}}
\expandafter\def\csname PY@tok@gh\endcsname{\let\PY@bf=\textbf\def\PY@tc##1{\textcolor[rgb]{0.00,0.00,0.50}{##1}}}
\expandafter\def\csname PY@tok@gu\endcsname{\let\PY@bf=\textbf\def\PY@tc##1{\textcolor[rgb]{0.50,0.00,0.50}{##1}}}
\expandafter\def\csname PY@tok@gd\endcsname{\def\PY@tc##1{\textcolor[rgb]{0.63,0.00,0.00}{##1}}}
\expandafter\def\csname PY@tok@gi\endcsname{\def\PY@tc##1{\textcolor[rgb]{0.00,0.63,0.00}{##1}}}
\expandafter\def\csname PY@tok@gr\endcsname{\def\PY@tc##1{\textcolor[rgb]{1.00,0.00,0.00}{##1}}}
\expandafter\def\csname PY@tok@ge\endcsname{\let\PY@it=\textit}
\expandafter\def\csname PY@tok@gs\endcsname{\let\PY@bf=\textbf}
\expandafter\def\csname PY@tok@gp\endcsname{\let\PY@bf=\textbf\def\PY@tc##1{\textcolor[rgb]{0.00,0.00,0.50}{##1}}}
\expandafter\def\csname PY@tok@go\endcsname{\def\PY@tc##1{\textcolor[rgb]{0.53,0.53,0.53}{##1}}}
\expandafter\def\csname PY@tok@gt\endcsname{\def\PY@tc##1{\textcolor[rgb]{0.00,0.27,0.87}{##1}}}
\expandafter\def\csname PY@tok@err\endcsname{\def\PY@bc##1{\setlength{\fboxsep}{0pt}\fcolorbox[rgb]{1.00,0.00,0.00}{1,1,1}{\strut ##1}}}
\expandafter\def\csname PY@tok@kc\endcsname{\let\PY@bf=\textbf\def\PY@tc##1{\textcolor[rgb]{0.00,0.50,0.00}{##1}}}
\expandafter\def\csname PY@tok@kd\endcsname{\let\PY@bf=\textbf\def\PY@tc##1{\textcolor[rgb]{0.00,0.50,0.00}{##1}}}
\expandafter\def\csname PY@tok@kn\endcsname{\let\PY@bf=\textbf\def\PY@tc##1{\textcolor[rgb]{0.00,0.50,0.00}{##1}}}
\expandafter\def\csname PY@tok@kr\endcsname{\let\PY@bf=\textbf\def\PY@tc##1{\textcolor[rgb]{0.00,0.50,0.00}{##1}}}
\expandafter\def\csname PY@tok@bp\endcsname{\def\PY@tc##1{\textcolor[rgb]{0.00,0.50,0.00}{##1}}}
\expandafter\def\csname PY@tok@fm\endcsname{\def\PY@tc##1{\textcolor[rgb]{0.00,0.00,1.00}{##1}}}
\expandafter\def\csname PY@tok@vc\endcsname{\def\PY@tc##1{\textcolor[rgb]{0.10,0.09,0.49}{##1}}}
\expandafter\def\csname PY@tok@vg\endcsname{\def\PY@tc##1{\textcolor[rgb]{0.10,0.09,0.49}{##1}}}
\expandafter\def\csname PY@tok@vi\endcsname{\def\PY@tc##1{\textcolor[rgb]{0.10,0.09,0.49}{##1}}}
\expandafter\def\csname PY@tok@vm\endcsname{\def\PY@tc##1{\textcolor[rgb]{0.10,0.09,0.49}{##1}}}
\expandafter\def\csname PY@tok@sa\endcsname{\def\PY@tc##1{\textcolor[rgb]{0.73,0.13,0.13}{##1}}}
\expandafter\def\csname PY@tok@sb\endcsname{\def\PY@tc##1{\textcolor[rgb]{0.73,0.13,0.13}{##1}}}
\expandafter\def\csname PY@tok@sc\endcsname{\def\PY@tc##1{\textcolor[rgb]{0.73,0.13,0.13}{##1}}}
\expandafter\def\csname PY@tok@dl\endcsname{\def\PY@tc##1{\textcolor[rgb]{0.73,0.13,0.13}{##1}}}
\expandafter\def\csname PY@tok@s2\endcsname{\def\PY@tc##1{\textcolor[rgb]{0.73,0.13,0.13}{##1}}}
\expandafter\def\csname PY@tok@sh\endcsname{\def\PY@tc##1{\textcolor[rgb]{0.73,0.13,0.13}{##1}}}
\expandafter\def\csname PY@tok@s1\endcsname{\def\PY@tc##1{\textcolor[rgb]{0.73,0.13,0.13}{##1}}}
\expandafter\def\csname PY@tok@mb\endcsname{\def\PY@tc##1{\textcolor[rgb]{0.40,0.40,0.40}{##1}}}
\expandafter\def\csname PY@tok@mf\endcsname{\def\PY@tc##1{\textcolor[rgb]{0.40,0.40,0.40}{##1}}}
\expandafter\def\csname PY@tok@mh\endcsname{\def\PY@tc##1{\textcolor[rgb]{0.40,0.40,0.40}{##1}}}
\expandafter\def\csname PY@tok@mi\endcsname{\def\PY@tc##1{\textcolor[rgb]{0.40,0.40,0.40}{##1}}}
\expandafter\def\csname PY@tok@il\endcsname{\def\PY@tc##1{\textcolor[rgb]{0.40,0.40,0.40}{##1}}}
\expandafter\def\csname PY@tok@mo\endcsname{\def\PY@tc##1{\textcolor[rgb]{0.40,0.40,0.40}{##1}}}
\expandafter\def\csname PY@tok@ch\endcsname{\let\PY@it=\textit\def\PY@tc##1{\textcolor[rgb]{0.25,0.50,0.50}{##1}}}
\expandafter\def\csname PY@tok@cm\endcsname{\let\PY@it=\textit\def\PY@tc##1{\textcolor[rgb]{0.25,0.50,0.50}{##1}}}
\expandafter\def\csname PY@tok@cpf\endcsname{\let\PY@it=\textit\def\PY@tc##1{\textcolor[rgb]{0.25,0.50,0.50}{##1}}}
\expandafter\def\csname PY@tok@c1\endcsname{\let\PY@it=\textit\def\PY@tc##1{\textcolor[rgb]{0.25,0.50,0.50}{##1}}}
\expandafter\def\csname PY@tok@cs\endcsname{\let\PY@it=\textit\def\PY@tc##1{\textcolor[rgb]{0.25,0.50,0.50}{##1}}}

\def\PYZbs{\char`\\}
\def\PYZus{\char`\_}
\def\PYZob{\char`\{}
\def\PYZcb{\char`\}}
\def\PYZca{\char`\^}
\def\PYZam{\char`\&}
\def\PYZlt{\char`\<}
\def\PYZgt{\char`\>}
\def\PYZsh{\char`\#}
\def\PYZpc{\char`\%}
\def\PYZdl{\char`\$}
\def\PYZhy{\char`\-}
\def\PYZsq{\char`\'}
\def\PYZdq{\char`\"}
\def\PYZti{\char`\~}
% for compatibility with earlier versions
\def\PYZat{@}
\def\PYZlb{[}
\def\PYZrb{]}
\makeatother


    % Exact colors from NB
    \definecolor{incolor}{rgb}{0.0, 0.0, 0.5}
    \definecolor{outcolor}{rgb}{0.545, 0.0, 0.0}



    
    % Prevent overflowing lines due to hard-to-break entities
    \sloppy 
    % Setup hyperref package
    \hypersetup{
      breaklinks=true,  % so long urls are correctly broken across lines
      colorlinks=true,
      urlcolor=urlcolor,
      linkcolor=linkcolor,
      citecolor=citecolor,
      }
    % Slightly bigger margins than the latex defaults
    
    \geometry{verbose,tmargin=1in,bmargin=1in,lmargin=1in,rmargin=1in}
    
    

    \begin{document}
    
    
    \maketitle
    
    

    
    \begin{Verbatim}[commandchars=\\\{\}]
{\color{incolor}In [{\color{incolor}1}]:} \PY{o}{\PYZpc{}}\PY{k}{precision} \PYZpc{}g
        \PY{o}{\PYZpc{}}\PY{k}{matplotlib} inline
        \PY{o}{\PYZpc{}}\PY{k}{config} InlineBackend.figure\PYZus{}format = \PYZsq{}retina\PYZsq{}
\end{Verbatim}

    \begin{Verbatim}[commandchars=\\\{\}]
{\color{incolor}In [{\color{incolor}2}]:} \PY{k+kn}{from} \PY{n+nn}{math} \PY{k}{import} \PY{n}{sqrt}\PY{p}{,} \PY{n}{pi}\PY{p}{,} \PY{n}{sin}\PY{p}{,} \PY{n}{cos}\PY{p}{,} \PY{n}{exp}\PY{p}{,} \PY{n}{inf}\PY{p}{,} \PY{n}{factorial}\PY{p}{,} \PY{n}{tan}
        \PY{c+c1}{\PYZsh{}from cmath import exp as cexp}
        \PY{k+kn}{import} \PY{n+nn}{numpy} \PY{k}{as} \PY{n+nn}{np}
        \PY{k+kn}{from} \PY{n+nn}{numpy} \PY{k}{import} \PY{n}{linalg} \PY{k}{as} \PY{n}{LA}
        \PY{k+kn}{from} \PY{n+nn}{scipy} \PY{k}{import} \PY{n}{constants} \PY{k}{as} \PY{n}{C}
        \PY{k+kn}{from} \PY{n+nn}{scipy} \PY{k}{import} \PY{n}{integrate}
        \PY{k+kn}{import} \PY{n+nn}{matplotlib}\PY{n+nn}{.}\PY{n+nn}{pyplot} \PY{k}{as} \PY{n+nn}{plt}
        
        \PY{c+c1}{\PYZsh{}from IPython.display import set\PYZus{}matplotlib\PYZus{}formats}
        \PY{c+c1}{\PYZsh{}set\PYZus{}matplotlib\PYZus{}formats(\PYZsq{}png\PYZsq{}, \PYZsq{}pdf\PYZsq{})}
\end{Verbatim}

    \section{CP 5.21 Electric field of a charge
distribution}\label{cp-5.21-electric-field-of-a-charge-distribution}

Electric potential due to a point charge is given by
\(\phi = \frac{q}{4\pi\epsilon_0r}.\)

The electric field can be found from this potential by taking partial
derivatives, such that

\[\mathbf{E} = -\nabla\phi\]

I will define two partial derivative functions to find the field at
respective points on the grid. The partial derivatives will be defined
as according to the central difference

\[\frac{\partial f(x,y)}{\partial x} = \frac{f(x+\frac{h}{2}, y) - f(x-\frac{h}{2}, y)}{h}\]

\[\frac{\partial f(x,y)}{\partial y} = \frac{f(x, y+\frac{h}{2}) - f(x, y-\frac{h}{2})}{h}\]

    \begin{Verbatim}[commandchars=\\\{\}]
{\color{incolor}In [{\color{incolor}3}]:} \PY{n}{q1} \PY{o}{=} \PY{l+m+mi}{1} \PY{c+c1}{\PYZsh{}charge 1 in coulombs}
        \PY{n}{q2} \PY{o}{=} \PY{o}{\PYZhy{}}\PY{l+m+mi}{1} \PY{c+c1}{\PYZsh{}charge 2 in coulombs}
        \PY{n}{e0} \PY{o}{=} \PY{n}{C}\PY{o}{.}\PY{n}{epsilon\PYZus{}0}
        
        \PY{k}{def} \PY{n+nf}{potential}\PY{p}{(}\PY{n}{q}\PY{p}{,} \PY{n}{r}\PY{p}{)}\PY{p}{:}
            \PY{k}{return} \PY{n}{q} \PY{o}{/} \PY{p}{(}\PY{l+m+mi}{4}\PY{o}{*} \PY{n}{pi}\PY{o}{*} \PY{n}{e0}\PY{o}{*} \PY{n}{r}\PY{p}{)}
        
        \PY{n}{side} \PY{o}{=} \PY{l+m+mf}{1.0} \PY{c+c1}{\PYZsh{}side length of the square (m)}
        \PY{n}{points} \PY{o}{=} \PY{l+m+mi}{100} \PY{c+c1}{\PYZsh{}number of grid points along each side}
        \PY{n}{spacing} \PY{o}{=} \PY{n}{side} \PY{o}{/} \PY{n}{points} \PY{c+c1}{\PYZsh{}spacing of points (m)}
        \PY{n}{separation} \PY{o}{=} \PY{l+m+mf}{0.1} \PY{c+c1}{\PYZsh{}separation of charges (m)}
        
        \PY{n}{x01} \PY{o}{=} \PY{n}{side}\PY{o}{/}\PY{l+m+mi}{2} \PY{o}{+} \PY{n}{separation}\PY{o}{/}\PY{l+m+mi}{2}
        \PY{n}{y01} \PY{o}{=} \PY{n}{side}\PY{o}{/}\PY{l+m+mi}{2} \PY{o}{+} \PY{n}{separation}\PY{o}{/}\PY{l+m+mi}{2}
        \PY{n}{x02} \PY{o}{=} \PY{n}{side}\PY{o}{/}\PY{l+m+mi}{2} \PY{o}{\PYZhy{}} \PY{n}{separation}\PY{o}{/}\PY{l+m+mi}{2}
        \PY{n}{y02} \PY{o}{=} \PY{n}{side}\PY{o}{/}\PY{l+m+mi}{2} \PY{o}{\PYZhy{}} \PY{n}{separation}\PY{o}{/}\PY{l+m+mi}{2}
\end{Verbatim}

    \begin{Verbatim}[commandchars=\\\{\}]
{\color{incolor}In [{\color{incolor}4}]:} \PY{c+c1}{\PYZsh{}np.seterr(divide=\PYZsq{}ignore\PYZsq{}, invalid=\PYZsq{}ignore\PYZsq{})}
        \PY{n}{phi} \PY{o}{=} \PY{n}{np}\PY{o}{.}\PY{n}{empty}\PY{p}{(}\PY{p}{[}\PY{n}{points}\PY{p}{,}\PY{n}{points}\PY{p}{]}\PY{p}{,}\PY{n+nb}{float}\PY{p}{)}
        \PY{c+c1}{\PYZsh{}calculate the values in the array}
        \PY{k}{for} \PY{n}{i} \PY{o+ow}{in} \PY{n+nb}{range}\PY{p}{(}\PY{n}{points}\PY{p}{)}\PY{p}{:}
            \PY{n}{y} \PY{o}{=} \PY{n}{spacing} \PY{o}{*} \PY{n}{i}
            \PY{k}{for} \PY{n}{j} \PY{o+ow}{in} \PY{n+nb}{range}\PY{p}{(}\PY{n}{points}\PY{p}{)}\PY{p}{:}
                \PY{n}{x} \PY{o}{=} \PY{n}{spacing} \PY{o}{*} \PY{n}{j}
                \PY{n}{r1} \PY{o}{=} \PY{n}{sqrt}\PY{p}{(}\PY{p}{(}\PY{n}{x}\PY{o}{\PYZhy{}}\PY{n}{x01}\PY{p}{)}\PY{o}{*}\PY{o}{*}\PY{l+m+mi}{2} \PY{o}{+} \PY{p}{(}\PY{n}{y}\PY{o}{\PYZhy{}}\PY{n}{y01}\PY{p}{)}\PY{o}{*}\PY{o}{*}\PY{l+m+mi}{2}\PY{p}{)}
                \PY{n}{r2} \PY{o}{=} \PY{n}{sqrt}\PY{p}{(}\PY{p}{(}\PY{n}{x}\PY{o}{\PYZhy{}}\PY{n}{x02}\PY{p}{)}\PY{o}{*}\PY{o}{*}\PY{l+m+mi}{2} \PY{o}{+} \PY{p}{(}\PY{n}{y}\PY{o}{\PYZhy{}}\PY{n}{y02}\PY{p}{)}\PY{o}{*}\PY{o}{*}\PY{l+m+mi}{2}\PY{p}{)}
                \PY{k}{if} \PY{n}{r1} \PY{o}{!=} \PY{l+m+mi}{0} \PY{o+ow}{and} \PY{n}{r2} \PY{o}{!=} \PY{l+m+mi}{0}\PY{p}{:}
                    \PY{n}{phi}\PY{p}{[}\PY{n}{i}\PY{p}{,}\PY{n}{j}\PY{p}{]} \PY{o}{=} \PY{n}{potential}\PY{p}{(}\PY{n}{q1}\PY{p}{,} \PY{n}{r1}\PY{p}{)} \PY{o}{+} \PY{n}{potential}\PY{p}{(}\PY{n}{q2}\PY{p}{,} \PY{n}{r2}\PY{p}{)}
        
        \PY{n}{fig1}\PY{p}{,} \PY{n}{ax1} \PY{o}{=} \PY{n}{plt}\PY{o}{.}\PY{n}{subplots}\PY{p}{(}\PY{l+m+mi}{1}\PY{p}{,} \PY{l+m+mi}{1}\PY{p}{,} \PY{n}{figsize} \PY{o}{=} \PY{p}{(}\PY{l+m+mi}{5}\PY{p}{,} \PY{l+m+mi}{5}\PY{p}{)}\PY{p}{)}
        
        \PY{c+c1}{\PYZsh{}increases readability of plot}
        \PY{n}{ax1}\PY{o}{.}\PY{n}{set\PYZus{}title}\PY{p}{(}\PY{l+s+s2}{\PYZdq{}}\PY{l+s+s2}{Electric Potential of Two Point Charges}\PY{l+s+s2}{\PYZdq{}}\PY{p}{)}
        \PY{n}{ax1}\PY{o}{.}\PY{n}{set\PYZus{}xlabel}\PY{p}{(}\PY{l+s+s2}{\PYZdq{}}\PY{l+s+s2}{x (\PYZdl{}m\PYZdl{})}\PY{l+s+s2}{\PYZdq{}}\PY{p}{)}
        \PY{n}{ax1}\PY{o}{.}\PY{n}{set\PYZus{}ylabel}\PY{p}{(}\PY{l+s+s2}{\PYZdq{}}\PY{l+s+s2}{y (\PYZdl{}m\PYZdl{})}\PY{l+s+s2}{\PYZdq{}}\PY{p}{)}
        \PY{n}{ax1}\PY{o}{.}\PY{n}{set\PYZus{}xticks}\PY{p}{(}\PY{p}{[}\PY{o}{\PYZhy{}}\PY{l+m+mf}{0.5}\PY{p}{,} \PY{l+m+mi}{0}\PY{p}{,} \PY{l+m+mf}{0.5}\PY{p}{]}\PY{p}{)}
        \PY{n}{ax1}\PY{o}{.}\PY{n}{set\PYZus{}yticks}\PY{p}{(}\PY{p}{[}\PY{o}{\PYZhy{}}\PY{l+m+mf}{0.5}\PY{p}{,} \PY{l+m+mi}{0}\PY{p}{,} \PY{l+m+mf}{0.5}\PY{p}{]}\PY{p}{)}
        
        \PY{n}{ax1}\PY{o}{.}\PY{n}{imshow}\PY{p}{(}\PY{n}{phi}\PY{p}{,}\PY{n}{origin}\PY{o}{=}\PY{l+s+s2}{\PYZdq{}}\PY{l+s+s2}{lower}\PY{l+s+s2}{\PYZdq{}}\PY{p}{,}\PY{n}{extent}\PY{o}{=}\PYZbs{}
                  \PY{p}{[}\PY{o}{\PYZhy{}}\PY{l+m+mf}{0.5}\PY{p}{,}\PY{l+m+mf}{0.5}\PY{p}{,}\PY{o}{\PYZhy{}}\PY{l+m+mf}{0.5}\PY{p}{,}\PY{l+m+mf}{0.5}\PY{p}{]}\PY{p}{,}\PYZbs{}
                  \PY{n}{cmap}\PY{o}{=}\PY{l+s+s2}{\PYZdq{}}\PY{l+s+s2}{seismic}\PY{l+s+s2}{\PYZdq{}}\PY{p}{,}\PY{n}{vmax}\PY{o}{=}\PY{l+m+mf}{7.5e11}\PY{p}{)}
\end{Verbatim}

\begin{Verbatim}[commandchars=\\\{\}]
{\color{outcolor}Out[{\color{outcolor}4}]:} <matplotlib.image.AxesImage at 0x10f42ec18>
\end{Verbatim}
            
    \begin{center}
    \adjustimage{max size={0.9\linewidth}{0.9\paperheight}}{output_4_1.png}
    \end{center}
    { \hspace*{\fill} \\}
    
    Because potential was only a function of distance from each point
charge, it was easier to define it in terms of \(r.\) However, because
we need to take the partial derivatives with respect to \(x\) and \(y\)
separately, it is easier to redefine this potential function and then
define partial derivative functions.

    \begin{Verbatim}[commandchars=\\\{\}]
{\color{incolor}In [{\color{incolor}5}]:} \PY{k}{def} \PY{n+nf}{potential}\PY{p}{(}\PY{n}{q}\PY{p}{,} \PY{n}{x}\PY{p}{,} \PY{n}{y}\PY{p}{)}\PY{p}{:}
            \PY{n}{r} \PY{o}{=} \PY{n}{sqrt}\PY{p}{(}\PY{n}{x}\PY{o}{*}\PY{o}{*}\PY{l+m+mi}{2} \PY{o}{+} \PY{n}{y}\PY{o}{*}\PY{o}{*}\PY{l+m+mi}{2}\PY{p}{)}
            \PY{k}{return} \PY{n}{q} \PY{o}{/} \PY{p}{(}\PY{l+m+mi}{4}\PY{o}{*} \PY{n}{pi}\PY{o}{*} \PY{n}{e0}\PY{o}{*} \PY{n}{r}\PY{p}{)}
        
        \PY{l+s+sd}{\PYZdq{}\PYZdq{}\PYZdq{}Creates partial derivative functions to}
        \PY{l+s+sd}{    find the field at points on the grid\PYZdq{}\PYZdq{}\PYZdq{}}
        \PY{k}{def} \PY{n+nf}{Ex}\PY{p}{(}\PY{n}{q}\PY{p}{,} \PY{n}{x}\PY{p}{,} \PY{n}{y}\PY{p}{,} \PY{n}{h}\PY{p}{)}\PY{p}{:}
        
            \PY{n}{partial} \PY{o}{=} \PY{p}{(}\PY{n}{potential}\PY{p}{(}\PY{n}{q}\PY{p}{,}\PY{n}{x}\PY{o}{+}\PY{n}{h}\PY{o}{/}\PY{l+m+mi}{2}\PY{p}{,}\PY{n}{y}\PY{p}{)} \PY{o}{\PYZhy{}} \PY{n}{potential}\PY{p}{(}\PY{n}{q}\PY{p}{,}\PY{n}{x}\PY{o}{\PYZhy{}}\PY{n}{h}\PY{o}{/}\PY{l+m+mi}{2}\PY{p}{,}\PY{n}{y}\PY{p}{)}\PY{p}{)} \PY{o}{/} \PY{n}{h}
            \PY{k}{return} \PY{o}{\PYZhy{}}\PY{n}{partial}
        
        \PY{k}{def} \PY{n+nf}{Ey}\PY{p}{(}\PY{n}{q}\PY{p}{,} \PY{n}{x}\PY{p}{,} \PY{n}{y}\PY{p}{,} \PY{n}{h}\PY{p}{)}\PY{p}{:}
        
            \PY{n}{partial} \PY{o}{=} \PY{p}{(}\PY{n}{potential}\PY{p}{(}\PY{n}{q}\PY{p}{,}\PY{n}{x}\PY{p}{,}\PY{n}{y}\PY{o}{+}\PY{n}{h}\PY{o}{/}\PY{l+m+mi}{2}\PY{p}{)} \PY{o}{\PYZhy{}} \PY{n}{potential}\PY{p}{(}\PY{n}{q}\PY{p}{,}\PY{n}{x}\PY{p}{,}\PY{n}{y}\PY{o}{\PYZhy{}}\PY{n}{h}\PY{o}{/}\PY{l+m+mi}{2}\PY{p}{)}\PY{p}{)} \PY{o}{/} \PY{n}{h}
            \PY{k}{return} \PY{o}{\PYZhy{}}\PY{n}{partial}
\end{Verbatim}

    \begin{Verbatim}[commandchars=\\\{\}]
{\color{incolor}In [{\color{incolor}6}]:} \PY{n}{h} \PY{o}{=} \PY{l+m+mf}{0.001}
        \PY{n}{side} \PY{o}{=} \PY{l+m+mf}{1.0} \PY{c+c1}{\PYZsh{}side length of the square (m)}
        \PY{n}{points} \PY{o}{=} \PY{l+m+mi}{20} \PY{c+c1}{\PYZsh{}number of grid points along each side}
        \PY{n}{spacing} \PY{o}{=} \PY{n}{side} \PY{o}{/} \PY{n}{points} \PY{c+c1}{\PYZsh{}spacing of points (m)}
        
        \PY{n}{x} \PY{o}{=} \PY{n}{np}\PY{o}{.}\PY{n}{linspace}\PY{p}{(}\PY{o}{\PYZhy{}}\PY{l+m+mf}{0.5}\PY{p}{,}\PY{l+m+mf}{0.5}\PY{p}{,}\PY{n}{points}\PY{p}{)}
        \PY{n}{y} \PY{o}{=} \PY{n}{np}\PY{o}{.}\PY{n}{linspace}\PY{p}{(}\PY{o}{\PYZhy{}}\PY{l+m+mf}{0.5}\PY{p}{,}\PY{l+m+mf}{0.5}\PY{p}{,}\PY{n}{points}\PY{p}{)}
        \PY{n}{E\PYZus{}x}\PY{p}{,} \PY{n}{E\PYZus{}y} \PY{o}{=} \PY{n}{np}\PY{o}{.}\PY{n}{meshgrid}\PY{p}{(}\PY{n}{x}\PY{p}{,}\PY{n}{y}\PY{p}{)}
        
        \PY{c+c1}{\PYZsh{}calculate the values in the array}
        \PY{k}{for} \PY{n}{i} \PY{o+ow}{in} \PY{n+nb}{range}\PY{p}{(}\PY{n}{points}\PY{p}{)}\PY{p}{:}
            \PY{n}{y} \PY{o}{=} \PY{n}{spacing} \PY{o}{*} \PY{n}{i}
            \PY{k}{for} \PY{n}{j} \PY{o+ow}{in} \PY{n+nb}{range}\PY{p}{(}\PY{n}{points}\PY{p}{)}\PY{p}{:}
                \PY{n}{x} \PY{o}{=} \PY{n}{spacing} \PY{o}{*} \PY{n}{j}
                
                \PY{n}{dx1} \PY{o}{=} \PY{n}{x} \PY{o}{\PYZhy{}} \PY{n}{x01}
                \PY{n}{dy1} \PY{o}{=} \PY{n}{y} \PY{o}{\PYZhy{}} \PY{n}{y01}
                \PY{n}{dx2} \PY{o}{=} \PY{n}{x} \PY{o}{\PYZhy{}} \PY{n}{x02}
                \PY{n}{dy2} \PY{o}{=} \PY{n}{y} \PY{o}{\PYZhy{}} \PY{n}{y02}
                
                \PY{n}{E\PYZus{}x}\PY{p}{[}\PY{n}{i}\PY{p}{,}\PY{n}{j}\PY{p}{]} \PY{o}{=} \PY{n}{Ex}\PY{p}{(}\PY{n}{q1}\PY{p}{,} \PY{n}{dx1}\PY{p}{,} \PY{n}{dy1}\PY{p}{,} \PY{n}{h}\PY{p}{)} \PY{o}{+} \PY{n}{Ex}\PY{p}{(}\PY{n}{q2}\PY{p}{,} \PY{n}{dx2}\PY{p}{,} \PY{n}{dy2}\PY{p}{,} \PY{n}{h}\PY{p}{)}
                \PY{n}{E\PYZus{}y}\PY{p}{[}\PY{n}{i}\PY{p}{,}\PY{n}{j}\PY{p}{]} \PY{o}{=} \PY{n}{Ey}\PY{p}{(}\PY{n}{q1}\PY{p}{,} \PY{n}{dx1}\PY{p}{,} \PY{n}{dy1}\PY{p}{,} \PY{n}{h}\PY{p}{)} \PY{o}{+} \PY{n}{Ey}\PY{p}{(}\PY{n}{q2}\PY{p}{,} \PY{n}{dx2}\PY{p}{,} \PY{n}{dy2}\PY{p}{,} \PY{n}{h}\PY{p}{)}
        
        \PY{c+c1}{\PYZsh{}E\PYZus{}x /= np.sqrt(E\PYZus{}x**2 + E\PYZus{}y**2)}
        \PY{c+c1}{\PYZsh{}E\PYZus{}y /= np.sqrt(E\PYZus{}x**2 + E\PYZus{}y**2)        }
                
        \PY{n}{fig2}\PY{p}{,} \PY{n}{ax2} \PY{o}{=} \PY{n}{plt}\PY{o}{.}\PY{n}{subplots}\PY{p}{(}\PY{l+m+mi}{1}\PY{p}{,} \PY{l+m+mi}{1}\PY{p}{,} \PY{n}{figsize} \PY{o}{=} \PY{p}{(}\PY{l+m+mi}{5}\PY{p}{,} \PY{l+m+mi}{5}\PY{p}{)}\PY{p}{)}
        \PY{n}{ax2}\PY{o}{.}\PY{n}{set\PYZus{}xlabel}\PY{p}{(}\PY{l+s+s2}{\PYZdq{}}\PY{l+s+s2}{x (m)}\PY{l+s+s2}{\PYZdq{}}\PY{p}{)}
        \PY{n}{ax2}\PY{o}{.}\PY{n}{set\PYZus{}ylabel}\PY{p}{(}\PY{l+s+s2}{\PYZdq{}}\PY{l+s+s2}{y (m)}\PY{l+s+s2}{\PYZdq{}}\PY{p}{)}
        \PY{n}{ax2}\PY{o}{.}\PY{n}{set\PYZus{}title}\PY{p}{(}\PY{l+s+s2}{\PYZdq{}}\PY{l+s+s2}{Field of Two Opposite Point Charges}\PY{l+s+se}{\PYZbs{}}
        \PY{l+s+s2}{ (arrow length\PYZdl{}}\PY{l+s+s2}{\PYZbs{}}\PY{l+s+s2}{propto |E|\PYZdl{})}\PY{l+s+s2}{\PYZdq{}}\PY{p}{)}
        \PY{n}{ax1}\PY{o}{.}\PY{n}{set\PYZus{}xticks}\PY{p}{(}\PY{p}{[}\PY{p}{]}\PY{p}{)}
        \PY{n}{ax1}\PY{o}{.}\PY{n}{set\PYZus{}yticks}\PY{p}{(}\PY{p}{[}\PY{o}{\PYZhy{}}\PY{l+m+mf}{0.5}\PY{p}{,} \PY{l+m+mi}{0}\PY{p}{,} \PY{l+m+mf}{0.5}\PY{p}{]}\PY{p}{)}
        \PY{n}{plt}\PY{o}{.}\PY{n}{quiver}\PY{p}{(}\PY{n}{E\PYZus{}x}\PY{p}{,} \PY{n}{E\PYZus{}y}\PY{p}{,}\PY{n}{headwidth}\PY{o}{=}\PY{l+m+mi}{3}\PY{p}{)}\PY{c+c1}{\PYZsh{}, scale=1)}
\end{Verbatim}

\begin{Verbatim}[commandchars=\\\{\}]
{\color{outcolor}Out[{\color{outcolor}6}]:} <matplotlib.quiver.Quiver at 0x10762ec50>
\end{Verbatim}
            
    \begin{center}
    \adjustimage{max size={0.9\linewidth}{0.9\paperheight}}{output_7_1.png}
    \end{center}
    { \hspace*{\fill} \\}
    
    \begin{Verbatim}[commandchars=\\\{\}]
{\color{incolor}In [{\color{incolor}7}]:} \PY{n}{points} \PY{o}{=} \PY{l+m+mi}{10} \PY{c+c1}{\PYZsh{}number of grid points along each side}
        \PY{n}{spacing} \PY{o}{=} \PY{n}{side} \PY{o}{/} \PY{n}{points} \PY{c+c1}{\PYZsh{}spacing of points (m)}
        
        \PY{n}{E\PYZus{}x} \PY{o}{/}\PY{o}{=} \PY{n}{np}\PY{o}{.}\PY{n}{sqrt}\PY{p}{(}\PY{n}{E\PYZus{}x}\PY{o}{*}\PY{o}{*}\PY{l+m+mi}{2} \PY{o}{+} \PY{n}{E\PYZus{}y}\PY{o}{*}\PY{o}{*}\PY{l+m+mi}{2}\PY{p}{)}
        \PY{n}{E\PYZus{}y} \PY{o}{/}\PY{o}{=} \PY{n}{np}\PY{o}{.}\PY{n}{sqrt}\PY{p}{(}\PY{n}{E\PYZus{}x}\PY{o}{*}\PY{o}{*}\PY{l+m+mi}{2} \PY{o}{+} \PY{n}{E\PYZus{}y}\PY{o}{*}\PY{o}{*}\PY{l+m+mi}{2}\PY{p}{)}        
                
        \PY{n}{fig2}\PY{p}{,} \PY{n}{ax2} \PY{o}{=} \PY{n}{plt}\PY{o}{.}\PY{n}{subplots}\PY{p}{(}\PY{l+m+mi}{1}\PY{p}{,} \PY{l+m+mi}{1}\PY{p}{,} \PY{n}{figsize} \PY{o}{=} \PY{p}{(}\PY{l+m+mi}{5}\PY{p}{,} \PY{l+m+mi}{5}\PY{p}{)}\PY{p}{)}
        \PY{n}{ax2}\PY{o}{.}\PY{n}{set\PYZus{}xlabel}\PY{p}{(}\PY{l+s+s2}{\PYZdq{}}\PY{l+s+s2}{x (m)}\PY{l+s+s2}{\PYZdq{}}\PY{p}{)}
        \PY{n}{ax2}\PY{o}{.}\PY{n}{set\PYZus{}ylabel}\PY{p}{(}\PY{l+s+s2}{\PYZdq{}}\PY{l+s+s2}{y (m)}\PY{l+s+s2}{\PYZdq{}}\PY{p}{)}
        \PY{n}{ax2}\PY{o}{.}\PY{n}{set\PYZus{}title}\PY{p}{(}\PY{l+s+s2}{\PYZdq{}}\PY{l+s+s2}{Field of Two Opposite Point Charges}\PY{l+s+se}{\PYZbs{}}
        \PY{l+s+s2}{ (normalized field arrows)}\PY{l+s+s2}{\PYZdq{}}\PY{p}{)}
        \PY{n}{ax1}\PY{o}{.}\PY{n}{set\PYZus{}xticks}\PY{p}{(}\PY{p}{[}\PY{o}{\PYZhy{}}\PY{l+m+mf}{0.5}\PY{p}{,} \PY{l+m+mi}{0}\PY{p}{,} \PY{l+m+mf}{0.5}\PY{p}{]}\PY{p}{)}
        \PY{n}{ax1}\PY{o}{.}\PY{n}{set\PYZus{}yticks}\PY{p}{(}\PY{p}{[}\PY{o}{\PYZhy{}}\PY{l+m+mf}{0.5}\PY{p}{,} \PY{l+m+mi}{0}\PY{p}{,} \PY{l+m+mf}{0.5}\PY{p}{]}\PY{p}{)}
        \PY{n}{plt}\PY{o}{.}\PY{n}{quiver}\PY{p}{(}\PY{n}{E\PYZus{}x}\PY{p}{,} \PY{n}{E\PYZus{}y}\PY{p}{,}\PY{n}{headwidth}\PY{o}{=}\PY{l+m+mi}{3}\PY{p}{,}\PY{n}{cmap}\PY{o}{=}\PY{l+s+s1}{\PYZsq{}}\PY{l+s+s1}{jet}\PY{l+s+s1}{\PYZsq{}}\PY{p}{)}
        \PY{c+c1}{\PYZsh{}plt.scatter(x,y, color=\PYZsq{}c\PYZsq{})}
\end{Verbatim}

\begin{Verbatim}[commandchars=\\\{\}]
{\color{outcolor}Out[{\color{outcolor}7}]:} <matplotlib.quiver.Quiver at 0x113bfaf28>
\end{Verbatim}
            
    \begin{center}
    \adjustimage{max size={0.9\linewidth}{0.9\paperheight}}{output_8_1.png}
    \end{center}
    { \hspace*{\fill} \\}
    
    While the quiver plot with varying arrow lengths contains more
information, it's also difficult to discern the pattern of the field,
which is clearly visible when all arrows are made the same length. I
wasn't able to set the axes of the plots to reflect the distances as
opposed to the number of points being used to create the vector field.

    \section{CP 6.1 Resistor circuit}\label{cp-6.1-resistor-circuit}

Given the circuit diagram in the book, we can solve a system of
equations to find the voltages at different points in the circuit. One
equation given is \[4V_1 - V_2 - V_3 - V_4 = V_+.\] By writing out
similar equations using Ohm's Law and Kirchoff's Junction Rule, we
obtain the following equations. \[3V_2 - V_1 - V_4 = 0\]
\[3V_3 - V_1 - V_4 = V_+\] \[4V_4 - V_1 - V_2 - V_3 = 0.\]

    We can write this in its vector-matrix form \(A\mathbf{x} = \mathbf{v}\)
with

\(A = \begin{pmatrix}  4 & -1 & -1 & -1 \\  -1 & 3 & 0 & -1 \\  -1 & 0 & 3 & -1 \\  -1 & -1 & -1 & 4 \end{pmatrix} \text{, }\)
\(\mathbf{x} = \begin{pmatrix} V_1 \\ V_2 \\ V_3 \\ V_4 \end{pmatrix} \text{, and }\)
\(\mathbf{v} = \begin{pmatrix} V_+ \\ 0 \\ V_+ \\ 0 \end{pmatrix}.\)

    \begin{Verbatim}[commandchars=\\\{\}]
{\color{incolor}In [{\color{incolor}8}]:} \PY{o}{\PYZpc{}\PYZpc{}}\PY{k}{time}
        
        Vp = 5 \PYZsh{}volts
        A = np.array([[4,\PYZhy{}1,\PYZhy{}1,\PYZhy{}1],
                      [\PYZhy{}1,3,0,\PYZhy{}1],
                      [\PYZhy{}1,0,3,\PYZhy{}1],
                      [\PYZhy{}1,\PYZhy{}1,\PYZhy{}1,4]], float)
        v = np.array([Vp,0,Vp,0], float)
        N = len(v)
        
        \PYZsh{}Gaussian elimination program w/out partial pivoting
        for m in range(N):
        
            \PYZsh{}Division by diagonal element
            div = A[m,m]
            A[m,:] /= div
            v[m] /= div
        
            \PYZsh{}subtract from lower rows
            for i in range(m+1, N):
                mult = A[i,m]
                A[i,:] \PYZhy{}= mult * A[m,:]
                v[i] \PYZhy{}= mult * v[m]
        
        \PYZsh{}backsubstitution
        x = np.empty(N, float)
        for m in range(N\PYZhy{}1, \PYZhy{}1, \PYZhy{}1):
            x[m] = v[m]
            for i in range(m+1, N):
                x[m] \PYZhy{}= A[m,i] * x[i]
         
        print(\PYZdq{}(V1, V2, V3, V4) = (\PYZob{}:2.2f\PYZcb{},\PYZob{}:2.2f\PYZcb{},\PYZob{}:2.2f\PYZcb{},\PYZob{}:2.2f\PYZcb{})\PYZdq{}\PYZbs{}
             .format(x[0],x[1],x[2],x[3]))
\end{Verbatim}

    \begin{Verbatim}[commandchars=\\\{\}]
(V1, V2, V3, V4) = (3.00,1.67,3.33,2.00)
CPU times: user 306 µs, sys: 113 µs, total: 419 µs
Wall time: 345 µs

    \end{Verbatim}

    \section{CP 6.2 Partial pivoting}\label{cp-6.2-partial-pivoting}

Incorporating partial pivoting into the Gaussian elimination algorithm
handles cases in which there are zeroes on the diagonal of our
coefficient matrix. This would normally be a problem because it would
create a divide by zero error, however rows can be exchanged without
effecting the solution, as long as they are changed consistently.

I will demonstrate that partial pivoting provides the same answers as
standard Gaussian elimination when applied to the system in 6.1:

\[2w+x+4y+z=-4\] \[3w+4x-y-z=3\] \[w-4x+y+5z=9\] \[2w-2x+y+3z=7\]

    \begin{Verbatim}[commandchars=\\\{\}]
{\color{incolor}In [{\color{incolor}9}]:} \PY{l+s+sd}{\PYZdq{}\PYZdq{}\PYZdq{}This cell prints the result of using partial pivoting}
        \PY{l+s+sd}{    with Gaussian elimination to solve the above system\PYZdq{}\PYZdq{}\PYZdq{}}
        \PY{n}{A} \PY{o}{=} \PY{n}{np}\PY{o}{.}\PY{n}{array}\PY{p}{(}\PY{p}{[}\PY{p}{[}\PY{l+m+mi}{2}\PY{p}{,}\PY{l+m+mi}{1}\PY{p}{,}\PY{l+m+mi}{4}\PY{p}{,}\PY{l+m+mi}{1}\PY{p}{]}\PY{p}{,}
                      \PY{p}{[}\PY{l+m+mi}{3}\PY{p}{,}\PY{l+m+mi}{4}\PY{p}{,}\PY{o}{\PYZhy{}}\PY{l+m+mi}{1}\PY{p}{,}\PY{o}{\PYZhy{}}\PY{l+m+mi}{1}\PY{p}{]}\PY{p}{,}
                      \PY{p}{[}\PY{l+m+mi}{1}\PY{p}{,}\PY{o}{\PYZhy{}}\PY{l+m+mi}{4}\PY{p}{,}\PY{l+m+mi}{1}\PY{p}{,}\PY{l+m+mi}{5}\PY{p}{]}\PY{p}{,}
                      \PY{p}{[}\PY{l+m+mi}{2}\PY{p}{,}\PY{o}{\PYZhy{}}\PY{l+m+mi}{2}\PY{p}{,}\PY{l+m+mi}{1}\PY{p}{,}\PY{l+m+mi}{3}\PY{p}{]}\PY{p}{]}\PY{p}{,} \PY{n+nb}{float}\PY{p}{)}
        \PY{n}{v} \PY{o}{=} \PY{n}{np}\PY{o}{.}\PY{n}{array}\PY{p}{(}\PY{p}{[}\PY{o}{\PYZhy{}}\PY{l+m+mi}{4}\PY{p}{,}\PY{l+m+mi}{3}\PY{p}{,}\PY{l+m+mi}{9}\PY{p}{,}\PY{l+m+mi}{7}\PY{p}{]}\PY{p}{,} \PY{n+nb}{float}\PY{p}{)}
        \PY{n}{N} \PY{o}{=} \PY{n+nb}{len}\PY{p}{(}\PY{n}{v}\PY{p}{)}
        
        \PY{c+c1}{\PYZsh{}Gaussian elimination program w/ partial pivoting}
        \PY{k}{for} \PY{n}{m} \PY{o+ow}{in} \PY{n+nb}{range}\PY{p}{(}\PY{n}{N}\PY{p}{)}\PY{p}{:}
        
            \PY{c+c1}{\PYZsh{}Partial pivoting}
            \PY{k}{if} \PY{n}{A}\PY{p}{[}\PY{n}{m}\PY{p}{,}\PY{n}{m}\PY{p}{]} \PY{o}{==} \PY{l+m+mi}{0}\PY{p}{:}
                \PY{n}{A}\PY{p}{[}\PY{p}{[}\PY{n}{m}\PY{p}{,}\PY{n}{m}\PY{o}{+}\PY{l+m+mi}{1}\PY{p}{]}\PY{p}{]} \PY{o}{=} \PY{n}{A}\PY{p}{[}\PY{p}{[}\PY{n}{m}\PY{o}{+}\PY{l+m+mi}{1}\PY{p}{,}\PY{n}{m}\PY{p}{]}\PY{p}{]}
                \PY{n}{v}\PY{p}{[}\PY{p}{[}\PY{n}{m}\PY{p}{,}\PY{n}{m}\PY{o}{+}\PY{l+m+mi}{1}\PY{p}{]}\PY{p}{]} \PY{o}{=} \PY{n}{v}\PY{p}{[}\PY{p}{[}\PY{n}{m}\PY{o}{+}\PY{l+m+mi}{1}\PY{p}{,}\PY{n}{m}\PY{p}{]}\PY{p}{]}
                
            \PY{c+c1}{\PYZsh{}Division by diagonal element}
            \PY{n}{div} \PY{o}{=} \PY{n}{A}\PY{p}{[}\PY{n}{m}\PY{p}{,}\PY{n}{m}\PY{p}{]}
            \PY{n}{A}\PY{p}{[}\PY{n}{m}\PY{p}{,}\PY{p}{:}\PY{p}{]} \PY{o}{/}\PY{o}{=} \PY{n}{div}
            \PY{n}{v}\PY{p}{[}\PY{n}{m}\PY{p}{]} \PY{o}{/}\PY{o}{=} \PY{n}{div}
        
            \PY{c+c1}{\PYZsh{}subtract from lower rows}
            \PY{k}{for} \PY{n}{i} \PY{o+ow}{in} \PY{n+nb}{range}\PY{p}{(}\PY{n}{m}\PY{o}{+}\PY{l+m+mi}{1}\PY{p}{,} \PY{n}{N}\PY{p}{)}\PY{p}{:}
                \PY{n}{mult} \PY{o}{=} \PY{n}{A}\PY{p}{[}\PY{n}{i}\PY{p}{,}\PY{n}{m}\PY{p}{]}
                \PY{n}{A}\PY{p}{[}\PY{n}{i}\PY{p}{,}\PY{p}{:}\PY{p}{]} \PY{o}{\PYZhy{}}\PY{o}{=} \PY{n}{mult} \PY{o}{*} \PY{n}{A}\PY{p}{[}\PY{n}{m}\PY{p}{,}\PY{p}{:}\PY{p}{]}
                \PY{n}{v}\PY{p}{[}\PY{n}{i}\PY{p}{]} \PY{o}{\PYZhy{}}\PY{o}{=} \PY{n}{mult} \PY{o}{*} \PY{n}{v}\PY{p}{[}\PY{n}{m}\PY{p}{]}
                
        
        \PY{c+c1}{\PYZsh{}backsubstitution}
        \PY{n}{x} \PY{o}{=} \PY{n}{np}\PY{o}{.}\PY{n}{empty}\PY{p}{(}\PY{n}{N}\PY{p}{,} \PY{n+nb}{float}\PY{p}{)}
        \PY{k}{for} \PY{n}{m} \PY{o+ow}{in} \PY{n+nb}{range}\PY{p}{(}\PY{n}{N}\PY{o}{\PYZhy{}}\PY{l+m+mi}{1}\PY{p}{,} \PY{o}{\PYZhy{}}\PY{l+m+mi}{1}\PY{p}{,} \PY{o}{\PYZhy{}}\PY{l+m+mi}{1}\PY{p}{)}\PY{p}{:}
            \PY{n}{x}\PY{p}{[}\PY{n}{m}\PY{p}{]} \PY{o}{=} \PY{n}{v}\PY{p}{[}\PY{n}{m}\PY{p}{]}
            \PY{k}{for} \PY{n}{i} \PY{o+ow}{in} \PY{n+nb}{range}\PY{p}{(}\PY{n}{m}\PY{o}{+}\PY{l+m+mi}{1}\PY{p}{,} \PY{n}{N}\PY{p}{)}\PY{p}{:}
                \PY{n}{x}\PY{p}{[}\PY{n}{m}\PY{p}{]} \PY{o}{\PYZhy{}}\PY{o}{=} \PY{n}{A}\PY{p}{[}\PY{n}{m}\PY{p}{,}\PY{n}{i}\PY{p}{]} \PY{o}{*} \PY{n}{x}\PY{p}{[}\PY{n}{i}\PY{p}{]}
        
        \PY{n+nb}{print}\PY{p}{(}\PY{l+s+s2}{\PYZdq{}}\PY{l+s+s2}{(w, x, y, z) = (}\PY{l+s+si}{\PYZob{}:2.2f\PYZcb{}}\PY{l+s+s2}{,}\PY{l+s+si}{\PYZob{}:2.2f\PYZcb{}}\PY{l+s+s2}{,}\PY{l+s+si}{\PYZob{}:2.2f\PYZcb{}}\PY{l+s+s2}{,}\PY{l+s+si}{\PYZob{}:2.2f\PYZcb{}}\PY{l+s+s2}{)}\PY{l+s+s2}{\PYZdq{}}\PYZbs{}
             \PY{o}{.}\PY{n}{format}\PY{p}{(}\PY{n}{x}\PY{p}{[}\PY{l+m+mi}{0}\PY{p}{]}\PY{p}{,}\PY{n}{x}\PY{p}{[}\PY{l+m+mi}{1}\PY{p}{]}\PY{p}{,}\PY{n}{x}\PY{p}{[}\PY{l+m+mi}{2}\PY{p}{]}\PY{p}{,}\PY{n}{x}\PY{p}{[}\PY{l+m+mi}{3}\PY{p}{]}\PY{p}{)}\PY{p}{)}
\end{Verbatim}

    \begin{Verbatim}[commandchars=\\\{\}]
(w, x, y, z) = (2.00,-1.00,-2.00,1.00)

    \end{Verbatim}

    \begin{Verbatim}[commandchars=\\\{\}]
{\color{incolor}In [{\color{incolor}10}]:} \PY{l+s+sd}{\PYZdq{}\PYZdq{}\PYZdq{}This cell solves the system of equations}
         \PY{l+s+sd}{    without using partial pivoting\PYZdq{}\PYZdq{}\PYZdq{}}
         \PY{n}{A} \PY{o}{=} \PY{n}{np}\PY{o}{.}\PY{n}{array}\PY{p}{(}\PY{p}{[}\PY{p}{[}\PY{l+m+mi}{2}\PY{p}{,}\PY{l+m+mi}{1}\PY{p}{,}\PY{l+m+mi}{4}\PY{p}{,}\PY{l+m+mi}{1}\PY{p}{]}\PY{p}{,}
                       \PY{p}{[}\PY{l+m+mi}{3}\PY{p}{,}\PY{l+m+mi}{4}\PY{p}{,}\PY{o}{\PYZhy{}}\PY{l+m+mi}{1}\PY{p}{,}\PY{o}{\PYZhy{}}\PY{l+m+mi}{1}\PY{p}{]}\PY{p}{,}
                       \PY{p}{[}\PY{l+m+mi}{1}\PY{p}{,}\PY{o}{\PYZhy{}}\PY{l+m+mi}{4}\PY{p}{,}\PY{l+m+mi}{1}\PY{p}{,}\PY{l+m+mi}{5}\PY{p}{]}\PY{p}{,}
                       \PY{p}{[}\PY{l+m+mi}{2}\PY{p}{,}\PY{o}{\PYZhy{}}\PY{l+m+mi}{2}\PY{p}{,}\PY{l+m+mi}{1}\PY{p}{,}\PY{l+m+mi}{3}\PY{p}{]}\PY{p}{]}\PY{p}{,} \PY{n+nb}{float}\PY{p}{)}
         \PY{n}{v} \PY{o}{=} \PY{n}{np}\PY{o}{.}\PY{n}{array}\PY{p}{(}\PY{p}{[}\PY{o}{\PYZhy{}}\PY{l+m+mi}{4}\PY{p}{,}\PY{l+m+mi}{3}\PY{p}{,}\PY{l+m+mi}{9}\PY{p}{,}\PY{l+m+mi}{7}\PY{p}{]}\PY{p}{,} \PY{n+nb}{float}\PY{p}{)}
         \PY{n}{N} \PY{o}{=} \PY{n+nb}{len}\PY{p}{(}\PY{n}{v}\PY{p}{)}
         
         \PY{c+c1}{\PYZsh{}Gaussian elimination program w/out partial pivoting}
         \PY{k}{for} \PY{n}{m} \PY{o+ow}{in} \PY{n+nb}{range}\PY{p}{(}\PY{n}{N}\PY{p}{)}\PY{p}{:}
         
             \PY{c+c1}{\PYZsh{}Division by diagonal element}
             \PY{n}{div} \PY{o}{=} \PY{n}{A}\PY{p}{[}\PY{n}{m}\PY{p}{,}\PY{n}{m}\PY{p}{]}
             \PY{n}{A}\PY{p}{[}\PY{n}{m}\PY{p}{,}\PY{p}{:}\PY{p}{]} \PY{o}{/}\PY{o}{=} \PY{n}{div}
             \PY{n}{v}\PY{p}{[}\PY{n}{m}\PY{p}{]} \PY{o}{/}\PY{o}{=} \PY{n}{div}
         
             \PY{c+c1}{\PYZsh{}subtract from lower rows}
             \PY{k}{for} \PY{n}{i} \PY{o+ow}{in} \PY{n+nb}{range}\PY{p}{(}\PY{n}{m}\PY{o}{+}\PY{l+m+mi}{1}\PY{p}{,} \PY{n}{N}\PY{p}{)}\PY{p}{:}
                 \PY{n}{mult} \PY{o}{=} \PY{n}{A}\PY{p}{[}\PY{n}{i}\PY{p}{,}\PY{n}{m}\PY{p}{]}
                 \PY{n}{A}\PY{p}{[}\PY{n}{i}\PY{p}{,}\PY{p}{:}\PY{p}{]} \PY{o}{\PYZhy{}}\PY{o}{=} \PY{n}{mult} \PY{o}{*} \PY{n}{A}\PY{p}{[}\PY{n}{m}\PY{p}{,}\PY{p}{:}\PY{p}{]}
                 \PY{n}{v}\PY{p}{[}\PY{n}{i}\PY{p}{]} \PY{o}{\PYZhy{}}\PY{o}{=} \PY{n}{mult} \PY{o}{*} \PY{n}{v}\PY{p}{[}\PY{n}{m}\PY{p}{]}
         
         \PY{c+c1}{\PYZsh{}backsubstitution}
         \PY{n}{x} \PY{o}{=} \PY{n}{np}\PY{o}{.}\PY{n}{empty}\PY{p}{(}\PY{n}{N}\PY{p}{,} \PY{n+nb}{float}\PY{p}{)}
         \PY{k}{for} \PY{n}{m} \PY{o+ow}{in} \PY{n+nb}{range}\PY{p}{(}\PY{n}{N}\PY{o}{\PYZhy{}}\PY{l+m+mi}{1}\PY{p}{,} \PY{o}{\PYZhy{}}\PY{l+m+mi}{1}\PY{p}{,} \PY{o}{\PYZhy{}}\PY{l+m+mi}{1}\PY{p}{)}\PY{p}{:}
             \PY{n}{x}\PY{p}{[}\PY{n}{m}\PY{p}{]} \PY{o}{=} \PY{n}{v}\PY{p}{[}\PY{n}{m}\PY{p}{]}
             \PY{k}{for} \PY{n}{i} \PY{o+ow}{in} \PY{n+nb}{range}\PY{p}{(}\PY{n}{m}\PY{o}{+}\PY{l+m+mi}{1}\PY{p}{,} \PY{n}{N}\PY{p}{)}\PY{p}{:}
                 \PY{n}{x}\PY{p}{[}\PY{n}{m}\PY{p}{]} \PY{o}{\PYZhy{}}\PY{o}{=} \PY{n}{A}\PY{p}{[}\PY{n}{m}\PY{p}{,}\PY{n}{i}\PY{p}{]} \PY{o}{*} \PY{n}{x}\PY{p}{[}\PY{n}{i}\PY{p}{]}
         
         \PY{n+nb}{print}\PY{p}{(}\PY{l+s+s2}{\PYZdq{}}\PY{l+s+s2}{(w, x, y, z) = (}\PY{l+s+si}{\PYZob{}:2.2f\PYZcb{}}\PY{l+s+s2}{,}\PY{l+s+si}{\PYZob{}:2.2f\PYZcb{}}\PY{l+s+s2}{,}\PY{l+s+si}{\PYZob{}:2.2f\PYZcb{}}\PY{l+s+s2}{,}\PY{l+s+si}{\PYZob{}:2.2f\PYZcb{}}\PY{l+s+s2}{)}\PY{l+s+s2}{\PYZdq{}}\PYZbs{}
              \PY{o}{.}\PY{n}{format}\PY{p}{(}\PY{n}{x}\PY{p}{[}\PY{l+m+mi}{0}\PY{p}{]}\PY{p}{,}\PY{n}{x}\PY{p}{[}\PY{l+m+mi}{1}\PY{p}{]}\PY{p}{,}\PY{n}{x}\PY{p}{[}\PY{l+m+mi}{2}\PY{p}{]}\PY{p}{,}\PY{n}{x}\PY{p}{[}\PY{l+m+mi}{3}\PY{p}{]}\PY{p}{)}\PY{p}{)}
\end{Verbatim}

    \begin{Verbatim}[commandchars=\\\{\}]
(w, x, y, z) = (2.00,-1.00,-2.00,1.00)

    \end{Verbatim}

    So, for that system, because there were no zeroes on the diagonal, the
methods using partial pivoting and not performed equally well. Now,
moving on to part (b), we attempt to solve the system in 6.17.

\(A = \begin{pmatrix}  0 & 1 & 4 & 1 \\  3 & 4 & -1 & -1 \\  1 & -4 & 1 & 5 \\  2 & -2 & 1 & 3 \end{pmatrix} \text{ and }\)
\(\mathbf{v} = \begin{pmatrix} -4 \\ 3 \\ 9 \\ 7 \end{pmatrix}.\)

Attempting to solve this, normal Gaussian elimination without partial
pivoting fails.

    \begin{Verbatim}[commandchars=\\\{\}]
{\color{incolor}In [{\color{incolor}11}]:} \PY{n}{A} \PY{o}{=} \PY{n}{np}\PY{o}{.}\PY{n}{array}\PY{p}{(}\PY{p}{[}\PY{p}{[}\PY{l+m+mi}{0}\PY{p}{,}\PY{l+m+mi}{1}\PY{p}{,}\PY{l+m+mi}{4}\PY{p}{,}\PY{l+m+mi}{1}\PY{p}{]}\PY{p}{,}
                       \PY{p}{[}\PY{l+m+mi}{3}\PY{p}{,}\PY{l+m+mi}{4}\PY{p}{,}\PY{o}{\PYZhy{}}\PY{l+m+mi}{1}\PY{p}{,}\PY{o}{\PYZhy{}}\PY{l+m+mi}{1}\PY{p}{]}\PY{p}{,}
                       \PY{p}{[}\PY{l+m+mi}{1}\PY{p}{,}\PY{o}{\PYZhy{}}\PY{l+m+mi}{4}\PY{p}{,}\PY{l+m+mi}{1}\PY{p}{,}\PY{l+m+mi}{5}\PY{p}{]}\PY{p}{,}
                       \PY{p}{[}\PY{l+m+mi}{2}\PY{p}{,}\PY{o}{\PYZhy{}}\PY{l+m+mi}{2}\PY{p}{,}\PY{l+m+mi}{1}\PY{p}{,}\PY{l+m+mi}{3}\PY{p}{]}\PY{p}{]}\PY{p}{,} \PY{n+nb}{float}\PY{p}{)}
         \PY{n}{v} \PY{o}{=} \PY{n}{np}\PY{o}{.}\PY{n}{array}\PY{p}{(}\PY{p}{[}\PY{o}{\PYZhy{}}\PY{l+m+mi}{4}\PY{p}{,}\PY{l+m+mi}{3}\PY{p}{,}\PY{l+m+mi}{9}\PY{p}{,}\PY{l+m+mi}{7}\PY{p}{]}\PY{p}{,} \PY{n+nb}{float}\PY{p}{)}
         \PY{n}{N} \PY{o}{=} \PY{n+nb}{len}\PY{p}{(}\PY{n}{v}\PY{p}{)}
         
         \PY{c+c1}{\PYZsh{}Gaussian elimination program w/out partial pivoting}
         \PY{k}{for} \PY{n}{m} \PY{o+ow}{in} \PY{n+nb}{range}\PY{p}{(}\PY{n}{N}\PY{p}{)}\PY{p}{:}
         
             \PY{c+c1}{\PYZsh{}Division by diagonal element}
             \PY{n}{div} \PY{o}{=} \PY{n}{A}\PY{p}{[}\PY{n}{m}\PY{p}{,}\PY{n}{m}\PY{p}{]}
             \PY{n}{A}\PY{p}{[}\PY{n}{m}\PY{p}{,}\PY{p}{:}\PY{p}{]} \PY{o}{/}\PY{o}{=} \PY{n}{div}
             \PY{n}{v}\PY{p}{[}\PY{n}{m}\PY{p}{]} \PY{o}{/}\PY{o}{=} \PY{n}{div}
         
             \PY{c+c1}{\PYZsh{}subtract from lower rows}
             \PY{k}{for} \PY{n}{i} \PY{o+ow}{in} \PY{n+nb}{range}\PY{p}{(}\PY{n}{m}\PY{o}{+}\PY{l+m+mi}{1}\PY{p}{,} \PY{n}{N}\PY{p}{)}\PY{p}{:}
                 \PY{n}{mult} \PY{o}{=} \PY{n}{A}\PY{p}{[}\PY{n}{i}\PY{p}{,}\PY{n}{m}\PY{p}{]}
                 \PY{n}{A}\PY{p}{[}\PY{n}{i}\PY{p}{,}\PY{p}{:}\PY{p}{]} \PY{o}{\PYZhy{}}\PY{o}{=} \PY{n}{mult} \PY{o}{*} \PY{n}{A}\PY{p}{[}\PY{n}{m}\PY{p}{,}\PY{p}{:}\PY{p}{]}
                 \PY{n}{v}\PY{p}{[}\PY{n}{i}\PY{p}{]} \PY{o}{\PYZhy{}}\PY{o}{=} \PY{n}{mult} \PY{o}{*} \PY{n}{v}\PY{p}{[}\PY{n}{m}\PY{p}{]}
         
         \PY{c+c1}{\PYZsh{}backsubstitution}
         \PY{n}{x} \PY{o}{=} \PY{n}{np}\PY{o}{.}\PY{n}{empty}\PY{p}{(}\PY{n}{N}\PY{p}{,} \PY{n+nb}{float}\PY{p}{)}
         \PY{k}{for} \PY{n}{m} \PY{o+ow}{in} \PY{n+nb}{range}\PY{p}{(}\PY{n}{N}\PY{o}{\PYZhy{}}\PY{l+m+mi}{1}\PY{p}{,} \PY{o}{\PYZhy{}}\PY{l+m+mi}{1}\PY{p}{,} \PY{o}{\PYZhy{}}\PY{l+m+mi}{1}\PY{p}{)}\PY{p}{:}
             \PY{n}{x}\PY{p}{[}\PY{n}{m}\PY{p}{]} \PY{o}{=} \PY{n}{v}\PY{p}{[}\PY{n}{m}\PY{p}{]}
             \PY{k}{for} \PY{n}{i} \PY{o+ow}{in} \PY{n+nb}{range}\PY{p}{(}\PY{n}{m}\PY{o}{+}\PY{l+m+mi}{1}\PY{p}{,} \PY{n}{N}\PY{p}{)}\PY{p}{:}
                 \PY{n}{x}\PY{p}{[}\PY{n}{m}\PY{p}{]} \PY{o}{\PYZhy{}}\PY{o}{=} \PY{n}{A}\PY{p}{[}\PY{n}{m}\PY{p}{,}\PY{n}{i}\PY{p}{]} \PY{o}{*} \PY{n}{x}\PY{p}{[}\PY{n}{i}\PY{p}{]}
                 
         \PY{n+nb}{print}\PY{p}{(}\PY{l+s+s2}{\PYZdq{}}\PY{l+s+s2}{(w, x, y, z) = (}\PY{l+s+si}{\PYZob{}:2.2f\PYZcb{}}\PY{l+s+s2}{,}\PY{l+s+si}{\PYZob{}:2.2f\PYZcb{}}\PY{l+s+s2}{,}\PY{l+s+si}{\PYZob{}:2.2f\PYZcb{}}\PY{l+s+s2}{,}\PY{l+s+si}{\PYZob{}:2.2f\PYZcb{}}\PY{l+s+s2}{)}\PY{l+s+s2}{\PYZdq{}}\PYZbs{}
              \PY{o}{.}\PY{n}{format}\PY{p}{(}\PY{n}{x}\PY{p}{[}\PY{l+m+mi}{0}\PY{p}{]}\PY{p}{,}\PY{n}{x}\PY{p}{[}\PY{l+m+mi}{1}\PY{p}{]}\PY{p}{,}\PY{n}{x}\PY{p}{[}\PY{l+m+mi}{2}\PY{p}{]}\PY{p}{,}\PY{n}{x}\PY{p}{[}\PY{l+m+mi}{3}\PY{p}{]}\PY{p}{)}\PY{p}{)}
\end{Verbatim}

    \begin{Verbatim}[commandchars=\\\{\}]
(w, x, y, z) = (nan,nan,nan,nan)

    \end{Verbatim}

    \begin{Verbatim}[commandchars=\\\{\}]
/Users/Varun/anaconda/lib/python3.6/site-packages/ipykernel\_launcher.py:13: RuntimeWarning: divide by zero encountered in true\_divide
  del sys.path[0]
/Users/Varun/anaconda/lib/python3.6/site-packages/ipykernel\_launcher.py:13: RuntimeWarning: invalid value encountered in true\_divide
  del sys.path[0]
/Users/Varun/anaconda/lib/python3.6/site-packages/ipykernel\_launcher.py:14: RuntimeWarning: divide by zero encountered in double\_scalars
  
/Users/Varun/anaconda/lib/python3.6/site-packages/ipykernel\_launcher.py:14: RuntimeWarning: invalid value encountered in double\_scalars
  

    \end{Verbatim}

    However, if we use the method of partial pivoting, the system is
solvable.

    \begin{Verbatim}[commandchars=\\\{\}]
{\color{incolor}In [{\color{incolor}12}]:} \PY{n}{A} \PY{o}{=} \PY{n}{np}\PY{o}{.}\PY{n}{array}\PY{p}{(}\PY{p}{[}\PY{p}{[}\PY{l+m+mi}{0}\PY{p}{,}\PY{l+m+mi}{1}\PY{p}{,}\PY{l+m+mi}{4}\PY{p}{,}\PY{l+m+mi}{1}\PY{p}{]}\PY{p}{,}
                       \PY{p}{[}\PY{l+m+mi}{3}\PY{p}{,}\PY{l+m+mi}{4}\PY{p}{,}\PY{o}{\PYZhy{}}\PY{l+m+mi}{1}\PY{p}{,}\PY{o}{\PYZhy{}}\PY{l+m+mi}{1}\PY{p}{]}\PY{p}{,}
                       \PY{p}{[}\PY{l+m+mi}{1}\PY{p}{,}\PY{o}{\PYZhy{}}\PY{l+m+mi}{4}\PY{p}{,}\PY{l+m+mi}{1}\PY{p}{,}\PY{l+m+mi}{5}\PY{p}{]}\PY{p}{,}
                       \PY{p}{[}\PY{l+m+mi}{2}\PY{p}{,}\PY{o}{\PYZhy{}}\PY{l+m+mi}{2}\PY{p}{,}\PY{l+m+mi}{1}\PY{p}{,}\PY{l+m+mi}{3}\PY{p}{]}\PY{p}{]}\PY{p}{,} \PY{n+nb}{float}\PY{p}{)}
         \PY{n}{v} \PY{o}{=} \PY{n}{np}\PY{o}{.}\PY{n}{array}\PY{p}{(}\PY{p}{[}\PY{o}{\PYZhy{}}\PY{l+m+mi}{4}\PY{p}{,}\PY{l+m+mi}{3}\PY{p}{,}\PY{l+m+mi}{9}\PY{p}{,}\PY{l+m+mi}{7}\PY{p}{]}\PY{p}{,} \PY{n+nb}{float}\PY{p}{)}
         \PY{n}{N} \PY{o}{=} \PY{n+nb}{len}\PY{p}{(}\PY{n}{v}\PY{p}{)}
         
         \PY{c+c1}{\PYZsh{}Gaussian elimination program w/ partial pivoting}
         \PY{k}{for} \PY{n}{m} \PY{o+ow}{in} \PY{n+nb}{range}\PY{p}{(}\PY{n}{N}\PY{p}{)}\PY{p}{:}
         
             \PY{c+c1}{\PYZsh{}Division by diagonal element}
             \PY{k}{if} \PY{n}{A}\PY{p}{[}\PY{n}{m}\PY{p}{,}\PY{n}{m}\PY{p}{]} \PY{o}{==} \PY{l+m+mi}{0}\PY{p}{:}
                 \PY{n}{A}\PY{p}{[}\PY{p}{[}\PY{n}{m}\PY{p}{,}\PY{n}{m}\PY{o}{+}\PY{l+m+mi}{1}\PY{p}{]}\PY{p}{]} \PY{o}{=} \PY{n}{A}\PY{p}{[}\PY{p}{[}\PY{n}{m}\PY{o}{+}\PY{l+m+mi}{1}\PY{p}{,}\PY{n}{m}\PY{p}{]}\PY{p}{]}
                 \PY{n}{v}\PY{p}{[}\PY{p}{[}\PY{n}{m}\PY{p}{,}\PY{n}{m}\PY{o}{+}\PY{l+m+mi}{1}\PY{p}{]}\PY{p}{]} \PY{o}{=} \PY{n}{v}\PY{p}{[}\PY{p}{[}\PY{n}{m}\PY{o}{+}\PY{l+m+mi}{1}\PY{p}{,}\PY{n}{m}\PY{p}{]}\PY{p}{]}
             \PY{n}{div} \PY{o}{=} \PY{n}{A}\PY{p}{[}\PY{n}{m}\PY{p}{,}\PY{n}{m}\PY{p}{]}
             \PY{n}{A}\PY{p}{[}\PY{n}{m}\PY{p}{,}\PY{p}{:}\PY{p}{]} \PY{o}{/}\PY{o}{=} \PY{n}{div}
             \PY{n}{v}\PY{p}{[}\PY{n}{m}\PY{p}{]} \PY{o}{/}\PY{o}{=} \PY{n}{div}
         
             \PY{c+c1}{\PYZsh{}subtract from lower rows}
             \PY{k}{for} \PY{n}{i} \PY{o+ow}{in} \PY{n+nb}{range}\PY{p}{(}\PY{n}{m}\PY{o}{+}\PY{l+m+mi}{1}\PY{p}{,} \PY{n}{N}\PY{p}{)}\PY{p}{:}
                 \PY{n}{mult} \PY{o}{=} \PY{n}{A}\PY{p}{[}\PY{n}{i}\PY{p}{,}\PY{n}{m}\PY{p}{]}
                 \PY{n}{A}\PY{p}{[}\PY{n}{i}\PY{p}{,}\PY{p}{:}\PY{p}{]} \PY{o}{\PYZhy{}}\PY{o}{=} \PY{n}{mult} \PY{o}{*} \PY{n}{A}\PY{p}{[}\PY{n}{m}\PY{p}{,}\PY{p}{:}\PY{p}{]}
                 \PY{n}{v}\PY{p}{[}\PY{n}{i}\PY{p}{]} \PY{o}{\PYZhy{}}\PY{o}{=} \PY{n}{mult} \PY{o}{*} \PY{n}{v}\PY{p}{[}\PY{n}{m}\PY{p}{]}
         
         \PY{c+c1}{\PYZsh{}backsubstitution}
         \PY{n}{x} \PY{o}{=} \PY{n}{np}\PY{o}{.}\PY{n}{empty}\PY{p}{(}\PY{n}{N}\PY{p}{,} \PY{n+nb}{float}\PY{p}{)}
         \PY{k}{for} \PY{n}{m} \PY{o+ow}{in} \PY{n+nb}{range}\PY{p}{(}\PY{n}{N}\PY{o}{\PYZhy{}}\PY{l+m+mi}{1}\PY{p}{,} \PY{o}{\PYZhy{}}\PY{l+m+mi}{1}\PY{p}{,} \PY{o}{\PYZhy{}}\PY{l+m+mi}{1}\PY{p}{)}\PY{p}{:}
             \PY{n}{x}\PY{p}{[}\PY{n}{m}\PY{p}{]} \PY{o}{=} \PY{n}{v}\PY{p}{[}\PY{n}{m}\PY{p}{]}
             \PY{k}{for} \PY{n}{i} \PY{o+ow}{in} \PY{n+nb}{range}\PY{p}{(}\PY{n}{m}\PY{o}{+}\PY{l+m+mi}{1}\PY{p}{,} \PY{n}{N}\PY{p}{)}\PY{p}{:}
                 \PY{n}{x}\PY{p}{[}\PY{n}{m}\PY{p}{]} \PY{o}{\PYZhy{}}\PY{o}{=} \PY{n}{A}\PY{p}{[}\PY{n}{m}\PY{p}{,}\PY{n}{i}\PY{p}{]} \PY{o}{*} \PY{n}{x}\PY{p}{[}\PY{n}{i}\PY{p}{]}
         
         \PY{n+nb}{print}\PY{p}{(}\PY{l+s+s2}{\PYZdq{}}\PY{l+s+s2}{(w, x, y, z) = (}\PY{l+s+si}{\PYZob{}:2.2f\PYZcb{}}\PY{l+s+s2}{,}\PY{l+s+si}{\PYZob{}:2.2f\PYZcb{}}\PY{l+s+s2}{,}\PY{l+s+si}{\PYZob{}:2.2f\PYZcb{}}\PY{l+s+s2}{,}\PY{l+s+si}{\PYZob{}:2.2f\PYZcb{}}\PY{l+s+s2}{)}\PY{l+s+s2}{\PYZdq{}}\PYZbs{}
              \PY{o}{.}\PY{n}{format}\PY{p}{(}\PY{n}{x}\PY{p}{[}\PY{l+m+mi}{0}\PY{p}{]}\PY{p}{,}\PY{n}{x}\PY{p}{[}\PY{l+m+mi}{1}\PY{p}{]}\PY{p}{,}\PY{n}{x}\PY{p}{[}\PY{l+m+mi}{2}\PY{p}{]}\PY{p}{,}\PY{n}{x}\PY{p}{[}\PY{l+m+mi}{3}\PY{p}{]}\PY{p}{)}\PY{p}{)}
\end{Verbatim}

    \begin{Verbatim}[commandchars=\\\{\}]
(w, x, y, z) = (1.62,-0.43,-1.24,1.38)

    \end{Verbatim}

    \section{CP 6.3 LU decomposition}\label{cp-6.3-lu-decomposition}

In \(LU\) decomposition, the goal is to factor the original matrix \(A\)
into lower and upper triangular matrices, respectively. This is
beneficial if we see multiple systems of equations where the set of
coefficients on the unknowns are the same, but result in different set
of solutions. By decomposing \(A\) into \(L\) and \(U,\) we don't need
to solve the system from scratch every time. In LU decomposition,
intermediate matrices \(L_0, L_1, ..., L_n\) are calculated. For the
final relation, the lower and upper triangular matrices are respectively

\[L = L_0^{-1}L_1^{-1}L_2^{-1}L_3^{-1} \cdots \qquad \text{ and } \qquad U = \cdots L_3L_2L_1L_0A.\]

    \begin{Verbatim}[commandchars=\\\{\}]
{\color{incolor}In [{\color{incolor}13}]:} \PY{c+c1}{\PYZsh{}matrix and vector from Eq 6.32}
         \PY{n}{A} \PY{o}{=} \PY{n}{np}\PY{o}{.}\PY{n}{array}\PY{p}{(}\PY{p}{[}\PY{p}{[}\PY{l+m+mi}{2}\PY{p}{,}\PY{l+m+mi}{1}\PY{p}{,}\PY{l+m+mi}{4}\PY{p}{,}\PY{l+m+mi}{1}\PY{p}{]}\PY{p}{,}
                       \PY{p}{[}\PY{l+m+mi}{3}\PY{p}{,}\PY{l+m+mi}{4}\PY{p}{,}\PY{o}{\PYZhy{}}\PY{l+m+mi}{1}\PY{p}{,}\PY{o}{\PYZhy{}}\PY{l+m+mi}{1}\PY{p}{]}\PY{p}{,}
                       \PY{p}{[}\PY{l+m+mi}{1}\PY{p}{,}\PY{o}{\PYZhy{}}\PY{l+m+mi}{4}\PY{p}{,}\PY{l+m+mi}{1}\PY{p}{,}\PY{l+m+mi}{5}\PY{p}{]}\PY{p}{,}
                       \PY{p}{[}\PY{l+m+mi}{2}\PY{p}{,}\PY{o}{\PYZhy{}}\PY{l+m+mi}{2}\PY{p}{,}\PY{l+m+mi}{1}\PY{p}{,}\PY{l+m+mi}{3}\PY{p}{]}\PY{p}{]}\PY{p}{,} \PY{n+nb}{float}\PY{p}{)}
         \PY{n}{v} \PY{o}{=} \PY{n}{np}\PY{o}{.}\PY{n}{array}\PY{p}{(}\PY{p}{[}\PY{o}{\PYZhy{}}\PY{l+m+mi}{4}\PY{p}{,}\PY{l+m+mi}{3}\PY{p}{,}\PY{l+m+mi}{9}\PY{p}{,}\PY{l+m+mi}{7}\PY{p}{]}\PY{p}{,} \PY{n+nb}{float}\PY{p}{)}
\end{Verbatim}

    \begin{Verbatim}[commandchars=\\\{\}]
{\color{incolor}In [{\color{incolor}14}]:} \PY{k}{def} \PY{n+nf}{LU}\PY{p}{(}\PY{n}{A}\PY{p}{)}\PY{p}{:}
             \PY{l+s+sd}{\PYZdq{}\PYZdq{}\PYZdq{}Defines a function to perform LU decomposition}
         \PY{l+s+sd}{        on a matrix A based on gausslim.py from book\PYZdq{}\PYZdq{}\PYZdq{}}
             
             \PY{n}{N} \PY{o}{=} \PY{n+nb}{len}\PY{p}{(}\PY{n}{A}\PY{p}{)}
             \PY{n}{U} \PY{o}{=} \PY{n}{np}\PY{o}{.}\PY{n}{copy}\PY{p}{(}\PY{n}{A}\PY{p}{)}
             \PY{n}{L} \PY{o}{=} \PY{n}{np}\PY{o}{.}\PY{n}{zeros}\PY{p}{(}\PY{p}{[}\PY{n}{N}\PY{p}{,}\PY{n}{N}\PY{p}{]}\PY{p}{,}\PY{n+nb}{float}\PY{p}{)}
             
             \PY{c+c1}{\PYZsh{}Gaussian elimination}
             \PY{k}{for} \PY{n}{m} \PY{o+ow}{in} \PY{n+nb}{range}\PY{p}{(}\PY{n}{N}\PY{p}{)}\PY{p}{:}
                 
                 \PY{k}{for} \PY{n}{i} \PY{o+ow}{in} \PY{n+nb}{range}\PY{p}{(}\PY{n}{m}\PY{p}{,} \PY{n}{N}\PY{p}{)}\PY{p}{:}
                     \PY{n}{L}\PY{p}{[}\PY{n}{i}\PY{p}{,}\PY{n}{m}\PY{p}{]} \PY{o}{=} \PY{n}{U}\PY{p}{[}\PY{n}{i}\PY{p}{,}\PY{n}{m}\PY{p}{]}
         
                 \PY{c+c1}{\PYZsh{}Division by diagonal element}
                 \PY{n}{div} \PY{o}{=} \PY{n}{U}\PY{p}{[}\PY{n}{m}\PY{p}{,}\PY{n}{m}\PY{p}{]}
                 \PY{n}{U}\PY{p}{[}\PY{n}{m}\PY{p}{,}\PY{p}{:}\PY{p}{]} \PY{o}{/}\PY{o}{=} \PY{n}{div}
         
                 \PY{c+c1}{\PYZsh{}subtract from lower rows}
                 \PY{k}{for} \PY{n}{i} \PY{o+ow}{in} \PY{n+nb}{range}\PY{p}{(}\PY{n}{m}\PY{o}{+}\PY{l+m+mi}{1}\PY{p}{,} \PY{n}{N}\PY{p}{)}\PY{p}{:}
                     \PY{n}{mult} \PY{o}{=} \PY{n}{U}\PY{p}{[}\PY{n}{i}\PY{p}{,}\PY{n}{m}\PY{p}{]}
                     \PY{n}{U}\PY{p}{[}\PY{n}{i}\PY{p}{,}\PY{p}{:}\PY{p}{]} \PY{o}{\PYZhy{}}\PY{o}{=} \PY{n}{mult} \PY{o}{*} \PY{n}{U}\PY{p}{[}\PY{n}{m}\PY{p}{,}\PY{p}{:}\PY{p}{]}
         
             \PY{k}{return} \PY{n}{L}\PY{p}{,}\PY{n}{U}
         
         \PY{n}{L}\PY{p}{,} \PY{n}{U} \PY{o}{=} \PY{n}{LU}\PY{p}{(}\PY{n}{A}\PY{p}{)}
         
         \PY{n}{L}\PY{n+nd}{@U}
\end{Verbatim}

\begin{Verbatim}[commandchars=\\\{\}]
{\color{outcolor}Out[{\color{outcolor}14}]:} array([[ 2.,  1.,  4.,  1.],
                [ 3.,  4., -1., -1.],
                [ 1., -4.,  1.,  5.],
                [ 2., -2.,  1.,  3.]])
\end{Verbatim}
            
    So, we recovered the original matrix \(A\) when the matrices \(L\) and
\(U\) are multiplied together. Now, implementing double
backsubstitution, the equation \(A\mathbf{x} = \mathbf{v}\) can be
solved as

\[LU\mathbf{x} = \mathbf{v}\] in two steps such that
\[U\mathbf{x} = \mathbf{y} \text{ and } L\mathbf{y} = \mathbf{v}.\]

    \begin{Verbatim}[commandchars=\\\{\}]
{\color{incolor}In [{\color{incolor}15}]:} \PY{c+c1}{\PYZsh{}LU decomposition with double backsubstitution}
         \PY{k}{def} \PY{n+nf}{LU\PYZus{}with\PYZus{}DB}\PY{p}{(}\PY{n}{A}\PY{p}{,} \PY{n}{v}\PY{p}{)}\PY{p}{:}
             \PY{l+s+sd}{\PYZdq{}\PYZdq{}\PYZdq{}Defines a function to perform LU decomposition}
         \PY{l+s+sd}{        on a matrix A based on gausslim.py from book.}
         \PY{l+s+sd}{        Then uses double backsubstitution to solve}
         \PY{l+s+sd}{        system LUx = v\PYZdq{}\PYZdq{}\PYZdq{}}
             
             \PY{n}{N} \PY{o}{=} \PY{n+nb}{len}\PY{p}{(}\PY{n}{A}\PY{p}{)}
             \PY{n}{U} \PY{o}{=} \PY{n}{np}\PY{o}{.}\PY{n}{copy}\PY{p}{(}\PY{n}{A}\PY{p}{)}
             \PY{n}{L} \PY{o}{=} \PY{n}{np}\PY{o}{.}\PY{n}{zeros}\PY{p}{(}\PY{p}{[}\PY{n}{N}\PY{p}{,}\PY{n}{N}\PY{p}{]}\PY{p}{,}\PY{n+nb}{float}\PY{p}{)}
             \PY{n}{w} \PY{o}{=} \PY{n}{np}\PY{o}{.}\PY{n}{copy}\PY{p}{(}\PY{n}{v}\PY{p}{)}
             
             \PY{c+c1}{\PYZsh{}Gaussian elimination}
             \PY{k}{for} \PY{n}{m} \PY{o+ow}{in} \PY{n+nb}{range}\PY{p}{(}\PY{n}{N}\PY{p}{)}\PY{p}{:}
                 
                 \PY{k}{for} \PY{n}{i} \PY{o+ow}{in} \PY{n+nb}{range}\PY{p}{(}\PY{n}{m}\PY{p}{,} \PY{n}{N}\PY{p}{)}\PY{p}{:}
                     \PY{n}{L}\PY{p}{[}\PY{n}{i}\PY{p}{,}\PY{n}{m}\PY{p}{]} \PY{o}{=} \PY{n}{U}\PY{p}{[}\PY{n}{i}\PY{p}{,}\PY{n}{m}\PY{p}{]}
         
                 \PY{c+c1}{\PYZsh{}Division by diagonal element}
                 \PY{n}{div} \PY{o}{=} \PY{n}{U}\PY{p}{[}\PY{n}{m}\PY{p}{,}\PY{n}{m}\PY{p}{]}
                 \PY{n}{U}\PY{p}{[}\PY{n}{m}\PY{p}{,}\PY{p}{:}\PY{p}{]} \PY{o}{/}\PY{o}{=} \PY{n}{div}
                 \PY{n}{w}\PY{p}{[}\PY{n}{m}\PY{p}{]} \PY{o}{/}\PY{o}{=} \PY{n}{div}
                 
                 \PY{c+c1}{\PYZsh{}subtract from lower rows}
                 \PY{k}{for} \PY{n}{i} \PY{o+ow}{in} \PY{n+nb}{range}\PY{p}{(}\PY{n}{m}\PY{o}{+}\PY{l+m+mi}{1}\PY{p}{,} \PY{n}{N}\PY{p}{)}\PY{p}{:}
                     \PY{n}{mult} \PY{o}{=} \PY{n}{U}\PY{p}{[}\PY{n}{i}\PY{p}{,}\PY{n}{m}\PY{p}{]}
                     \PY{n}{U}\PY{p}{[}\PY{n}{i}\PY{p}{,}\PY{p}{:}\PY{p}{]} \PY{o}{\PYZhy{}}\PY{o}{=} \PY{n}{mult} \PY{o}{*} \PY{n}{U}\PY{p}{[}\PY{n}{m}\PY{p}{,}\PY{p}{:}\PY{p}{]}
                     \PY{n}{w}\PY{p}{[}\PY{n}{i}\PY{p}{]} \PY{o}{\PYZhy{}}\PY{o}{=} \PY{n}{mult} \PY{o}{*} \PY{n}{w}\PY{p}{[}\PY{n}{m}\PY{p}{]}
         
             \PY{c+c1}{\PYZsh{}first backsubstitution}
             \PY{n}{y} \PY{o}{=} \PY{n}{np}\PY{o}{.}\PY{n}{empty}\PY{p}{(}\PY{n}{N}\PY{p}{,} \PY{n+nb}{float}\PY{p}{)}
             \PY{k}{for} \PY{n}{m} \PY{o+ow}{in} \PY{n+nb}{range}\PY{p}{(}\PY{n}{N}\PY{o}{\PYZhy{}}\PY{l+m+mi}{1}\PY{p}{,} \PY{o}{\PYZhy{}}\PY{l+m+mi}{1}\PY{p}{,} \PY{o}{\PYZhy{}}\PY{l+m+mi}{1}\PY{p}{)}\PY{p}{:}
                 \PY{n}{y}\PY{p}{[}\PY{n}{m}\PY{p}{]} \PY{o}{=} \PY{n}{w}\PY{p}{[}\PY{n}{m}\PY{p}{]}
                 \PY{k}{for} \PY{n}{i} \PY{o+ow}{in} \PY{n+nb}{range}\PY{p}{(}\PY{n}{m}\PY{o}{+}\PY{l+m+mi}{1}\PY{p}{,} \PY{n}{N}\PY{p}{)}\PY{p}{:}
                     \PY{n}{y}\PY{p}{[}\PY{n}{m}\PY{p}{]} \PY{o}{\PYZhy{}}\PY{o}{=} \PY{n}{U}\PY{p}{[}\PY{n}{m}\PY{p}{,}\PY{n}{i}\PY{p}{]} \PY{o}{*} \PY{n}{y}\PY{p}{[}\PY{n}{i}\PY{p}{]}
                     
             \PY{c+c1}{\PYZsh{}second backsubstitution}
             \PY{n}{x} \PY{o}{=} \PY{n}{np}\PY{o}{.}\PY{n}{empty}\PY{p}{(}\PY{n}{N}\PY{p}{,} \PY{n+nb}{float}\PY{p}{)}
             \PY{k}{for} \PY{n}{m} \PY{o+ow}{in} \PY{n+nb}{range}\PY{p}{(}\PY{n}{N}\PY{o}{\PYZhy{}}\PY{l+m+mi}{1}\PY{p}{,} \PY{o}{\PYZhy{}}\PY{l+m+mi}{1}\PY{p}{,} \PY{o}{\PYZhy{}}\PY{l+m+mi}{1}\PY{p}{)}\PY{p}{:}
                 \PY{n}{x}\PY{p}{[}\PY{n}{m}\PY{p}{]} \PY{o}{=} \PY{n}{y}\PY{p}{[}\PY{n}{m}\PY{p}{]}
                 \PY{k}{for} \PY{n}{i} \PY{o+ow}{in} \PY{n+nb}{range}\PY{p}{(}\PY{n}{m}\PY{o}{+}\PY{l+m+mi}{1}\PY{p}{,} \PY{n}{N}\PY{p}{)}\PY{p}{:}
                     \PY{n}{x}\PY{p}{[}\PY{n}{m}\PY{p}{]} \PY{o}{\PYZhy{}}\PY{o}{=} \PY{n}{L}\PY{p}{[}\PY{n}{m}\PY{p}{,}\PY{n}{i}\PY{p}{]} \PY{o}{*} \PY{n}{x}\PY{p}{[}\PY{n}{i}\PY{p}{]}
                       
             \PY{k}{return} \PY{n}{x}
         
         \PY{n}{LU\PYZus{}with\PYZus{}DB}\PY{p}{(}\PY{n}{A}\PY{p}{,}\PY{n}{v}\PY{p}{)}
\end{Verbatim}

\begin{Verbatim}[commandchars=\\\{\}]
{\color{outcolor}Out[{\color{outcolor}15}]:} array([ 2., -1., -2.,  1.])
\end{Verbatim}
            
    So we arrive at the solution \(\mathbf{x} = \begin{pmatrix} 2, \
-1, \
-2, \
1 \end{pmatrix}^\text{T},\) which is confirmed by the solve function
from the numpy library below.

    \begin{Verbatim}[commandchars=\\\{\}]
{\color{incolor}In [{\color{incolor}16}]:} \PY{n}{LA}\PY{o}{.}\PY{n}{solve}\PY{p}{(}\PY{n}{A}\PY{p}{,}\PY{n}{v}\PY{p}{)}
\end{Verbatim}

\begin{Verbatim}[commandchars=\\\{\}]
{\color{outcolor}Out[{\color{outcolor}16}]:} array([ 2., -1., -2.,  1.])
\end{Verbatim}
            
    \section{CP 6.4 A circuit of resistors
(revisited)}\label{cp-6.4-a-circuit-of-resistors-revisited}

Now, I'll solve the same circuit from exercise 6.1 using the built-in
functionality of numpy. The physical equations I derived still hold and
will form the matrix and solution space we solve for. The system of
equations is \[4V_1 - V_2 - V_3 - V_4 = V_+\] \[3V_2 - V_1 - V_4 = 0\]
\[3V_3 - V_1 - V_4 = V_+\] \[4V_4 - V_1 - V_2 - V_3 = 0\] where again we
have it in matrix form given by \(A\mathbf{x} = \mathbf{v}\) with

\(A = \begin{pmatrix}  4 & -1 & -1 & -1 \\  -1 & 3 & 0 & -1 \\  -1 & 0 & 3 & -1 \\  -1 & -1 & -1 & 4 \end{pmatrix} \text{, }\)
\(\mathbf{x} = \begin{pmatrix} V_1 \\ V_2 \\ V_3 \\ V_4 \end{pmatrix} \text{, and }\)
\(\mathbf{v} = \begin{pmatrix} V_+ \\ 0 \\ V_+ \\ 0 \end{pmatrix}.\)

    \begin{Verbatim}[commandchars=\\\{\}]
{\color{incolor}In [{\color{incolor}17}]:} \PY{o}{\PYZpc{}\PYZpc{}}\PY{k}{time}
         
         Vp = 5 \PYZsh{}volts
         A = np.array([[4,\PYZhy{}1,\PYZhy{}1,\PYZhy{}1],
                       [\PYZhy{}1,3,0,\PYZhy{}1],
                       [\PYZhy{}1,0,3,\PYZhy{}1],
                       [\PYZhy{}1,\PYZhy{}1,\PYZhy{}1,4]], float)
         v = np.array([Vp,0,Vp,0], float)
         
         x = LA.solve(A,v)
         x
\end{Verbatim}

    \begin{Verbatim}[commandchars=\\\{\}]
CPU times: user 463 µs, sys: 270 µs, total: 733 µs
Wall time: 596 µs

    \end{Verbatim}

    This is the same answer given to us by the "hand made" Gaussian
elimination function whereas in this case it took approximately a fourth
to a fifth of the time to run, which was negligible for such a small
system, but could prove importantly different for larger cases.

    \section{CP 6.7 A chain of resistors}\label{cp-6.7-a-chain-of-resistors}

Ohm's law tells us that \(V = IR\), and Kirchhoff's current junction law
says that \(\sum_i I_i = 0\) for a given junction in a circuit. So by
Ohm's law, the currents flowing into the junction 1 are
\(\frac{V_1 - V_+}{R},\) \(\frac{V_1 - V_2}{R},\) and
\(\frac{V_1 - V_3}{R}.\) So the Kirchhoff's junction rule for 1 gives
the equation

\[\frac{V_1 - V_+}{R} + \frac{V_1 - V_2}{R} + \frac{V_1 - V_3}{R} = 0.\]

Multiplying through by the common denominator \(R\) and collecting like
terms and unknowns gives the first equation in the system provided in
the book:

\[3V_1 - V_2 -V_3 = V_+.\]

The \(1^\text{st}\) and \(N^\text{th}\) junction are the edge cases
where only three wires connect to the junction and hence only three
currents flow in/out. Every other junction in the chain will have four
currents flowing in/out that correspond to currents flowing from the
\(i^\text{th} -2\), \(i^\text{th} -1\), \(i^\text{th} +1\), and
\(i^\text{th}+2\) junctions. Similarly, \(V_+\) will not be the
potential at an adjacent junction for any other than the first and
second, so the second also becomes an edge case. We can write the
general equation as

\[\frac{V_i - V_{i-2}}{R} + \frac{V_i - V_{i-1}}{R} + \frac{V_i - V_{i+1}}{R} + \frac{V_i - V_{i+2}}{R} = 0.\]

Once we have this, the system provided in the book is clearly reached
where

\[-V_{i-2} - V_{i-1} + 4V_i - V_{i+1} - V_{i+2} = 0.\]

We can rewrite this system of equations in matrix form that satisfies
the relation \(A\mathbf{v} = \mathbf{w}.\) These are given by
\(A\mathbf{x} = \mathbf{v}\) with

\(A = \begin{pmatrix}  3 & -1 & -1 & 0 & 0 & 0 & \cdots & 0 \\  -1 & 4 & -1 & -1 & 0 & 0 & \cdots & 0 \\  -1 & -1 & 4 & -1 & -1& 0 & \cdots & 0 \\  0 &-1 & -1 & 4 & -1 & -1& \cdots & 0 \\  \vdots & \vdots & \vdots & \vdots & \vdots &\vdots & \ddots & \vdots \\  0 &\cdots& -1&-1&4&-1&-1&0\\  0&\cdots&0&-1&-1&4&-1&-1 \\  0&\cdots&0&0&-1 & -1 & 4 & -1 \\  0 & \cdots &0&0& 0 & -1 & -1 & 3 \end{pmatrix} \text{, }\)
\(\mathbf{v} = \begin{pmatrix} V_1 \\ V_2 \\ V_3 \\ V_4 \\ \vdots \\ V_{N-3} \\ V_{N-2} \\ V_{N-1} \\ V_N \end{pmatrix} \text{, and }\)
\(\mathbf{w} = \begin{pmatrix} V_+ \\ V_+ \\ 0 \\ 0 \\ 0 \\ \vdots \\ 0 \\ 0 \\ 0 \end{pmatrix}.\)

Because all the points in the circuit are connected to at most 4 other
points, many of the off-diagonal elements are zero. We see that are
matrix of coefficients \(A\) is banded and can thus be solved in a less
computationally intensive manner than iterating through every element.

    \begin{Verbatim}[commandchars=\\\{\}]
{\color{incolor}In [{\color{incolor}29}]:} \PY{l+s+sd}{\PYZdq{}\PYZdq{}\PYZdq{}Solving a banded system of linear equations as defined}
         \PY{l+s+sd}{    }
         \PY{l+s+sd}{    by Mark Newman used for the following problems in the}
         \PY{l+s+sd}{    }
         \PY{l+s+sd}{    homework. I edited the numpy functions, adding the}
         \PY{l+s+sd}{    }
         \PY{l+s+sd}{    prefix \PYZsq{}np.\PYZsq{} so it would run properly in my notebook\PYZsq{}\PYZdq{}\PYZdq{}\PYZdq{}}
         
         \PY{k}{def} \PY{n+nf}{banded}\PY{p}{(}\PY{n}{Aa}\PY{p}{,}\PY{n}{va}\PY{p}{,}\PY{n}{up}\PY{p}{,}\PY{n}{down}\PY{p}{)}\PY{p}{:}
         
             \PY{c+c1}{\PYZsh{} Copy the inputs and determine the size of the system}
             \PY{n}{A} \PY{o}{=} \PY{n}{np}\PY{o}{.}\PY{n}{copy}\PY{p}{(}\PY{n}{Aa}\PY{p}{)}
             \PY{n}{v} \PY{o}{=} \PY{n}{np}\PY{o}{.}\PY{n}{copy}\PY{p}{(}\PY{n}{va}\PY{p}{)}
             \PY{n}{N} \PY{o}{=} \PY{n+nb}{len}\PY{p}{(}\PY{n}{v}\PY{p}{)}
         
             \PY{c+c1}{\PYZsh{} Gaussian elimination}
             \PY{k}{for} \PY{n}{m} \PY{o+ow}{in} \PY{n+nb}{range}\PY{p}{(}\PY{n}{N}\PY{p}{)}\PY{p}{:}
         
                 \PY{c+c1}{\PYZsh{} Normalization factor}
                 \PY{n}{div} \PY{o}{=} \PY{n}{A}\PY{p}{[}\PY{n}{up}\PY{p}{,}\PY{n}{m}\PY{p}{]}
         
                 \PY{c+c1}{\PYZsh{} Update the vector first}
                 \PY{n}{v}\PY{p}{[}\PY{n}{m}\PY{p}{]} \PY{o}{/}\PY{o}{=} \PY{n}{div}
                 \PY{k}{for} \PY{n}{k} \PY{o+ow}{in} \PY{n+nb}{range}\PY{p}{(}\PY{l+m+mi}{1}\PY{p}{,}\PY{n}{down}\PY{o}{+}\PY{l+m+mi}{1}\PY{p}{)}\PY{p}{:}
                     \PY{k}{if} \PY{n}{m}\PY{o}{+}\PY{n}{k}\PY{o}{\PYZlt{}}\PY{n}{N}\PY{p}{:}
                         \PY{n}{v}\PY{p}{[}\PY{n}{m}\PY{o}{+}\PY{n}{k}\PY{p}{]} \PY{o}{\PYZhy{}}\PY{o}{=} \PY{n}{A}\PY{p}{[}\PY{n}{up}\PY{o}{+}\PY{n}{k}\PY{p}{,}\PY{n}{m}\PY{p}{]}\PY{o}{*}\PY{n}{v}\PY{p}{[}\PY{n}{m}\PY{p}{]}
         
                 \PY{c+c1}{\PYZsh{} Now normalize the pivot row of A and subtract from lower ones}
                 \PY{k}{for} \PY{n}{i} \PY{o+ow}{in} \PY{n+nb}{range}\PY{p}{(}\PY{n}{up}\PY{p}{)}\PY{p}{:}
                     \PY{n}{j} \PY{o}{=} \PY{n}{m} \PY{o}{+} \PY{n}{up} \PY{o}{\PYZhy{}} \PY{n}{i}
                     \PY{k}{if} \PY{n}{j}\PY{o}{\PYZlt{}}\PY{n}{N}\PY{p}{:}
                         \PY{n}{A}\PY{p}{[}\PY{n}{i}\PY{p}{,}\PY{n}{j}\PY{p}{]} \PY{o}{/}\PY{o}{=} \PY{n}{div}
                         \PY{k}{for} \PY{n}{k} \PY{o+ow}{in} \PY{n+nb}{range}\PY{p}{(}\PY{l+m+mi}{1}\PY{p}{,}\PY{n}{down}\PY{o}{+}\PY{l+m+mi}{1}\PY{p}{)}\PY{p}{:}
                             \PY{n}{A}\PY{p}{[}\PY{n}{i}\PY{o}{+}\PY{n}{k}\PY{p}{,}\PY{n}{j}\PY{p}{]} \PY{o}{\PYZhy{}}\PY{o}{=} \PY{n}{A}\PY{p}{[}\PY{n}{up}\PY{o}{+}\PY{n}{k}\PY{p}{,}\PY{n}{m}\PY{p}{]}\PY{o}{*}\PY{n}{A}\PY{p}{[}\PY{n}{i}\PY{p}{,}\PY{n}{j}\PY{p}{]}
         
             \PY{c+c1}{\PYZsh{} Backsubstitution}
             \PY{k}{for} \PY{n}{m} \PY{o+ow}{in} \PY{n+nb}{range}\PY{p}{(}\PY{n}{N}\PY{o}{\PYZhy{}}\PY{l+m+mi}{2}\PY{p}{,}\PY{o}{\PYZhy{}}\PY{l+m+mi}{1}\PY{p}{,}\PY{o}{\PYZhy{}}\PY{l+m+mi}{1}\PY{p}{)}\PY{p}{:}
                 \PY{k}{for} \PY{n}{i} \PY{o+ow}{in} \PY{n+nb}{range}\PY{p}{(}\PY{n}{up}\PY{p}{)}\PY{p}{:}
                     \PY{n}{j} \PY{o}{=} \PY{n}{m} \PY{o}{+} \PY{n}{up} \PY{o}{\PYZhy{}} \PY{n}{i}
                     \PY{k}{if} \PY{n}{j}\PY{o}{\PYZlt{}}\PY{n}{N}\PY{p}{:}
                         \PY{n}{v}\PY{p}{[}\PY{n}{m}\PY{p}{]} \PY{o}{\PYZhy{}}\PY{o}{=} \PY{n}{A}\PY{p}{[}\PY{n}{i}\PY{p}{,}\PY{n}{j}\PY{p}{]}\PY{o}{*}\PY{n}{v}\PY{p}{[}\PY{n}{j}\PY{p}{]}
         
             \PY{k}{return} \PY{n}{v}
\end{Verbatim}

    \begin{Verbatim}[commandchars=\\\{\}]
{\color{incolor}In [{\color{incolor}30}]:} \PY{c+c1}{\PYZsh{}my banded system function solver}
         \PY{k}{def} \PY{n+nf}{banded\PYZus{}solve}\PY{p}{(}\PY{n}{Aa}\PY{p}{,}\PY{n}{va}\PY{p}{,}\PY{n}{bw}\PY{p}{)}\PY{p}{:}
             \PY{l+s+sd}{\PYZdq{}\PYZdq{}\PYZdq{}Solves banded system of linear equations}
         \PY{l+s+sd}{        with bandwidth bw above and below diagonal.}
         \PY{l+s+sd}{        Takes an NxN matrix, vector, and number of}
         \PY{l+s+sd}{        elements above and below diagonal to look at}
         \PY{l+s+sd}{        as arguments.\PYZdq{}\PYZdq{}\PYZdq{}}
         
             \PY{c+c1}{\PYZsh{} Copy the inputs and determine the size of the system}
             \PY{n}{A} \PY{o}{=} \PY{n}{np}\PY{o}{.}\PY{n}{copy}\PY{p}{(}\PY{n}{Aa}\PY{p}{)}
             \PY{n}{v} \PY{o}{=} \PY{n}{np}\PY{o}{.}\PY{n}{copy}\PY{p}{(}\PY{n}{va}\PY{p}{)}
             \PY{n}{N} \PY{o}{=} \PY{n+nb}{len}\PY{p}{(}\PY{n}{v}\PY{p}{)}
             \PY{n}{s} \PY{o}{=} \PY{l+m+mi}{0} \PY{c+c1}{\PYZsh{}variable to have something to do in except clause}
             
             \PY{c+c1}{\PYZsh{}Gaussian elimination program w/out partial pivoting}
             \PY{k}{for} \PY{n}{m} \PY{o+ow}{in} \PY{n+nb}{range}\PY{p}{(}\PY{n}{N}\PY{p}{)}\PY{p}{:}
         
                 \PY{c+c1}{\PYZsh{}Division by diagonal element}
                 \PY{n}{div} \PY{o}{=} \PY{n}{A}\PY{p}{[}\PY{n}{m}\PY{p}{,}\PY{n}{m}\PY{p}{]}
                 \PY{n}{A}\PY{p}{[}\PY{n}{m}\PY{p}{,}\PY{n}{m}\PY{p}{]} \PY{o}{/}\PY{o}{=} \PY{n}{div}
                 
                 \PY{k}{try}\PY{p}{:}
                     \PY{k}{for} \PY{n}{j} \PY{o+ow}{in} \PY{n+nb}{range}\PY{p}{(}\PY{n}{m}\PY{o}{\PYZhy{}}\PY{n}{bw}\PY{p}{,}\PY{n}{m}\PY{o}{+}\PY{n}{bw}\PY{o}{+}\PY{l+m+mi}{1}\PY{p}{)}\PY{p}{:}
                         \PY{n}{A}\PY{p}{[}\PY{n}{m}\PY{p}{,} \PY{n}{j}\PY{p}{]} \PY{o}{/}\PY{o}{=} \PY{n}{div}
                 \PY{k}{except}\PY{p}{:}
                     \PY{n}{s} \PY{o}{+}\PY{o}{=} \PY{l+m+mi}{1}
                     
                 \PY{n}{v}\PY{p}{[}\PY{n}{m}\PY{p}{]} \PY{o}{/}\PY{o}{=} \PY{n}{div}
         
                 \PY{c+c1}{\PYZsh{}subtract from lower rows}
                 \PY{k}{for} \PY{n}{i} \PY{o+ow}{in} \PY{n+nb}{range}\PY{p}{(}\PY{n}{m}\PY{o}{+}\PY{l+m+mi}{1}\PY{p}{,} \PY{n}{N}\PY{p}{)}\PY{p}{:}
                     \PY{n}{mult} \PY{o}{=} \PY{n}{A}\PY{p}{[}\PY{n}{i}\PY{p}{,}\PY{n}{m}\PY{p}{]}
                     \PY{n}{A}\PY{p}{[}\PY{n}{i}\PY{p}{,}\PY{n}{m}\PY{p}{]} \PY{o}{\PYZhy{}}\PY{o}{=} \PY{n}{mult} \PY{o}{*} \PY{n}{A}\PY{p}{[}\PY{n}{m}\PY{p}{,}\PY{n}{m}\PY{p}{]}
                     
                     \PY{k}{try}\PY{p}{:}
                         \PY{k}{for} \PY{n}{j} \PY{o+ow}{in} \PY{n+nb}{range}\PY{p}{(}\PY{n}{m}\PY{o}{\PYZhy{}}\PY{n}{bw}\PY{p}{,}\PY{n}{m}\PY{o}{+}\PY{n}{bw}\PY{o}{+}\PY{l+m+mi}{1}\PY{p}{)}\PY{p}{:}
                             \PY{n}{A}\PY{p}{[}\PY{n}{i}\PY{p}{,}\PY{n}{j}\PY{p}{]} \PY{o}{\PYZhy{}}\PY{o}{=} \PY{n}{mult} \PY{o}{*} \PY{n}{A}\PY{p}{[}\PY{n}{m}\PY{p}{,}\PY{n}{j}\PY{p}{]}
                     \PY{k}{except}\PY{p}{:}
                         \PY{n}{s} \PY{o}{+}\PY{o}{=} \PY{l+m+mi}{1}
                         
                     \PY{n}{v}\PY{p}{[}\PY{n}{i}\PY{p}{]} \PY{o}{\PYZhy{}}\PY{o}{=} \PY{n}{mult} \PY{o}{*} \PY{n}{v}\PY{p}{[}\PY{n}{m}\PY{p}{]}
         
             \PY{c+c1}{\PYZsh{}backsubstitution}
             \PY{n}{x} \PY{o}{=} \PY{n}{np}\PY{o}{.}\PY{n}{empty}\PY{p}{(}\PY{n}{N}\PY{p}{,} \PY{n+nb}{float}\PY{p}{)}
             \PY{k}{for} \PY{n}{m} \PY{o+ow}{in} \PY{n+nb}{range}\PY{p}{(}\PY{n}{N}\PY{o}{\PYZhy{}}\PY{l+m+mi}{1}\PY{p}{,} \PY{o}{\PYZhy{}}\PY{l+m+mi}{1}\PY{p}{,} \PY{o}{\PYZhy{}}\PY{l+m+mi}{1}\PY{p}{)}\PY{p}{:}
                 \PY{n}{x}\PY{p}{[}\PY{n}{m}\PY{p}{]} \PY{o}{=} \PY{n}{v}\PY{p}{[}\PY{n}{m}\PY{p}{]}
                 \PY{k}{for} \PY{n}{i} \PY{o+ow}{in} \PY{n+nb}{range}\PY{p}{(}\PY{n}{m}\PY{o}{+}\PY{l+m+mi}{1}\PY{p}{,} \PY{n}{N}\PY{p}{)}\PY{p}{:}
                     \PY{n}{x}\PY{p}{[}\PY{n}{m}\PY{p}{]} \PY{o}{\PYZhy{}}\PY{o}{=} \PY{n}{A}\PY{p}{[}\PY{n}{m}\PY{p}{,}\PY{n}{i}\PY{p}{]} \PY{o}{*} \PY{n}{x}\PY{p}{[}\PY{n}{i}\PY{p}{]}
         
             \PY{k}{return} \PY{n}{x}
\end{Verbatim}

    \begin{Verbatim}[commandchars=\\\{\}]
{\color{incolor}In [{\color{incolor}31}]:} \PY{c+c1}{\PYZsh{}cell to define problem specific matrices and vectors}
         \PY{n}{Vp} \PY{o}{=} \PY{l+m+mi}{5} \PY{c+c1}{\PYZsh{}volts}
         \PY{n}{N} \PY{o}{=} \PY{l+m+mi}{6}
         
         \PY{c+c1}{\PYZsh{}matrix for my function}
         \PY{n}{A} \PY{o}{=} \PY{n}{np}\PY{o}{.}\PY{n}{zeros}\PY{p}{(}\PY{p}{[}\PY{n}{N}\PY{p}{,}\PY{n}{N}\PY{p}{]}\PY{p}{,}\PY{n+nb}{float}\PY{p}{)}
         \PY{k}{for} \PY{n}{i} \PY{o+ow}{in} \PY{n+nb}{range}\PY{p}{(}\PY{n}{N}\PY{o}{\PYZhy{}}\PY{l+m+mi}{2}\PY{p}{)}\PY{p}{:}
             \PY{n}{A}\PY{p}{[}\PY{n}{i}\PY{p}{,}\PY{n}{i}\PY{p}{]} \PY{o}{=} \PY{l+m+mi}{4}
             \PY{n}{A}\PY{p}{[}\PY{n}{i}\PY{p}{,}\PY{n}{i}\PY{o}{+}\PY{l+m+mi}{1}\PY{p}{]} \PY{o}{=} \PY{o}{\PYZhy{}}\PY{l+m+mi}{1}
             \PY{n}{A}\PY{p}{[}\PY{n}{i}\PY{p}{,}\PY{n}{i}\PY{o}{+}\PY{l+m+mi}{2}\PY{p}{]} \PY{o}{=} \PY{o}{\PYZhy{}}\PY{l+m+mi}{1}
             \PY{n}{A}\PY{p}{[}\PY{n}{i}\PY{o}{+}\PY{l+m+mi}{1}\PY{p}{,}\PY{n}{i}\PY{p}{]} \PY{o}{=} \PY{o}{\PYZhy{}}\PY{l+m+mi}{1}
             \PY{n}{A}\PY{p}{[}\PY{n}{i}\PY{o}{+}\PY{l+m+mi}{2}\PY{p}{,}\PY{n}{i}\PY{p}{]} \PY{o}{=} \PY{o}{\PYZhy{}}\PY{l+m+mi}{1}
         
         \PY{n}{A}\PY{p}{[}\PY{l+m+mi}{0}\PY{p}{,}\PY{l+m+mi}{0}\PY{p}{]} \PY{o}{=} \PY{l+m+mi}{3}
         \PY{n}{A}\PY{p}{[}\PY{n}{N}\PY{o}{\PYZhy{}}\PY{l+m+mi}{1}\PY{p}{,}\PY{n}{N}\PY{o}{\PYZhy{}}\PY{l+m+mi}{1}\PY{p}{]} \PY{o}{=} \PY{l+m+mi}{3}
         \PY{n}{A}\PY{p}{[}\PY{n}{N}\PY{o}{\PYZhy{}}\PY{l+m+mi}{2}\PY{p}{,}\PY{n}{N}\PY{o}{\PYZhy{}}\PY{l+m+mi}{2}\PY{p}{]} \PY{o}{=} \PY{l+m+mi}{4}
         \PY{n}{A}\PY{p}{[}\PY{n}{N}\PY{o}{\PYZhy{}}\PY{l+m+mi}{2}\PY{p}{,}\PY{n}{N}\PY{o}{\PYZhy{}}\PY{l+m+mi}{1}\PY{p}{]} \PY{o}{=} \PY{o}{\PYZhy{}}\PY{l+m+mi}{1}
         \PY{n}{A}\PY{p}{[}\PY{n}{N}\PY{o}{\PYZhy{}}\PY{l+m+mi}{1}\PY{p}{,}\PY{n}{N}\PY{o}{\PYZhy{}}\PY{l+m+mi}{2}\PY{p}{]} \PY{o}{=} \PY{o}{\PYZhy{}}\PY{l+m+mi}{1}
         
         
         \PY{c+c1}{\PYZsh{}matrix for banded.py function}
         \PY{n}{B} \PY{o}{=} \PY{n}{np}\PY{o}{.}\PY{n}{empty}\PY{p}{(}\PY{p}{[}\PY{l+m+mi}{5}\PY{p}{,}\PY{n}{N}\PY{p}{]}\PY{p}{,}\PY{n+nb}{float}\PY{p}{)}
         \PY{n}{B}\PY{p}{[}\PY{l+m+mi}{0}\PY{p}{,}\PY{p}{:}\PY{p}{]} \PY{o}{=} \PY{o}{\PYZhy{}}\PY{l+m+mi}{1}
         \PY{n}{B}\PY{p}{[}\PY{l+m+mi}{1}\PY{p}{,}\PY{p}{:}\PY{p}{]} \PY{o}{=} \PY{o}{\PYZhy{}}\PY{l+m+mi}{1}
         \PY{n}{B}\PY{p}{[}\PY{l+m+mi}{2}\PY{p}{,}\PY{p}{:}\PY{p}{]} \PY{o}{=} \PY{l+m+mi}{4}
         \PY{n}{B}\PY{p}{[}\PY{l+m+mi}{3}\PY{p}{,}\PY{p}{:}\PY{p}{]} \PY{o}{=} \PY{o}{\PYZhy{}}\PY{l+m+mi}{1}
         \PY{n}{B}\PY{p}{[}\PY{l+m+mi}{4}\PY{p}{,}\PY{p}{:}\PY{p}{]} \PY{o}{=} \PY{o}{\PYZhy{}}\PY{l+m+mi}{1}
         
         \PY{n}{B}\PY{p}{[}\PY{l+m+mi}{2}\PY{p}{,}\PY{l+m+mi}{0}\PY{p}{]} \PY{o}{=} \PY{l+m+mi}{3}
         \PY{n}{B}\PY{p}{[}\PY{l+m+mi}{2}\PY{p}{,}\PY{n}{N}\PY{o}{\PYZhy{}}\PY{l+m+mi}{1}\PY{p}{]} \PY{o}{=} \PY{l+m+mi}{3}
         
         \PY{n}{w} \PY{o}{=} \PY{n}{np}\PY{o}{.}\PY{n}{zeros}\PY{p}{(}\PY{n}{N}\PY{p}{,}\PY{n+nb}{float}\PY{p}{)}
         \PY{n}{w}\PY{p}{[}\PY{l+m+mi}{0}\PY{p}{]} \PY{o}{=} \PY{n}{Vp}
         \PY{n}{w}\PY{p}{[}\PY{l+m+mi}{1}\PY{p}{]} \PY{o}{=} \PY{n}{Vp}
         
         \PY{n+nb}{print}\PY{p}{(}\PY{n}{banded}\PY{p}{(}\PY{n}{B}\PY{p}{,}\PY{n}{w}\PY{p}{,}\PY{l+m+mi}{2}\PY{p}{,}\PY{l+m+mi}{2}\PY{p}{)}\PY{p}{)}
         \PY{n+nb}{print}\PY{p}{(}\PY{n}{banded\PYZus{}solve}\PY{p}{(}\PY{n}{A}\PY{p}{,}\PY{n}{w}\PY{p}{,}\PY{l+m+mi}{2}\PY{p}{)}\PY{p}{)}
         \PY{n+nb}{print}\PY{p}{(}\PY{n}{LA}\PY{o}{.}\PY{n}{solve}\PY{p}{(}\PY{n}{A}\PY{p}{,}\PY{n}{w}\PY{p}{)}\PY{p}{)}
\end{Verbatim}

    \begin{Verbatim}[commandchars=\\\{\}]
[ 3.7254902   3.43137255  2.74509804  2.25490196  1.56862745  1.2745098 ]
[ 3.7254902   3.43137255  2.74509804  2.25490196  1.56862745  1.2745098 ]
[ 3.7254902   3.43137255  2.74509804  2.25490196  1.56862745  1.2745098 ]

    \end{Verbatim}

    So for the circuit where \(N=6\), the results of the function I wrote to
solve banded systems of linear equations match the results of Newman's
function and the function in the numpy.linalg library. For the case
where \(N=10000,\) the results also match the results of the other
functions. I'll graph the voltages at each point next.

    \begin{Verbatim}[commandchars=\\\{\}]
{\color{incolor}In [{\color{incolor}32}]:} \PY{c+c1}{\PYZsh{}cell to define problem specific matrices and vectors}
         \PY{n}{Vp} \PY{o}{=} \PY{l+m+mi}{5} \PY{c+c1}{\PYZsh{}volts}
         \PY{n}{N} \PY{o}{=} \PY{l+m+mi}{10000}
         
         \PY{c+c1}{\PYZsh{}matrix for my function}
         \PY{n}{A} \PY{o}{=} \PY{n}{np}\PY{o}{.}\PY{n}{zeros}\PY{p}{(}\PY{p}{[}\PY{n}{N}\PY{p}{,}\PY{n}{N}\PY{p}{]}\PY{p}{,}\PY{n+nb}{float}\PY{p}{)}
         \PY{k}{for} \PY{n}{i} \PY{o+ow}{in} \PY{n+nb}{range}\PY{p}{(}\PY{n}{N}\PY{o}{\PYZhy{}}\PY{l+m+mi}{2}\PY{p}{)}\PY{p}{:}
             \PY{n}{A}\PY{p}{[}\PY{n}{i}\PY{p}{,}\PY{n}{i}\PY{p}{]} \PY{o}{=} \PY{l+m+mi}{4}
             \PY{n}{A}\PY{p}{[}\PY{n}{i}\PY{p}{,}\PY{n}{i}\PY{o}{+}\PY{l+m+mi}{1}\PY{p}{]} \PY{o}{=} \PY{o}{\PYZhy{}}\PY{l+m+mi}{1}
             \PY{n}{A}\PY{p}{[}\PY{n}{i}\PY{p}{,}\PY{n}{i}\PY{o}{+}\PY{l+m+mi}{2}\PY{p}{]} \PY{o}{=} \PY{o}{\PYZhy{}}\PY{l+m+mi}{1}
             \PY{n}{A}\PY{p}{[}\PY{n}{i}\PY{o}{+}\PY{l+m+mi}{1}\PY{p}{,}\PY{n}{i}\PY{p}{]} \PY{o}{=} \PY{o}{\PYZhy{}}\PY{l+m+mi}{1}
             \PY{n}{A}\PY{p}{[}\PY{n}{i}\PY{o}{+}\PY{l+m+mi}{2}\PY{p}{,}\PY{n}{i}\PY{p}{]} \PY{o}{=} \PY{o}{\PYZhy{}}\PY{l+m+mi}{1}
         
         \PY{n}{A}\PY{p}{[}\PY{l+m+mi}{0}\PY{p}{,}\PY{l+m+mi}{0}\PY{p}{]} \PY{o}{=} \PY{l+m+mi}{3}
         \PY{n}{A}\PY{p}{[}\PY{n}{N}\PY{o}{\PYZhy{}}\PY{l+m+mi}{1}\PY{p}{,}\PY{n}{N}\PY{o}{\PYZhy{}}\PY{l+m+mi}{1}\PY{p}{]} \PY{o}{=} \PY{l+m+mi}{3}
         \PY{n}{A}\PY{p}{[}\PY{n}{N}\PY{o}{\PYZhy{}}\PY{l+m+mi}{2}\PY{p}{,}\PY{n}{N}\PY{o}{\PYZhy{}}\PY{l+m+mi}{2}\PY{p}{]} \PY{o}{=} \PY{l+m+mi}{4}
         \PY{n}{A}\PY{p}{[}\PY{n}{N}\PY{o}{\PYZhy{}}\PY{l+m+mi}{2}\PY{p}{,}\PY{n}{N}\PY{o}{\PYZhy{}}\PY{l+m+mi}{1}\PY{p}{]} \PY{o}{=} \PY{o}{\PYZhy{}}\PY{l+m+mi}{1}
         \PY{n}{A}\PY{p}{[}\PY{n}{N}\PY{o}{\PYZhy{}}\PY{l+m+mi}{1}\PY{p}{,}\PY{n}{N}\PY{o}{\PYZhy{}}\PY{l+m+mi}{2}\PY{p}{]} \PY{o}{=} \PY{o}{\PYZhy{}}\PY{l+m+mi}{1}
         
         \PY{n}{w} \PY{o}{=} \PY{n}{np}\PY{o}{.}\PY{n}{zeros}\PY{p}{(}\PY{n}{N}\PY{p}{,}\PY{n+nb}{float}\PY{p}{)}
         \PY{n}{w}\PY{p}{[}\PY{l+m+mi}{0}\PY{p}{]} \PY{o}{=} \PY{n}{Vp}
         \PY{n}{w}\PY{p}{[}\PY{l+m+mi}{1}\PY{p}{]} \PY{o}{=} \PY{n}{Vp}
         
         \PY{n}{x} \PY{o}{=} \PY{n}{np}\PY{o}{.}\PY{n}{arange}\PY{p}{(}\PY{l+m+mi}{1}\PY{p}{,}\PY{l+m+mi}{10001}\PY{p}{,}\PY{l+m+mi}{1}\PY{p}{)}
         \PY{n}{y} \PY{o}{=} \PY{n}{banded\PYZus{}solve}\PY{p}{(}\PY{n}{A}\PY{p}{,}\PY{n}{w}\PY{p}{,}\PY{l+m+mi}{2}\PY{p}{)}
         
         \PY{n}{fig3}\PY{p}{,} \PY{n}{ax3} \PY{o}{=} \PY{n}{plt}\PY{o}{.}\PY{n}{subplots}\PY{p}{(}\PY{l+m+mi}{1}\PY{p}{,} \PY{l+m+mi}{1}\PY{p}{,} \PY{n}{figsize} \PY{o}{=} \PY{p}{(}\PY{l+m+mi}{10}\PY{p}{,} \PY{l+m+mi}{5}\PY{p}{)}\PY{p}{)}
         
         \PY{c+c1}{\PYZsh{}increases readability of plot}
         \PY{n}{ax3}\PY{o}{.}\PY{n}{set\PYZus{}title}\PY{p}{(}\PY{l+s+s2}{\PYZdq{}}\PY{l+s+s2}{A Chain of Resistors}\PY{l+s+s2}{\PYZdq{}}\PY{p}{)}
         \PY{n}{ax3}\PY{o}{.}\PY{n}{set\PYZus{}xlabel}\PY{p}{(}\PY{l+s+s2}{\PYZdq{}}\PY{l+s+s2}{Internal Point}\PY{l+s+s2}{\PYZdq{}}\PY{p}{)}
         \PY{n}{ax3}\PY{o}{.}\PY{n}{set\PYZus{}ylabel}\PY{p}{(}\PY{l+s+s2}{\PYZdq{}}\PY{l+s+s2}{Internal Voltage (V)}\PY{l+s+s2}{\PYZdq{}}\PY{p}{)}
         \PY{n}{plt}\PY{o}{.}\PY{n}{plot}\PY{p}{(}\PY{n}{x}\PY{p}{,}\PY{n}{y}\PY{p}{)}
         \PY{n}{plt}\PY{o}{.}\PY{n}{show}\PY{p}{(}\PY{p}{)}
\end{Verbatim}

    \begin{center}
    \adjustimage{max size={0.9\linewidth}{0.9\paperheight}}{output_35_0.png}
    \end{center}
    { \hspace*{\fill} \\}
    
    \section{CP 6.8 The QR algorithm}\label{cp-6.8-the-qr-algorithm}

The problem introduces the following relations given the column vectors
\(\mathbf{a}_0, \mathbf{a}_1, ..., \mathbf{a}_{N-1}\) of the matrix
\(A\):

\[\mathbf{u}_i = \mathbf{a}_i - \sum_{j=0}^{i-1} (\mathbf{q}_j\cdot\mathbf{a}_i) \mathbf{q}_j \qquad \text{ and } \qquad \mathbf{q}_i = {\mathbf{u}_i\over|\mathbf{u}_i|}.\]

From this we can rearrange the equations to give us a system of
equations that begins as

\[\mathbf{a}_0 = |\mathbf{u}_0|\,\mathbf{q}_0, \qquad \qquad \qquad \qquad \\
\mathbf{a}_1 = |\mathbf{u}_1|\,\mathbf{q}_1 + (\mathbf{q}_0\cdot\mathbf{a}_1) \mathbf{q}_0, \qquad \qquad\\
\mathbf{a}_2 = |\mathbf{u}_2|\,\mathbf{q}_2 + (\mathbf{q}_0\cdot\mathbf{a}_2) \mathbf{q}_0
              + (\mathbf{q}_1\cdot\mathbf{a}_2) \mathbf{q}_1.\]

These can be written in matrix form as \[A = \begin{pmatrix}
            | & | & | & \cdots \\
            \mathbf{a}_0 & \mathbf{a}_1 & \mathbf{a}_2 & \cdots \\
            | & | & | & \cdots
          \end{pmatrix}
 =
\begin{pmatrix}
  | & | & | & \cdots \\
  \mathbf{q}_0 & \mathbf{q}_1 & \mathbf{q}_2 & \cdots \\
  | & | & | & \cdots
\end{pmatrix}
\begin{pmatrix}
|\mathbf{u}_0| & \mathbf{q}_0\cdot\mathbf{a}_1 & \mathbf{q}_0\cdot\mathbf{a}_2 & \cdots \\
0           & |\mathbf{u}_1| & \mathbf{q}_1\cdot\mathbf{a}_2 & \cdots \\
0           & 0           & |\mathbf{u}_2| & \cdots
\end{pmatrix} = QR.\]

Where \(Q\) is the orthonormal and \(R\) is upper triangular, so we have
successfully completed the QR decomposition. We can test the QR
decomposition on the matrix \(A\) to make sure we get back the same
matrix.

\[{A} = \begin{pmatrix}
            1 & 4 & 8 & 4 \\
            4 & 2 & 3 & 7 \\
            8 & 3 & 6 & 9 \\
            4 & 7 & 9 & 2
          \end{pmatrix}.\]

    \begin{Verbatim}[commandchars=\\\{\}]
{\color{incolor}In [{\color{incolor}22}]:} \PY{n}{A} \PY{o}{=} \PY{n}{np}\PY{o}{.}\PY{n}{array}\PY{p}{(}\PY{p}{[}\PY{p}{[}\PY{l+m+mi}{1}\PY{p}{,}\PY{l+m+mi}{4}\PY{p}{,}\PY{l+m+mi}{8}\PY{p}{,}\PY{l+m+mi}{4}\PY{p}{]}\PY{p}{,}
                       \PY{p}{[}\PY{l+m+mi}{4}\PY{p}{,}\PY{l+m+mi}{2}\PY{p}{,}\PY{l+m+mi}{3}\PY{p}{,}\PY{l+m+mi}{7}\PY{p}{]}\PY{p}{,}
                       \PY{p}{[}\PY{l+m+mi}{8}\PY{p}{,}\PY{l+m+mi}{3}\PY{p}{,}\PY{l+m+mi}{6}\PY{p}{,}\PY{l+m+mi}{9}\PY{p}{]}\PY{p}{,}
                       \PY{p}{[}\PY{l+m+mi}{4}\PY{p}{,}\PY{l+m+mi}{7}\PY{p}{,}\PY{l+m+mi}{9}\PY{p}{,}\PY{l+m+mi}{2}\PY{p}{]}\PY{p}{]}\PY{p}{,} \PY{n+nb}{float}\PY{p}{)}
         
         \PY{k}{def} \PY{n+nf}{QR}\PY{p}{(}\PY{n}{A}\PY{p}{)}\PY{p}{:}
             \PY{l+s+sd}{\PYZdq{}\PYZdq{}\PYZdq{}With a matrix A as its argument, finds QR decomposition}
         \PY{l+s+sd}{        and returns those two matrices\PYZdq{}\PYZdq{}\PYZdq{}}
             
             \PY{n}{N} \PY{o}{=} \PY{n+nb}{len}\PY{p}{(}\PY{n}{A}\PY{p}{)}
             \PY{n}{a} \PY{o}{=} \PY{n}{np}\PY{o}{.}\PY{n}{copy}\PY{p}{(}\PY{n}{A}\PY{p}{)}
             \PY{n}{u} \PY{o}{=} \PY{n}{np}\PY{o}{.}\PY{n}{zeros}\PY{p}{(}\PY{p}{(}\PY{n}{N}\PY{p}{,}\PY{n}{N}\PY{p}{)}\PY{p}{,} \PY{n+nb}{float}\PY{p}{)}
             \PY{n}{q} \PY{o}{=} \PY{n}{np}\PY{o}{.}\PY{n}{zeros}\PY{p}{(}\PY{p}{(}\PY{n}{N}\PY{p}{,}\PY{n}{N}\PY{p}{)}\PY{p}{,} \PY{n+nb}{float}\PY{p}{)}
             
             \PY{c+c1}{\PYZsh{}calculates orthonormal Q and intermediate U}
             \PY{k}{for} \PY{n}{i} \PY{o+ow}{in} \PY{n+nb}{range}\PY{p}{(}\PY{n}{N}\PY{p}{)}\PY{p}{:}
                 \PY{n}{s} \PY{o}{=} \PY{l+m+mi}{0}
                 \PY{k}{for} \PY{n}{j} \PY{o+ow}{in} \PY{n+nb}{range}\PY{p}{(}\PY{n}{i}\PY{p}{)}\PY{p}{:}
                     \PY{n}{s} \PY{o}{+}\PY{o}{=} \PY{p}{(}\PY{n}{np}\PY{o}{.}\PY{n}{dot}\PY{p}{(}\PY{n}{q}\PY{p}{[}\PY{p}{:}\PY{p}{,}\PY{n}{j}\PY{p}{]}\PY{p}{,}\PY{n}{a}\PY{p}{[}\PY{p}{:}\PY{p}{,}\PY{n}{i}\PY{p}{]}\PY{p}{)}\PY{p}{)} \PY{o}{*} \PY{n}{q}\PY{p}{[}\PY{p}{:}\PY{p}{,}\PY{n}{j}\PY{p}{]}
                     
                 \PY{n}{u}\PY{p}{[}\PY{p}{:}\PY{p}{,}\PY{n}{i}\PY{p}{]} \PY{o}{=} \PY{n}{a}\PY{p}{[}\PY{p}{:}\PY{p}{,}\PY{n}{i}\PY{p}{]} \PY{o}{\PYZhy{}} \PY{n}{s}
                 \PY{n}{q}\PY{p}{[}\PY{p}{:}\PY{p}{,}\PY{n}{i}\PY{p}{]} \PY{o}{=} \PY{n}{u}\PY{p}{[}\PY{p}{:}\PY{p}{,}\PY{n}{i}\PY{p}{]} \PY{o}{/} \PY{n}{LA}\PY{o}{.}\PY{n}{norm}\PY{p}{(}\PY{n}{u}\PY{p}{[}\PY{p}{:}\PY{p}{,}\PY{n}{i}\PY{p}{]}\PY{p}{)}
             \PY{c+c1}{\PYZsh{}q[:,0] *= \PYZhy{}1}
             
             \PY{n}{r} \PY{o}{=} \PY{n}{np}\PY{o}{.}\PY{n}{zeros}\PY{p}{(}\PY{p}{(}\PY{n}{N}\PY{p}{,}\PY{n}{N}\PY{p}{)}\PY{p}{,} \PY{n+nb}{float}\PY{p}{)}
             
             \PY{c+c1}{\PYZsh{}calculates upper triangular R}
             \PY{k}{for} \PY{n}{i} \PY{o+ow}{in} \PY{n+nb}{range}\PY{p}{(}\PY{n}{N}\PY{p}{)}\PY{p}{:}
                 \PY{n}{r}\PY{p}{[}\PY{n}{i}\PY{p}{,}\PY{n}{i}\PY{p}{]} \PY{o}{=} \PY{n}{LA}\PY{o}{.}\PY{n}{norm}\PY{p}{(}\PY{n}{u}\PY{p}{[}\PY{p}{:}\PY{p}{,}\PY{n}{i}\PY{p}{]}\PY{p}{)}
                 \PY{k}{for} \PY{n}{j} \PY{o+ow}{in} \PY{n+nb}{range}\PY{p}{(}\PY{n}{i}\PY{p}{)}\PY{p}{:}
                     \PY{n}{r}\PY{p}{[}\PY{n}{j}\PY{p}{,}\PY{n}{i}\PY{p}{]} \PY{o}{=} \PY{n}{np}\PY{o}{.}\PY{n}{dot}\PY{p}{(}\PY{n}{q}\PY{p}{[}\PY{p}{:}\PY{p}{,}\PY{n}{j}\PY{p}{]}\PY{p}{,}\PY{n}{a}\PY{p}{[}\PY{p}{:}\PY{p}{,}\PY{n}{i}\PY{p}{]}\PY{p}{)}
                 
             \PY{k}{return} \PY{n}{q}\PY{p}{,}\PY{n}{r}
         
         \PY{n}{q}\PY{p}{,} \PY{n}{r} \PY{o}{=} \PY{n}{QR}\PY{p}{(}\PY{n}{A}\PY{p}{)}
         \PY{n+nb}{print}\PY{p}{(}\PY{l+s+s2}{\PYZdq{}}\PY{l+s+s2}{The orthonormal matrix Q is}\PY{l+s+s2}{\PYZdq{}}\PY{p}{)}\PY{p}{,} \PY{n+nb}{print}\PY{p}{(}\PY{n}{q}\PY{p}{)}
         \PY{n+nb}{print}\PY{p}{(}\PY{l+s+s2}{\PYZdq{}}\PY{l+s+se}{\PYZbs{}n}\PY{l+s+s2}{\PYZdq{}}\PY{p}{)}
         \PY{n+nb}{print}\PY{p}{(}\PY{l+s+s2}{\PYZdq{}}\PY{l+s+s2}{The upper triangular matrix R is}\PY{l+s+s2}{\PYZdq{}}\PY{p}{)}\PY{p}{,} \PY{n+nb}{print}\PY{p}{(}\PY{n}{r}\PY{p}{)}
         \PY{n+nb}{print}\PY{p}{(}\PY{l+s+s2}{\PYZdq{}}\PY{l+s+se}{\PYZbs{}n}\PY{l+s+s2}{\PYZdq{}}\PY{p}{)}
         \PY{n+nb}{print}\PY{p}{(}\PY{l+s+s2}{\PYZdq{}}\PY{l+s+s2}{Multiplied together, they return A.}\PY{l+s+s2}{\PYZdq{}}\PY{p}{)}
         \PY{n+nb}{print}\PY{p}{(}\PY{n}{q}\PY{n+nd}{@r}\PY{p}{)}
\end{Verbatim}

    \begin{Verbatim}[commandchars=\\\{\}]
The orthonormal matrix Q is
[[ 0.10153462  0.558463    0.80981107  0.1483773 ]
 [ 0.40613847 -0.10686638 -0.14147555  0.8964462 ]
 [ 0.81227693 -0.38092692  0.22995024 -0.37712564]
 [ 0.40613847  0.72910447 -0.5208777  -0.17928924]]


The upper triangular matrix R is
[[  9.8488578    6.49821546  10.55960012  11.37187705]
 [  0.           5.98106979   8.4234836   -0.484346  ]
 [  0.           0.           2.74586406   3.27671222]
 [  0.           0.           0.           3.11592335]]


Multiplied together, they return A.
[[ 1.  4.  8.  4.]
 [ 4.  2.  3.  7.]
 [ 8.  3.  6.  9.]
 [ 4.  7.  9.  2.]]

    \end{Verbatim}

    \begin{Verbatim}[commandchars=\\\{\}]
{\color{incolor}In [{\color{incolor}23}]:} \PY{k}{def} \PY{n+nf}{eigenvalues}\PY{p}{(}\PY{n}{A}\PY{p}{)}\PY{p}{:}
             \PY{l+s+sd}{\PYZdq{}\PYZdq{}\PYZdq{}This function returns the eigenvalues of a matrix}
         \PY{l+s+sd}{        through QR decomposition and then diagonalization.}
         \PY{l+s+sd}{        It\PYZsq{}s tolerance is 10\PYZca{}\PYZhy{}6 for non\PYZhy{}diagonal elements\PYZdq{}\PYZdq{}\PYZdq{}}
         
             \PY{n}{a} \PY{o}{=} \PY{n}{np}\PY{o}{.}\PY{n}{copy}\PY{p}{(}\PY{n}{A}\PY{p}{)} \PY{c+c1}{\PYZsh{}will be the diagonalized matrix}
             \PY{n}{N} \PY{o}{=} \PY{n+nb}{len}\PY{p}{(}\PY{n}{a}\PY{p}{)}
             \PY{n}{epsilon} \PY{o}{=} \PY{l+m+mf}{10e\PYZhy{}6}
             \PY{n}{V} \PY{o}{=} \PY{n}{np}\PY{o}{.}\PY{n}{identity}\PY{p}{(}\PY{n}{N}\PY{p}{)}
             \PY{n}{max\PYZus{}value} \PY{o}{=} \PY{l+m+mi}{1}
             
             \PY{c+c1}{\PYZsh{}runs until a certain accuracy has been achieved}
             \PY{k}{while} \PY{n}{max\PYZus{}value} \PY{o}{\PYZgt{}} \PY{n}{epsilon}\PY{p}{:}
                 \PY{n}{Q}\PY{p}{,}\PY{n}{R} \PY{o}{=} \PY{n}{QR}\PY{p}{(}\PY{n}{a}\PY{p}{)}
                 \PY{n}{a} \PY{o}{=} \PY{n}{R}\PY{n+nd}{@Q}
                 \PY{n}{V} \PY{o}{=} \PY{n}{V}\PY{n+nd}{@Q}
         
                 \PY{c+c1}{\PYZsh{}finds largest off\PYZhy{}diagonal element by masking diagonal}
                 \PY{n}{mask} \PY{o}{=} \PY{n}{np}\PY{o}{.}\PY{n}{ones}\PY{p}{(}\PY{n}{a}\PY{o}{.}\PY{n}{shape}\PY{p}{,} \PY{n}{dtype}\PY{o}{=}\PY{n+nb}{bool}\PY{p}{)}
                 \PY{n}{np}\PY{o}{.}\PY{n}{fill\PYZus{}diagonal}\PY{p}{(}\PY{n}{mask}\PY{p}{,} \PY{l+m+mi}{0}\PY{p}{)}
                 \PY{n}{max\PYZus{}value} \PY{o}{=} \PY{n}{a}\PY{p}{[}\PY{n}{mask}\PY{p}{]}\PY{o}{.}\PY{n}{max}\PY{p}{(}\PY{p}{)}
             
             \PY{n}{eigvals} \PY{o}{=} \PY{n}{np}\PY{o}{.}\PY{n}{zeros}\PY{p}{(}\PY{n}{N}\PY{p}{,}\PY{n+nb}{float}\PY{p}{)}
             \PY{k}{for} \PY{n}{i} \PY{o+ow}{in} \PY{n+nb}{range}\PY{p}{(}\PY{n}{N}\PY{p}{)}\PY{p}{:}
                 \PY{n}{eigvals}\PY{p}{[}\PY{n}{i}\PY{p}{]} \PY{o}{=} \PY{n}{a}\PY{p}{[}\PY{n}{i}\PY{p}{,}\PY{n}{i}\PY{p}{]}
                 
             \PY{k}{return} \PY{n}{eigvals}
         
         \PY{n+nb}{print}\PY{p}{(}\PY{l+s+s2}{\PYZdq{}}\PY{l+s+s2}{The eigenvalues of A match the expected values.}\PY{l+s+s2}{\PYZdq{}}\PY{p}{)}
         \PY{n+nb}{print}\PY{p}{(}\PY{n}{eigenvalues}\PY{p}{(}\PY{n}{A}\PY{p}{)}\PY{p}{)} 
\end{Verbatim}

    \begin{Verbatim}[commandchars=\\\{\}]
The eigenvalues of A match the expected values.
[ 21.  -8.  -3.   1.]

    \end{Verbatim}

    \section{CP 6.9 Asymmetric quantum
well}\label{cp-6.9-asymmetric-quantum-well}

Unlike the previous qunatum well problem, the asymmetry of this problem
prevents it from being solved analytically, however formatting it as a
system of equations allows it to be solved numerically. The particle in
the given well obeys the relation \(\mathbf{H}\psi(x) = E\psi(x)\) where
\[\mathbf{H} = - {\hbar^2\over2M}\,{d^2\over d x^2} + V(x).\]

We can assume the wavefunction goes to zero outside the well because the
potential is infinitely high for \(x<0\) and \(x>L.\) Thus we can
Fourier transform the the wavefunction

\[\psi(x) = \sum_{n=1}^\infty \psi_n \sin {\pi n x\over L}.\]

The Schrodinger equation \(\mathbf{H}\psi = E\psi\) implies that
\[\sum_{n=1}^\infty \psi_n \int_0^L \sin{\pi m x\over L} \mathbf{H}\sin{\pi n x\over L} \> d x = \tfrac12 L E \psi_m.\]

    This can be derived by expanding the Schrodinger equation so that it
reads
\[\mathbf{H}\biggr(\sum_{n=1}^\infty \psi_n \sin {\pi n x\over L}\biggr) = E\biggr(\sum_{m=1}^\infty \psi_m \sin {\pi m x\over L}\biggr).\]

From here, we can multiply both sides by \(\sin{\pi m x \over L}\) and
reorder the terms on each side so that we have

\[\biggr(\sum_{n=1}^\infty \psi_n \sin {\pi m x\over L} \mathbf{H} \sin {\pi n x\over L}\biggr) = E\biggr(\sum_{m=1}^\infty \psi_m \sin {\pi m x\over L}\sin {\pi m x\over L}\biggr).\]

We can integrate both sides from \(0\) to \(L\) with respect to \(x\)
giving

\[\sum_{n=1}^\infty \psi_n \int_0^L \biggr(\sin {\pi m x\over L} \mathbf{H} \sin {\pi n x\over L}\biggr)\ dx = E \sum_{m=1}^\infty \psi_m \int_0^L \sin {\pi m x\over L}\sin {\pi m x\over L}\ dx = E \sum_{m=1}^\infty \psi_m {L \over 2}\]

by properties of Fourier Series. Thus we have our final result. The
Schrodinger equation combined with the fact that the potential is
infinite outside of the bounds of the quantum well together imply that

\[\sum_{n=1}^\infty \psi_n \int_0^L \sin {\pi m x\over L} \mathbf{H} \sin {\pi n x\over L}\ dx = \tfrac12 LE \psi_m.\]

    If the Hamiltonian matrix \(\mathbf{H}\) is defined according to

\[H_{mn} = {2\over L} \int_0^L \sin{\pi m x\over L} \mathbf{H}\sin{\pi n x\over L} \ d x\]
which can be expanded to include the value of the Hamiltonian

\[H_{mn} = {2\over L} \int_0^L \sin{\pi m x\over L} \biggl[ -{\hbar^2\over2M}\,{d^2\over d x^2} + V(x) \biggr] \sin{\pi n x\over L} \ d x.\]

Taking the case where \(V(x) = {ax \over L}\) gives the expression

\[H_{mn} = {2\over L} \int_0^L \sin{\pi m x\over L} \biggl[ -{\hbar^2\over2M}\,{d^2\over d x^2} + {ax \over L} \biggr] \sin{\pi n x\over L} \ d x.\]

This can be expanded so
\[H_{mn} = {2\over L}\Biggr[ \int_0^L \frac{-\hbar^2}{2M}\sin{\pi m x\over L} \biggr({d^2\over d x^2} \sin{\pi n x\over L}\biggr) \ d x + \int_0^L {ax \over L}\sin{\pi m x\over L} \sin{\pi n x\over L} \ d x \Biggr].\]

\[H_{mn} = {2\over L}\Biggr[\frac{-\hbar^2}{2M}\int_0^L \sin{\pi m x\over L} \biggr(-\biggr(\frac{\pi n}{L}\biggr)^2 \sin{\pi n x\over L}\biggr) \ d x
+ {a \over L}\int_0^L x\sin{\pi m x\over L} \sin{\pi n x\over L} \ d x \Biggr].\]

\[H_{mn} = {2\over L}\Biggr[\frac{\hbar^2}{2M}\biggr(\frac{\pi n}{L}\biggr)^2\int_0^L \sin{\pi m x\over L} \sin{\pi n x\over L} \ d x
+ {a \over L}\int_0^L x\sin{\pi m x\over L} \sin{\pi n x\over L} \ d x \Biggr].\]

This is the point where we create separate expressions for the cases
where \(m=n\) and where \(m \ne n\) but one is even and one is odd. This
is because the integrals take on different values based on the
relationship between \(m\) and \(n.\) From the givens in the problem

\[\hspace{-2em}
\int_0^L x \sin{\pi m x\over L} \sin{\pi n x\over L} \ d x
  = \begin{cases}
      0 & \quad\mbox{if $m\ne n$ and both even or both odd,} \\
      {\displaystyle-\biggl({2L\over\pi}\biggr)^2{mn\over(m^2-n^2)^2}}
        & \quad\mbox{if $m\ne n$ and one is even, one is odd,} \\
      L^2/4 & \quad\mbox{if $m=n$}
    \end{cases}\]

and

\[\int_0^L \sin {\pi m x\over L} \,\sin{\pi n x\over L} \ d x
  = \begin{cases}
      L/2 & \qquad\mbox{if $m=n$,} \\
      0          & \qquad\mbox{otherwise.}
    \end{cases}\]

    So after doing the algebra for the cases, we are left with three
expressions expressions for the three different cases (the second
integral expression's "otherwise" case is split between two of the above
cases.

\[H_{mn} = {2\over L}\Biggr[\frac{\hbar^2}{2M}\biggr(\frac{\pi n}{L}\biggr)^2\int_0^L \sin{\pi m x\over L} \sin{\pi n x\over L} \ d x
+ {a \over L}\int_0^L x\sin{\pi m x\over L} \sin{\pi n x\over L} \ d x \Biggr].\]

The diagonal elements of the Hamiltonian (when \(m=n\)) are given by

\[H_{mn} = {2\over L}\Biggr[\frac{\hbar^2}{2M}\biggr(\frac{\pi n}{L}\biggr)^2 \frac L2 + \frac{aL^2}{4L} \Biggr]\ \forall\ m=n\]

\[H_{mn} = \frac{\hbar^2}{2M}\biggr(\frac{\pi n}{L}\biggr)^2 + \frac a2 \ \forall\ m=n. \qquad \qquad \qquad \qquad (1)\]

The off-diagonal elements of the Hamiltonian (for the cases \(m\ne n\)
but only one is even) are gven by

\[H_{mn} = -\frac 2L\biggr({a \over L}\biggr)\biggl({2L\over\pi}\biggr)^2{mn\over(m^2-n^2)^2} \ \forall\ m \ne n \text{ and one is even, one is odd}\]

\[H_{mn} = \frac{-8amn}{\pi^2(m^2-n^2)^2} \ \forall\ m \ne n \text{ and one is even, one is odd}. \qquad (2)\]

Trivially, we have that

\[H_{mn} = 0 \ \forall \ m\ne n \text{ and both even, or both odd}. \qquad \qquad \qquad (3)\]

And thus with equations (1), (2) and (3), the Hamiltonian can be
populated. After, the diagonalized Hamiltonian of it yields the
eigenvalues, which are the observable energy states we can find the
particle in if exposed to this potential in a quantum well.

    \begin{Verbatim}[commandchars=\\\{\}]
{\color{incolor}In [{\color{incolor}36}]:} \PY{c+c1}{\PYZsh{}defines constants}
         \PY{n}{L} \PY{o}{=} \PY{l+m+mi}{5} \PY{c+c1}{\PYZsh{}width of well (m) = 5 angstroms}
         \PY{n}{a} \PY{o}{=} \PY{l+m+mi}{10} \PY{c+c1}{\PYZsh{}potential parameter (eV)}
         \PY{n}{m} \PY{o}{=} \PY{n}{C}\PY{o}{.}\PY{n}{m\PYZus{}e} \PY{c+c1}{\PYZsh{}mass of electron in the well (kg)}
         \PY{n}{h} \PY{o}{=} \PY{n}{C}\PY{o}{.}\PY{n}{hbar} \PY{c+c1}{\PYZsh{}Planck\PYZsq{}s reduced constant}
\end{Verbatim}

    \begin{Verbatim}[commandchars=\\\{\}]
{\color{incolor}In [{\color{incolor}37}]:} \PY{k}{def} \PY{n+nf}{Hmn}\PY{p}{(}\PY{n}{m}\PY{p}{,}\PY{n}{n}\PY{p}{)}\PY{p}{:}
             \PY{l+s+sd}{\PYZdq{}\PYZdq{}\PYZdq{}Returns the mn element of the Hamiltonian H}
         \PY{l+s+sd}{        based on the parameters specified above\PYZdq{}\PYZdq{}\PYZdq{}}
             
             \PY{k}{if} \PY{n}{m}\PY{o}{==}\PY{n}{n}\PY{p}{:}
                 \PY{k}{return} \PY{p}{(}\PY{p}{(}\PY{n}{h}\PY{o}{*}\PY{o}{*}\PY{l+m+mi}{2}\PY{p}{)}\PY{o}{/}\PY{p}{(}\PY{l+m+mi}{2}\PY{o}{*}\PY{n}{m}\PY{p}{)}\PY{p}{)}\PY{o}{*}\PY{p}{(}\PY{p}{(}\PY{n}{pi}\PY{o}{*}\PY{n}{n}\PY{p}{)}\PY{o}{/}\PY{p}{(}\PY{n}{L}\PY{p}{)}\PY{p}{)}\PY{o}{*}\PY{o}{*}\PY{l+m+mi}{2} \PY{o}{+} \PY{n}{a}\PY{o}{/}\PY{l+m+mi}{2}
             \PY{k}{elif} \PY{n}{m}\PY{o}{!=}\PY{n}{n}\PY{p}{:}
                 \PY{k}{if} \PY{p}{(}\PY{n}{m}\PY{o}{+}\PY{n}{n}\PY{p}{)} \PY{o}{\PYZpc{}} \PY{l+m+mi}{2} \PY{o}{==} \PY{l+m+mi}{1}\PY{p}{:}
                     \PY{n}{num} \PY{o}{=} \PY{o}{\PYZhy{}}\PY{l+m+mi}{8}\PY{o}{*}\PY{n}{a}\PY{o}{*}\PY{n}{m}\PY{o}{*}\PY{n}{n}
                     \PY{n}{den} \PY{o}{=} \PY{p}{(}\PY{n}{pi}\PY{o}{*}\PY{o}{*}\PY{l+m+mi}{2}\PY{p}{)} \PY{o}{*} \PY{p}{(}\PY{p}{(}\PY{n}{m}\PY{o}{*}\PY{o}{*}\PY{l+m+mi}{2} \PY{o}{\PYZhy{}} \PY{n}{n}\PY{o}{*}\PY{o}{*}\PY{l+m+mi}{2}\PY{p}{)}\PY{o}{*}\PY{o}{*}\PY{l+m+mi}{2}\PY{p}{)}
                     \PY{k}{return} \PY{n}{num} \PY{o}{/} \PY{n}{den}
                 \PY{k}{else}\PY{p}{:}
                     \PY{k}{return} \PY{l+m+mi}{0}
         
         \PY{k}{def} \PY{n+nf}{Hamiltonian}\PY{p}{(}\PY{n}{N}\PY{p}{)}\PY{p}{:}
             \PY{l+s+sd}{\PYZdq{}\PYZdq{}\PYZdq{}Creates the Hamiltonian for the given system}
         \PY{l+s+sd}{        with dimensions NxN\PYZdq{}\PYZdq{}\PYZdq{}}
         
             \PY{n}{H} \PY{o}{=} \PY{n}{np}\PY{o}{.}\PY{n}{zeros}\PY{p}{(}\PY{p}{[}\PY{n}{N}\PY{p}{,}\PY{n}{N}\PY{p}{]}\PY{p}{,} \PY{n+nb}{float}\PY{p}{)}
             \PY{k}{for} \PY{n}{m} \PY{o+ow}{in} \PY{n+nb}{range}\PY{p}{(}\PY{l+m+mi}{1}\PY{p}{,}\PY{n}{N}\PY{o}{+}\PY{l+m+mi}{1}\PY{p}{)}\PY{p}{:}
                 \PY{k}{for} \PY{n}{n} \PY{o+ow}{in} \PY{n+nb}{range}\PY{p}{(}\PY{l+m+mi}{1}\PY{p}{,}\PY{n}{N}\PY{o}{+}\PY{l+m+mi}{1}\PY{p}{)}\PY{p}{:}
                     \PY{n}{H}\PY{p}{[}\PY{n}{m}\PY{o}{\PYZhy{}}\PY{l+m+mi}{1}\PY{p}{,}\PY{n}{n}\PY{o}{\PYZhy{}}\PY{l+m+mi}{1}\PY{p}{]} \PY{o}{=} \PY{n}{Hmn}\PY{p}{(}\PY{n}{m}\PY{p}{,}\PY{n}{n}\PY{p}{)}
                     
             \PY{k}{return} \PY{n}{H}
                     
         \PY{n}{LA}\PY{o}{.}\PY{n}{eigvals}\PY{p}{(}\PY{n}{Hamiltonian}\PY{p}{(}\PY{l+m+mi}{10}\PY{p}{)}\PY{p}{)}
         
         \PY{n+nb}{print}\PY{p}{(}\PY{l+s+s2}{\PYZdq{}}\PY{l+s+s2}{The ground state eigenvalue is }\PY{l+s+si}{\PYZob{}:4.2f\PYZcb{}}\PY{l+s+s2}{ eV.}\PY{l+s+s2}{\PYZdq{}}\PYZbs{}
              \PY{o}{.}\PY{n}{format}\PY{p}{(}\PY{n}{np}\PY{o}{.}\PY{n}{min}\PY{p}{(}\PY{n}{LA}\PY{o}{.}\PY{n}{eigvals}\PY{p}{(}\PY{n}{Hamiltonian}\PY{p}{(}\PY{l+m+mi}{10}\PY{p}{)}\PY{p}{)}\PY{p}{)}\PY{p}{)}\PY{p}{)}
\end{Verbatim}

    \begin{Verbatim}[commandchars=\\\{\}]
The ground state eigenvalue is 5.84 eV.

    \end{Verbatim}

    \section{CP 6.16 The Lagrange point}\label{cp-6.16-the-lagrange-point}

The distance between the Earth and the Moon where a satellite remains in
synchronous orbit with the two is called the Lagrange point. At this
point, the gravitational pull towards each of the Earth and the Moon
creates the right balance of centripetal force for the orbit to be
synchronous. If we assume circular orbits and that the mass of the Earth
is much larger than either the Moon or the satellite, the Lagrange point
\(L_1\) must satisfy this equation for \(r\)

\[{GM\over r^2} - {Gm\over(R-r)^2} = \omega^2 r.\]

    Say that the Earth and Moon have mass \(M\) and \(m\), respectively. Say
the satellite in question has mass \(m_1\). Then, as can be seen in the
diagram, the distance from the Earth to the satellite and from the Moon
to the satellite are \(r\) and \(R-r\), respectively. So we can write
the two forces acting on the satellite as
\(F_\text{Earth} = \frac{GMm_1}{r^2}\) and
\(F_\text{Moon} = \frac{Gmm_1}{(R-r)^2}.\) We can sum these forces to
give the total force acting on the satellite as

\[\frac{GMm_1}{r^2} - \frac{Gmm_1}{(R-r)^2} = m_1a_c = m_1 \biggr(\frac{v^2}{r}\biggr) = m_1 \omega^2 r\]

because \(\omega = \frac vr.\) Now, dividing by \(m_1\) gives us the
desired equation:

\[{GM\over r^2} - {Gm\over(R-r)^2} = \omega^2 r.\]

The general form of the secant method is to repeatedly solve the
equation \[x_3 = x_2 - f(x_2)\frac{x_2 -x_1}{f(x_2)-f(x_1)}\]

until the difference between \(x_3\) and \(x_2\) remains smaller than
the desired accuracy for a few iterations. So in this case, we will be
searching for \(r_3\) and comparing to the known value of the Lagrange
point \(L_1 = 326045 \text{ km}\). If we call our error \(\delta,\) then
\(\delta = r_3 - r_2.\) So we can write the error for the secant method
as

\[\delta = -f(r_2)\frac{r_2 -r_1}{f(r_2)-f(r_1)}.\]

However, to solve this equation, it would be more useful if we write it
as a function \(f(r)\), so multiplying out the fraction on the left hand
side leaves us with

\[{GM(R-r)^2} - {Gmr^2} = \omega^2 r^3(R-r)^2.\]

We can then move everything to one side of the equation so that

\[0 = {GM(R-r)^2} - {Gmr^2} - \omega^2 r^3(R-r)^2.\]

    \begin{Verbatim}[commandchars=\\\{\}]
{\color{incolor}In [{\color{incolor}26}]:} \PY{c+c1}{\PYZsh{}defining constants}
         \PY{n}{G} \PY{o}{=} \PY{n}{C}\PY{o}{.}\PY{n}{G} \PY{c+c1}{\PYZsh{}gravitational constant}
         \PY{n}{M} \PY{o}{=} \PY{l+m+mf}{5.974e24} \PY{c+c1}{\PYZsh{}earth mass (kg)}
         \PY{n}{m} \PY{o}{=} \PY{l+m+mf}{7.348e22} \PY{c+c1}{\PYZsh{}moon mass (kg)}
         \PY{n}{R} \PY{o}{=} \PY{l+m+mf}{3.844e8} \PY{c+c1}{\PYZsh{}distance to moon (m)}
         \PY{n}{omega} \PY{o}{=} \PY{l+m+mf}{2.662e\PYZhy{}6} \PY{c+c1}{\PYZsh{}angular velocity (s\PYZca{}\PYZhy{}1)}
         
         \PY{c+c1}{\PYZsh{}secant method test function}
         \PY{k}{def} \PY{n+nf}{f}\PY{p}{(}\PY{n}{x}\PY{p}{)}\PY{p}{:}
             \PY{k}{return} \PY{n}{sin}\PY{p}{(}\PY{n}{x}\PY{p}{)}\PY{o}{*}\PY{n}{tan}\PY{p}{(}\PY{l+m+mi}{3}\PY{o}{*}\PY{n}{x}\PY{p}{)} \PY{o}{+} \PY{n}{exp}\PY{p}{(}\PY{n}{x}\PY{p}{)} \PY{o}{+} \PY{n}{x}\PY{o}{*}\PY{o}{*}\PY{l+m+mi}{3}
         
         \PY{c+c1}{\PYZsh{}function for Lagrange point}
         \PY{k}{def} \PY{n+nf}{g}\PY{p}{(}\PY{n}{r}\PY{p}{)}\PY{p}{:}
             \PY{n}{p1} \PY{o}{=} \PY{n}{G}\PY{o}{*}\PY{n}{M}\PY{o}{*}\PY{p}{(}\PY{p}{(}\PY{n}{R}\PY{o}{\PYZhy{}}\PY{n}{r}\PY{p}{)}\PY{o}{*}\PY{o}{*}\PY{l+m+mi}{2}\PY{p}{)}
             \PY{n}{p2} \PY{o}{=} \PY{n}{G}\PY{o}{*}\PY{n}{m}\PY{o}{*}\PY{p}{(}\PY{n}{r}\PY{o}{*}\PY{o}{*}\PY{l+m+mi}{2}\PY{p}{)}
             \PY{n}{p3} \PY{o}{=} \PY{p}{(}\PY{n}{omega}\PY{o}{*}\PY{o}{*}\PY{l+m+mi}{2}\PY{p}{)}\PY{o}{*}\PY{p}{(}\PY{n}{r}\PY{o}{*}\PY{o}{*}\PY{l+m+mi}{3}\PY{p}{)}\PY{o}{*}\PY{p}{(}\PY{p}{(}\PY{n}{R}\PY{o}{\PYZhy{}}\PY{n}{r}\PY{p}{)}\PY{o}{*}\PY{o}{*}\PY{l+m+mi}{2}\PY{p}{)}
             \PY{k}{return} \PY{n}{p1} \PY{o}{\PYZhy{}} \PY{n}{p2} \PY{o}{\PYZhy{}} \PY{n}{p3}
\end{Verbatim}

    \begin{Verbatim}[commandchars=\\\{\}]
{\color{incolor}In [{\color{incolor}27}]:} \PY{k}{def} \PY{n+nf}{secant}\PY{p}{(}\PY{n}{r1}\PY{p}{,} \PY{n}{r2}\PY{p}{)}\PY{p}{:}
             \PY{l+s+sd}{\PYZdq{}\PYZdq{}\PYZdq{}Given two starting points x1 and x2, this}
         \PY{l+s+sd}{        approximates the root of a function to}
         \PY{l+s+sd}{        four significant digits using the secant}
         \PY{l+s+sd}{        method as described in the book\PYZdq{}\PYZdq{}\PYZdq{}}
             
             \PY{n}{accuracy} \PY{o}{=} \PY{l+m+mf}{1e\PYZhy{}5}
             \PY{n}{delta} \PY{o}{=} \PY{l+m+mi}{1}
             \PY{k}{while} \PY{n+nb}{abs}\PY{p}{(}\PY{n}{delta}\PY{p}{)} \PY{o}{\PYZgt{}} \PY{n}{accuracy}\PY{p}{:}
                 \PY{n}{delta} \PY{o}{=} \PY{n}{g}\PY{p}{(}\PY{n}{r2}\PY{p}{)}\PY{o}{*}\PY{p}{(} \PY{p}{(}\PY{n}{r2}\PY{o}{\PYZhy{}}\PY{n}{r1}\PY{p}{)} \PY{o}{/} \PY{p}{(}\PY{n}{g}\PY{p}{(}\PY{n}{r2}\PY{p}{)}\PY{o}{\PYZhy{}}\PY{n}{g}\PY{p}{(}\PY{n}{r1}\PY{p}{)}\PY{p}{)} \PY{p}{)}
                 \PY{n}{r1} \PY{o}{=} \PY{n}{r2}
                 \PY{n}{r2} \PY{o}{\PYZhy{}}\PY{o}{=} \PY{n}{delta}
                 
             \PY{k}{return} \PY{n}{r2}
\end{Verbatim}

    \begin{Verbatim}[commandchars=\\\{\}]
{\color{incolor}In [{\color{incolor}28}]:} \PY{n+nb}{print}\PY{p}{(}\PY{l+s+s2}{\PYZdq{}}\PY{l+s+s2}{The Lagrange point is }\PY{l+s+si}{\PYZob{}:4.4f\PYZcb{}}\PY{l+s+s2}{ m from Earth.}\PY{l+s+s2}{\PYZdq{}}\PYZbs{}
               \PY{o}{.}\PY{n}{format}\PY{p}{(}\PY{n}{secant}\PY{p}{(}\PY{l+m+mi}{100000}\PY{p}{,}\PY{l+m+mi}{500}\PY{p}{)}\PY{p}{)}\PY{p}{)}
\end{Verbatim}

    \begin{Verbatim}[commandchars=\\\{\}]
The Lagrange point is 326045420.2403 m from Earth.

    \end{Verbatim}

    This matches the known value of \(r_{L_1} = 326045 \text{ km}\).


    % Add a bibliography block to the postdoc
    
    
    
    \end{document}

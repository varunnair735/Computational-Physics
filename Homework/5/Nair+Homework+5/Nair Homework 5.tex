
% Default to the notebook output style

    


% Inherit from the specified cell style.




    
\documentclass[11pt]{article}

    
    
    \usepackage[T1]{fontenc}
    % Nicer default font (+ math font) than Computer Modern for most use cases
    \usepackage{mathpazo}

    % Basic figure setup, for now with no caption control since it's done
    % automatically by Pandoc (which extracts ![](path) syntax from Markdown).
    \usepackage{graphicx}
    % We will generate all images so they have a width \maxwidth. This means
    % that they will get their normal width if they fit onto the page, but
    % are scaled down if they would overflow the margins.
    \makeatletter
    \def\maxwidth{\ifdim\Gin@nat@width>\linewidth\linewidth
    \else\Gin@nat@width\fi}
    \makeatother
    \let\Oldincludegraphics\includegraphics
    % Set max figure width to be 80% of text width, for now hardcoded.
    \renewcommand{\includegraphics}[1]{\Oldincludegraphics[width=.8\maxwidth]{#1}}
    % Ensure that by default, figures have no caption (until we provide a
    % proper Figure object with a Caption API and a way to capture that
    % in the conversion process - todo).
    \usepackage{caption}
    \DeclareCaptionLabelFormat{nolabel}{}
    \captionsetup{labelformat=nolabel}

    \usepackage{adjustbox} % Used to constrain images to a maximum size 
    \usepackage{xcolor} % Allow colors to be defined
    \usepackage{enumerate} % Needed for markdown enumerations to work
    \usepackage{geometry} % Used to adjust the document margins
    \usepackage{amsmath} % Equations
    \usepackage{amssymb} % Equations
    \usepackage{textcomp} % defines textquotesingle
    % Hack from http://tex.stackexchange.com/a/47451/13684:
    \AtBeginDocument{%
        \def\PYZsq{\textquotesingle}% Upright quotes in Pygmentized code
    }
    \usepackage{upquote} % Upright quotes for verbatim code
    \usepackage{eurosym} % defines \euro
    \usepackage[mathletters]{ucs} % Extended unicode (utf-8) support
    \usepackage[utf8x]{inputenc} % Allow utf-8 characters in the tex document
    \usepackage{fancyvrb} % verbatim replacement that allows latex
    \usepackage{grffile} % extends the file name processing of package graphics 
                         % to support a larger range 
    % The hyperref package gives us a pdf with properly built
    % internal navigation ('pdf bookmarks' for the table of contents,
    % internal cross-reference links, web links for URLs, etc.)
    \usepackage{hyperref}
    \usepackage{longtable} % longtable support required by pandoc >1.10
    \usepackage{booktabs}  % table support for pandoc > 1.12.2
    \usepackage[inline]{enumitem} % IRkernel/repr support (it uses the enumerate* environment)
    \usepackage[normalem]{ulem} % ulem is needed to support strikethroughs (\sout)
                                % normalem makes italics be italics, not underlines
    \usepackage{mathrsfs}
    

    
    
    % Colors for the hyperref package
    \definecolor{urlcolor}{rgb}{0,.145,.698}
    \definecolor{linkcolor}{rgb}{.71,0.21,0.01}
    \definecolor{citecolor}{rgb}{.12,.54,.11}

    % ANSI colors
    \definecolor{ansi-black}{HTML}{3E424D}
    \definecolor{ansi-black-intense}{HTML}{282C36}
    \definecolor{ansi-red}{HTML}{E75C58}
    \definecolor{ansi-red-intense}{HTML}{B22B31}
    \definecolor{ansi-green}{HTML}{00A250}
    \definecolor{ansi-green-intense}{HTML}{007427}
    \definecolor{ansi-yellow}{HTML}{DDB62B}
    \definecolor{ansi-yellow-intense}{HTML}{B27D12}
    \definecolor{ansi-blue}{HTML}{208FFB}
    \definecolor{ansi-blue-intense}{HTML}{0065CA}
    \definecolor{ansi-magenta}{HTML}{D160C4}
    \definecolor{ansi-magenta-intense}{HTML}{A03196}
    \definecolor{ansi-cyan}{HTML}{60C6C8}
    \definecolor{ansi-cyan-intense}{HTML}{258F8F}
    \definecolor{ansi-white}{HTML}{C5C1B4}
    \definecolor{ansi-white-intense}{HTML}{A1A6B2}
    \definecolor{ansi-default-inverse-fg}{HTML}{FFFFFF}
    \definecolor{ansi-default-inverse-bg}{HTML}{000000}

    % commands and environments needed by pandoc snippets
    % extracted from the output of `pandoc -s`
    \providecommand{\tightlist}{%
      \setlength{\itemsep}{0pt}\setlength{\parskip}{0pt}}
    \DefineVerbatimEnvironment{Highlighting}{Verbatim}{commandchars=\\\{\}}
    % Add ',fontsize=\small' for more characters per line
    \newenvironment{Shaded}{}{}
    \newcommand{\KeywordTok}[1]{\textcolor[rgb]{0.00,0.44,0.13}{\textbf{{#1}}}}
    \newcommand{\DataTypeTok}[1]{\textcolor[rgb]{0.56,0.13,0.00}{{#1}}}
    \newcommand{\DecValTok}[1]{\textcolor[rgb]{0.25,0.63,0.44}{{#1}}}
    \newcommand{\BaseNTok}[1]{\textcolor[rgb]{0.25,0.63,0.44}{{#1}}}
    \newcommand{\FloatTok}[1]{\textcolor[rgb]{0.25,0.63,0.44}{{#1}}}
    \newcommand{\CharTok}[1]{\textcolor[rgb]{0.25,0.44,0.63}{{#1}}}
    \newcommand{\StringTok}[1]{\textcolor[rgb]{0.25,0.44,0.63}{{#1}}}
    \newcommand{\CommentTok}[1]{\textcolor[rgb]{0.38,0.63,0.69}{\textit{{#1}}}}
    \newcommand{\OtherTok}[1]{\textcolor[rgb]{0.00,0.44,0.13}{{#1}}}
    \newcommand{\AlertTok}[1]{\textcolor[rgb]{1.00,0.00,0.00}{\textbf{{#1}}}}
    \newcommand{\FunctionTok}[1]{\textcolor[rgb]{0.02,0.16,0.49}{{#1}}}
    \newcommand{\RegionMarkerTok}[1]{{#1}}
    \newcommand{\ErrorTok}[1]{\textcolor[rgb]{1.00,0.00,0.00}{\textbf{{#1}}}}
    \newcommand{\NormalTok}[1]{{#1}}
    
    % Additional commands for more recent versions of Pandoc
    \newcommand{\ConstantTok}[1]{\textcolor[rgb]{0.53,0.00,0.00}{{#1}}}
    \newcommand{\SpecialCharTok}[1]{\textcolor[rgb]{0.25,0.44,0.63}{{#1}}}
    \newcommand{\VerbatimStringTok}[1]{\textcolor[rgb]{0.25,0.44,0.63}{{#1}}}
    \newcommand{\SpecialStringTok}[1]{\textcolor[rgb]{0.73,0.40,0.53}{{#1}}}
    \newcommand{\ImportTok}[1]{{#1}}
    \newcommand{\DocumentationTok}[1]{\textcolor[rgb]{0.73,0.13,0.13}{\textit{{#1}}}}
    \newcommand{\AnnotationTok}[1]{\textcolor[rgb]{0.38,0.63,0.69}{\textbf{\textit{{#1}}}}}
    \newcommand{\CommentVarTok}[1]{\textcolor[rgb]{0.38,0.63,0.69}{\textbf{\textit{{#1}}}}}
    \newcommand{\VariableTok}[1]{\textcolor[rgb]{0.10,0.09,0.49}{{#1}}}
    \newcommand{\ControlFlowTok}[1]{\textcolor[rgb]{0.00,0.44,0.13}{\textbf{{#1}}}}
    \newcommand{\OperatorTok}[1]{\textcolor[rgb]{0.40,0.40,0.40}{{#1}}}
    \newcommand{\BuiltInTok}[1]{{#1}}
    \newcommand{\ExtensionTok}[1]{{#1}}
    \newcommand{\PreprocessorTok}[1]{\textcolor[rgb]{0.74,0.48,0.00}{{#1}}}
    \newcommand{\AttributeTok}[1]{\textcolor[rgb]{0.49,0.56,0.16}{{#1}}}
    \newcommand{\InformationTok}[1]{\textcolor[rgb]{0.38,0.63,0.69}{\textbf{\textit{{#1}}}}}
    \newcommand{\WarningTok}[1]{\textcolor[rgb]{0.38,0.63,0.69}{\textbf{\textit{{#1}}}}}
    
    
    % Define a nice break command that doesn't care if a line doesn't already
    % exist.
    \def\br{\hspace*{\fill} \\* }
    % Math Jax compatibility definitions
    \def\gt{>}
    \def\lt{<}
    \let\Oldtex\TeX
    \let\Oldlatex\LaTeX
    \renewcommand{\TeX}{\textrm{\Oldtex}}
    \renewcommand{\LaTeX}{\textrm{\Oldlatex}}
    % Document parameters
    % Document title
    \title{Homework 5 \\ \vspace{10mm}
    {\large Varun Nair}}    
    
    
    
    

    % Pygments definitions
    
\makeatletter
\def\PY@reset{\let\PY@it=\relax \let\PY@bf=\relax%
    \let\PY@ul=\relax \let\PY@tc=\relax%
    \let\PY@bc=\relax \let\PY@ff=\relax}
\def\PY@tok#1{\csname PY@tok@#1\endcsname}
\def\PY@toks#1+{\ifx\relax#1\empty\else%
    \PY@tok{#1}\expandafter\PY@toks\fi}
\def\PY@do#1{\PY@bc{\PY@tc{\PY@ul{%
    \PY@it{\PY@bf{\PY@ff{#1}}}}}}}
\def\PY#1#2{\PY@reset\PY@toks#1+\relax+\PY@do{#2}}

\expandafter\def\csname PY@tok@w\endcsname{\def\PY@tc##1{\textcolor[rgb]{0.73,0.73,0.73}{##1}}}
\expandafter\def\csname PY@tok@c\endcsname{\let\PY@it=\textit\def\PY@tc##1{\textcolor[rgb]{0.25,0.50,0.50}{##1}}}
\expandafter\def\csname PY@tok@cp\endcsname{\def\PY@tc##1{\textcolor[rgb]{0.74,0.48,0.00}{##1}}}
\expandafter\def\csname PY@tok@k\endcsname{\let\PY@bf=\textbf\def\PY@tc##1{\textcolor[rgb]{0.00,0.50,0.00}{##1}}}
\expandafter\def\csname PY@tok@kp\endcsname{\def\PY@tc##1{\textcolor[rgb]{0.00,0.50,0.00}{##1}}}
\expandafter\def\csname PY@tok@kt\endcsname{\def\PY@tc##1{\textcolor[rgb]{0.69,0.00,0.25}{##1}}}
\expandafter\def\csname PY@tok@o\endcsname{\def\PY@tc##1{\textcolor[rgb]{0.40,0.40,0.40}{##1}}}
\expandafter\def\csname PY@tok@ow\endcsname{\let\PY@bf=\textbf\def\PY@tc##1{\textcolor[rgb]{0.67,0.13,1.00}{##1}}}
\expandafter\def\csname PY@tok@nb\endcsname{\def\PY@tc##1{\textcolor[rgb]{0.00,0.50,0.00}{##1}}}
\expandafter\def\csname PY@tok@nf\endcsname{\def\PY@tc##1{\textcolor[rgb]{0.00,0.00,1.00}{##1}}}
\expandafter\def\csname PY@tok@nc\endcsname{\let\PY@bf=\textbf\def\PY@tc##1{\textcolor[rgb]{0.00,0.00,1.00}{##1}}}
\expandafter\def\csname PY@tok@nn\endcsname{\let\PY@bf=\textbf\def\PY@tc##1{\textcolor[rgb]{0.00,0.00,1.00}{##1}}}
\expandafter\def\csname PY@tok@ne\endcsname{\let\PY@bf=\textbf\def\PY@tc##1{\textcolor[rgb]{0.82,0.25,0.23}{##1}}}
\expandafter\def\csname PY@tok@nv\endcsname{\def\PY@tc##1{\textcolor[rgb]{0.10,0.09,0.49}{##1}}}
\expandafter\def\csname PY@tok@no\endcsname{\def\PY@tc##1{\textcolor[rgb]{0.53,0.00,0.00}{##1}}}
\expandafter\def\csname PY@tok@nl\endcsname{\def\PY@tc##1{\textcolor[rgb]{0.63,0.63,0.00}{##1}}}
\expandafter\def\csname PY@tok@ni\endcsname{\let\PY@bf=\textbf\def\PY@tc##1{\textcolor[rgb]{0.60,0.60,0.60}{##1}}}
\expandafter\def\csname PY@tok@na\endcsname{\def\PY@tc##1{\textcolor[rgb]{0.49,0.56,0.16}{##1}}}
\expandafter\def\csname PY@tok@nt\endcsname{\let\PY@bf=\textbf\def\PY@tc##1{\textcolor[rgb]{0.00,0.50,0.00}{##1}}}
\expandafter\def\csname PY@tok@nd\endcsname{\def\PY@tc##1{\textcolor[rgb]{0.67,0.13,1.00}{##1}}}
\expandafter\def\csname PY@tok@s\endcsname{\def\PY@tc##1{\textcolor[rgb]{0.73,0.13,0.13}{##1}}}
\expandafter\def\csname PY@tok@sd\endcsname{\let\PY@it=\textit\def\PY@tc##1{\textcolor[rgb]{0.73,0.13,0.13}{##1}}}
\expandafter\def\csname PY@tok@si\endcsname{\let\PY@bf=\textbf\def\PY@tc##1{\textcolor[rgb]{0.73,0.40,0.53}{##1}}}
\expandafter\def\csname PY@tok@se\endcsname{\let\PY@bf=\textbf\def\PY@tc##1{\textcolor[rgb]{0.73,0.40,0.13}{##1}}}
\expandafter\def\csname PY@tok@sr\endcsname{\def\PY@tc##1{\textcolor[rgb]{0.73,0.40,0.53}{##1}}}
\expandafter\def\csname PY@tok@ss\endcsname{\def\PY@tc##1{\textcolor[rgb]{0.10,0.09,0.49}{##1}}}
\expandafter\def\csname PY@tok@sx\endcsname{\def\PY@tc##1{\textcolor[rgb]{0.00,0.50,0.00}{##1}}}
\expandafter\def\csname PY@tok@m\endcsname{\def\PY@tc##1{\textcolor[rgb]{0.40,0.40,0.40}{##1}}}
\expandafter\def\csname PY@tok@gh\endcsname{\let\PY@bf=\textbf\def\PY@tc##1{\textcolor[rgb]{0.00,0.00,0.50}{##1}}}
\expandafter\def\csname PY@tok@gu\endcsname{\let\PY@bf=\textbf\def\PY@tc##1{\textcolor[rgb]{0.50,0.00,0.50}{##1}}}
\expandafter\def\csname PY@tok@gd\endcsname{\def\PY@tc##1{\textcolor[rgb]{0.63,0.00,0.00}{##1}}}
\expandafter\def\csname PY@tok@gi\endcsname{\def\PY@tc##1{\textcolor[rgb]{0.00,0.63,0.00}{##1}}}
\expandafter\def\csname PY@tok@gr\endcsname{\def\PY@tc##1{\textcolor[rgb]{1.00,0.00,0.00}{##1}}}
\expandafter\def\csname PY@tok@ge\endcsname{\let\PY@it=\textit}
\expandafter\def\csname PY@tok@gs\endcsname{\let\PY@bf=\textbf}
\expandafter\def\csname PY@tok@gp\endcsname{\let\PY@bf=\textbf\def\PY@tc##1{\textcolor[rgb]{0.00,0.00,0.50}{##1}}}
\expandafter\def\csname PY@tok@go\endcsname{\def\PY@tc##1{\textcolor[rgb]{0.53,0.53,0.53}{##1}}}
\expandafter\def\csname PY@tok@gt\endcsname{\def\PY@tc##1{\textcolor[rgb]{0.00,0.27,0.87}{##1}}}
\expandafter\def\csname PY@tok@err\endcsname{\def\PY@bc##1{\setlength{\fboxsep}{0pt}\fcolorbox[rgb]{1.00,0.00,0.00}{1,1,1}{\strut ##1}}}
\expandafter\def\csname PY@tok@kc\endcsname{\let\PY@bf=\textbf\def\PY@tc##1{\textcolor[rgb]{0.00,0.50,0.00}{##1}}}
\expandafter\def\csname PY@tok@kd\endcsname{\let\PY@bf=\textbf\def\PY@tc##1{\textcolor[rgb]{0.00,0.50,0.00}{##1}}}
\expandafter\def\csname PY@tok@kn\endcsname{\let\PY@bf=\textbf\def\PY@tc##1{\textcolor[rgb]{0.00,0.50,0.00}{##1}}}
\expandafter\def\csname PY@tok@kr\endcsname{\let\PY@bf=\textbf\def\PY@tc##1{\textcolor[rgb]{0.00,0.50,0.00}{##1}}}
\expandafter\def\csname PY@tok@bp\endcsname{\def\PY@tc##1{\textcolor[rgb]{0.00,0.50,0.00}{##1}}}
\expandafter\def\csname PY@tok@fm\endcsname{\def\PY@tc##1{\textcolor[rgb]{0.00,0.00,1.00}{##1}}}
\expandafter\def\csname PY@tok@vc\endcsname{\def\PY@tc##1{\textcolor[rgb]{0.10,0.09,0.49}{##1}}}
\expandafter\def\csname PY@tok@vg\endcsname{\def\PY@tc##1{\textcolor[rgb]{0.10,0.09,0.49}{##1}}}
\expandafter\def\csname PY@tok@vi\endcsname{\def\PY@tc##1{\textcolor[rgb]{0.10,0.09,0.49}{##1}}}
\expandafter\def\csname PY@tok@vm\endcsname{\def\PY@tc##1{\textcolor[rgb]{0.10,0.09,0.49}{##1}}}
\expandafter\def\csname PY@tok@sa\endcsname{\def\PY@tc##1{\textcolor[rgb]{0.73,0.13,0.13}{##1}}}
\expandafter\def\csname PY@tok@sb\endcsname{\def\PY@tc##1{\textcolor[rgb]{0.73,0.13,0.13}{##1}}}
\expandafter\def\csname PY@tok@sc\endcsname{\def\PY@tc##1{\textcolor[rgb]{0.73,0.13,0.13}{##1}}}
\expandafter\def\csname PY@tok@dl\endcsname{\def\PY@tc##1{\textcolor[rgb]{0.73,0.13,0.13}{##1}}}
\expandafter\def\csname PY@tok@s2\endcsname{\def\PY@tc##1{\textcolor[rgb]{0.73,0.13,0.13}{##1}}}
\expandafter\def\csname PY@tok@sh\endcsname{\def\PY@tc##1{\textcolor[rgb]{0.73,0.13,0.13}{##1}}}
\expandafter\def\csname PY@tok@s1\endcsname{\def\PY@tc##1{\textcolor[rgb]{0.73,0.13,0.13}{##1}}}
\expandafter\def\csname PY@tok@mb\endcsname{\def\PY@tc##1{\textcolor[rgb]{0.40,0.40,0.40}{##1}}}
\expandafter\def\csname PY@tok@mf\endcsname{\def\PY@tc##1{\textcolor[rgb]{0.40,0.40,0.40}{##1}}}
\expandafter\def\csname PY@tok@mh\endcsname{\def\PY@tc##1{\textcolor[rgb]{0.40,0.40,0.40}{##1}}}
\expandafter\def\csname PY@tok@mi\endcsname{\def\PY@tc##1{\textcolor[rgb]{0.40,0.40,0.40}{##1}}}
\expandafter\def\csname PY@tok@il\endcsname{\def\PY@tc##1{\textcolor[rgb]{0.40,0.40,0.40}{##1}}}
\expandafter\def\csname PY@tok@mo\endcsname{\def\PY@tc##1{\textcolor[rgb]{0.40,0.40,0.40}{##1}}}
\expandafter\def\csname PY@tok@ch\endcsname{\let\PY@it=\textit\def\PY@tc##1{\textcolor[rgb]{0.25,0.50,0.50}{##1}}}
\expandafter\def\csname PY@tok@cm\endcsname{\let\PY@it=\textit\def\PY@tc##1{\textcolor[rgb]{0.25,0.50,0.50}{##1}}}
\expandafter\def\csname PY@tok@cpf\endcsname{\let\PY@it=\textit\def\PY@tc##1{\textcolor[rgb]{0.25,0.50,0.50}{##1}}}
\expandafter\def\csname PY@tok@c1\endcsname{\let\PY@it=\textit\def\PY@tc##1{\textcolor[rgb]{0.25,0.50,0.50}{##1}}}
\expandafter\def\csname PY@tok@cs\endcsname{\let\PY@it=\textit\def\PY@tc##1{\textcolor[rgb]{0.25,0.50,0.50}{##1}}}

\def\PYZbs{\char`\\}
\def\PYZus{\char`\_}
\def\PYZob{\char`\{}
\def\PYZcb{\char`\}}
\def\PYZca{\char`\^}
\def\PYZam{\char`\&}
\def\PYZlt{\char`\<}
\def\PYZgt{\char`\>}
\def\PYZsh{\char`\#}
\def\PYZpc{\char`\%}
\def\PYZdl{\char`\$}
\def\PYZhy{\char`\-}
\def\PYZsq{\char`\'}
\def\PYZdq{\char`\"}
\def\PYZti{\char`\~}
% for compatibility with earlier versions
\def\PYZat{@}
\def\PYZlb{[}
\def\PYZrb{]}
\makeatother


    % Exact colors from NB
    \definecolor{incolor}{rgb}{0.0, 0.0, 0.5}
    \definecolor{outcolor}{rgb}{0.545, 0.0, 0.0}



    
    % Prevent overflowing lines due to hard-to-break entities
    \sloppy 
    % Setup hyperref package
    \hypersetup{
      breaklinks=true,  % so long urls are correctly broken across lines
      colorlinks=true,
      urlcolor=urlcolor,
      linkcolor=linkcolor,
      citecolor=citecolor,
      }
    % Slightly bigger margins than the latex defaults
    
    \geometry{verbose,tmargin=1in,bmargin=1in,lmargin=1in,rmargin=1in}
    
    

    \begin{document}
    
    
    \maketitle
    
    

    
    \begin{Verbatim}[commandchars=\\\{\}]
{\color{incolor}In [{\color{incolor}1}]:} \PY{o}{\PYZpc{}}\PY{k}{precision} \PYZpc{}g
        \PY{o}{\PYZpc{}}\PY{k}{matplotlib} inline
        \PY{o}{\PYZpc{}}\PY{k}{config} InlineBackend.figure\PYZus{}format = \PYZsq{}retina\PYZsq{}
\end{Verbatim}

    \begin{Verbatim}[commandchars=\\\{\}]
{\color{incolor}In [{\color{incolor}2}]:} \PY{k+kn}{from} \PY{n+nn}{math} \PY{k}{import} \PY{n}{sqrt}\PY{p}{,} \PY{n}{pi}\PY{p}{,} \PY{n}{sin}\PY{p}{,} \PY{n}{cos}\PY{p}{,} \PY{n}{floor}\PY{p}{,} \PY{n}{exp}
        \PY{c+c1}{\PYZsh{}from cmath import exp}
        \PY{k+kn}{import} \PY{n+nn}{numpy} \PY{k}{as} \PY{n+nn}{np}
        \PY{k+kn}{from} \PY{n+nn}{numpy} \PY{k}{import} \PY{n}{linalg} \PY{k}{as} \PY{n}{LA}
        \PY{k+kn}{from} \PY{n+nn}{numpy}\PY{n+nn}{.}\PY{n+nn}{fft} \PY{k}{import} \PY{n}{rfft}\PY{p}{,} \PY{n}{irfft}\PY{p}{,} \PY{n}{rfft2}\PY{p}{,} \PY{n}{irfft2}
        \PY{k+kn}{from} \PY{n+nn}{scipy} \PY{k}{import} \PY{n}{constants} \PY{k}{as} \PY{n}{C}
        \PY{k+kn}{import} \PY{n+nn}{matplotlib}\PY{n+nn}{.}\PY{n+nn}{pyplot} \PY{k}{as} \PY{n+nn}{plt}
        
        \PY{c+c1}{\PYZsh{}from IPython.display import set\PYZus{}matplotlib\PYZus{}formats}
        \PY{c+c1}{\PYZsh{}set\PYZus{}matplotlib\PYZus{}formats(\PYZsq{}png\PYZsq{}, \PYZsq{}pdf\PYZsq{})}
\end{Verbatim}

    \section{CP 7.1 Fourier transforms of simple
functions}\label{cp-7.1-fourier-transforms-of-simple-functions}

This problem examines the Fourier transforms of simple periodic
functions. It involves the following functions - a single cycle of a
square wave (amplitude 1) - a sawtooth wave \(y_n = n\) - a modulated
sine wave \(y_n = \sin{\pi n \over N} \sin{20\pi n\over N}\)

    \begin{Verbatim}[commandchars=\\\{\}]
{\color{incolor}In [{\color{incolor}3}]:} \PY{n}{N} \PY{o}{=} \PY{l+m+mi}{500}
        
        \PY{c+c1}{\PYZsh{}defines square wave}
        \PY{n}{y} \PY{o}{=} \PY{n}{np}\PY{o}{.}\PY{n}{zeros}\PY{p}{(}\PY{n}{N}\PY{p}{)}
        \PY{n}{y}\PY{p}{[}\PY{p}{:}\PY{n+nb}{int}\PY{p}{(}\PY{n}{N}\PY{o}{/}\PY{l+m+mi}{2}\PY{p}{)}\PY{p}{]} \PY{o}{=} \PY{l+m+mi}{1}
        \PY{n}{y}\PY{p}{[}\PY{n+nb}{int}\PY{p}{(}\PY{n}{N}\PY{o}{/}\PY{l+m+mi}{2}\PY{p}{)}\PY{p}{:}\PY{p}{]} \PY{o}{=} \PY{o}{\PYZhy{}}\PY{l+m+mi}{1}
        
        \PY{n}{c} \PY{o}{=} \PY{n}{rfft}\PY{p}{(}\PY{n}{y}\PY{p}{)}
        
        \PY{n}{fig1}\PY{p}{,} \PY{n}{ax1} \PY{o}{=} \PY{n}{plt}\PY{o}{.}\PY{n}{subplots}\PY{p}{(}\PY{l+m+mi}{1}\PY{p}{,} \PY{l+m+mi}{1}\PY{p}{,} \PY{n}{figsize} \PY{o}{=} \PY{p}{(}\PY{l+m+mi}{8}\PY{p}{,} \PY{l+m+mi}{4}\PY{p}{)}\PY{p}{)}
        \PY{c+c1}{\PYZsh{}increases readability of plot}
        \PY{n}{ax1}\PY{o}{.}\PY{n}{set\PYZus{}title}\PY{p}{(}\PY{l+s+s2}{\PYZdq{}}\PY{l+s+s2}{Fourier Transform Coefficients of a Square Wave}\PY{l+s+s2}{\PYZdq{}}\PY{p}{)}
        \PY{n}{ax1}\PY{o}{.}\PY{n}{plot}\PY{p}{(}\PY{n+nb}{abs}\PY{p}{(}\PY{n}{c}\PY{p}{)}\PY{p}{)}
        \PY{n}{plt}\PY{o}{.}\PY{n}{xlim}\PY{p}{(}\PY{l+m+mi}{0}\PY{p}{,}\PY{l+m+mi}{200}\PY{p}{)}
        \PY{n}{plt}\PY{o}{.}\PY{n}{show}\PY{p}{(}\PY{p}{)}
\end{Verbatim}

    \begin{center}
    \adjustimage{max size={0.9\linewidth}{0.9\paperheight}}{output_3_0.png}
    \end{center}
    { \hspace*{\fill} \\}
    
    \begin{Verbatim}[commandchars=\\\{\}]
{\color{incolor}In [{\color{incolor}4}]:} \PY{n}{N} \PY{o}{=} \PY{l+m+mi}{250}
        
        \PY{c+c1}{\PYZsh{}defines sawtooth wave}
        \PY{n}{y} \PY{o}{=} \PY{n}{np}\PY{o}{.}\PY{n}{zeros}\PY{p}{(}\PY{n}{N}\PY{p}{)}
        \PY{k}{for} \PY{n}{n} \PY{o+ow}{in} \PY{n+nb}{range}\PY{p}{(}\PY{n+nb}{len}\PY{p}{(}\PY{n}{y}\PY{p}{)}\PY{p}{)}\PY{p}{:}
            \PY{n}{y}\PY{p}{[}\PY{n}{n}\PY{p}{]} \PY{o}{=} \PY{n}{n}
        
        \PY{n}{c} \PY{o}{=} \PY{n}{rfft}\PY{p}{(}\PY{n}{y}\PY{p}{)}
        
        \PY{n}{fig2}\PY{p}{,} \PY{n}{ax2} \PY{o}{=} \PY{n}{plt}\PY{o}{.}\PY{n}{subplots}\PY{p}{(}\PY{l+m+mi}{1}\PY{p}{,} \PY{l+m+mi}{1}\PY{p}{,} \PY{n}{figsize} \PY{o}{=} \PY{p}{(}\PY{l+m+mi}{8}\PY{p}{,} \PY{l+m+mi}{4}\PY{p}{)}\PY{p}{)}
        \PY{c+c1}{\PYZsh{}increases readability of plot}
        \PY{n}{ax2}\PY{o}{.}\PY{n}{set\PYZus{}title}\PY{p}{(}\PY{l+s+s2}{\PYZdq{}}\PY{l+s+s2}{Fourier Transform Coefficients of a Sawtooth Wave}\PY{l+s+s2}{\PYZdq{}}\PY{p}{)}
        \PY{n}{ax2}\PY{o}{.}\PY{n}{plot}\PY{p}{(}\PY{n+nb}{abs}\PY{p}{(}\PY{n}{c}\PY{p}{)}\PY{p}{)}
        \PY{n}{plt}\PY{o}{.}\PY{n}{xlim}\PY{p}{(}\PY{l+m+mi}{0}\PY{p}{,}\PY{n+nb}{int}\PY{p}{(}\PY{n}{N}\PY{o}{/}\PY{l+m+mi}{2}\PY{p}{)}\PY{p}{)}
        \PY{n}{plt}\PY{o}{.}\PY{n}{show}\PY{p}{(}\PY{p}{)}
\end{Verbatim}

    \begin{center}
    \adjustimage{max size={0.9\linewidth}{0.9\paperheight}}{output_4_0.png}
    \end{center}
    { \hspace*{\fill} \\}
    
    \begin{Verbatim}[commandchars=\\\{\}]
{\color{incolor}In [{\color{incolor}5}]:} \PY{n}{N} \PY{o}{=} \PY{l+m+mi}{1000}
        
        \PY{c+c1}{\PYZsh{}defines sawtooth wave}
        \PY{n}{x} \PY{o}{=} \PY{n}{np}\PY{o}{.}\PY{n}{linspace}\PY{p}{(}\PY{o}{\PYZhy{}}\PY{l+m+mi}{1}\PY{p}{,}\PY{l+m+mi}{1}\PY{p}{,}\PY{n}{N}\PY{p}{)}
        \PY{n}{y} \PY{o}{=} \PY{n}{np}\PY{o}{.}\PY{n}{zeros}\PY{p}{(}\PY{n}{N}\PY{p}{)}
        \PY{k}{for} \PY{n}{n} \PY{o+ow}{in} \PY{n+nb}{range}\PY{p}{(}\PY{n+nb}{len}\PY{p}{(}\PY{n}{y}\PY{p}{)}\PY{p}{)}\PY{p}{:}
            \PY{n}{p1} \PY{o}{=} \PY{n}{sin}\PY{p}{(}\PY{n}{pi}\PY{o}{*}\PY{n}{n} \PY{o}{/} \PY{n}{N}\PY{p}{)}
            \PY{n}{p2} \PY{o}{=} \PY{n}{sin}\PY{p}{(}\PY{l+m+mi}{20}\PY{o}{*}\PY{n}{pi}\PY{o}{*}\PY{n}{n} \PY{o}{/} \PY{n}{N}\PY{p}{)}
            \PY{n}{y}\PY{p}{[}\PY{n}{n}\PY{p}{]} \PY{o}{=} \PY{n}{p1}\PY{o}{*}\PY{n}{p2}
        
        \PY{n}{c} \PY{o}{=} \PY{n}{rfft}\PY{p}{(}\PY{n}{y}\PY{p}{)}
        
        \PY{n}{fig3}\PY{p}{,} \PY{n}{ax3} \PY{o}{=} \PY{n}{plt}\PY{o}{.}\PY{n}{subplots}\PY{p}{(}\PY{l+m+mi}{1}\PY{p}{,} \PY{l+m+mi}{1}\PY{p}{,} \PY{n}{figsize} \PY{o}{=} \PY{p}{(}\PY{l+m+mi}{8}\PY{p}{,} \PY{l+m+mi}{4}\PY{p}{)}\PY{p}{)}
        \PY{c+c1}{\PYZsh{}increases readability of plot}
        \PY{n}{ax3}\PY{o}{.}\PY{n}{set\PYZus{}title}\PY{p}{(}\PY{l+s+s2}{\PYZdq{}}\PY{l+s+s2}{Fourier Transform Coefficients of a Modulated Sine Wave}\PY{l+s+s2}{\PYZdq{}}\PY{p}{)}
        \PY{n}{ax3}\PY{o}{.}\PY{n}{plot}\PY{p}{(}\PY{n+nb}{abs}\PY{p}{(}\PY{n}{c}\PY{p}{)}\PY{p}{)}
        \PY{n}{plt}\PY{o}{.}\PY{n}{xlim}\PY{p}{(}\PY{l+m+mi}{0}\PY{p}{,}\PY{l+m+mi}{200}\PY{p}{)}
        \PY{n}{plt}\PY{o}{.}\PY{n}{show}\PY{p}{(}\PY{p}{)}
\end{Verbatim}

    \begin{center}
    \adjustimage{max size={0.9\linewidth}{0.9\paperheight}}{output_5_0.png}
    \end{center}
    { \hspace*{\fill} \\}
    
    \section{CP 7.2 Detecting
periodicity}\label{cp-7.2-detecting-periodicity}

Looking at data from January 1749 onwards on the number of monthly
sunspots on the sun, we can examine the period of certain fluctuations
by applying Fourier transforms to the data. Clearly, it would be
difficult to analyze the period based on the graph below alone. It looks
as if the distance from peak to peak is roughly 125 months which
corresponds to a period of a little under 10.5 years, but this is a
fairly imprecise way of doing it.

    \begin{Verbatim}[commandchars=\\\{\}]
{\color{incolor}In [{\color{incolor}6}]:} \PY{n}{sunspots} \PY{o}{=} \PY{n}{np}\PY{o}{.}\PY{n}{loadtxt}\PY{p}{(}\PY{l+s+s2}{\PYZdq{}}\PY{l+s+s2}{sunspots.txt}\PY{l+s+s2}{\PYZdq{}}\PY{p}{,} \PY{n+nb}{float}\PY{p}{)}
        
        \PY{n}{fig4}\PY{p}{,} \PY{n}{ax4} \PY{o}{=} \PY{n}{plt}\PY{o}{.}\PY{n}{subplots}\PY{p}{(}\PY{l+m+mi}{1}\PY{p}{,} \PY{l+m+mi}{1}\PY{p}{,} \PY{n}{figsize} \PY{o}{=} \PY{p}{(}\PY{l+m+mi}{12}\PY{p}{,} \PY{l+m+mi}{4}\PY{p}{)}\PY{p}{)}
        \PY{n}{x} \PY{o}{=} \PY{n}{sunspots}\PY{p}{[}\PY{p}{:}\PY{p}{,}\PY{l+m+mi}{0}\PY{p}{]}
        \PY{n}{y} \PY{o}{=} \PY{n}{sunspots}\PY{p}{[}\PY{p}{:}\PY{p}{,}\PY{l+m+mi}{1}\PY{p}{]}
        \PY{c+c1}{\PYZsh{}plt.plot(x[:500],y[:500])}
        \PY{n}{ax4}\PY{o}{.}\PY{n}{plot}\PY{p}{(}\PY{n}{x}\PY{p}{,}\PY{n}{y}\PY{p}{)}
        \PY{n}{plt}\PY{o}{.}\PY{n}{show}\PY{p}{(}\PY{p}{)}
\end{Verbatim}

    \begin{center}
    \adjustimage{max size={0.9\linewidth}{0.9\paperheight}}{output_7_0.png}
    \end{center}
    { \hspace*{\fill} \\}
    
    \begin{Verbatim}[commandchars=\\\{\}]
{\color{incolor}In [{\color{incolor}7}]:} \PY{n}{c} \PY{o}{=} \PY{n}{rfft}\PY{p}{(}\PY{n}{y}\PY{p}{)}
        \PY{n}{ck2} \PY{o}{=} \PY{n+nb}{abs}\PY{p}{(}\PY{n}{c}\PY{p}{)}\PY{o}{*}\PY{o}{*}\PY{l+m+mi}{2} \PY{c+c1}{\PYZsh{}magnitude squared for coefficients}
        
        \PY{n}{fig5}\PY{p}{,} \PY{n}{ax5} \PY{o}{=} \PY{n}{plt}\PY{o}{.}\PY{n}{subplots}\PY{p}{(}\PY{l+m+mi}{1}\PY{p}{,} \PY{l+m+mi}{1}\PY{p}{,} \PY{n}{figsize} \PY{o}{=} \PY{p}{(}\PY{l+m+mi}{8}\PY{p}{,} \PY{l+m+mi}{4}\PY{p}{)}\PY{p}{)}
        \PY{c+c1}{\PYZsh{}increases readability of plot}
        \PY{n}{ax5}\PY{o}{.}\PY{n}{set\PYZus{}title}\PY{p}{(}\PY{l+s+s2}{\PYZdq{}}\PY{l+s+s2}{Power Spectrum of the Sunspots}\PY{l+s+s2}{\PYZdq{}}\PY{p}{)}
        \PY{n}{ax5}\PY{o}{.}\PY{n}{plot}\PY{p}{(}\PY{n}{ck2}\PY{p}{)}
        \PY{n}{ax5}\PY{o}{.}\PY{n}{set\PYZus{}ylabel}\PY{p}{(}\PY{l+s+s2}{\PYZdq{}}\PY{l+s+s2}{\PYZdl{}|c\PYZus{}k|\PYZca{}2\PYZdl{}}\PY{l+s+s2}{\PYZdq{}}\PY{p}{)}
        \PY{n}{plt}\PY{o}{.}\PY{n}{xlim}\PY{p}{(}\PY{o}{\PYZhy{}}\PY{l+m+mi}{5}\PY{p}{,}\PY{l+m+mi}{250}\PY{p}{)}
        \PY{n}{plt}\PY{o}{.}\PY{n}{show}\PY{p}{(}\PY{p}{)}
\end{Verbatim}

    \begin{center}
    \adjustimage{max size={0.9\linewidth}{0.9\paperheight}}{output_8_0.png}
    \end{center}
    { \hspace*{\fill} \\}
    
    As mentioned in the problem, there is a significant spike in
\(|c_k|^2 \approx 2 \times 10^9\) for some \(k \ne 0.\) While a much
smaller spike than the one at \(k=0,\) this implies the existence of
periodicity in the amount of sunspots. For the coefficients

\[c_k = \sum_{n=0}^{N-1}\ y_n \text{exp}{\bigr(-\text{i}\frac{2\pi k n}{N}\bigr)},\]

the period is determined by the additional factors in the exponent.
Without any additional factors, \(e^{-\text{i}x}\) has a period of
\(T= 2\pi.\) The new period \(T'\) can be found by multiplying (or
dividing) any additional factors that are added to the argument of the
exponential. So, for the exponential we have here, \(T' = \frac Nk\)
where \(N\) is the number of data points observed and \(k\) is the
frequency the specific coefficient is at.

    \begin{Verbatim}[commandchars=\\\{\}]
{\color{incolor}In [{\color{incolor}8}]:} \PY{c+c1}{\PYZsh{}finding k for secondary peak}
        
        \PY{c+c1}{\PYZsh{}start searching for maximum after initial peak}
        \PY{c+c1}{\PYZsh{}then add index where slicing started to find k}
        \PY{n}{k} \PY{o}{=} \PY{n}{np}\PY{o}{.}\PY{n}{argmax}\PY{p}{(}\PY{n}{ck2}\PY{p}{[}\PY{l+m+mi}{15}\PY{p}{:}\PY{p}{]}\PY{p}{)} \PY{o}{+} \PY{l+m+mi}{15}
        \PY{n}{N} \PY{o}{=} \PY{n+nb}{len}\PY{p}{(}\PY{n}{y}\PY{p}{)}
        
        \PY{n}{T} \PY{o}{=} \PY{n}{N} \PY{o}{/} \PY{n}{k}
        \PY{n}{T}
\end{Verbatim}

\begin{Verbatim}[commandchars=\\\{\}]
{\color{outcolor}Out[{\color{outcolor}8}]:} 130.958
\end{Verbatim}
            
    Slicing the data around the secondary peak above, it can be found that
the spike of \(|c_k|^2\) is located at \(k=24.\) Combined with the fact
that \(N = 3143,\) we have \(T' = 130.958\) months which corresponds to
10.91 years. This is close to the result found previously by comparing
the locations of successive peaks.

    \section{CP 7.3 Fourier transforms of musical
instruments}\label{cp-7.3-fourier-transforms-of-musical-instruments}

To plot Fourier coefficients from larger datasets, it is necessary to
use the fast Fourier transform (FFT) which performs significantly fewer
operations to calculate the results. While discrete Fourier transforms
are \(\mathcal{O}(\tfrac12 N^2),\) FFTs repeatedly splits the given set
of points into subgroups to reduce the number of calculations, making it
\(\mathcal{O}(N \log_2 N).\)

For the following data from "trumpets.txt" and "piano.txt," there are
enough points to necessitate the use of an FFT.

    \begin{Verbatim}[commandchars=\\\{\}]
{\color{incolor}In [{\color{incolor}9}]:} \PY{k}{def} \PY{n+nf}{Ck}\PY{p}{(}\PY{n}{filename}\PY{p}{)}\PY{p}{:}
            \PY{l+s+sd}{\PYZdq{}\PYZdq{}\PYZdq{}Plots the magnitudes of FFT coefficients}
        \PY{l+s+sd}{        based on data read from the given file\PYZdq{}\PYZdq{}\PYZdq{}}
            
            \PY{c+c1}{\PYZsh{}loads file and does rfft}
            \PY{n}{file} \PY{o}{=} \PY{n}{np}\PY{o}{.}\PY{n}{loadtxt}\PY{p}{(}\PY{n}{filename}\PY{p}{,} \PY{n+nb}{float}\PY{p}{)}
            \PY{n}{y} \PY{o}{=} \PY{n}{file}
            \PY{n}{c} \PY{o}{=} \PY{n}{rfft}\PY{p}{(}\PY{n}{y}\PY{p}{)}
        
            \PY{n}{fig}\PY{p}{,} \PY{n}{ax} \PY{o}{=} \PY{n}{plt}\PY{o}{.}\PY{n}{subplots}\PY{p}{(}\PY{l+m+mi}{2}\PY{p}{,} \PY{l+m+mi}{1}\PY{p}{,} \PY{n}{figsize} \PY{o}{=} \PY{p}{(}\PY{l+m+mi}{12}\PY{p}{,} \PY{l+m+mi}{8}\PY{p}{)}\PY{p}{)}
            \PY{n}{ax}\PY{p}{[}\PY{l+m+mi}{0}\PY{p}{]}\PY{o}{.}\PY{n}{plot}\PY{p}{(}\PY{n}{y}\PY{p}{,}\PY{l+s+s1}{\PYZsq{}}\PY{l+s+s1}{m,}\PY{l+s+s1}{\PYZsq{}}\PY{p}{)}
            \PY{n}{ax}\PY{p}{[}\PY{l+m+mi}{1}\PY{p}{]}\PY{o}{.}\PY{n}{plot}\PY{p}{(}\PY{n+nb}{abs}\PY{p}{(}\PY{n}{c}\PY{p}{)}\PY{p}{,} \PY{l+s+s1}{\PYZsq{}}\PY{l+s+s1}{c}\PY{l+s+s1}{\PYZsq{}}\PY{p}{)}
            
            \PY{c+c1}{\PYZsh{}creates title for plot}
            \PY{n}{title} \PY{o}{=} \PY{n}{filename}\PY{p}{[}\PY{p}{:}\PY{o}{\PYZhy{}}\PY{l+m+mi}{4}\PY{p}{]}\PY{o}{.}\PY{n}{capitalize}\PY{p}{(}\PY{p}{)}
            \PY{n}{ax}\PY{p}{[}\PY{l+m+mi}{0}\PY{p}{]}\PY{o}{.}\PY{n}{set\PYZus{}title}\PY{p}{(}\PY{l+s+s2}{\PYZdq{}}\PY{l+s+s2}{Waveform for }\PY{l+s+si}{\PYZpc{}s}\PY{l+s+s2}{ Sound}\PY{l+s+s2}{\PYZdq{}} \PY{o}{\PYZpc{}} \PY{n}{title}\PY{p}{)}
            \PY{n}{ax}\PY{p}{[}\PY{l+m+mi}{0}\PY{p}{]}\PY{o}{.}\PY{n}{set\PYZus{}ylabel}\PY{p}{(}\PY{l+s+s2}{\PYZdq{}}\PY{l+s+s2}{Amplitude}\PY{l+s+s2}{\PYZdq{}}\PY{p}{)}
            \PY{n}{ax}\PY{p}{[}\PY{l+m+mi}{1}\PY{p}{]}\PY{o}{.}\PY{n}{set\PYZus{}ylabel}\PY{p}{(}\PY{l+s+s2}{\PYZdq{}}\PY{l+s+s2}{\PYZdl{}|c\PYZus{}k|\PYZdl{}}\PY{l+s+s2}{\PYZdq{}}\PY{p}{)}
            \PY{n}{ax}\PY{p}{[}\PY{l+m+mi}{1}\PY{p}{]}\PY{o}{.}\PY{n}{set\PYZus{}xlabel}\PY{p}{(}\PY{l+s+s2}{\PYZdq{}}\PY{l+s+s2}{Sample Number}\PY{l+s+s2}{\PYZdq{}}\PY{p}{)}
            \PY{n}{ax}\PY{p}{[}\PY{l+m+mi}{1}\PY{p}{]}\PY{o}{.}\PY{n}{set\PYZus{}title}\PY{p}{(}\PY{l+s+s2}{\PYZdq{}}\PY{l+s+s2}{FFT Coefficients for }\PY{l+s+si}{\PYZpc{}s}\PY{l+s+s2}{ Sound}\PY{l+s+s2}{\PYZdq{}} \PY{o}{\PYZpc{}} \PY{n}{title}\PY{p}{)}
            
            \PY{n}{plt}\PY{o}{.}\PY{n}{xlim}\PY{p}{(}\PY{l+m+mi}{0}\PY{p}{,}\PY{l+m+mi}{20000}\PY{p}{)}
            \PY{n}{plt}\PY{o}{.}\PY{n}{show}\PY{p}{(}\PY{p}{)}
            
            \PY{k}{return} \PY{n}{c}
        
        \PY{c+c1}{\PYZsh{}saving coefficients for both}
        \PY{n}{ckp} \PY{o}{=} \PY{n}{Ck}\PY{p}{(}\PY{l+s+s2}{\PYZdq{}}\PY{l+s+s2}{piano.txt}\PY{l+s+s2}{\PYZdq{}}\PY{p}{)}
        \PY{n}{ckt} \PY{o}{=} \PY{n}{Ck}\PY{p}{(}\PY{l+s+s2}{\PYZdq{}}\PY{l+s+s2}{trumpet.txt}\PY{l+s+s2}{\PYZdq{}}\PY{p}{)}
\end{Verbatim}

    \begin{center}
    \adjustimage{max size={0.9\linewidth}{0.9\paperheight}}{output_13_0.png}
    \end{center}
    { \hspace*{\fill} \\}
    
    \begin{center}
    \adjustimage{max size={0.9\linewidth}{0.9\paperheight}}{output_13_1.png}
    \end{center}
    { \hspace*{\fill} \\}
    
    Because the magnitude of the FFT coefficients calculated from the piano
data is more concentrated at one frequency, its functional form is
closer to a pure sine wave than the form of the trumpet's sound. The
successive spikes in both the piano and trumpet data represent the
harmonics of the peak spike. Because there aren't a lot of random spikes
in the plots, we can deduce that there was very little background noise
and almost everything seen is purely the instruments' sounds.

Knowing that the sampling rate of these recordings is 44100 allows us to
calculate the numerical frequency that \(k\) corresponds to for each of
the \(|c_k|\) coefficients. In turn, knowing that middle C corresponds
to the frequency 261 Hz will allow us to calculate the note played by
the trumpet and the piano. If we find \(k\) for each of the spikes, we
can calculate the frequency knowing they are harmonics of each other.
Let the locations of the peaks be stored in lists \(k_p\) and \(k_t\)
(for piano and trumpet respectively) where the harmonic frequencies are
the entries.

    \begin{Verbatim}[commandchars=\\\{\}]
{\color{incolor}In [{\color{incolor}10}]:} \PY{n}{M} \PY{o}{=} \PY{l+m+mi}{5}
         \PY{c+c1}{\PYZsh{}finding k that corresponds to peaks}
         
         \PY{n}{kp} \PY{o}{=} \PY{n}{np}\PY{o}{.}\PY{n}{zeros}\PY{p}{(}\PY{n}{M}\PY{o}{+}\PY{l+m+mi}{1}\PY{p}{,}\PY{n+nb}{float}\PY{p}{)}
         \PY{c+c1}{\PYZsh{}difference between successive harmonics}
         \PY{n}{dkp} \PY{o}{=} \PY{n}{np}\PY{o}{.}\PY{n}{zeros}\PY{p}{(}\PY{n}{M}\PY{p}{,}\PY{n+nb}{float}\PY{p}{)}
         \PY{k}{for} \PY{n}{i} \PY{o+ow}{in} \PY{n+nb}{range}\PY{p}{(}\PY{n}{M}\PY{o}{+}\PY{l+m+mi}{1}\PY{p}{)}\PY{p}{:}
             \PY{n}{kp}\PY{p}{[}\PY{n}{i}\PY{p}{]} \PY{o}{=} \PY{n}{np}\PY{o}{.}\PY{n}{argmax}\PY{p}{(}\PY{n}{ckp}\PY{p}{[}\PY{n}{i}\PY{o}{*}\PY{l+m+mi}{1250}\PY{p}{:}\PY{p}{(}\PY{n}{i}\PY{o}{+}\PY{l+m+mi}{1}\PY{p}{)}\PY{o}{*}\PY{l+m+mi}{1250}\PY{p}{]}\PY{p}{)} \PY{o}{+} \PY{n}{i}\PY{o}{*}\PY{l+m+mi}{1250}
             
         \PY{n}{N} \PY{o}{=} \PY{l+m+mi}{9}
         
         \PY{n}{kt} \PY{o}{=} \PY{n}{np}\PY{o}{.}\PY{n}{zeros}\PY{p}{(}\PY{n}{N}\PY{o}{+}\PY{l+m+mi}{1}\PY{p}{,}\PY{n+nb}{float}\PY{p}{)}
         \PY{c+c1}{\PYZsh{}difference between successive harmonics}
         \PY{n}{dkt} \PY{o}{=} \PY{n}{np}\PY{o}{.}\PY{n}{zeros}\PY{p}{(}\PY{n}{N}\PY{p}{,} \PY{n+nb}{float}\PY{p}{)}
         \PY{k}{for} \PY{n}{i} \PY{o+ow}{in} \PY{n+nb}{range}\PY{p}{(}\PY{n}{N}\PY{o}{+}\PY{l+m+mi}{1}\PY{p}{)}\PY{p}{:}
             \PY{n}{kt}\PY{p}{[}\PY{n}{i}\PY{p}{]} \PY{o}{=} \PY{n}{np}\PY{o}{.}\PY{n}{argmax}\PY{p}{(}\PY{n}{ckt}\PY{p}{[}\PY{n}{i}\PY{o}{*}\PY{l+m+mi}{1300}\PY{p}{:}\PY{p}{(}\PY{n}{i}\PY{o}{+}\PY{l+m+mi}{1}\PY{p}{)}\PY{o}{*}\PY{l+m+mi}{1300}\PY{p}{]}\PY{p}{)} \PY{o}{+} \PY{n}{i}\PY{o}{*}\PY{l+m+mi}{1300}
         
         \PY{n+nb}{print}\PY{p}{(}\PY{n}{kp}\PY{p}{)}
         \PY{n+nb}{print}\PY{p}{(}\PY{n}{kt}\PY{p}{)}
         
         \PY{c+c1}{\PYZsh{}calculations outlined described below}
         \PY{n}{factor} \PY{o}{=} \PY{l+m+mf}{1e5}\PY{o}{/}\PY{l+m+mi}{44100}
         \PY{k}{for} \PY{n}{j} \PY{o+ow}{in} \PY{n+nb}{range}\PY{p}{(}\PY{n}{N}\PY{o}{+}\PY{l+m+mi}{1}\PY{p}{)}\PY{p}{:}
             \PY{k}{try}\PY{p}{:}
                 \PY{n}{k} \PY{o}{=} \PY{p}{(}\PY{n}{kp}\PY{p}{[}\PY{n}{j}\PY{p}{]} \PY{o}{+} \PY{n}{kt}\PY{p}{[}\PY{n}{j}\PY{p}{]}\PY{p}{)} \PY{o}{/} \PY{l+m+mi}{2}
                 \PY{n+nb}{print}\PY{p}{(}\PY{l+s+s2}{\PYZdq{}}\PY{l+s+s2}{The frequency is }\PY{l+s+si}{\PYZob{}:4.3f\PYZcb{}}\PY{l+s+s2}{ Hz.}\PY{l+s+s2}{\PYZdq{}}\PYZbs{}
                      \PY{o}{.}\PY{n}{format}\PY{p}{(}\PY{n}{k}\PY{o}{/}\PY{n}{factor}\PY{p}{)}\PY{p}{)}
             \PY{k}{except}\PY{p}{:}
                 \PY{n}{s}\PY{o}{=}\PY{l+m+mi}{0}
\end{Verbatim}

    \begin{Verbatim}[commandchars=\\\{\}]
[ 1191.  2384.  3579.  4814.  6072.  7272.]
[  1181.   2367.   3551.   4731.   5908.   7101.   8286.   9458.  10652.
  11837.]
The frequency is 523.026 Hz.
The frequency is 1047.595 Hz.
The frequency is 1572.165 Hz.
The frequency is 2104.673 Hz.
The frequency is 2641.590 Hz.
The frequency is 3169.247 Hz.

    \end{Verbatim}

    Thus, we can see that the separate spikes, do indeed represent harmonic
frequencies because of the highly rigid pattern they appear in. Based on
the \(k\) we are looking at a specific coefficient for, we can calculate
that frequency in Hz based on the number of points sampled and the
sampling rate. Because the frequency \(k\) we found is related to the
number of data points by way of the sampling rate, we know the frequency
of the note played (in Hz) is

\[ f = k\ \biggr/ \frac{100000}{44100}.\]

where 100000 is the number of data points we see in the original file.
So for our first spike we see a frequency of 523 Hz which is quite close
to twice the frequency in the hint of 261 Hz. So 523 Hz corresponds to C
one octave above middle C (C5). Similarly, the second frequency we see
(1048 Hz) corresponds to C6. The third frequency of 1572 Hz matches that
of G6. This makes sense because G is an overtone of C, which means when
C is played G is also heard.

    \section{CP 7.4 Fourier filtering and
smoothing}\label{cp-7.4-fourier-filtering-and-smoothing}

This problem examines stock market data for about four years (late 2006
to late 2010) via the Dow Jones Industrial Average.

    \begin{Verbatim}[commandchars=\\\{\}]
{\color{incolor}In [{\color{incolor}11}]:} \PY{c+c1}{\PYZsh{}loading DJI data}
         \PY{n}{y} \PY{o}{=} \PY{n}{np}\PY{o}{.}\PY{n}{loadtxt}\PY{p}{(}\PY{l+s+s2}{\PYZdq{}}\PY{l+s+s2}{dow.txt}\PY{l+s+s2}{\PYZdq{}}\PY{p}{,} \PY{n+nb}{float}\PY{p}{)}
         \PY{n}{N} \PY{o}{=} \PY{n+nb}{len}\PY{p}{(}\PY{n}{y}\PY{p}{)}
         \PY{n}{x} \PY{o}{=} \PY{n}{np}\PY{o}{.}\PY{n}{arange}\PY{p}{(}\PY{n}{N}\PY{p}{)}
         
         \PY{c+c1}{\PYZsh{}doing FFT on DJI data}
         \PY{n}{c} \PY{o}{=} \PY{n}{rfft}\PY{p}{(}\PY{n}{y}\PY{p}{)}
         \PY{n}{M} \PY{o}{=} \PY{n+nb}{len}\PY{p}{(}\PY{n}{c}\PY{p}{)}
         
         \PY{n}{fig6}\PY{p}{,} \PY{n}{ax6} \PY{o}{=} \PY{n}{plt}\PY{o}{.}\PY{n}{subplots}\PY{p}{(}\PY{l+m+mi}{2}\PY{p}{,} \PY{l+m+mi}{1}\PY{p}{,} \PY{n}{figsize} \PY{o}{=} \PY{p}{(}\PY{l+m+mi}{12}\PY{p}{,} \PY{l+m+mi}{8}\PY{p}{)}\PY{p}{)}
         
         \PY{c+c1}{\PYZsh{}10\PYZpc{} smoothed}
         \PY{n}{smooth} \PY{o}{=} \PY{l+m+mf}{0.1}
         \PY{n}{c}\PY{p}{[}\PY{n+nb}{int}\PY{p}{(}\PY{n}{smooth}\PY{o}{*}\PY{n}{M}\PY{p}{)}\PY{p}{:}\PY{p}{]} \PY{o}{=} \PY{l+m+mi}{0}
         \PY{n}{d} \PY{o}{=} \PY{n}{irfft}\PY{p}{(}\PY{n}{c}\PY{p}{)}
         \PY{n}{ax6}\PY{p}{[}\PY{l+m+mi}{0}\PY{p}{]}\PY{o}{.}\PY{n}{plot}\PY{p}{(}\PY{n}{x}\PY{p}{,} \PY{n}{y}\PY{p}{,} \PY{l+s+s1}{\PYZsq{}}\PY{l+s+s1}{m}\PY{l+s+s1}{\PYZsq{}}\PY{p}{,} \PY{n}{label}\PY{o}{=}\PY{l+s+s1}{\PYZsq{}}\PY{l+s+s1}{DJI}\PY{l+s+s1}{\PYZsq{}}\PY{p}{)}
         \PY{n}{ax6}\PY{p}{[}\PY{l+m+mi}{0}\PY{p}{]}\PY{o}{.}\PY{n}{plot}\PY{p}{(}\PY{n}{x}\PY{p}{,} \PY{n}{d}\PY{p}{,} \PY{l+s+s1}{\PYZsq{}}\PY{l+s+s1}{c}\PY{l+s+s1}{\PYZsq{}}\PY{p}{,} \PY{n}{label}\PY{o}{=}\PY{l+s+s1}{\PYZsq{}}\PY{l+s+s1}{Smoothed (10}\PY{l+s+s1}{\PYZpc{}}\PY{l+s+s1}{) DJI}\PY{l+s+s1}{\PYZsq{}}\PY{p}{)}
         
         \PY{c+c1}{\PYZsh{}2\PYZpc{} smoothed}
         \PY{n}{smooth} \PY{o}{=} \PY{l+m+mf}{0.02}
         \PY{n}{c}\PY{p}{[}\PY{n+nb}{int}\PY{p}{(}\PY{n}{smooth}\PY{o}{*}\PY{n}{M}\PY{p}{)}\PY{p}{:}\PY{p}{]} \PY{o}{=} \PY{l+m+mi}{0}
         \PY{n}{d} \PY{o}{=} \PY{n}{irfft}\PY{p}{(}\PY{n}{c}\PY{p}{)}
         \PY{n}{ax6}\PY{p}{[}\PY{l+m+mi}{1}\PY{p}{]}\PY{o}{.}\PY{n}{plot}\PY{p}{(}\PY{n}{x}\PY{p}{,} \PY{n}{y}\PY{p}{,} \PY{l+s+s1}{\PYZsq{}}\PY{l+s+s1}{m}\PY{l+s+s1}{\PYZsq{}}\PY{p}{,} \PY{n}{label}\PY{o}{=}\PY{l+s+s1}{\PYZsq{}}\PY{l+s+s1}{DJI}\PY{l+s+s1}{\PYZsq{}}\PY{p}{)}
         \PY{n}{ax6}\PY{p}{[}\PY{l+m+mi}{1}\PY{p}{]}\PY{o}{.}\PY{n}{plot}\PY{p}{(}\PY{n}{x}\PY{p}{,} \PY{n}{d}\PY{p}{,} \PY{l+s+s1}{\PYZsq{}}\PY{l+s+s1}{c}\PY{l+s+s1}{\PYZsq{}}\PY{p}{,} \PY{n}{label}\PY{o}{=}\PY{l+s+s1}{\PYZsq{}}\PY{l+s+s1}{Smoothed (2}\PY{l+s+s1}{\PYZpc{}}\PY{l+s+s1}{) DJI}\PY{l+s+s1}{\PYZsq{}}\PY{p}{)}
         
         \PY{c+c1}{\PYZsh{}titles, axes, labels}
         \PY{n}{ax6}\PY{p}{[}\PY{l+m+mi}{0}\PY{p}{]}\PY{o}{.}\PY{n}{set\PYZus{}title}\PY{p}{(}\PY{l+s+s2}{\PYZdq{}}\PY{l+s+s2}{DJI Average Over Time}\PY{l+s+s2}{\PYZdq{}}\PY{p}{)}
         \PY{n}{ax6}\PY{p}{[}\PY{l+m+mi}{0}\PY{p}{]}\PY{o}{.}\PY{n}{set\PYZus{}ylabel}\PY{p}{(}\PY{l+s+s2}{\PYZdq{}}\PY{l+s+s2}{Price (\PYZdl{})}\PY{l+s+s2}{\PYZdq{}}\PY{p}{)}
         \PY{n}{ax6}\PY{p}{[}\PY{l+m+mi}{1}\PY{p}{]}\PY{o}{.}\PY{n}{set\PYZus{}ylabel}\PY{p}{(}\PY{l+s+s2}{\PYZdq{}}\PY{l+s+s2}{Price (\PYZdl{})}\PY{l+s+s2}{\PYZdq{}}\PY{p}{)}
         \PY{n}{ax6}\PY{p}{[}\PY{l+m+mi}{1}\PY{p}{]}\PY{o}{.}\PY{n}{set\PYZus{}xlabel}\PY{p}{(}\PY{l+s+s2}{\PYZdq{}}\PY{l+s+s2}{Time (days)}\PY{l+s+s2}{\PYZdq{}}\PY{p}{)}
         \PY{n}{ax6}\PY{p}{[}\PY{l+m+mi}{0}\PY{p}{]}\PY{o}{.}\PY{n}{legend}\PY{p}{(}\PY{p}{)}
         \PY{n}{ax6}\PY{p}{[}\PY{l+m+mi}{1}\PY{p}{]}\PY{o}{.}\PY{n}{legend}\PY{p}{(}\PY{p}{)}
         \PY{n}{plt}\PY{o}{.}\PY{n}{show}\PY{p}{(}\PY{p}{)}
\end{Verbatim}

    \begin{center}
    \adjustimage{max size={0.9\linewidth}{0.9\paperheight}}{output_18_0.png}
    \end{center}
    { \hspace*{\fill} \\}
    
    By setting the later values in the FFT coefficient array to 0, we are
essentially deleting the higher frequencies that the FFT calculated. So,
by taking the inverse after this, we are able to get back the data
without any of the higher frequency oscillations (noise) from before. By
starting to set coefficients \(c_k = 0\) earlier in the array, we begin
including only the largest scale oscillations from the original data at
lower frequencies, which is why keeping only the first 2\% of
coefficients results in much greater smoothing than at the 10\% level.

    \section{CP 7.5 Artifacts}\label{cp-7.5-artifacts}

If we have a square wave of amplitude 1 and frequency 1 Hz defined as

\[f(t) = \begin{cases}
    1  & \quad\mbox{if } \lfloor 2t \rfloor \mbox{ is even,} \\
    -1 & \quad\mbox{if } \lfloor 2t \rfloor \mbox{ is odd,}
    \end{cases}\]

then we can attempt a process similar to in exercise 7.4 to smooth this
function.

    \begin{Verbatim}[commandchars=\\\{\}]
{\color{incolor}In [{\color{incolor}12}]:} \PY{k}{def} \PY{n+nf}{f}\PY{p}{(}\PY{n}{t}\PY{p}{)}\PY{p}{:}
             \PY{l+s+sd}{\PYZdq{}\PYZdq{}\PYZdq{}Calculates function as defined above\PYZdq{}\PYZdq{}\PYZdq{}}
             \PY{n}{y} \PY{o}{=} \PY{n}{np}\PY{o}{.}\PY{n}{zeros}\PY{p}{(}\PY{n}{N}\PY{p}{)}
             
             \PY{k}{for} \PY{n}{i} \PY{o+ow}{in} \PY{n+nb}{range}\PY{p}{(}\PY{n+nb}{len}\PY{p}{(}\PY{n}{t}\PY{p}{)}\PY{p}{)}\PY{p}{:}
                 \PY{k}{if} \PY{n}{floor}\PY{p}{(}\PY{l+m+mi}{2}\PY{o}{*}\PY{n}{t}\PY{p}{[}\PY{n}{i}\PY{p}{]}\PY{p}{)} \PY{o}{\PYZpc{}} \PY{l+m+mi}{2} \PY{o}{==} \PY{l+m+mi}{0}\PY{p}{:}
                     \PY{n}{y}\PY{p}{[}\PY{n}{i}\PY{p}{]} \PY{o}{=} \PY{l+m+mi}{1}
                 \PY{k}{elif} \PY{n}{floor}\PY{p}{(}\PY{l+m+mi}{2}\PY{o}{*}\PY{n}{t}\PY{p}{[}\PY{n}{i}\PY{p}{]}\PY{p}{)}\PY{o}{\PYZpc{}}\PY{k}{2} == 1:
                     \PY{n}{y}\PY{p}{[}\PY{n}{i}\PY{p}{]} \PY{o}{=} \PY{o}{\PYZhy{}}\PY{l+m+mi}{1}
             \PY{k}{return} \PY{n}{y}
\end{Verbatim}

    \begin{Verbatim}[commandchars=\\\{\}]
{\color{incolor}In [{\color{incolor}13}]:} \PY{n}{N} \PY{o}{=} \PY{l+m+mi}{1000}
         \PY{n}{x} \PY{o}{=} \PY{n}{np}\PY{o}{.}\PY{n}{linspace}\PY{p}{(}\PY{o}{\PYZhy{}}\PY{l+m+mi}{1}\PY{p}{,}\PY{l+m+mi}{1}\PY{p}{,} \PY{n}{N}\PY{p}{)}
         
         \PY{n}{y} \PY{o}{=} \PY{n}{f}\PY{p}{(}\PY{n}{x}\PY{p}{)}
         \PY{n}{c} \PY{o}{=} \PY{n}{rfft}\PY{p}{(}\PY{n}{y}\PY{p}{)}
         \PY{n}{M} \PY{o}{=} \PY{n+nb}{len}\PY{p}{(}\PY{n}{c}\PY{p}{)}
         
         \PY{n}{fig7}\PY{p}{,} \PY{n}{ax7} \PY{o}{=} \PY{n}{plt}\PY{o}{.}\PY{n}{subplots}\PY{p}{(}\PY{l+m+mi}{1}\PY{p}{,} \PY{l+m+mi}{1}\PY{p}{,} \PY{n}{figsize} \PY{o}{=} \PY{p}{(}\PY{l+m+mi}{10}\PY{p}{,} \PY{l+m+mi}{5}\PY{p}{)}\PY{p}{)}
         
         \PY{n}{c}\PY{p}{[}\PY{l+m+mi}{10}\PY{p}{:}\PY{p}{]} \PY{o}{=} \PY{l+m+mi}{0}
         \PY{n}{d} \PY{o}{=} \PY{n}{irfft}\PY{p}{(}\PY{n}{c}\PY{p}{)}
         \PY{n}{ax7}\PY{o}{.}\PY{n}{plot}\PY{p}{(}\PY{n}{x}\PY{p}{,} \PY{n}{y}\PY{p}{,} \PY{l+s+s1}{\PYZsq{}}\PY{l+s+s1}{c}\PY{l+s+s1}{\PYZsq{}}\PY{p}{,} \PY{n}{label}\PY{o}{=}\PY{l+s+s1}{\PYZsq{}}\PY{l+s+s1}{Square Wave}\PY{l+s+s1}{\PYZsq{}}\PY{p}{)}
         \PY{n}{ax7}\PY{o}{.}\PY{n}{plot}\PY{p}{(}\PY{n}{x}\PY{p}{,} \PY{n}{d}\PY{p}{,} \PY{l+s+s1}{\PYZsq{}}\PY{l+s+s1}{k}\PY{l+s+s1}{\PYZsq{}}\PY{p}{,} \PY{n}{label}\PY{o}{=}\PY{l+s+s1}{\PYZsq{}}\PY{l+s+s1}{Smoothed Square Wave}\PY{l+s+s1}{\PYZsq{}}\PY{p}{)}
         
         \PY{c+c1}{\PYZsh{}titles, axes, labels}
         \PY{n}{ax7}\PY{o}{.}\PY{n}{set\PYZus{}title}\PY{p}{(}\PY{l+s+s2}{\PYZdq{}}\PY{l+s+s2}{Smoothing a Square Wave}\PY{l+s+s2}{\PYZdq{}}\PY{p}{)}
         \PY{n}{ax7}\PY{o}{.}\PY{n}{set\PYZus{}ylabel}\PY{p}{(}\PY{l+s+s2}{\PYZdq{}}\PY{l+s+s2}{Amplitude}\PY{l+s+s2}{\PYZdq{}}\PY{p}{)}
         \PY{n}{ax7}\PY{o}{.}\PY{n}{set\PYZus{}xlabel}\PY{p}{(}\PY{l+s+s2}{\PYZdq{}}\PY{l+s+s2}{x}\PY{l+s+s2}{\PYZdq{}}\PY{p}{)}
         \PY{n}{ax7}\PY{o}{.}\PY{n}{legend}\PY{p}{(}\PY{p}{)}
         \PY{n}{plt}\PY{o}{.}\PY{n}{show}\PY{p}{(}\PY{p}{)}
\end{Verbatim}

    \begin{center}
    \adjustimage{max size={0.9\linewidth}{0.9\paperheight}}{output_22_0.png}
    \end{center}
    { \hspace*{\fill} \\}
    
    After applying the same smoothing technique as before to the square
wave, we can see that it oscillates at the peaks and troughs of the
square wave. Because the FFT is based on an oscillating function (sine),
it can't take on a constant value like a square wave does. So, the best
way for it to approximate the square wave becomes to oscillate as
tightly around the square wave extrema so it averages out. If enough of
the coefficients are included (at the higher frequencies and thus values
of \(k\)), then these oscillations become almost imperceptible. However,
when forced to use only the lowest frequency oscillations, the smoothed
curve sweeps out wider arcs when approximating the function. This can be
seen just by plotting the same function and smoothing transformation but
including the first 50 coefficients instead of just the first 10. And
now, the artifacts (wiggles) have reduced in amplitude, but have become
more numerous. Overall, this estimates the funtion better.

    \begin{Verbatim}[commandchars=\\\{\}]
{\color{incolor}In [{\color{incolor}14}]:} \PY{n}{c} \PY{o}{=} \PY{n}{rfft}\PY{p}{(}\PY{n}{y}\PY{p}{)}
         \PY{n}{M} \PY{o}{=} \PY{n+nb}{len}\PY{p}{(}\PY{n}{c}\PY{p}{)}
         
         \PY{n}{fig8}\PY{p}{,} \PY{n}{ax8} \PY{o}{=} \PY{n}{plt}\PY{o}{.}\PY{n}{subplots}\PY{p}{(}\PY{l+m+mi}{1}\PY{p}{,} \PY{l+m+mi}{1}\PY{p}{,} \PY{n}{figsize} \PY{o}{=} \PY{p}{(}\PY{l+m+mi}{8}\PY{p}{,} \PY{l+m+mi}{4}\PY{p}{)}\PY{p}{)}
         
         \PY{n}{c}\PY{p}{[}\PY{l+m+mi}{50}\PY{p}{:}\PY{p}{]} \PY{o}{=} \PY{l+m+mi}{0}
         \PY{n}{d} \PY{o}{=} \PY{n}{irfft}\PY{p}{(}\PY{n}{c}\PY{p}{)}
         \PY{n}{ax8}\PY{o}{.}\PY{n}{plot}\PY{p}{(}\PY{n}{x}\PY{p}{,} \PY{n}{y}\PY{p}{,} \PY{l+s+s1}{\PYZsq{}}\PY{l+s+s1}{c}\PY{l+s+s1}{\PYZsq{}}\PY{p}{,} \PY{n}{label}\PY{o}{=}\PY{l+s+s1}{\PYZsq{}}\PY{l+s+s1}{Square Wave}\PY{l+s+s1}{\PYZsq{}}\PY{p}{)}
         \PY{n}{ax8}\PY{o}{.}\PY{n}{plot}\PY{p}{(}\PY{n}{x}\PY{p}{,} \PY{n}{d}\PY{p}{,} \PY{l+s+s1}{\PYZsq{}}\PY{l+s+s1}{k}\PY{l+s+s1}{\PYZsq{}}\PY{p}{,} \PY{n}{label}\PY{o}{=}\PY{l+s+s1}{\PYZsq{}}\PY{l+s+s1}{Smoothed Square Wave}\PY{l+s+s1}{\PYZsq{}}\PY{p}{)}
         
         \PY{c+c1}{\PYZsh{}titles, axes, labels}
         \PY{n}{ax8}\PY{o}{.}\PY{n}{set\PYZus{}title}\PY{p}{(}\PY{l+s+s2}{\PYZdq{}}\PY{l+s+s2}{Smoothing a Square Wave}\PY{l+s+s2}{\PYZdq{}}\PY{p}{)}
         \PY{n}{ax8}\PY{o}{.}\PY{n}{set\PYZus{}ylabel}\PY{p}{(}\PY{l+s+s2}{\PYZdq{}}\PY{l+s+s2}{Amplitude}\PY{l+s+s2}{\PYZdq{}}\PY{p}{)}
         \PY{n}{ax8}\PY{o}{.}\PY{n}{set\PYZus{}xlabel}\PY{p}{(}\PY{l+s+s2}{\PYZdq{}}\PY{l+s+s2}{x}\PY{l+s+s2}{\PYZdq{}}\PY{p}{)}
         \PY{n}{ax8}\PY{o}{.}\PY{n}{legend}\PY{p}{(}\PY{p}{)}
         \PY{n}{plt}\PY{o}{.}\PY{n}{show}\PY{p}{(}\PY{p}{)}
\end{Verbatim}

    \begin{center}
    \adjustimage{max size={0.9\linewidth}{0.9\paperheight}}{output_24_0.png}
    \end{center}
    { \hspace*{\fill} \\}
    
    \section{CP 7.9 Image deconvolution}\label{cp-7.9-image-deconvolution}

Looking at the one-dimensional problem first, the brightness can be
represented as

\[b(x) = \int_0^L a(x') f(x-x') \ d x.\]

This can be expanded, Fourier transformed and moved around, so that we
have an expression for the Fourier coefficients that is a product of the
original blurry data and the point spread function

\[\tilde{b}_k = \int_0^L a(x')
              \exp\biggl( -\text{i} {2\pi k x'\over L} \biggr)
              \tilde{f}_k \ d x'
            = L\,\tilde{a}_k\tilde{f}_{k}.\]

Thus by dividing the Fourier transform of the blurred picture by that of
the point spread function, we can find the Fourier coefficients for a
de-blurred version of the picture we're attempting to enhance. This can
then be inverted to generate the image. In two dimensions, we can
recover an analogous relation

\[\tilde{b}_{kl} =  K\tilde{L}{a}_{kl}\tilde{f}_{kl} \implies \frac{\tilde{b}_{kl}}{K\tilde{L}{f}_{kl}} = \tilde{a}_{kl}.\]

The main hurdle we run into is determining the point spread function
that we want to use to deblur the image. A good starting point is to
assume that it's Gaussian

\[f(x,y) = \text{exp}\biggr(-\frac{x^2 + y^2}{\sigma^2}\biggr),\]

which means we can direct our search to determining the width of the
Gaussian function \(\sigma.\)

    \begin{Verbatim}[commandchars=\\\{\}]
{\color{incolor}In [{\color{incolor}15}]:} \PY{l+s+sd}{\PYZdq{}\PYZdq{}\PYZdq{}(a) Loading the blurred image\PYZdq{}\PYZdq{}\PYZdq{}}
         \PY{n}{raw\PYZus{}image} \PY{o}{=} \PY{n}{np}\PY{o}{.}\PY{n}{loadtxt}\PY{p}{(}\PY{l+s+s2}{\PYZdq{}}\PY{l+s+s2}{blur.txt}\PY{l+s+s2}{\PYZdq{}}\PY{p}{,}\PY{n+nb}{float}\PY{p}{)}
         
         \PY{l+s+sd}{\PYZdq{}\PYZdq{}\PYZdq{}(b) Writing and plotting the point spread function\PYZdq{}\PYZdq{}\PYZdq{}}
         \PY{n}{N} \PY{o}{=} \PY{n+nb}{len}\PY{p}{(}\PY{n}{raw\PYZus{}image}\PY{p}{)} \PY{c+c1}{\PYZsh{}number of grid points}
         \PY{k}{def} \PY{n+nf}{point\PYZus{}spread}\PY{p}{(}\PY{n}{x}\PY{p}{,}\PY{n}{y}\PY{p}{)}\PY{p}{:}
             \PY{n}{sigma} \PY{o}{=} \PY{l+m+mi}{25}
             \PY{k}{return} \PY{n}{exp}\PY{p}{(}\PY{o}{\PYZhy{}}\PY{p}{(}\PY{n}{x}\PY{o}{*}\PY{o}{*}\PY{l+m+mi}{2}\PY{o}{+}\PY{n}{y}\PY{o}{*}\PY{o}{*}\PY{l+m+mi}{2}\PY{p}{)}\PY{o}{/}\PY{p}{(}\PY{l+m+mi}{2}\PY{o}{*}\PY{n}{sigma}\PY{o}{*}\PY{o}{*}\PY{l+m+mi}{2}\PY{p}{)}\PY{p}{)}
         
         \PY{n}{U} \PY{o}{=} \PY{n}{np}\PY{o}{.}\PY{n}{zeros}\PY{p}{(}\PY{p}{(}\PY{n}{N}\PY{p}{,}\PY{n}{N}\PY{p}{)}\PY{p}{)}
         
         \PY{k}{for} \PY{n}{i} \PY{o+ow}{in} \PY{n+nb}{range}\PY{p}{(}\PY{n+nb}{int}\PY{p}{(}\PY{n}{N}\PY{o}{/}\PY{l+m+mi}{2}\PY{p}{)}\PY{p}{)}\PY{p}{:}
             \PY{k}{for} \PY{n}{j} \PY{o+ow}{in} \PY{n+nb}{range}\PY{p}{(}\PY{n+nb}{int}\PY{p}{(}\PY{n}{N}\PY{o}{/}\PY{l+m+mi}{2}\PY{p}{)}\PY{p}{)}\PY{p}{:}
                 \PY{n}{x} \PY{o}{=} \PY{n}{point\PYZus{}spread}\PY{p}{(}\PY{n}{i}\PY{p}{,}\PY{n}{j}\PY{p}{)}    
                 \PY{n}{U}\PY{p}{[}\PY{n}{i}\PY{p}{,}\PY{n}{j}\PY{p}{]}   \PY{o}{=} \PY{n}{x}
                 \PY{n}{U}\PY{p}{[}\PY{o}{\PYZhy{}}\PY{n}{i}\PY{p}{,}\PY{o}{\PYZhy{}}\PY{n}{j}\PY{p}{]} \PY{o}{=} \PY{n}{x}
                 \PY{n}{U}\PY{p}{[}\PY{o}{\PYZhy{}}\PY{n}{i}\PY{p}{,}\PY{n}{j}\PY{p}{]}  \PY{o}{=} \PY{n}{x}
                 \PY{n}{U}\PY{p}{[}\PY{n}{i}\PY{p}{,}\PY{o}{\PYZhy{}}\PY{n}{j}\PY{p}{]}  \PY{o}{=} \PY{n}{x}
         
         \PY{l+s+sd}{\PYZdq{}\PYZdq{}\PYZdq{}(c) Using point spread function to unblur image\PYZdq{}\PYZdq{}\PYZdq{}}
         
         \PY{c+c1}{\PYZsh{}fourier coefficients of blurred image}
         \PY{n}{c} \PY{o}{=} \PY{n}{rfft2}\PY{p}{(}\PY{n}{raw\PYZus{}image}\PY{p}{)}
         \PY{c+c1}{\PYZsh{}fourier coefficients of point spread}
         \PY{n}{psk} \PY{o}{=} \PY{n}{rfft2}\PY{p}{(}\PY{n}{U}\PY{p}{)}
         
         \PY{c+c1}{\PYZsh{}coefficients for deblurred image}
         \PY{n}{A} \PY{o}{=} \PY{n}{c} \PY{o}{/} \PY{p}{(}\PY{n}{N}\PY{o}{*}\PY{o}{*}\PY{l+m+mi}{2}\PY{o}{*}\PY{n}{psk}\PY{p}{)}
         
         \PY{n}{tol} \PY{o}{=} \PY{l+m+mf}{10e\PYZhy{}3}
         \PY{k}{for} \PY{n}{i} \PY{o+ow}{in} \PY{n+nb}{range}\PY{p}{(}\PY{n}{N}\PY{p}{)}\PY{p}{:}
             \PY{k}{for} \PY{n}{j} \PY{o+ow}{in} \PY{n+nb}{range}\PY{p}{(}\PY{n+nb}{int}\PY{p}{(}\PY{n}{N}\PY{o}{/}\PY{l+m+mi}{2}\PY{p}{)}\PY{o}{+}\PY{l+m+mi}{1}\PY{p}{)}\PY{p}{:}
                 \PY{k}{if} \PY{n}{psk}\PY{p}{[}\PY{n}{i}\PY{p}{,}\PY{n}{j}\PY{p}{]} \PY{o}{\PYZlt{}} \PY{n}{tol}\PY{p}{:}
                     \PY{n}{A}\PY{p}{[}\PY{n}{i}\PY{p}{,}\PY{n}{j}\PY{p}{]} \PY{o}{=} \PY{n}{c}\PY{p}{[}\PY{n}{i}\PY{p}{,}\PY{n}{j}\PY{p}{]} \PY{o}{/} \PY{p}{(}\PY{n}{N}\PY{o}{*}\PY{o}{*}\PY{l+m+mi}{2}\PY{p}{)}
                 
         \PY{n}{deblurred} \PY{o}{=} \PY{n}{irfft2}\PY{p}{(}\PY{n}{A}\PY{p}{)}
         
         \PY{n}{fig9}\PY{p}{,} \PY{n}{ax9} \PY{o}{=} \PY{n}{plt}\PY{o}{.}\PY{n}{subplots}\PY{p}{(}\PY{l+m+mi}{2}\PY{p}{,} \PY{l+m+mi}{1}\PY{p}{,} \PY{n}{figsize} \PY{o}{=} \PY{p}{(}\PY{l+m+mi}{8}\PY{p}{,} \PY{l+m+mi}{16}\PY{p}{)}\PY{p}{)}
         \PY{n}{ax9}\PY{p}{[}\PY{l+m+mi}{0}\PY{p}{]}\PY{o}{.}\PY{n}{set\PYZus{}xticks}\PY{p}{(}\PY{p}{[}\PY{p}{]}\PY{p}{)}
         \PY{n}{ax9}\PY{p}{[}\PY{l+m+mi}{1}\PY{p}{]}\PY{o}{.}\PY{n}{set\PYZus{}xticks}\PY{p}{(}\PY{p}{[}\PY{p}{]}\PY{p}{)}
         \PY{n}{ax9}\PY{p}{[}\PY{l+m+mi}{0}\PY{p}{]}\PY{o}{.}\PY{n}{set\PYZus{}yticks}\PY{p}{(}\PY{p}{[}\PY{p}{]}\PY{p}{)}
         \PY{n}{ax9}\PY{p}{[}\PY{l+m+mi}{1}\PY{p}{]}\PY{o}{.}\PY{n}{set\PYZus{}yticks}\PY{p}{(}\PY{p}{[}\PY{p}{]}\PY{p}{)}
         \PY{n}{ax9}\PY{p}{[}\PY{l+m+mi}{0}\PY{p}{]}\PY{o}{.}\PY{n}{imshow}\PY{p}{(}\PY{n}{raw\PYZus{}image}\PY{p}{,}\PY{n}{cmap}\PY{o}{=}\PY{l+s+s1}{\PYZsq{}}\PY{l+s+s1}{gray}\PY{l+s+s1}{\PYZsq{}}\PY{p}{)}
         \PY{n}{ax9}\PY{p}{[}\PY{l+m+mi}{1}\PY{p}{]}\PY{o}{.}\PY{n}{imshow}\PY{p}{(}\PY{n}{deblurred}\PY{p}{,}\PY{n}{cmap}\PY{o}{=}\PY{l+s+s1}{\PYZsq{}}\PY{l+s+s1}{gray}\PY{l+s+s1}{\PYZsq{}}\PY{p}{)}
         \PY{n}{ax9}\PY{p}{[}\PY{l+m+mi}{0}\PY{p}{]}\PY{o}{.}\PY{n}{set\PYZus{}title}\PY{p}{(}\PY{l+s+s2}{\PYZdq{}}\PY{l+s+s2}{Blurry Image}\PY{l+s+s2}{\PYZdq{}}\PY{p}{)}
         \PY{n}{ax9}\PY{p}{[}\PY{l+m+mi}{1}\PY{p}{]}\PY{o}{.}\PY{n}{set\PYZus{}title}\PY{p}{(}\PY{l+s+s2}{\PYZdq{}}\PY{l+s+s2}{Deconvoluted Image}\PY{l+s+s2}{\PYZdq{}}\PY{p}{)}
\end{Verbatim}

    \begin{Verbatim}[commandchars=\\\{\}]
/Users/Varun/anaconda/lib/python3.6/site-packages/ipykernel\_launcher.py:28: RuntimeWarning: divide by zero encountered in true\_divide
/Users/Varun/anaconda/lib/python3.6/site-packages/ipykernel\_launcher.py:28: RuntimeWarning: invalid value encountered in true\_divide

    \end{Verbatim}

\begin{Verbatim}[commandchars=\\\{\}]
{\color{outcolor}Out[{\color{outcolor}15}]:} <matplotlib.text.Text at 0x1138648d0>
\end{Verbatim}
            
    \begin{center}
    \adjustimage{max size={0.9\linewidth}{0.9\paperheight}}{output_26_2.png}
    \end{center}
    { \hspace*{\fill} \\}
    
    Our ability to deblur a photo comes inherently from the fact that the
"blur" and the unblurry image are stored in the same set of numbers.
Thus it is impossible to fully separate these two things. Additionally,
the process outlined above and then programmed ignores the potential for
noise in the file. It assumes the equation

\[b_{kl} =  KL{a}_{kl}f_{kl}\]

is what governs the image where \(b_{kl}\) are the FFT coefficients for
the original, blurry image. However, the image could have an additional
noise term that would make the unveiling of the true image even more
difficult. The relation for this scenario could be

\[b_{kl} =  KL{a}_{kl}f_{kl} + n.\]

As far the point spread function is concerned, having zeros or very
small values is clearly a problem. Because, if we use these values (for
the very small one), it will distort the deconvoluted image from what it
should be, however if we ignore these values, we're losing information
from the original image since the data can't be separated. So, our
deconvolution runs into trouble for both of those workarounds.

Of course, this all assumes that the point spread function in question
is accurate and is what is governing the blur. If the function is
shifted, weaker, or simply not the blur pattern, this won't work
properly. Fortunately, for uses such as satellite imaging and similar
scenarios, the light can be treated as originating from point sources
and thus this function can be assumed reasonably accurate.

    \begin{Verbatim}[commandchars=\\\{\}]
{\color{incolor}In [{\color{incolor} }]:} 
\end{Verbatim}


    % Add a bibliography block to the postdoc
    
    
    
    \end{document}


% Default to the notebook output style

    


% Inherit from the specified cell style.




    
\documentclass[11pt]{article}

    
    
    \usepackage[T1]{fontenc}
    % Nicer default font (+ math font) than Computer Modern for most use cases
    \usepackage{mathpazo}

    % Basic figure setup, for now with no caption control since it's done
    % automatically by Pandoc (which extracts ![](path) syntax from Markdown).
    \usepackage{graphicx}
    % We will generate all images so they have a width \maxwidth. This means
    % that they will get their normal width if they fit onto the page, but
    % are scaled down if they would overflow the margins.
    \makeatletter
    \def\maxwidth{\ifdim\Gin@nat@width>\linewidth\linewidth
    \else\Gin@nat@width\fi}
    \makeatother
    \let\Oldincludegraphics\includegraphics
    % Set max figure width to be 80% of text width, for now hardcoded.
    \renewcommand{\includegraphics}[1]{\Oldincludegraphics[width=.8\maxwidth]{#1}}
    % Ensure that by default, figures have no caption (until we provide a
    % proper Figure object with a Caption API and a way to capture that
    % in the conversion process - todo).
    \usepackage{caption}
    \DeclareCaptionLabelFormat{nolabel}{}
    \captionsetup{labelformat=nolabel}

    \usepackage{adjustbox} % Used to constrain images to a maximum size 
    \usepackage{xcolor} % Allow colors to be defined
    \usepackage{enumerate} % Needed for markdown enumerations to work
    \usepackage{geometry} % Used to adjust the document margins
    \usepackage{amsmath} % Equations
    \usepackage{amssymb} % Equations
    \usepackage{textcomp} % defines textquotesingle
    % Hack from http://tex.stackexchange.com/a/47451/13684:
    \AtBeginDocument{%
        \def\PYZsq{\textquotesingle}% Upright quotes in Pygmentized code
    }
    \usepackage{upquote} % Upright quotes for verbatim code
    \usepackage{eurosym} % defines \euro
    \usepackage[mathletters]{ucs} % Extended unicode (utf-8) support
    \usepackage[utf8x]{inputenc} % Allow utf-8 characters in the tex document
    \usepackage{fancyvrb} % verbatim replacement that allows latex
    \usepackage{grffile} % extends the file name processing of package graphics 
                         % to support a larger range 
    % The hyperref package gives us a pdf with properly built
    % internal navigation ('pdf bookmarks' for the table of contents,
    % internal cross-reference links, web links for URLs, etc.)
    \usepackage{hyperref}
    \usepackage{longtable} % longtable support required by pandoc >1.10
    \usepackage{booktabs}  % table support for pandoc > 1.12.2
    \usepackage[inline]{enumitem} % IRkernel/repr support (it uses the enumerate* environment)
    \usepackage[normalem]{ulem} % ulem is needed to support strikethroughs (\sout)
                                % normalem makes italics be italics, not underlines
    \usepackage{mathrsfs}
    

    
    
    % Colors for the hyperref package
    \definecolor{urlcolor}{rgb}{0,.145,.698}
    \definecolor{linkcolor}{rgb}{.71,0.21,0.01}
    \definecolor{citecolor}{rgb}{.12,.54,.11}

    % ANSI colors
    \definecolor{ansi-black}{HTML}{3E424D}
    \definecolor{ansi-black-intense}{HTML}{282C36}
    \definecolor{ansi-red}{HTML}{E75C58}
    \definecolor{ansi-red-intense}{HTML}{B22B31}
    \definecolor{ansi-green}{HTML}{00A250}
    \definecolor{ansi-green-intense}{HTML}{007427}
    \definecolor{ansi-yellow}{HTML}{DDB62B}
    \definecolor{ansi-yellow-intense}{HTML}{B27D12}
    \definecolor{ansi-blue}{HTML}{208FFB}
    \definecolor{ansi-blue-intense}{HTML}{0065CA}
    \definecolor{ansi-magenta}{HTML}{D160C4}
    \definecolor{ansi-magenta-intense}{HTML}{A03196}
    \definecolor{ansi-cyan}{HTML}{60C6C8}
    \definecolor{ansi-cyan-intense}{HTML}{258F8F}
    \definecolor{ansi-white}{HTML}{C5C1B4}
    \definecolor{ansi-white-intense}{HTML}{A1A6B2}
    \definecolor{ansi-default-inverse-fg}{HTML}{FFFFFF}
    \definecolor{ansi-default-inverse-bg}{HTML}{000000}

    % commands and environments needed by pandoc snippets
    % extracted from the output of `pandoc -s`
    \providecommand{\tightlist}{%
      \setlength{\itemsep}{0pt}\setlength{\parskip}{0pt}}
    \DefineVerbatimEnvironment{Highlighting}{Verbatim}{commandchars=\\\{\}}
    % Add ',fontsize=\small' for more characters per line
    \newenvironment{Shaded}{}{}
    \newcommand{\KeywordTok}[1]{\textcolor[rgb]{0.00,0.44,0.13}{\textbf{{#1}}}}
    \newcommand{\DataTypeTok}[1]{\textcolor[rgb]{0.56,0.13,0.00}{{#1}}}
    \newcommand{\DecValTok}[1]{\textcolor[rgb]{0.25,0.63,0.44}{{#1}}}
    \newcommand{\BaseNTok}[1]{\textcolor[rgb]{0.25,0.63,0.44}{{#1}}}
    \newcommand{\FloatTok}[1]{\textcolor[rgb]{0.25,0.63,0.44}{{#1}}}
    \newcommand{\CharTok}[1]{\textcolor[rgb]{0.25,0.44,0.63}{{#1}}}
    \newcommand{\StringTok}[1]{\textcolor[rgb]{0.25,0.44,0.63}{{#1}}}
    \newcommand{\CommentTok}[1]{\textcolor[rgb]{0.38,0.63,0.69}{\textit{{#1}}}}
    \newcommand{\OtherTok}[1]{\textcolor[rgb]{0.00,0.44,0.13}{{#1}}}
    \newcommand{\AlertTok}[1]{\textcolor[rgb]{1.00,0.00,0.00}{\textbf{{#1}}}}
    \newcommand{\FunctionTok}[1]{\textcolor[rgb]{0.02,0.16,0.49}{{#1}}}
    \newcommand{\RegionMarkerTok}[1]{{#1}}
    \newcommand{\ErrorTok}[1]{\textcolor[rgb]{1.00,0.00,0.00}{\textbf{{#1}}}}
    \newcommand{\NormalTok}[1]{{#1}}
    
    % Additional commands for more recent versions of Pandoc
    \newcommand{\ConstantTok}[1]{\textcolor[rgb]{0.53,0.00,0.00}{{#1}}}
    \newcommand{\SpecialCharTok}[1]{\textcolor[rgb]{0.25,0.44,0.63}{{#1}}}
    \newcommand{\VerbatimStringTok}[1]{\textcolor[rgb]{0.25,0.44,0.63}{{#1}}}
    \newcommand{\SpecialStringTok}[1]{\textcolor[rgb]{0.73,0.40,0.53}{{#1}}}
    \newcommand{\ImportTok}[1]{{#1}}
    \newcommand{\DocumentationTok}[1]{\textcolor[rgb]{0.73,0.13,0.13}{\textit{{#1}}}}
    \newcommand{\AnnotationTok}[1]{\textcolor[rgb]{0.38,0.63,0.69}{\textbf{\textit{{#1}}}}}
    \newcommand{\CommentVarTok}[1]{\textcolor[rgb]{0.38,0.63,0.69}{\textbf{\textit{{#1}}}}}
    \newcommand{\VariableTok}[1]{\textcolor[rgb]{0.10,0.09,0.49}{{#1}}}
    \newcommand{\ControlFlowTok}[1]{\textcolor[rgb]{0.00,0.44,0.13}{\textbf{{#1}}}}
    \newcommand{\OperatorTok}[1]{\textcolor[rgb]{0.40,0.40,0.40}{{#1}}}
    \newcommand{\BuiltInTok}[1]{{#1}}
    \newcommand{\ExtensionTok}[1]{{#1}}
    \newcommand{\PreprocessorTok}[1]{\textcolor[rgb]{0.74,0.48,0.00}{{#1}}}
    \newcommand{\AttributeTok}[1]{\textcolor[rgb]{0.49,0.56,0.16}{{#1}}}
    \newcommand{\InformationTok}[1]{\textcolor[rgb]{0.38,0.63,0.69}{\textbf{\textit{{#1}}}}}
    \newcommand{\WarningTok}[1]{\textcolor[rgb]{0.38,0.63,0.69}{\textbf{\textit{{#1}}}}}
    
    
    % Define a nice break command that doesn't care if a line doesn't already
    % exist.
    \def\br{\hspace*{\fill} \\* }
    % Math Jax compatibility definitions
    \def\gt{>}
    \def\lt{<}
    \let\Oldtex\TeX
    \let\Oldlatex\LaTeX
    \renewcommand{\TeX}{\textrm{\Oldtex}}
    \renewcommand{\LaTeX}{\textrm{\Oldlatex}}
    % Document parameters
    % Document title
    \title{Homework 1 \\ \vspace{10mm}
    {\large Varun Nair}}
    
    
    
    
    

    % Pygments definitions
    
\makeatletter
\def\PY@reset{\let\PY@it=\relax \let\PY@bf=\relax%
    \let\PY@ul=\relax \let\PY@tc=\relax%
    \let\PY@bc=\relax \let\PY@ff=\relax}
\def\PY@tok#1{\csname PY@tok@#1\endcsname}
\def\PY@toks#1+{\ifx\relax#1\empty\else%
    \PY@tok{#1}\expandafter\PY@toks\fi}
\def\PY@do#1{\PY@bc{\PY@tc{\PY@ul{%
    \PY@it{\PY@bf{\PY@ff{#1}}}}}}}
\def\PY#1#2{\PY@reset\PY@toks#1+\relax+\PY@do{#2}}

\expandafter\def\csname PY@tok@w\endcsname{\def\PY@tc##1{\textcolor[rgb]{0.73,0.73,0.73}{##1}}}
\expandafter\def\csname PY@tok@c\endcsname{\let\PY@it=\textit\def\PY@tc##1{\textcolor[rgb]{0.25,0.50,0.50}{##1}}}
\expandafter\def\csname PY@tok@cp\endcsname{\def\PY@tc##1{\textcolor[rgb]{0.74,0.48,0.00}{##1}}}
\expandafter\def\csname PY@tok@k\endcsname{\let\PY@bf=\textbf\def\PY@tc##1{\textcolor[rgb]{0.00,0.50,0.00}{##1}}}
\expandafter\def\csname PY@tok@kp\endcsname{\def\PY@tc##1{\textcolor[rgb]{0.00,0.50,0.00}{##1}}}
\expandafter\def\csname PY@tok@kt\endcsname{\def\PY@tc##1{\textcolor[rgb]{0.69,0.00,0.25}{##1}}}
\expandafter\def\csname PY@tok@o\endcsname{\def\PY@tc##1{\textcolor[rgb]{0.40,0.40,0.40}{##1}}}
\expandafter\def\csname PY@tok@ow\endcsname{\let\PY@bf=\textbf\def\PY@tc##1{\textcolor[rgb]{0.67,0.13,1.00}{##1}}}
\expandafter\def\csname PY@tok@nb\endcsname{\def\PY@tc##1{\textcolor[rgb]{0.00,0.50,0.00}{##1}}}
\expandafter\def\csname PY@tok@nf\endcsname{\def\PY@tc##1{\textcolor[rgb]{0.00,0.00,1.00}{##1}}}
\expandafter\def\csname PY@tok@nc\endcsname{\let\PY@bf=\textbf\def\PY@tc##1{\textcolor[rgb]{0.00,0.00,1.00}{##1}}}
\expandafter\def\csname PY@tok@nn\endcsname{\let\PY@bf=\textbf\def\PY@tc##1{\textcolor[rgb]{0.00,0.00,1.00}{##1}}}
\expandafter\def\csname PY@tok@ne\endcsname{\let\PY@bf=\textbf\def\PY@tc##1{\textcolor[rgb]{0.82,0.25,0.23}{##1}}}
\expandafter\def\csname PY@tok@nv\endcsname{\def\PY@tc##1{\textcolor[rgb]{0.10,0.09,0.49}{##1}}}
\expandafter\def\csname PY@tok@no\endcsname{\def\PY@tc##1{\textcolor[rgb]{0.53,0.00,0.00}{##1}}}
\expandafter\def\csname PY@tok@nl\endcsname{\def\PY@tc##1{\textcolor[rgb]{0.63,0.63,0.00}{##1}}}
\expandafter\def\csname PY@tok@ni\endcsname{\let\PY@bf=\textbf\def\PY@tc##1{\textcolor[rgb]{0.60,0.60,0.60}{##1}}}
\expandafter\def\csname PY@tok@na\endcsname{\def\PY@tc##1{\textcolor[rgb]{0.49,0.56,0.16}{##1}}}
\expandafter\def\csname PY@tok@nt\endcsname{\let\PY@bf=\textbf\def\PY@tc##1{\textcolor[rgb]{0.00,0.50,0.00}{##1}}}
\expandafter\def\csname PY@tok@nd\endcsname{\def\PY@tc##1{\textcolor[rgb]{0.67,0.13,1.00}{##1}}}
\expandafter\def\csname PY@tok@s\endcsname{\def\PY@tc##1{\textcolor[rgb]{0.73,0.13,0.13}{##1}}}
\expandafter\def\csname PY@tok@sd\endcsname{\let\PY@it=\textit\def\PY@tc##1{\textcolor[rgb]{0.73,0.13,0.13}{##1}}}
\expandafter\def\csname PY@tok@si\endcsname{\let\PY@bf=\textbf\def\PY@tc##1{\textcolor[rgb]{0.73,0.40,0.53}{##1}}}
\expandafter\def\csname PY@tok@se\endcsname{\let\PY@bf=\textbf\def\PY@tc##1{\textcolor[rgb]{0.73,0.40,0.13}{##1}}}
\expandafter\def\csname PY@tok@sr\endcsname{\def\PY@tc##1{\textcolor[rgb]{0.73,0.40,0.53}{##1}}}
\expandafter\def\csname PY@tok@ss\endcsname{\def\PY@tc##1{\textcolor[rgb]{0.10,0.09,0.49}{##1}}}
\expandafter\def\csname PY@tok@sx\endcsname{\def\PY@tc##1{\textcolor[rgb]{0.00,0.50,0.00}{##1}}}
\expandafter\def\csname PY@tok@m\endcsname{\def\PY@tc##1{\textcolor[rgb]{0.40,0.40,0.40}{##1}}}
\expandafter\def\csname PY@tok@gh\endcsname{\let\PY@bf=\textbf\def\PY@tc##1{\textcolor[rgb]{0.00,0.00,0.50}{##1}}}
\expandafter\def\csname PY@tok@gu\endcsname{\let\PY@bf=\textbf\def\PY@tc##1{\textcolor[rgb]{0.50,0.00,0.50}{##1}}}
\expandafter\def\csname PY@tok@gd\endcsname{\def\PY@tc##1{\textcolor[rgb]{0.63,0.00,0.00}{##1}}}
\expandafter\def\csname PY@tok@gi\endcsname{\def\PY@tc##1{\textcolor[rgb]{0.00,0.63,0.00}{##1}}}
\expandafter\def\csname PY@tok@gr\endcsname{\def\PY@tc##1{\textcolor[rgb]{1.00,0.00,0.00}{##1}}}
\expandafter\def\csname PY@tok@ge\endcsname{\let\PY@it=\textit}
\expandafter\def\csname PY@tok@gs\endcsname{\let\PY@bf=\textbf}
\expandafter\def\csname PY@tok@gp\endcsname{\let\PY@bf=\textbf\def\PY@tc##1{\textcolor[rgb]{0.00,0.00,0.50}{##1}}}
\expandafter\def\csname PY@tok@go\endcsname{\def\PY@tc##1{\textcolor[rgb]{0.53,0.53,0.53}{##1}}}
\expandafter\def\csname PY@tok@gt\endcsname{\def\PY@tc##1{\textcolor[rgb]{0.00,0.27,0.87}{##1}}}
\expandafter\def\csname PY@tok@err\endcsname{\def\PY@bc##1{\setlength{\fboxsep}{0pt}\fcolorbox[rgb]{1.00,0.00,0.00}{1,1,1}{\strut ##1}}}
\expandafter\def\csname PY@tok@kc\endcsname{\let\PY@bf=\textbf\def\PY@tc##1{\textcolor[rgb]{0.00,0.50,0.00}{##1}}}
\expandafter\def\csname PY@tok@kd\endcsname{\let\PY@bf=\textbf\def\PY@tc##1{\textcolor[rgb]{0.00,0.50,0.00}{##1}}}
\expandafter\def\csname PY@tok@kn\endcsname{\let\PY@bf=\textbf\def\PY@tc##1{\textcolor[rgb]{0.00,0.50,0.00}{##1}}}
\expandafter\def\csname PY@tok@kr\endcsname{\let\PY@bf=\textbf\def\PY@tc##1{\textcolor[rgb]{0.00,0.50,0.00}{##1}}}
\expandafter\def\csname PY@tok@bp\endcsname{\def\PY@tc##1{\textcolor[rgb]{0.00,0.50,0.00}{##1}}}
\expandafter\def\csname PY@tok@fm\endcsname{\def\PY@tc##1{\textcolor[rgb]{0.00,0.00,1.00}{##1}}}
\expandafter\def\csname PY@tok@vc\endcsname{\def\PY@tc##1{\textcolor[rgb]{0.10,0.09,0.49}{##1}}}
\expandafter\def\csname PY@tok@vg\endcsname{\def\PY@tc##1{\textcolor[rgb]{0.10,0.09,0.49}{##1}}}
\expandafter\def\csname PY@tok@vi\endcsname{\def\PY@tc##1{\textcolor[rgb]{0.10,0.09,0.49}{##1}}}
\expandafter\def\csname PY@tok@vm\endcsname{\def\PY@tc##1{\textcolor[rgb]{0.10,0.09,0.49}{##1}}}
\expandafter\def\csname PY@tok@sa\endcsname{\def\PY@tc##1{\textcolor[rgb]{0.73,0.13,0.13}{##1}}}
\expandafter\def\csname PY@tok@sb\endcsname{\def\PY@tc##1{\textcolor[rgb]{0.73,0.13,0.13}{##1}}}
\expandafter\def\csname PY@tok@sc\endcsname{\def\PY@tc##1{\textcolor[rgb]{0.73,0.13,0.13}{##1}}}
\expandafter\def\csname PY@tok@dl\endcsname{\def\PY@tc##1{\textcolor[rgb]{0.73,0.13,0.13}{##1}}}
\expandafter\def\csname PY@tok@s2\endcsname{\def\PY@tc##1{\textcolor[rgb]{0.73,0.13,0.13}{##1}}}
\expandafter\def\csname PY@tok@sh\endcsname{\def\PY@tc##1{\textcolor[rgb]{0.73,0.13,0.13}{##1}}}
\expandafter\def\csname PY@tok@s1\endcsname{\def\PY@tc##1{\textcolor[rgb]{0.73,0.13,0.13}{##1}}}
\expandafter\def\csname PY@tok@mb\endcsname{\def\PY@tc##1{\textcolor[rgb]{0.40,0.40,0.40}{##1}}}
\expandafter\def\csname PY@tok@mf\endcsname{\def\PY@tc##1{\textcolor[rgb]{0.40,0.40,0.40}{##1}}}
\expandafter\def\csname PY@tok@mh\endcsname{\def\PY@tc##1{\textcolor[rgb]{0.40,0.40,0.40}{##1}}}
\expandafter\def\csname PY@tok@mi\endcsname{\def\PY@tc##1{\textcolor[rgb]{0.40,0.40,0.40}{##1}}}
\expandafter\def\csname PY@tok@il\endcsname{\def\PY@tc##1{\textcolor[rgb]{0.40,0.40,0.40}{##1}}}
\expandafter\def\csname PY@tok@mo\endcsname{\def\PY@tc##1{\textcolor[rgb]{0.40,0.40,0.40}{##1}}}
\expandafter\def\csname PY@tok@ch\endcsname{\let\PY@it=\textit\def\PY@tc##1{\textcolor[rgb]{0.25,0.50,0.50}{##1}}}
\expandafter\def\csname PY@tok@cm\endcsname{\let\PY@it=\textit\def\PY@tc##1{\textcolor[rgb]{0.25,0.50,0.50}{##1}}}
\expandafter\def\csname PY@tok@cpf\endcsname{\let\PY@it=\textit\def\PY@tc##1{\textcolor[rgb]{0.25,0.50,0.50}{##1}}}
\expandafter\def\csname PY@tok@c1\endcsname{\let\PY@it=\textit\def\PY@tc##1{\textcolor[rgb]{0.25,0.50,0.50}{##1}}}
\expandafter\def\csname PY@tok@cs\endcsname{\let\PY@it=\textit\def\PY@tc##1{\textcolor[rgb]{0.25,0.50,0.50}{##1}}}

\def\PYZbs{\char`\\}
\def\PYZus{\char`\_}
\def\PYZob{\char`\{}
\def\PYZcb{\char`\}}
\def\PYZca{\char`\^}
\def\PYZam{\char`\&}
\def\PYZlt{\char`\<}
\def\PYZgt{\char`\>}
\def\PYZsh{\char`\#}
\def\PYZpc{\char`\%}
\def\PYZdl{\char`\$}
\def\PYZhy{\char`\-}
\def\PYZsq{\char`\'}
\def\PYZdq{\char`\"}
\def\PYZti{\char`\~}
% for compatibility with earlier versions
\def\PYZat{@}
\def\PYZlb{[}
\def\PYZrb{]}
\makeatother


    % Exact colors from NB
    \definecolor{incolor}{rgb}{0.0, 0.0, 0.5}
    \definecolor{outcolor}{rgb}{0.545, 0.0, 0.0}



    
    % Prevent overflowing lines due to hard-to-break entities
    \sloppy 
    % Setup hyperref package
    \hypersetup{
      breaklinks=true,  % so long urls are correctly broken across lines
      colorlinks=true,
      urlcolor=urlcolor,
      linkcolor=linkcolor,
      citecolor=citecolor,
      }
    % Slightly bigger margins than the latex defaults
    
    \geometry{verbose,tmargin=1in,bmargin=1in,lmargin=1in,rmargin=1in}
    
    

    \begin{document}
    
    
    \maketitle
    
    

    
    \begin{Verbatim}[commandchars=\\\{\}]
{\color{incolor}In [{\color{incolor}229}]:} \PY{o}{\PYZpc{}}\PY{k}{precision} \PYZpc{}g
          \PY{o}{\PYZpc{}}\PY{k}{matplotlib} inline
          \PY{o}{\PYZpc{}}\PY{k}{config} InlineBackend.figure\PYZus{}format = \PYZsq{}retina\PYZsq{}
\end{Verbatim}

    \begin{Verbatim}[commandchars=\\\{\}]
{\color{incolor}In [{\color{incolor}230}]:} \PY{k+kn}{from} \PY{n+nn}{math} \PY{k}{import} \PY{n}{sqrt}\PY{p}{,} \PY{n}{pi}
          \PY{k+kn}{from} \PY{n+nn}{cmath} \PY{k}{import} \PY{n}{sqrt} \PY{k}{as} \PY{n}{csqrt}
          \PY{k+kn}{import} \PY{n+nn}{numpy} \PY{k}{as} \PY{n+nn}{np}
          \PY{k+kn}{from} \PY{n+nn}{scipy} \PY{k}{import} \PY{n}{constants} \PY{k}{as} \PY{n}{C}
          \PY{k+kn}{import} \PY{n+nn}{matplotlib}\PY{n+nn}{.}\PY{n+nn}{pyplot} \PY{k}{as} \PY{n+nn}{pyplot}
\end{Verbatim}

    \subsection{CP 2.5 Quantum Potential
Step}\label{cp-2.5-quantum-potential-step}

This problem is to find the transmission (T) and reflection (R)
probabilities for a given particle (with mass and energy) and potential
step based on the wavevector forms of the probabilities:

\[T = \frac{4k_1k_2}{(k_1+k_2)^2} \ \textrm{ and } \ R = \left( \frac{k_1-k_2}{k_1+k_2} \right)^2.\]

The wavevectors \(k_1\) and \(k_2\) are defined in their respective
regions as

\[k_1 = \frac{\sqrt{2mE}}{\hbar} \ \textrm{ and } \ k_2 = \frac{\sqrt{2m(E-V)}}{\hbar}\]

    \begin{Verbatim}[commandchars=\\\{\}]
{\color{incolor}In [{\color{incolor}231}]:} \PY{c+c1}{\PYZsh{}This cell defines constants}
          
          \PY{n}{h} \PY{o}{=} \PY{l+m+mf}{6.626e\PYZhy{}34} \PY{c+c1}{\PYZsh{}in kg m\PYZca{}2 s\PYZca{}\PYZhy{}1}
          \PY{n}{hbar} \PY{o}{=} \PY{n}{h} \PY{o}{/} \PY{p}{(}\PY{l+m+mi}{2}\PY{o}{*}\PY{n}{pi}\PY{p}{)}
\end{Verbatim}

    \begin{Verbatim}[commandchars=\\\{\}]
{\color{incolor}In [{\color{incolor}232}]:} \PY{c+c1}{\PYZsh{}defines wavevector functions}
          
          \PY{k}{def} \PY{n+nf}{k1}\PY{p}{(}\PY{n}{m}\PY{p}{,} \PY{n}{E}\PY{p}{)}\PY{p}{:} \PY{c+c1}{\PYZsh{}m is mass and E is energy}
              \PY{k}{return} \PY{p}{(}\PY{n}{sqrt}\PY{p}{(}\PY{l+m+mi}{2}\PY{o}{*}\PY{n}{m}\PY{o}{*}\PY{n}{E}\PY{p}{)}\PY{p}{)} \PY{o}{/} \PY{n}{hbar}
          
          \PY{k}{def} \PY{n+nf}{k2}\PY{p}{(}\PY{n}{m}\PY{p}{,} \PY{n}{E}\PY{p}{,} \PY{n}{V}\PY{p}{)}\PY{p}{:} \PY{c+c1}{\PYZsh{}V is height of potential step}
              \PY{k}{return} \PY{p}{(}\PY{n}{sqrt}\PY{p}{(}\PY{l+m+mi}{2}\PY{o}{*}\PY{n}{m}\PY{o}{*}\PY{p}{(}\PY{n}{E}\PY{o}{\PYZhy{}}\PY{n}{V}\PY{p}{)}\PY{p}{)}\PY{p}{)} \PY{o}{/} \PY{n}{hbar}
          
          
          \PY{c+c1}{\PYZsh{}probability functions defined below}
          
          \PY{k}{def} \PY{n+nf}{T}\PY{p}{(}\PY{n}{m}\PY{p}{,} \PY{n}{E}\PY{p}{,} \PY{n}{V}\PY{p}{)}\PY{p}{:} \PY{c+c1}{\PYZsh{}transmission probability}
              \PY{n}{num} \PY{o}{=} \PY{l+m+mi}{4} \PY{o}{*} \PY{n}{k1}\PY{p}{(}\PY{n}{m}\PY{p}{,} \PY{n}{E}\PY{p}{)} \PY{o}{*} \PY{n}{k2}\PY{p}{(}\PY{n}{m}\PY{p}{,} \PY{n}{E}\PY{p}{,} \PY{n}{V}\PY{p}{)}
              \PY{n}{den} \PY{o}{=} \PY{p}{(}\PY{n}{k1}\PY{p}{(}\PY{n}{m}\PY{p}{,} \PY{n}{E}\PY{p}{)} \PY{o}{+} \PY{n}{k2}\PY{p}{(}\PY{n}{m}\PY{p}{,} \PY{n}{E}\PY{p}{,} \PY{n}{V}\PY{p}{)}\PY{p}{)}\PY{o}{*}\PY{o}{*}\PY{l+m+mi}{2}
              \PY{k}{return} \PY{n}{num} \PY{o}{/} \PY{n}{den}
          
          \PY{k}{def} \PY{n+nf}{R}\PY{p}{(}\PY{n}{m}\PY{p}{,} \PY{n}{E}\PY{p}{,} \PY{n}{V}\PY{p}{)}\PY{p}{:} \PY{c+c1}{\PYZsh{}reflection probability}
              \PY{n}{num} \PY{o}{=} \PY{n}{k1}\PY{p}{(}\PY{n}{m}\PY{p}{,} \PY{n}{E}\PY{p}{)} \PY{o}{\PYZhy{}} \PY{n}{k2}\PY{p}{(}\PY{n}{m}\PY{p}{,} \PY{n}{E}\PY{p}{,} \PY{n}{V}\PY{p}{)}
              \PY{n}{den} \PY{o}{=} \PY{n}{k1}\PY{p}{(}\PY{n}{m}\PY{p}{,} \PY{n}{E}\PY{p}{)} \PY{o}{+} \PY{n}{k2}\PY{p}{(}\PY{n}{m}\PY{p}{,} \PY{n}{E}\PY{p}{,} \PY{n}{V}\PY{p}{)}
              \PY{k}{return} \PY{p}{(}\PY{n}{num} \PY{o}{/} \PY{n}{den}\PY{p}{)}\PY{o}{*}\PY{o}{*}\PY{l+m+mi}{2}
\end{Verbatim}

    \begin{Verbatim}[commandchars=\\\{\}]
{\color{incolor}In [{\color{incolor}233}]:} \PY{n+nb}{print}\PY{p}{(}\PY{l+s+s2}{\PYZdq{}}\PY{l+s+s2}{For a particle of mass 9.11e\PYZhy{}31 kg and energy 10 eV, it will have }\PY{l+s+se}{\PYZbs{}}
          \PY{l+s+s2}{transmission probability T = }\PY{l+s+si}{\PYZob{}:4.3f\PYZcb{}}\PY{l+s+s2}{ and }\PY{l+s+se}{\PYZbs{}}
          \PY{l+s+s2}{reflection probability R = }\PY{l+s+si}{\PYZob{}:4.3f\PYZcb{}}\PY{l+s+s2}{ }\PY{l+s+se}{\PYZbs{}}
          \PY{l+s+s2}{if it encounters a potential step of height 9 eV.}\PY{l+s+s2}{\PYZdq{}}\PY{o}{.}\PY{n}{format}\PY{p}{(}\PY{n}{T}\PY{p}{(}\PY{l+m+mf}{9.11e\PYZhy{}31}\PY{p}{,} \PY{l+m+mi}{10}\PY{p}{,} \PY{l+m+mi}{9}\PY{p}{)}\PY{p}{,} \PY{n}{R}\PY{p}{(}\PY{l+m+mf}{9.11e\PYZhy{}31}\PY{p}{,} \PY{l+m+mi}{10}\PY{p}{,} \PY{l+m+mi}{9}\PY{p}{)}\PY{p}{)}\PY{p}{)}
\end{Verbatim}

    \begin{Verbatim}[commandchars=\\\{\}]
For a particle of mass 9.11e-31 kg and energy 10 eV, it will have transmission probability T = 0.730 and reflection probability R = 0.270 if it encounters a potential step of height 9 eV.

    \end{Verbatim}

    \subsection{CP 2.6 Planetary Orbits}\label{cp-2.6-planetary-orbits}

Knowing the distance that a planet's perihelion and aphelion are from
the sun and its linear velocity at one of those points is powerful,
allowing one to calculate many facets about the planet's orbit.

Among other things, you can calculate the following

\begin{align*}
\textrm{Semi-major axis:} \qquad a &= {\tiny\frac{1}{2}}(\ell_1+\ell_2), \\
\textrm{Semi-minor axis:} \qquad b &= \sqrt{\ell_1\ell_2}\,, \\
\textrm{Orbital period:} \hspace{1.80em}
  T &= {2\pi ab\over\ell_1 v_1}\,, \\
\textrm{Orbital eccentricity:} \qquad e &=
  {\ell_2-\ell_1\over\ell_2+\ell_1}.
\end{align*}

    \begin{Verbatim}[commandchars=\\\{\}]
{\color{incolor}In [{\color{incolor}234}]:} \PY{c+c1}{\PYZsh{}this defines constants}
          
          \PY{n}{M} \PY{o}{=} \PY{l+m+mf}{1.9891e30} \PY{c+c1}{\PYZsh{}mass of the sun in kg}
          \PY{n}{G} \PY{o}{=} \PY{l+m+mf}{6.6738e\PYZhy{}11} \PY{c+c1}{\PYZsh{} gravitational constant in m\PYZca{}3 kg\PYZca{}\PYZhy{}1 s\PYZca{}\PYZhy{}2}
\end{Verbatim}

    \begin{Verbatim}[commandchars=\\\{\}]
{\color{incolor}In [{\color{incolor}235}]:} \PY{c+c1}{\PYZsh{}defining function to calculate period and eccentricity}
          
          \PY{k}{def} \PY{n+nf}{findv2}\PY{p}{(}\PY{n}{l1}\PY{p}{,} \PY{n}{v1}\PY{p}{)}\PY{p}{:}
              \PY{l+s+sd}{\PYZsq{}\PYZsq{}\PYZsq{}This function calculates the linear velocity of an object at its aphelion given distance and velocity at perihelion \PYZbs{}}
          \PY{l+s+sd}{       by solving the quadratic equation ax**2 + bx + c = 0\PYZsq{}\PYZsq{}\PYZsq{}}
              \PY{n}{a} \PY{o}{=} \PY{l+m+mi}{1}
              \PY{n}{b} \PY{o}{=} \PY{p}{(}\PY{o}{\PYZhy{}}\PY{l+m+mi}{2}\PY{o}{*}\PY{n}{G}\PY{o}{*}\PY{n}{M}\PY{p}{)} \PY{o}{/} \PY{p}{(}\PY{n}{l1} \PY{o}{*} \PY{n}{v1}\PY{p}{)}
              \PY{n}{c} \PY{o}{=} \PY{o}{\PYZhy{}}\PY{p}{(}\PY{n}{v1}\PY{o}{*}\PY{o}{*}\PY{l+m+mi}{2} \PY{o}{\PYZhy{}} \PY{p}{(}\PY{p}{(}\PY{l+m+mi}{2}\PY{o}{*}\PY{n}{G}\PY{o}{*}\PY{n}{M}\PY{p}{)} \PY{o}{/} \PY{n}{l1}\PY{p}{)}\PY{p}{)}
              
              \PY{c+c1}{\PYZsh{}the discriminant in the quadratic equation}
              \PY{n}{disc} \PY{o}{=} \PY{n}{b}\PY{o}{*}\PY{o}{*}\PY{l+m+mi}{2} \PY{o}{\PYZhy{}} \PY{p}{(}\PY{l+m+mi}{4}\PY{o}{*}\PY{n}{a}\PY{o}{*}\PY{n}{c}\PY{p}{)}
              
              \PY{c+c1}{\PYZsh{}only returns 1 solutions because we are only interested in the smaller solution}
              \PY{k}{return} \PY{p}{(}\PY{o}{\PYZhy{}}\PY{n}{b} \PY{o}{\PYZhy{}} \PY{n}{sqrt}\PY{p}{(}\PY{n}{disc}\PY{p}{)}\PY{p}{)} \PY{o}{/} \PY{p}{(}\PY{l+m+mi}{2}\PY{o}{*}\PY{n}{a}\PY{p}{)}
          
          \PY{k}{def} \PY{n+nf}{findl2}\PY{p}{(}\PY{n}{l1}\PY{p}{,} \PY{n}{v1}\PY{p}{,} \PY{n}{v2}\PY{p}{)}\PY{p}{:}
              \PY{l+s+sd}{\PYZsq{}\PYZsq{}\PYZsq{}Finds distance from Sun to aphelion. Set equal to l2\PYZsq{}\PYZsq{}\PYZsq{}}
              \PY{k}{return} \PY{p}{(}\PY{n}{l1}\PY{o}{*}\PY{n}{v1}\PY{p}{)} \PY{o}{/} \PY{n}{v2}
          
          \PY{k}{def} \PY{n+nf}{semimajor}\PY{p}{(}\PY{n}{l1}\PY{p}{,} \PY{n}{l2}\PY{p}{)}\PY{p}{:}
              \PY{l+s+sd}{\PYZsq{}\PYZsq{}\PYZsq{}Finds semimajor axis of the orbit. Set equal to a.\PYZsq{}\PYZsq{}\PYZsq{}}
              \PY{k}{return} \PY{l+m+mf}{0.5} \PY{o}{*} \PY{p}{(}\PY{n}{l1}\PY{o}{+}\PY{n}{l2}\PY{p}{)}
          
          \PY{k}{def} \PY{n+nf}{semiminor}\PY{p}{(}\PY{n}{l1}\PY{p}{,} \PY{n}{l2}\PY{p}{)}\PY{p}{:}
              \PY{l+s+sd}{\PYZsq{}\PYZsq{}\PYZsq{}Finds semiminor axis of the orbit. Set equal to b.\PYZsq{}\PYZsq{}\PYZsq{}}
              \PY{k}{return} \PY{n}{sqrt}\PY{p}{(}\PY{n}{l1}\PY{o}{*}\PY{n}{l2}\PY{p}{)}
          
          \PY{k}{def} \PY{n+nf}{period}\PY{p}{(}\PY{n}{l1}\PY{p}{,} \PY{n}{v1}\PY{p}{,} \PY{n}{a}\PY{p}{,} \PY{n}{b}\PY{p}{)}\PY{p}{:}
              \PY{l+s+sd}{\PYZsq{}\PYZsq{}\PYZsq{}Find the period of an object orbitting the Sun. Set equal to T.\PYZsq{}\PYZsq{}\PYZsq{}}
              \PY{k}{return} \PY{p}{(}\PY{l+m+mi}{2}\PY{o}{*}\PY{n}{pi}\PY{o}{*}\PY{n}{a}\PY{o}{*}\PY{n}{b}\PY{p}{)} \PY{o}{/} \PY{p}{(}\PY{n}{l1}\PY{o}{*}\PY{n}{v1}\PY{p}{)}
          
          \PY{k}{def} \PY{n+nf}{eccentricity}\PY{p}{(}\PY{n}{l1}\PY{p}{,} \PY{n}{l2}\PY{p}{)}\PY{p}{:}
              \PY{l+s+sd}{\PYZdq{}\PYZdq{}\PYZdq{}Finds the orbit\PYZsq{}s eccentricity. Set equal to ecc.\PYZdq{}\PYZdq{}\PYZdq{}}
              \PY{k}{return} \PY{p}{(}\PY{n}{l2} \PY{o}{\PYZhy{}} \PY{n}{l1}\PY{p}{)} \PY{o}{/} \PY{p}{(}\PY{n}{l1} \PY{o}{+} \PY{n}{l2}\PY{p}{)}
              
\end{Verbatim}

    \begin{Verbatim}[commandchars=\\\{\}]
{\color{incolor}In [{\color{incolor}236}]:} \PY{k}{def} \PY{n+nf}{process}\PY{p}{(}\PY{n}{l1}\PY{p}{,} \PY{n}{v1}\PY{p}{)}\PY{p}{:}
              \PY{l+s+sd}{\PYZsq{}\PYZsq{}\PYZsq{}Puts together individual functions to find period and eccentricity at once\PYZsq{}\PYZsq{}\PYZsq{}}
              \PY{n}{v2} \PY{o}{=} \PY{n}{findv2}\PY{p}{(}\PY{n}{l1}\PY{p}{,} \PY{n}{v1}\PY{p}{)} \PY{c+c1}{\PYZsh{}units: m/s}
              \PY{n}{l2} \PY{o}{=} \PY{n}{findl2}\PY{p}{(}\PY{n}{l1}\PY{p}{,} \PY{n}{v1}\PY{p}{,} \PY{n}{v2}\PY{p}{)} \PY{c+c1}{\PYZsh{}units: m}
              \PY{n}{a} \PY{o}{=} \PY{n}{semimajor}\PY{p}{(}\PY{n}{l1}\PY{p}{,} \PY{n}{l2}\PY{p}{)} \PY{c+c1}{\PYZsh{}units: m}
              \PY{n}{b} \PY{o}{=} \PY{n}{semiminor}\PY{p}{(}\PY{n}{l1}\PY{p}{,} \PY{n}{l2}\PY{p}{)} \PY{c+c1}{\PYZsh{}units: m}
              
              \PY{n}{T} \PY{o}{=} \PY{n}{period}\PY{p}{(}\PY{n}{l1}\PY{p}{,} \PY{n}{v1}\PY{p}{,} \PY{n}{a}\PY{p}{,} \PY{n}{b}\PY{p}{)} \PY{c+c1}{\PYZsh{}units: s}
              \PY{n}{years} \PY{o}{=} \PY{n}{T} \PY{o}{/} \PY{l+m+mi}{31536000}
              \PY{n}{ecc} \PY{o}{=} \PY{n}{eccentricity}\PY{p}{(}\PY{n}{l1}\PY{p}{,} \PY{n}{l2}\PY{p}{)}
              
              \PY{n}{output} \PY{o}{=} \PY{p}{[}\PY{n}{l2}\PY{p}{,} \PY{n}{v2}\PY{p}{,} \PY{n}{years}\PY{p}{,} \PY{n}{ecc}\PY{p}{]}
              \PY{k}{return} \PY{n}{output}
\end{Verbatim}

    \begin{Verbatim}[commandchars=\\\{\}]
{\color{incolor}In [{\color{incolor}237}]:} \PY{n+nb}{print}\PY{p}{(}\PY{l+s+s2}{\PYZdq{}}\PY{l+s+s2}{For the Earth...}\PY{l+s+se}{\PYZbs{}n}\PY{l+s+s2}{ the distance to the aphelion is }\PY{l+s+si}{\PYZob{}:5.3f\PYZcb{}}\PY{l+s+s2}{ m }\PY{l+s+se}{\PYZbs{}n}\PY{l+s+s2}{ velocity is }\PY{l+s+si}{\PYZob{}:5.3f\PYZcb{}}\PY{l+s+s2}{ m/s }\PY{l+s+se}{\PYZbs{}n}\PY{l+s+s2}{ }\PY{l+s+se}{\PYZbs{}}
          \PY{l+s+s2}{orbital period is }\PY{l+s+si}{\PYZob{}:5.5f\PYZcb{}}\PY{l+s+s2}{ years }\PY{l+s+se}{\PYZbs{}n}\PY{l+s+s2}{ orbital eccentricity is }\PY{l+s+si}{\PYZob{}:5.5f\PYZcb{}}\PY{l+s+s2}{\PYZdq{}}\PYZbs{}
                \PY{o}{.}\PY{n}{format}\PY{p}{(}\PY{n}{process}\PY{p}{(}\PY{l+m+mf}{1.4710e11}\PY{p}{,} \PY{l+m+mf}{3.0287e4}\PY{p}{)}\PY{p}{[}\PY{l+m+mi}{0}\PY{p}{]}\PY{p}{,}\PYZbs{}
                       \PY{n}{process}\PY{p}{(}\PY{l+m+mf}{1.4710e11}\PY{p}{,} \PY{l+m+mf}{3.0287e4}\PY{p}{)}\PY{p}{[}\PY{l+m+mi}{1}\PY{p}{]}\PY{p}{,}\PYZbs{}
                       \PY{n}{process}\PY{p}{(}\PY{l+m+mf}{1.4710e11}\PY{p}{,} \PY{l+m+mf}{3.0287e4}\PY{p}{)}\PY{p}{[}\PY{l+m+mi}{2}\PY{p}{]}\PY{p}{,}\PYZbs{}
                       \PY{n}{process}\PY{p}{(}\PY{l+m+mf}{1.4710e11}\PY{p}{,} \PY{l+m+mf}{3.0287e4}\PY{p}{)}\PY{p}{[}\PY{l+m+mi}{3}\PY{p}{]}\PY{p}{)}\PY{p}{)} \PY{c+c1}{\PYZsh{}Earth\PYZsq{}s numbers}
\end{Verbatim}

    \begin{Verbatim}[commandchars=\\\{\}]
For the Earth{\ldots}
 the distance to the aphelion is 152027197208.660 m 
 velocity is 29305.399 m/s 
 orbital period is 1.00022 years 
 orbital eccentricity is 0.01647

    \end{Verbatim}

    \begin{Verbatim}[commandchars=\\\{\}]
{\color{incolor}In [{\color{incolor}238}]:} \PY{n+nb}{print}\PY{p}{(}\PY{l+s+s2}{\PYZdq{}}\PY{l+s+s2}{For Halley}\PY{l+s+s2}{\PYZsq{}}\PY{l+s+s2}{s comet...}\PY{l+s+se}{\PYZbs{}n}\PY{l+s+s2}{ the distance to the aphelion is }\PY{l+s+si}{\PYZob{}:5.3f\PYZcb{}}\PY{l+s+s2}{ m }\PY{l+s+se}{\PYZbs{}n}\PY{l+s+s2}{ velocity is }\PY{l+s+si}{\PYZob{}:5.3f\PYZcb{}}\PY{l+s+s2}{ m/s }\PY{l+s+se}{\PYZbs{}n}\PY{l+s+s2}{ }\PY{l+s+se}{\PYZbs{}}
          \PY{l+s+s2}{orbital period is }\PY{l+s+si}{\PYZob{}:5.5f\PYZcb{}}\PY{l+s+s2}{ years }\PY{l+s+se}{\PYZbs{}n}\PY{l+s+s2}{ orbital eccentricity is }\PY{l+s+si}{\PYZob{}:5.5f\PYZcb{}}\PY{l+s+s2}{\PYZdq{}}\PYZbs{}
                \PY{o}{.}\PY{n}{format}\PY{p}{(}\PY{n}{process}\PY{p}{(}\PY{l+m+mf}{8.7830e10}\PY{p}{,} \PY{l+m+mf}{5.4529e4}\PY{p}{)}\PY{p}{[}\PY{l+m+mi}{0}\PY{p}{]}\PY{p}{,}\PYZbs{}
                       \PY{n}{process}\PY{p}{(}\PY{l+m+mf}{8.7830e10}\PY{p}{,} \PY{l+m+mf}{5.4529e4}\PY{p}{)}\PY{p}{[}\PY{l+m+mi}{1}\PY{p}{]}\PY{p}{,}\PYZbs{}
                       \PY{n}{process}\PY{p}{(}\PY{l+m+mf}{8.7830e10}\PY{p}{,} \PY{l+m+mf}{5.4529e4}\PY{p}{)}\PY{p}{[}\PY{l+m+mi}{2}\PY{p}{]}\PY{p}{,}\PYZbs{}
                       \PY{n}{process}\PY{p}{(}\PY{l+m+mf}{8.7830e10}\PY{p}{,} \PY{l+m+mf}{5.4529e4}\PY{p}{)}\PY{p}{[}\PY{l+m+mi}{3}\PY{p}{]}\PY{p}{)}\PY{p}{)} \PY{c+c1}{\PYZsh{}Halley\PYZsq{}s numbers}
\end{Verbatim}

    \begin{Verbatim}[commandchars=\\\{\}]
For Halley's comet{\ldots}
 the distance to the aphelion is 5282214660876.441 m 
 velocity is 906.681 m/s 
 orbital period is 76.08170 years 
 orbital eccentricity is 0.96729

    \end{Verbatim}

    \subsection{CP 2.9 The Madelung
constant}\label{cp-2.9-the-madelung-constant}

The Madelung constant gives the electric potential in a solid based on
the surrounding atoms. This is found by summing the potential due to all
atoms as the number of atoms in all directions approaches infinity. The
equation relating the Madelung constant to the total potential is

\[V_\textrm{total} = \sum_{\substack{i,j,k=-L\\ \textrm{not }i=j=k=0}}^L
                   \hspace{-0.5em} V(i,j,k)
                 = {e\over4\pi\epsilon_0 a}\,M.\]

Thus, we can write (omitting the conditions of the sum for brevity)

\[M = \sum V(i,j,k) \ \frac{4\pi\epsilon_0 a}{e} = \sum \pm {e\over4\pi\epsilon_0 a\sqrt{i^2+j^2+k^2}} \ \frac{4\pi\epsilon_0 a}{e} = \pm\frac{1}{\sqrt{i^2+j^2+k^2}}.\]

This allows us to see that the Madelung constant M (for the case where
each atom is of unit charge) does not depend on anything other than the
relative positions of the atoms to each other, i.e., the spacing between
adjacent atoms does not affect its value.

    \begin{Verbatim}[commandchars=\\\{\}]
{\color{incolor}In [{\color{incolor}239}]:} \PY{k}{def} \PY{n+nf}{Madelung}\PY{p}{(}\PY{n}{L}\PY{p}{)}\PY{p}{:}
              \PY{l+s+sd}{\PYZsq{}\PYZsq{}\PYZsq{}The entire function for finding the Madelung constant for sodium chloride.}
          \PY{l+s+sd}{       L is the number of atoms extending from origin\PYZsq{}\PYZsq{}\PYZsq{}}
              \PY{n+nb}{sum} \PY{o}{=} \PY{l+m+mi}{0}
              
              \PY{k}{for} \PY{n}{i} \PY{o+ow}{in} \PY{n+nb}{range}\PY{p}{(}\PY{o}{\PYZhy{}}\PY{n}{L}\PY{p}{,} \PY{n}{L}\PY{p}{)}\PY{p}{:}
                  \PY{k}{for} \PY{n}{j} \PY{o+ow}{in} \PY{n+nb}{range}\PY{p}{(}\PY{o}{\PYZhy{}}\PY{n}{L}\PY{p}{,} \PY{n}{L}\PY{p}{)}\PY{p}{:}
                      \PY{k}{for} \PY{n}{k} \PY{o+ow}{in} \PY{n+nb}{range}\PY{p}{(}\PY{o}{\PYZhy{}}\PY{n}{L}\PY{p}{,} \PY{n}{L}\PY{p}{)}\PY{p}{:}
                          \PY{k}{if} \PY{n}{i} \PY{o}{==} \PY{l+m+mi}{0} \PY{o+ow}{and} \PY{n}{j} \PY{o}{==} \PY{l+m+mi}{0} \PY{o+ow}{and} \PY{n}{k} \PY{o}{==} \PY{l+m+mi}{0}\PY{p}{:}
                              \PY{k}{pass}
                          \PY{k}{else}\PY{p}{:}
                              \PY{k}{if} \PY{p}{(}\PY{n}{i} \PY{o}{+} \PY{n}{j} \PY{o}{+} \PY{n}{k}\PY{p}{)} \PY{o}{\PYZpc{}} \PY{l+m+mi}{2} \PY{o}{==} \PY{l+m+mi}{0}\PY{p}{:} \PY{c+c1}{\PYZsh{}adds all the potentials from sodium atoms}
                                  \PY{n+nb}{sum} \PY{o}{+}\PY{o}{=} \PY{o}{\PYZhy{}}\PY{l+m+mi}{1} \PY{o}{/} \PY{p}{(}\PY{n}{sqrt}\PY{p}{(}\PY{n}{i}\PY{o}{*}\PY{o}{*}\PY{l+m+mi}{2} \PY{o}{+} \PY{n}{j}\PY{o}{*}\PY{o}{*}\PY{l+m+mi}{2} \PY{o}{+} \PY{n}{k}\PY{o}{*}\PY{o}{*}\PY{l+m+mi}{2}\PY{p}{)}\PY{p}{)}
                              \PY{k}{elif} \PY{p}{(}\PY{n}{i} \PY{o}{+} \PY{n}{j} \PY{o}{+} \PY{n}{k}\PY{p}{)} \PY{o}{\PYZpc{}} \PY{l+m+mi}{2} \PY{o}{==} \PY{l+m+mi}{1}\PY{p}{:} \PY{c+c1}{\PYZsh{}adds all the potentials from chlorine atoms}
                                  \PY{n+nb}{sum} \PY{o}{+}\PY{o}{=} \PY{o}{+}\PY{l+m+mi}{1} \PY{o}{/} \PY{p}{(}\PY{n}{sqrt}\PY{p}{(}\PY{n}{i}\PY{o}{*}\PY{o}{*}\PY{l+m+mi}{2} \PY{o}{+} \PY{n}{j}\PY{o}{*}\PY{o}{*}\PY{l+m+mi}{2} \PY{o}{+} \PY{n}{k}\PY{o}{*}\PY{o}{*}\PY{l+m+mi}{2}\PY{p}{)}\PY{p}{)}
                          
              \PY{k}{return} \PY{n+nb}{sum}
\end{Verbatim}

    \begin{Verbatim}[commandchars=\\\{\}]
{\color{incolor}In [{\color{incolor}240}]:} \PY{o}{\PYZpc{}\PYZpc{}}\PY{k}{time}
          
          Ls = [10, 100] \PYZsh{}values of L to calculate the Madelung constant for
          
          [print(\PYZdq{}The Madelung constant for L =\PYZob{}:3.0f\PYZcb{} is \PYZob{}:1.3f\PYZcb{}. \PYZbs{}n\PYZdq{}.format(x, Madelung(x))) for x in Ls]
\end{Verbatim}

    \begin{Verbatim}[commandchars=\\\{\}]
The Madelung constant for L = 10 is 1.748. 

The Madelung constant for L =100 is 1.748. 

CPU times: user 13.1 s, sys: 181 ms, total: 13.2 s
Wall time: 15.3 s

    \end{Verbatim}

    \subsection{CP 3.6 Deterministic chaos and the Feigenbaum
plot}\label{cp-3.6-deterministic-chaos-and-the-feigenbaum-plot}

For certain values of \(r\) in the logistic equation

\[x' = rx(1-x),\]

the iterative map of the results will appear to be random. This is
deterministic chaos.

    \begin{Verbatim}[commandchars=\\\{\}]
{\color{incolor}In [{\color{incolor}241}]:} \PY{c+c1}{\PYZsh{}the logistic map function}
          
          \PY{k}{def} \PY{n+nf}{logistic}\PY{p}{(}\PY{n}{r}\PY{p}{,} \PY{n}{x}\PY{p}{)}\PY{p}{:}
              \PY{k}{return} \PY{n}{r} \PY{o}{*} \PY{n}{x} \PY{o}{*} \PY{p}{(}\PY{l+m+mi}{1}\PY{o}{\PYZhy{}}\PY{n}{x}\PY{p}{)}
\end{Verbatim}

    \begin{Verbatim}[commandchars=\\\{\}]
{\color{incolor}In [{\color{incolor}242}]:} \PY{c+c1}{\PYZsh{}x = np.linspace(0, 1, 100)}
          \PY{n}{fig}\PY{p}{,} \PY{n}{axis} \PY{o}{=} \PY{n}{pyplot}\PY{o}{.}\PY{n}{subplots}\PY{p}{(}\PY{l+m+mi}{1}\PY{p}{,} \PY{l+m+mi}{1}\PY{p}{)}
          \PY{n}{axis}\PY{o}{.}\PY{n}{plot}\PY{p}{(}\PY{n}{x}\PY{p}{,} \PY{n}{logistic}\PY{p}{(}\PY{l+m+mi}{2}\PY{p}{,} \PY{n}{x}\PY{p}{)}\PY{p}{,} \PY{l+s+s1}{\PYZsq{}}\PY{l+s+s1}{ko}\PY{l+s+s1}{\PYZsq{}}\PY{p}{)}
\end{Verbatim}

\begin{Verbatim}[commandchars=\\\{\}]
{\color{outcolor}Out[{\color{outcolor}242}]:} [<matplotlib.lines.Line2D at 0x11c23f438>]
\end{Verbatim}
            
    \begin{center}
    \adjustimage{max size={0.9\linewidth}{0.9\paperheight}}{output_17_1.png}
    \end{center}
    { \hspace*{\fill} \\}
    
    \begin{Verbatim}[commandchars=\\\{\}]
{\color{incolor}In [{\color{incolor}243}]:} \PY{c+c1}{\PYZsh{}for a single value of r, starts with x and iterates a certain number of times}
          \PY{k}{def} \PY{n+nf}{iterate}\PY{p}{(}\PY{n}{r}\PY{p}{,} \PY{n}{N}\PY{p}{)}\PY{p}{:}
              \PY{l+s+sd}{\PYZsq{}\PYZsq{}\PYZsq{}applies logistic map with N iterations to starting value of xi for increasing r\PYZsq{}\PYZsq{}\PYZsq{}}
              
              \PY{n}{fig}\PY{p}{,} \PY{n}{figplot} \PY{o}{=} \PY{n}{pyplot}\PY{o}{.}\PY{n}{subplots}\PY{p}{(}\PY{l+m+mi}{1}\PY{p}{,} \PY{l+m+mi}{1}\PY{p}{,} \PY{n}{figsize} \PY{o}{=} \PY{p}{(}\PY{l+m+mi}{7}\PY{p}{,} \PY{l+m+mi}{7}\PY{p}{)}\PY{p}{)}
              \PY{n}{figplot}\PY{o}{.}\PY{n}{set\PYZus{}title}\PY{p}{(}\PY{l+s+s2}{\PYZdq{}}\PY{l+s+s2}{Feigenbaum Plot}\PY{l+s+s2}{\PYZdq{}}\PY{p}{)}
              
              \PY{n}{xi} \PY{o}{=} \PY{l+m+mf}{0.5} \PY{c+c1}{\PYZsh{}starting value of x    }
              
              \PY{k}{for} \PY{n}{i} \PY{o+ow}{in} \PY{n+nb}{range}\PY{p}{(}\PY{n}{N}\PY{p}{)}\PY{p}{:}
                  \PY{n}{step} \PY{o}{=} \PY{n}{logistic}\PY{p}{(}\PY{n}{r}\PY{p}{,} \PY{n}{xi}\PY{p}{)} \PY{c+c1}{\PYZsh{}utilizes logistic map function to find output}
                  \PY{k}{if} \PY{n}{i} \PY{o}{\PYZhy{}} \PY{l+m+mi}{1000} \PY{o}{\PYZgt{}}\PY{o}{=} \PY{l+m+mi}{0}\PY{p}{:}
                      \PY{n}{figplot}\PY{o}{.}\PY{n}{plot}\PY{p}{(}\PY{n}{r}\PY{p}{,} \PY{n}{step}\PY{p}{,} \PY{l+s+s1}{\PYZsq{}}\PY{l+s+s1}{,k}\PY{l+s+s1}{\PYZsq{}}\PY{p}{)}
                      
                  \PY{n}{xi} \PY{o}{=} \PY{n}{step} \PY{c+c1}{\PYZsh{}with this, the output of logistic will become the next input}
\end{Verbatim}

    \begin{Verbatim}[commandchars=\\\{\}]
{\color{incolor}In [{\color{incolor}244}]:} \PY{n}{r} \PY{o}{=} \PY{n}{np}\PY{o}{.}\PY{n}{linspace}\PY{p}{(}\PY{l+m+mi}{1}\PY{p}{,} \PY{l+m+mi}{4}\PY{p}{,} \PY{l+m+mi}{400}\PY{p}{)} \PY{c+c1}{\PYZsh{}array of possible r valus}
          \PY{n}{iterate}\PY{p}{(}\PY{n}{r}\PY{p}{,} \PY{l+m+mi}{2000}\PY{p}{)} \PY{c+c1}{\PYZsh{}calls function}
\end{Verbatim}

    \begin{center}
    \adjustimage{max size={0.9\linewidth}{0.9\paperheight}}{output_19_0.png}
    \end{center}
    { \hspace*{\fill} \\}
    
    \begin{enumerate}
\def\labelenumi{\alph{enumi})}
\item
  For a given r on the Feigenbaum Plot, you get a fixed point if there
  is only a single x value corresponding to it. Likewise you can read
  off a limit cycle if there are multiple (but countable) x values to
  read off after branches. Chaos corresponds to an infinite amount of
  possible x values, and can be seen as the darker, almost fully-shaded
  regions.
\item
  The edge of chaos appears to be at r = 3.5. The logistic map
  transitions from a fixed point to limit cycles at r = 3.0 and splits
  into double the number of branches a few times before r = 3.5. After
  this, you start to see full shading along a vertical line passing
  through a given r value.
\end{enumerate}

    \subsection{CP 3.7 The Mandelbrot set}\label{cp-3.7-the-mandelbrot-set}

Like the logistic map, the Mandelbrot set is recursively applied. If
after some large number of iterations, \(|z'| < 2\) for every one of the
iterations, then that point is in the set.

    \begin{Verbatim}[commandchars=\\\{\}]
{\color{incolor}In [{\color{incolor}245}]:} \PY{k}{def} \PY{n+nf}{mandelbrot}\PY{p}{(}\PY{n}{z}\PY{p}{,} \PY{n}{c}\PY{p}{)}\PY{p}{:} \PY{c+c1}{\PYZsh{}defines the mandelbrot function}
              \PY{k}{return} \PY{n+nb}{pow}\PY{p}{(}\PY{n}{z}\PY{p}{,} \PY{l+m+mi}{2}\PY{p}{)} \PY{o}{+} \PY{n}{c}
\end{Verbatim}

    \begin{Verbatim}[commandchars=\\\{\}]
{\color{incolor}In [{\color{incolor}246}]:} \PY{k}{def} \PY{n+nf}{plot\PYZus{}mandelbrot}\PY{p}{(}\PY{n}{N}\PY{p}{,} \PY{n}{threshold}\PY{p}{,} \PY{n}{grid}\PY{p}{)}\PY{p}{:}
              \PY{l+s+sd}{\PYZsq{}\PYZsq{}\PYZsq{}This function will iterate the mandelbrot function \PYZlt{}N\PYZgt{} times,}
          \PY{l+s+sd}{       compare the magnitude of the resulting complex number to \PYZlt{}threshold\PYZgt{},}
          \PY{l+s+sd}{       then plot values under the threshold on a specified size \PYZlt{}grid\PYZgt{}.\PYZsq{}\PYZsq{}\PYZsq{}}
          
              \PY{c+c1}{\PYZsh{}fig, ax = pyplot.subplots(1, 1, figsize = (10, 10))}
              \PY{c+c1}{\PYZsh{}ax.imshow(\PYZhy{}\PYZhy{}\PYZhy{}\PYZhy{}\PYZhy{})}
              \PY{n}{x}\PY{p}{,} \PY{n}{y} \PY{o}{=} \PY{n}{np}\PY{o}{.}\PY{n}{ogrid}\PY{p}{[}\PY{o}{\PYZhy{}}\PY{l+m+mi}{2}\PY{p}{:}\PY{l+m+mi}{2}\PY{p}{:}\PY{l+m+mi}{1}\PY{n}{j}\PY{o}{*}\PY{n}{grid}\PY{p}{,} \PY{o}{\PYZhy{}}\PY{l+m+mi}{2}\PY{p}{:}\PY{l+m+mi}{2}\PY{p}{:}\PY{l+m+mi}{1}\PY{n}{j}\PY{o}{*}\PY{n}{grid}\PY{p}{]} \PY{c+c1}{\PYZsh{}creates the grid}
          
              \PY{n}{c} \PY{o}{=} \PY{n}{x} \PY{o}{+} \PY{n}{y}\PY{o}{*}\PY{l+m+mi}{1}\PY{n}{j} \PY{c+c1}{\PYZsh{}defines c as a complex number based on location on grid}
              \PY{n}{z} \PY{o}{=} \PY{l+m+mi}{0} \PY{c+c1}{\PYZsh{}starting value of z}
              
              \PY{c+c1}{\PYZsh{}iterates Mandelbrot function set number of times}
              \PY{k}{for} \PY{n}{i} \PY{o+ow}{in} \PY{n+nb}{range}\PY{p}{(}\PY{n}{N}\PY{p}{)}\PY{p}{:}
                  \PY{n}{z} \PY{o}{=} \PY{n}{mandelbrot}\PY{p}{(}\PY{n}{z}\PY{p}{,} \PY{n}{c}\PY{p}{)}
              \PY{c+c1}{\PYZsh{}[z = mandelbrot(z, c) for i in range(N)] }
              
              \PY{c+c1}{\PYZsh{}the values that are in the Mandelbrot set}
              \PY{n}{inset} \PY{o}{=} \PY{n}{np}\PY{o}{.}\PY{n}{abs}\PY{p}{(}\PY{n}{z}\PY{p}{)} \PY{o}{\PYZlt{}} \PY{n}{threshold}
              
              \PY{n}{pyplot}\PY{o}{.}\PY{n}{imshow}\PY{p}{(}\PY{n}{inset}\PY{o}{.}\PY{n}{T}\PY{p}{,} \PY{n}{extent} \PY{o}{=} \PY{p}{[}\PY{o}{\PYZhy{}}\PY{l+m+mi}{2}\PY{p}{,} \PY{l+m+mi}{2}\PY{p}{,} \PY{o}{\PYZhy{}}\PY{l+m+mi}{2}\PY{p}{,} \PY{l+m+mi}{2}\PY{p}{]}\PY{p}{,} \PY{n}{aspect} \PY{o}{=} \PY{l+s+s1}{\PYZsq{}}\PY{l+s+s1}{equal}\PY{l+s+s1}{\PYZsq{}}\PY{p}{)}\PY{c+c1}{\PYZsh{} ,cmap = \PYZsq{}jet\PYZsq{})}
              \PY{c+c1}{\PYZsh{}pyplot.gray}
              \PY{c+c1}{\PYZsh{}ax = pyplot.figure(figsize = (10,10))}
              
              \PY{n}{color} \PY{o}{=} \PY{n}{np}\PY{o}{.}\PY{n}{ogrid}\PY{p}{[}\PY{o}{\PYZhy{}}\PY{l+m+mi}{2}\PY{p}{:}\PY{l+m+mi}{2}\PY{p}{:}\PY{l+m+mi}{1}\PY{n}{j}\PY{o}{*}\PY{n}{grid}\PY{p}{,} \PY{o}{\PYZhy{}}\PY{l+m+mi}{2}\PY{p}{:}\PY{l+m+mi}{2}\PY{p}{:}\PY{l+m+mi}{1}\PY{n}{j}\PY{o}{*}\PY{n}{grid}\PY{p}{]} \PY{c+c1}{\PYZsh{}initializing array for colors}
              
              \PY{n}{pyplot}\PY{o}{.}\PY{n}{show}\PY{p}{(}\PY{p}{)}
\end{Verbatim}

    \begin{Verbatim}[commandchars=\\\{\}]
{\color{incolor}In [{\color{incolor}247}]:} \PY{n}{plot\PYZus{}mandelbrot}\PY{p}{(}\PY{l+m+mi}{100}\PY{p}{,} \PY{l+m+mi}{2}\PY{p}{,} \PY{l+m+mi}{4000}\PY{p}{)}
\end{Verbatim}

    \begin{Verbatim}[commandchars=\\\{\}]
/Users/Varun/anaconda/lib/python3.6/site-packages/ipykernel\_launcher.py:2: RuntimeWarning: overflow encountered in square
  
/Users/Varun/anaconda/lib/python3.6/site-packages/ipykernel\_launcher.py:2: RuntimeWarning: invalid value encountered in square
  
/Users/Varun/anaconda/lib/python3.6/site-packages/ipykernel\_launcher.py:19: RuntimeWarning: invalid value encountered in less

    \end{Verbatim}

    \begin{center}
    \adjustimage{max size={0.9\linewidth}{0.9\paperheight}}{output_24_1.png}
    \end{center}
    { \hspace*{\fill} \\}
    

    % Add a bibliography block to the postdoc
    
    
    
    \end{document}

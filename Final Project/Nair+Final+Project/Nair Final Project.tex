
% Default to the notebook output style

    


% Inherit from the specified cell style.




    
\documentclass[11pt]{article}

    
    
    \usepackage[T1]{fontenc}
    % Nicer default font (+ math font) than Computer Modern for most use cases
    \usepackage{mathpazo}

    % Basic figure setup, for now with no caption control since it's done
    % automatically by Pandoc (which extracts ![](path) syntax from Markdown).
    \usepackage{graphicx}
    % We will generate all images so they have a width \maxwidth. This means
    % that they will get their normal width if they fit onto the page, but
    % are scaled down if they would overflow the margins.
    \makeatletter
    \def\maxwidth{\ifdim\Gin@nat@width>\linewidth\linewidth
    \else\Gin@nat@width\fi}
    \makeatother
    \let\Oldincludegraphics\includegraphics
    % Set max figure width to be 80% of text width, for now hardcoded.
    \renewcommand{\includegraphics}[1]{\Oldincludegraphics[width=.8\maxwidth]{#1}}
    % Ensure that by default, figures have no caption (until we provide a
    % proper Figure object with a Caption API and a way to capture that
    % in the conversion process - todo).
    \usepackage{caption}
    \DeclareCaptionLabelFormat{nolabel}{}
    \captionsetup{labelformat=nolabel}

    \usepackage{adjustbox} % Used to constrain images to a maximum size 
    \usepackage{xcolor} % Allow colors to be defined
    \usepackage{enumerate} % Needed for markdown enumerations to work
    \usepackage{geometry} % Used to adjust the document margins
    \usepackage{amsmath} % Equations
    \usepackage{amssymb} % Equations
    \usepackage{textcomp} % defines textquotesingle
    % Hack from http://tex.stackexchange.com/a/47451/13684:
    \AtBeginDocument{%
        \def\PYZsq{\textquotesingle}% Upright quotes in Pygmentized code
    }
    \usepackage{upquote} % Upright quotes for verbatim code
    \usepackage{eurosym} % defines \euro
    \usepackage[mathletters]{ucs} % Extended unicode (utf-8) support
    \usepackage[utf8x]{inputenc} % Allow utf-8 characters in the tex document
    \usepackage{fancyvrb} % verbatim replacement that allows latex
    \usepackage{grffile} % extends the file name processing of package graphics 
                         % to support a larger range 
    % The hyperref package gives us a pdf with properly built
    % internal navigation ('pdf bookmarks' for the table of contents,
    % internal cross-reference links, web links for URLs, etc.)
    \usepackage{hyperref}
    \usepackage{longtable} % longtable support required by pandoc >1.10
    \usepackage{booktabs}  % table support for pandoc > 1.12.2
    \usepackage[inline]{enumitem} % IRkernel/repr support (it uses the enumerate* environment)
    \usepackage[normalem]{ulem} % ulem is needed to support strikethroughs (\sout)
                                % normalem makes italics be italics, not underlines
    \usepackage{mathrsfs}
    \usepackage{mathtools}

    
    
    % Colors for the hyperref package
    \definecolor{urlcolor}{rgb}{0,.145,.698}
    \definecolor{linkcolor}{rgb}{.71,0.21,0.01}
    \definecolor{citecolor}{rgb}{.12,.54,.11}

    % ANSI colors
    \definecolor{ansi-black}{HTML}{3E424D}
    \definecolor{ansi-black-intense}{HTML}{282C36}
    \definecolor{ansi-red}{HTML}{E75C58}
    \definecolor{ansi-red-intense}{HTML}{B22B31}
    \definecolor{ansi-green}{HTML}{00A250}
    \definecolor{ansi-green-intense}{HTML}{007427}
    \definecolor{ansi-yellow}{HTML}{DDB62B}
    \definecolor{ansi-yellow-intense}{HTML}{B27D12}
    \definecolor{ansi-blue}{HTML}{208FFB}
    \definecolor{ansi-blue-intense}{HTML}{0065CA}
    \definecolor{ansi-magenta}{HTML}{D160C4}
    \definecolor{ansi-magenta-intense}{HTML}{A03196}
    \definecolor{ansi-cyan}{HTML}{60C6C8}
    \definecolor{ansi-cyan-intense}{HTML}{258F8F}
    \definecolor{ansi-white}{HTML}{C5C1B4}
    \definecolor{ansi-white-intense}{HTML}{A1A6B2}
    \definecolor{ansi-default-inverse-fg}{HTML}{FFFFFF}
    \definecolor{ansi-default-inverse-bg}{HTML}{000000}

    % commands and environments needed by pandoc snippets
    % extracted from the output of `pandoc -s`
    \providecommand{\tightlist}{%
      \setlength{\itemsep}{0pt}\setlength{\parskip}{0pt}}
    \DefineVerbatimEnvironment{Highlighting}{Verbatim}{commandchars=\\\{\}}
    % Add ',fontsize=\small' for more characters per line
    \newenvironment{Shaded}{}{}
    \newcommand{\KeywordTok}[1]{\textcolor[rgb]{0.00,0.44,0.13}{\textbf{{#1}}}}
    \newcommand{\DataTypeTok}[1]{\textcolor[rgb]{0.56,0.13,0.00}{{#1}}}
    \newcommand{\DecValTok}[1]{\textcolor[rgb]{0.25,0.63,0.44}{{#1}}}
    \newcommand{\BaseNTok}[1]{\textcolor[rgb]{0.25,0.63,0.44}{{#1}}}
    \newcommand{\FloatTok}[1]{\textcolor[rgb]{0.25,0.63,0.44}{{#1}}}
    \newcommand{\CharTok}[1]{\textcolor[rgb]{0.25,0.44,0.63}{{#1}}}
    \newcommand{\StringTok}[1]{\textcolor[rgb]{0.25,0.44,0.63}{{#1}}}
    \newcommand{\CommentTok}[1]{\textcolor[rgb]{0.38,0.63,0.69}{\textit{{#1}}}}
    \newcommand{\OtherTok}[1]{\textcolor[rgb]{0.00,0.44,0.13}{{#1}}}
    \newcommand{\AlertTok}[1]{\textcolor[rgb]{1.00,0.00,0.00}{\textbf{{#1}}}}
    \newcommand{\FunctionTok}[1]{\textcolor[rgb]{0.02,0.16,0.49}{{#1}}}
    \newcommand{\RegionMarkerTok}[1]{{#1}}
    \newcommand{\ErrorTok}[1]{\textcolor[rgb]{1.00,0.00,0.00}{\textbf{{#1}}}}
    \newcommand{\NormalTok}[1]{{#1}}
    
    % Additional commands for more recent versions of Pandoc
    \newcommand{\ConstantTok}[1]{\textcolor[rgb]{0.53,0.00,0.00}{{#1}}}
    \newcommand{\SpecialCharTok}[1]{\textcolor[rgb]{0.25,0.44,0.63}{{#1}}}
    \newcommand{\VerbatimStringTok}[1]{\textcolor[rgb]{0.25,0.44,0.63}{{#1}}}
    \newcommand{\SpecialStringTok}[1]{\textcolor[rgb]{0.73,0.40,0.53}{{#1}}}
    \newcommand{\ImportTok}[1]{{#1}}
    \newcommand{\DocumentationTok}[1]{\textcolor[rgb]{0.73,0.13,0.13}{\textit{{#1}}}}
    \newcommand{\AnnotationTok}[1]{\textcolor[rgb]{0.38,0.63,0.69}{\textbf{\textit{{#1}}}}}
    \newcommand{\CommentVarTok}[1]{\textcolor[rgb]{0.38,0.63,0.69}{\textbf{\textit{{#1}}}}}
    \newcommand{\VariableTok}[1]{\textcolor[rgb]{0.10,0.09,0.49}{{#1}}}
    \newcommand{\ControlFlowTok}[1]{\textcolor[rgb]{0.00,0.44,0.13}{\textbf{{#1}}}}
    \newcommand{\OperatorTok}[1]{\textcolor[rgb]{0.40,0.40,0.40}{{#1}}}
    \newcommand{\BuiltInTok}[1]{{#1}}
    \newcommand{\ExtensionTok}[1]{{#1}}
    \newcommand{\PreprocessorTok}[1]{\textcolor[rgb]{0.74,0.48,0.00}{{#1}}}
    \newcommand{\AttributeTok}[1]{\textcolor[rgb]{0.49,0.56,0.16}{{#1}}}
    \newcommand{\InformationTok}[1]{\textcolor[rgb]{0.38,0.63,0.69}{\textbf{\textit{{#1}}}}}
    \newcommand{\WarningTok}[1]{\textcolor[rgb]{0.38,0.63,0.69}{\textbf{\textit{{#1}}}}}
    
    
    % Define a nice break command that doesn't care if a line doesn't already
    % exist.
    \def\br{\hspace*{\fill} \\* }
    % Math Jax compatibility definitions
    \def\gt{>}
    \def\lt{<}
    \let\Oldtex\TeX
    \let\Oldlatex\LaTeX
    \renewcommand{\TeX}{\textrm{\Oldtex}}
    \renewcommand{\LaTeX}{\textrm{\Oldlatex}}
    % Document parameters
    % Document title
    \title{Physics 566: Computational Physics Final Project\\ \vspace{5mm}
    {\Large Modeling Channel Flow with the Navier-Stokes Equations}\\ \vspace{10mm}
    {\large Varun Nair}}     
    
    
    
    

    % Pygments definitions
    
\makeatletter
\def\PY@reset{\let\PY@it=\relax \let\PY@bf=\relax%
    \let\PY@ul=\relax \let\PY@tc=\relax%
    \let\PY@bc=\relax \let\PY@ff=\relax}
\def\PY@tok#1{\csname PY@tok@#1\endcsname}
\def\PY@toks#1+{\ifx\relax#1\empty\else%
    \PY@tok{#1}\expandafter\PY@toks\fi}
\def\PY@do#1{\PY@bc{\PY@tc{\PY@ul{%
    \PY@it{\PY@bf{\PY@ff{#1}}}}}}}
\def\PY#1#2{\PY@reset\PY@toks#1+\relax+\PY@do{#2}}

\expandafter\def\csname PY@tok@w\endcsname{\def\PY@tc##1{\textcolor[rgb]{0.73,0.73,0.73}{##1}}}
\expandafter\def\csname PY@tok@c\endcsname{\let\PY@it=\textit\def\PY@tc##1{\textcolor[rgb]{0.25,0.50,0.50}{##1}}}
\expandafter\def\csname PY@tok@cp\endcsname{\def\PY@tc##1{\textcolor[rgb]{0.74,0.48,0.00}{##1}}}
\expandafter\def\csname PY@tok@k\endcsname{\let\PY@bf=\textbf\def\PY@tc##1{\textcolor[rgb]{0.00,0.50,0.00}{##1}}}
\expandafter\def\csname PY@tok@kp\endcsname{\def\PY@tc##1{\textcolor[rgb]{0.00,0.50,0.00}{##1}}}
\expandafter\def\csname PY@tok@kt\endcsname{\def\PY@tc##1{\textcolor[rgb]{0.69,0.00,0.25}{##1}}}
\expandafter\def\csname PY@tok@o\endcsname{\def\PY@tc##1{\textcolor[rgb]{0.40,0.40,0.40}{##1}}}
\expandafter\def\csname PY@tok@ow\endcsname{\let\PY@bf=\textbf\def\PY@tc##1{\textcolor[rgb]{0.67,0.13,1.00}{##1}}}
\expandafter\def\csname PY@tok@nb\endcsname{\def\PY@tc##1{\textcolor[rgb]{0.00,0.50,0.00}{##1}}}
\expandafter\def\csname PY@tok@nf\endcsname{\def\PY@tc##1{\textcolor[rgb]{0.00,0.00,1.00}{##1}}}
\expandafter\def\csname PY@tok@nc\endcsname{\let\PY@bf=\textbf\def\PY@tc##1{\textcolor[rgb]{0.00,0.00,1.00}{##1}}}
\expandafter\def\csname PY@tok@nn\endcsname{\let\PY@bf=\textbf\def\PY@tc##1{\textcolor[rgb]{0.00,0.00,1.00}{##1}}}
\expandafter\def\csname PY@tok@ne\endcsname{\let\PY@bf=\textbf\def\PY@tc##1{\textcolor[rgb]{0.82,0.25,0.23}{##1}}}
\expandafter\def\csname PY@tok@nv\endcsname{\def\PY@tc##1{\textcolor[rgb]{0.10,0.09,0.49}{##1}}}
\expandafter\def\csname PY@tok@no\endcsname{\def\PY@tc##1{\textcolor[rgb]{0.53,0.00,0.00}{##1}}}
\expandafter\def\csname PY@tok@nl\endcsname{\def\PY@tc##1{\textcolor[rgb]{0.63,0.63,0.00}{##1}}}
\expandafter\def\csname PY@tok@ni\endcsname{\let\PY@bf=\textbf\def\PY@tc##1{\textcolor[rgb]{0.60,0.60,0.60}{##1}}}
\expandafter\def\csname PY@tok@na\endcsname{\def\PY@tc##1{\textcolor[rgb]{0.49,0.56,0.16}{##1}}}
\expandafter\def\csname PY@tok@nt\endcsname{\let\PY@bf=\textbf\def\PY@tc##1{\textcolor[rgb]{0.00,0.50,0.00}{##1}}}
\expandafter\def\csname PY@tok@nd\endcsname{\def\PY@tc##1{\textcolor[rgb]{0.67,0.13,1.00}{##1}}}
\expandafter\def\csname PY@tok@s\endcsname{\def\PY@tc##1{\textcolor[rgb]{0.73,0.13,0.13}{##1}}}
\expandafter\def\csname PY@tok@sd\endcsname{\let\PY@it=\textit\def\PY@tc##1{\textcolor[rgb]{0.73,0.13,0.13}{##1}}}
\expandafter\def\csname PY@tok@si\endcsname{\let\PY@bf=\textbf\def\PY@tc##1{\textcolor[rgb]{0.73,0.40,0.53}{##1}}}
\expandafter\def\csname PY@tok@se\endcsname{\let\PY@bf=\textbf\def\PY@tc##1{\textcolor[rgb]{0.73,0.40,0.13}{##1}}}
\expandafter\def\csname PY@tok@sr\endcsname{\def\PY@tc##1{\textcolor[rgb]{0.73,0.40,0.53}{##1}}}
\expandafter\def\csname PY@tok@ss\endcsname{\def\PY@tc##1{\textcolor[rgb]{0.10,0.09,0.49}{##1}}}
\expandafter\def\csname PY@tok@sx\endcsname{\def\PY@tc##1{\textcolor[rgb]{0.00,0.50,0.00}{##1}}}
\expandafter\def\csname PY@tok@m\endcsname{\def\PY@tc##1{\textcolor[rgb]{0.40,0.40,0.40}{##1}}}
\expandafter\def\csname PY@tok@gh\endcsname{\let\PY@bf=\textbf\def\PY@tc##1{\textcolor[rgb]{0.00,0.00,0.50}{##1}}}
\expandafter\def\csname PY@tok@gu\endcsname{\let\PY@bf=\textbf\def\PY@tc##1{\textcolor[rgb]{0.50,0.00,0.50}{##1}}}
\expandafter\def\csname PY@tok@gd\endcsname{\def\PY@tc##1{\textcolor[rgb]{0.63,0.00,0.00}{##1}}}
\expandafter\def\csname PY@tok@gi\endcsname{\def\PY@tc##1{\textcolor[rgb]{0.00,0.63,0.00}{##1}}}
\expandafter\def\csname PY@tok@gr\endcsname{\def\PY@tc##1{\textcolor[rgb]{1.00,0.00,0.00}{##1}}}
\expandafter\def\csname PY@tok@ge\endcsname{\let\PY@it=\textit}
\expandafter\def\csname PY@tok@gs\endcsname{\let\PY@bf=\textbf}
\expandafter\def\csname PY@tok@gp\endcsname{\let\PY@bf=\textbf\def\PY@tc##1{\textcolor[rgb]{0.00,0.00,0.50}{##1}}}
\expandafter\def\csname PY@tok@go\endcsname{\def\PY@tc##1{\textcolor[rgb]{0.53,0.53,0.53}{##1}}}
\expandafter\def\csname PY@tok@gt\endcsname{\def\PY@tc##1{\textcolor[rgb]{0.00,0.27,0.87}{##1}}}
\expandafter\def\csname PY@tok@err\endcsname{\def\PY@bc##1{\setlength{\fboxsep}{0pt}\fcolorbox[rgb]{1.00,0.00,0.00}{1,1,1}{\strut ##1}}}
\expandafter\def\csname PY@tok@kc\endcsname{\let\PY@bf=\textbf\def\PY@tc##1{\textcolor[rgb]{0.00,0.50,0.00}{##1}}}
\expandafter\def\csname PY@tok@kd\endcsname{\let\PY@bf=\textbf\def\PY@tc##1{\textcolor[rgb]{0.00,0.50,0.00}{##1}}}
\expandafter\def\csname PY@tok@kn\endcsname{\let\PY@bf=\textbf\def\PY@tc##1{\textcolor[rgb]{0.00,0.50,0.00}{##1}}}
\expandafter\def\csname PY@tok@kr\endcsname{\let\PY@bf=\textbf\def\PY@tc##1{\textcolor[rgb]{0.00,0.50,0.00}{##1}}}
\expandafter\def\csname PY@tok@bp\endcsname{\def\PY@tc##1{\textcolor[rgb]{0.00,0.50,0.00}{##1}}}
\expandafter\def\csname PY@tok@fm\endcsname{\def\PY@tc##1{\textcolor[rgb]{0.00,0.00,1.00}{##1}}}
\expandafter\def\csname PY@tok@vc\endcsname{\def\PY@tc##1{\textcolor[rgb]{0.10,0.09,0.49}{##1}}}
\expandafter\def\csname PY@tok@vg\endcsname{\def\PY@tc##1{\textcolor[rgb]{0.10,0.09,0.49}{##1}}}
\expandafter\def\csname PY@tok@vi\endcsname{\def\PY@tc##1{\textcolor[rgb]{0.10,0.09,0.49}{##1}}}
\expandafter\def\csname PY@tok@vm\endcsname{\def\PY@tc##1{\textcolor[rgb]{0.10,0.09,0.49}{##1}}}
\expandafter\def\csname PY@tok@sa\endcsname{\def\PY@tc##1{\textcolor[rgb]{0.73,0.13,0.13}{##1}}}
\expandafter\def\csname PY@tok@sb\endcsname{\def\PY@tc##1{\textcolor[rgb]{0.73,0.13,0.13}{##1}}}
\expandafter\def\csname PY@tok@sc\endcsname{\def\PY@tc##1{\textcolor[rgb]{0.73,0.13,0.13}{##1}}}
\expandafter\def\csname PY@tok@dl\endcsname{\def\PY@tc##1{\textcolor[rgb]{0.73,0.13,0.13}{##1}}}
\expandafter\def\csname PY@tok@s2\endcsname{\def\PY@tc##1{\textcolor[rgb]{0.73,0.13,0.13}{##1}}}
\expandafter\def\csname PY@tok@sh\endcsname{\def\PY@tc##1{\textcolor[rgb]{0.73,0.13,0.13}{##1}}}
\expandafter\def\csname PY@tok@s1\endcsname{\def\PY@tc##1{\textcolor[rgb]{0.73,0.13,0.13}{##1}}}
\expandafter\def\csname PY@tok@mb\endcsname{\def\PY@tc##1{\textcolor[rgb]{0.40,0.40,0.40}{##1}}}
\expandafter\def\csname PY@tok@mf\endcsname{\def\PY@tc##1{\textcolor[rgb]{0.40,0.40,0.40}{##1}}}
\expandafter\def\csname PY@tok@mh\endcsname{\def\PY@tc##1{\textcolor[rgb]{0.40,0.40,0.40}{##1}}}
\expandafter\def\csname PY@tok@mi\endcsname{\def\PY@tc##1{\textcolor[rgb]{0.40,0.40,0.40}{##1}}}
\expandafter\def\csname PY@tok@il\endcsname{\def\PY@tc##1{\textcolor[rgb]{0.40,0.40,0.40}{##1}}}
\expandafter\def\csname PY@tok@mo\endcsname{\def\PY@tc##1{\textcolor[rgb]{0.40,0.40,0.40}{##1}}}
\expandafter\def\csname PY@tok@ch\endcsname{\let\PY@it=\textit\def\PY@tc##1{\textcolor[rgb]{0.25,0.50,0.50}{##1}}}
\expandafter\def\csname PY@tok@cm\endcsname{\let\PY@it=\textit\def\PY@tc##1{\textcolor[rgb]{0.25,0.50,0.50}{##1}}}
\expandafter\def\csname PY@tok@cpf\endcsname{\let\PY@it=\textit\def\PY@tc##1{\textcolor[rgb]{0.25,0.50,0.50}{##1}}}
\expandafter\def\csname PY@tok@c1\endcsname{\let\PY@it=\textit\def\PY@tc##1{\textcolor[rgb]{0.25,0.50,0.50}{##1}}}
\expandafter\def\csname PY@tok@cs\endcsname{\let\PY@it=\textit\def\PY@tc##1{\textcolor[rgb]{0.25,0.50,0.50}{##1}}}

\def\PYZbs{\char`\\}
\def\PYZus{\char`\_}
\def\PYZob{\char`\{}
\def\PYZcb{\char`\}}
\def\PYZca{\char`\^}
\def\PYZam{\char`\&}
\def\PYZlt{\char`\<}
\def\PYZgt{\char`\>}
\def\PYZsh{\char`\#}
\def\PYZpc{\char`\%}
\def\PYZdl{\char`\$}
\def\PYZhy{\char`\-}
\def\PYZsq{\char`\'}
\def\PYZdq{\char`\"}
\def\PYZti{\char`\~}
% for compatibility with earlier versions
\def\PYZat{@}
\def\PYZlb{[}
\def\PYZrb{]}
\makeatother


    % Exact colors from NB
    \definecolor{incolor}{rgb}{0.0, 0.0, 0.5}
    \definecolor{outcolor}{rgb}{0.545, 0.0, 0.0}



    
    % Prevent overflowing lines due to hard-to-break entities
    \sloppy 
    % Setup hyperref package
    \hypersetup{
      breaklinks=true,  % so long urls are correctly broken across lines
      colorlinks=true,
      urlcolor=urlcolor,
      linkcolor=linkcolor,
      citecolor=citecolor,
      }
    % Slightly bigger margins than the latex defaults
    
    \geometry{verbose,tmargin=1in,bmargin=1in,lmargin=1in,rmargin=1in}
    
    

    \begin{document}
    
    
    \maketitle
    
    

    
    \begin{Verbatim}[commandchars=\\\{\}]
{\color{incolor}In [{\color{incolor}1}]:} \PY{o}{\PYZpc{}}\PY{k}{precision} \PYZpc{}g
        \PY{o}{\PYZpc{}}\PY{k}{matplotlib} inline
        \PY{o}{\PYZpc{}}\PY{k}{config} InlineBackend.figure\PYZus{}format = \PYZsq{}retina\PYZsq{}
        
        \PY{k+kn}{from} \PY{n+nn}{math} \PY{k}{import} \PY{n}{sqrt}\PY{p}{,} \PY{n}{pi}\PY{p}{,} \PY{n}{sin}\PY{p}{,} \PY{n}{cos}\PY{p}{,} \PY{n}{floor}\PY{p}{,} \PY{n}{exp}
        \PY{k+kn}{import} \PY{n+nn}{numpy} \PY{k}{as} \PY{n+nn}{np}
        \PY{k+kn}{from} \PY{n+nn}{numpy} \PY{k}{import} \PY{n}{linalg} \PY{k}{as} \PY{n}{LA}
        \PY{k+kn}{from} \PY{n+nn}{scipy} \PY{k}{import} \PY{n}{constants} \PY{k}{as} \PY{n}{con}
        \PY{k+kn}{import} \PY{n+nn}{matplotlib}\PY{n+nn}{.}\PY{n+nn}{pyplot} \PY{k}{as} \PY{n+nn}{plt}
        
        \PY{k+kn}{from} \PY{n+nn}{channel\PYZus{}flow} \PY{k}{import} \PY{n}{solver}\PY{p}{,} \PY{n}{solved\PYZus{}u}
\end{Verbatim}

    \section{Theory}\label{theory}

\subsection{Introduction}\label{introduction}

The Navier-Stokes equations (NSE) can only be solved analytically for a
few systems. One of these is for steady-state, fully developed laminar
flow in a single direction between two plates. Laminar flow occurs when
fluids move in parallel layers, so that movement in one layer doesn't
disturb movement in another layer. The equation of motion for this
system can be written as

\begin{equation}
\frac{\partial \mathbf{u}}{\partial t} + \bigr(\mathbf{u} \cdot \nabla\bigr)\mathbf{u} = \nu\Delta\mathbf{u} - \frac 1\rho\nabla P,
\end{equation}

where \(\mathbf{u}\) is the fluid velocity vector,
\(\nu = \frac{\textrm{viscosity}}{\textrm{density}}\) is the kinematic
viscosity, and \(P \equiv p + \Phi.\) The fluid is propelled through the
channel by a force \(F = -\nabla\Phi.\) This can be rewritten in terms
of fluid density (\(\rho\)) and shear viscosity (\(\eta\)), giving the
partial differential equation

\begin{equation}
\rho\frac{\partial \mathbf{u}}{\partial t} + \rho\bigr(\mathbf{u} \cdot \nabla\bigr)\mathbf{u} = - \nabla p + \eta\Delta\mathbf{u},
\end{equation}

where \(\mathbf{u}\) is again the velocity vector and \(P = P(x,t)\) is
the pressure as a function of position and time. For incompressible
fluid flow, I can get a Poisson equation for pressure by setting the
divergence of the momentum equation zero (this ends up taking the
divergence of velocity to be zero because mass is constant for
incompressible fluids) and assuming the pressure must be continuous over
the region.

\begin{equation}
\nabla \cdot \mathbf{u} = 0
\end{equation}

\begin{equation}
\frac{\partial u}{\partial x} + \frac{\partial v}{\partial y} = 0
\end{equation}

These conditions allow us to write three equations that govern the fluid
flow through the channel. From now on, I use \(u\) and \(v\) to
designate the velocities in the \(x\) and \(y\) directions,
respectively. \(\rho\) and \(\nu\) are still the fluid density and
viscosity, respectively.

\begin{equation*}
\frac{\partial u}{\partial t} + u \frac{\partial u}{\partial x} + v\frac{\partial u}{\partial y} = -\frac 1\rho\frac{\partial p}{\partial x} + \nu\biggr(\frac{\partial^2 u}{\partial x^2} + \frac{\partial^2 u}{\partial y^2}\biggr)
\end{equation*}

\begin{equation*}
\frac{\partial v}{\partial t} + u \frac{\partial v}{\partial x} + v\frac{\partial v}{\partial y} = -\frac 1\rho\frac{\partial p}{\partial y} + \nu\biggr(\frac{\partial^2 v}{\partial x^2} + \frac{\partial^2 v}{\partial y^2}\biggr)
\end{equation*}

\begin{equation*}
-\frac 1\rho \biggr(\frac{\partial^2 p}{\partial x^2} + \frac{\partial^2 p}{\partial y^2}\biggr) = \biggr(\frac{\partial u}{\partial x}\biggr)^2 + 2\frac{\partial u}{\partial y}\frac{\partial v}{\partial x} + \biggr(\frac{\partial v}{\partial y}\biggr)^2
\end{equation*}

I now have a system of 2nd order, nonlinear partial differential
equations that fully describe a fluid's behavior in a channel. However,
these equations are only analytically solvable for a closed form
solution under the following conditions: - laminar flow - steady flow -
incompressible fluid - 2-D geometry - between two smooth surfaces of
constant width.

\subsection{More Complicated Flows}\label{more-complicated-flows}

I'll use computational physics to numerically solve these equations when
assumptions are relaxed. First I will relax the steady flow assumption
for fluid velocity in the \(x\) direction by writing pressure in terms
of steady and unsteady components. I write the total pressure

\[p = p_0 + p_1\]

as the sum of steady (\(p_0\)) and unsteady (\(p_1\)) pressures.
Previously, \(p_1 = 0\), but if it varied with time, it would alter the
flow.

\begin{equation}
\frac{\partial u}{\partial t} + u \frac{\partial u}{\partial x} + v\frac{\partial u}{\partial y} = -\frac 1\rho\biggr(\frac{\partial p_0}{\partial x} + \frac{\partial p_1}{\partial x}\biggr) + \nu\biggr(\frac{\partial^2 u}{\partial x^2} + \frac{\partial^2 u}{\partial y^2}\biggr)
\end{equation}

\begin{equation}
\frac{\partial v}{\partial t} + u \frac{\partial v}{\partial x} + v\frac{\partial v}{\partial y} = -\frac 1\rho\frac{\partial p}{\partial y} + \nu\biggr(\frac{\partial^2 v}{\partial x^2} + \frac{\partial^2 v}{\partial y^2}\biggr)
\end{equation}

\begin{equation}
-\frac 1\rho \biggr(\frac{\partial^2 p}{\partial x^2} + \frac{\partial^2 p}{\partial y^2}\biggr) = \biggr(\frac{\partial u}{\partial x}\biggr)^2 + 2\frac{\partial u}{\partial y}\frac{\partial v}{\partial x} + \biggr(\frac{\partial v}{\partial y}\biggr)^2
\end{equation}

\subsection{Initial/Boundary
Conditions}\label{initialboundary-conditions}

The initial conditions I use for the different cases remain the same.
The initial velocity of the fluid at all points in the channel (for both
\(x\) and \(y\) directions is zero). The fluid is also at equal pressure
throughout the region. \[ u(x,y,0) = v(x,y,0) = 0\] \[p(x,y,0) = 5\]

The boundary conditions for the most basic case are that fluid velocity
along the upper and lower boundary is zero due to the no slip condition.
Further the first derivative of pressure is zero.

\begin{subequations}
\begin{align}
u(x,0,t) = u(x,\tfrac L2,t) = v(x,0,t) = v(x,\tfrac L2,t) = 0\\
\frac{\partial p}{\partial y} \biggr\vert_{y=0} = \frac{\partial p}{\partial y} \biggr\vert_{y=\frac{L}{2}} = 0
\end{align}
\end{subequations}

\subsection{Modelling the Flow}\label{modelling-the-flow}

Given this, we can solve for the velocity of the fluid throughout the
region. I'll write out the expression for \(v_x\) of the fluid on the
interior of the region explicitly as

\begin{align}
        \phantom{u(x,y)}
        &\begin{aligned}
          \mathllap{u(x,y)} &\leftarrow u(x,y) - \frac{u(x,y)\text{d}t}{a}\biggr[u(x,y)- u(x-a,y)\biggr]\\
          &\quad - \frac{v(x,y)\text{d}t}{a}\biggr[u(x,y) - u(x,y-a)\biggr]- \frac{\text{d}t}{2\rho a}\biggr[p(x+a,y) - p(x-a,y)\biggr]\\
            &\quad + \nu\biggr[\frac{\text{d}t}{a^2}\biggr(\bigr(u(x+a,y) - 2u(x,y) + u(x-a,y)\bigr) + \bigr(u(x,y+a) - 2u(x,y) + u(x,y-a)\bigr)\biggr)\biggr]\\
            &\quad + P\text{d}t\\
        \end{aligned}
\end{align}

    \begin{Verbatim}[commandchars=\\\{\}]
{\color{incolor}In [{\color{incolor}2}]:} \PY{n}{tend} \PY{o}{=} \PY{l+m+mf}{0.5}
        \PY{n}{N} \PY{o}{=} \PY{l+m+mi}{100}
        \PY{n}{L} \PY{o}{=} \PY{l+m+mi}{1}
        \PY{n}{res} \PY{o}{=} \PY{l+m+mi}{100} \PY{c+c1}{\PYZsh{}image resolution}
        
        \PY{c+c1}{\PYZsh{}initiates quiver plot}
        \PY{n}{x} \PY{o}{=} \PY{n}{np}\PY{o}{.}\PY{n}{linspace}\PY{p}{(}\PY{l+m+mi}{0}\PY{p}{,} \PY{n}{L}\PY{p}{,} \PY{n}{N}\PY{p}{)}
        \PY{n}{y} \PY{o}{=} \PY{n}{np}\PY{o}{.}\PY{n}{linspace}\PY{p}{(}\PY{l+m+mi}{0}\PY{p}{,} \PY{n}{L}\PY{o}{/}\PY{l+m+mi}{2}\PY{p}{,} \PY{n}{N}\PY{p}{)}
        \PY{n}{X}\PY{p}{,} \PY{n}{Y} \PY{o}{=} \PY{n}{np}\PY{o}{.}\PY{n}{meshgrid}\PY{p}{(}\PY{n}{x}\PY{p}{,} \PY{n}{y}\PY{p}{)}
        
        \PY{n}{u1}\PY{p}{,}\PY{n}{v1}\PY{p}{,}\PY{n}{p1} \PY{o}{=} \PY{n}{solver}\PY{p}{(}\PY{n}{tend}\PY{p}{,}\PY{n}{N}\PY{p}{,}\PY{n}{L}\PY{p}{)}\PY{c+c1}{\PYZsh{},obstacle=\PYZsq{}ball\PYZsq{})}
\end{Verbatim}

    \begin{Verbatim}[commandchars=\\\{\}]
{\color{incolor}In [{\color{incolor}3}]:} \PY{n}{fig1}\PY{p}{,} \PY{n}{ax1} \PY{o}{=} \PY{n}{plt}\PY{o}{.}\PY{n}{subplots}\PY{p}{(}\PY{l+m+mi}{1}\PY{p}{,} \PY{l+m+mi}{1}\PY{p}{,} \PY{n}{figsize} \PY{o}{=} \PY{p}{(}\PY{l+m+mi}{10}\PY{p}{,} \PY{l+m+mi}{5}\PY{p}{)}\PY{p}{,}
                                 \PY{n}{dpi}\PY{o}{=}\PY{n}{res}\PY{p}{)}
        
        \PY{n}{a} \PY{o}{=} \PY{l+m+mi}{2} \PY{c+c1}{\PYZsh{}frequency of arrows}
        \PY{n}{ax1}\PY{o}{.}\PY{n}{quiver}\PY{p}{(}\PY{n}{X}\PY{p}{[}\PY{p}{:}\PY{p}{:}\PY{n}{a}\PY{p}{,} \PY{p}{:}\PY{p}{:}\PY{n}{a}\PY{p}{]}\PY{p}{,} \PY{n}{Y}\PY{p}{[}\PY{p}{:}\PY{p}{:}\PY{n}{a}\PY{p}{,} \PY{p}{:}\PY{p}{:}\PY{n}{a}\PY{p}{]}\PY{p}{,}
                   \PY{n}{u1}\PY{p}{[}\PY{p}{:}\PY{p}{:}\PY{n}{a}\PY{p}{,} \PY{p}{:}\PY{p}{:}\PY{n}{a}\PY{p}{]}\PY{p}{,} \PY{n}{v1}\PY{p}{[}\PY{p}{:}\PY{p}{:}\PY{n}{a}\PY{p}{,} \PY{p}{:}\PY{p}{:}\PY{n}{a}\PY{p}{]}\PY{p}{)}
        \PY{n}{ax1}\PY{o}{.}\PY{n}{set\PYZus{}title}\PY{p}{(}\PY{l+s+s2}{\PYZdq{}}\PY{l+s+s2}{Fluid Velocity (\PYZdl{}v\PYZus{}x (m/s)\PYZdl{})}\PY{l+s+s2}{\PYZdq{}}\PY{p}{)}
        \PY{n}{ax1}\PY{o}{.}\PY{n}{set\PYZus{}xlabel}\PY{p}{(}\PY{l+s+s2}{\PYZdq{}}\PY{l+s+s2}{\PYZdl{}x}\PY{l+s+s2}{\PYZbs{}}\PY{l+s+s2}{ (m)\PYZdl{}}\PY{l+s+s2}{\PYZdq{}}\PY{p}{)}
        \PY{n}{ax1}\PY{o}{.}\PY{n}{set\PYZus{}ylabel}\PY{p}{(}\PY{l+s+s2}{\PYZdq{}}\PY{l+s+s2}{\PYZdl{}y}\PY{l+s+s2}{\PYZbs{}}\PY{l+s+s2}{ (m)\PYZdl{}}\PY{l+s+s2}{\PYZdq{}}\PY{p}{)}
        \PY{c+c1}{\PYZsh{}plt.savefig(\PYZdq{}\PYZob{}:3.0f\PYZcb{}\PYZus{}basic\PYZus{}quiver.png\PYZdq{}.format(N))}
        
        
        \PY{n}{fig2}\PY{p}{,} \PY{n}{ax2} \PY{o}{=} \PY{n}{plt}\PY{o}{.}\PY{n}{subplots}\PY{p}{(}\PY{l+m+mi}{1}\PY{p}{,} \PY{l+m+mi}{1}\PY{p}{,} \PY{n}{figsize} \PY{o}{=} \PY{p}{(}\PY{l+m+mi}{10}\PY{p}{,} \PY{l+m+mi}{5}\PY{p}{)}\PY{p}{,}
                                 \PY{n}{dpi}\PY{o}{=}\PY{n}{res}\PY{p}{)}
        
        \PY{n}{ax2}\PY{o}{.}\PY{n}{imshow}\PY{p}{(}\PY{n}{u1}\PY{p}{,}\PY{n}{origin}\PY{o}{=}\PY{l+s+s1}{\PYZsq{}}\PY{l+s+s1}{lower}\PY{l+s+s1}{\PYZsq{}}\PY{p}{,}\PY{n}{cmap}\PY{o}{=}\PY{l+s+s1}{\PYZsq{}}\PY{l+s+s1}{gray}\PY{l+s+s1}{\PYZsq{}}\PY{p}{)}
        \PY{n}{ax2}\PY{o}{.}\PY{n}{set\PYZus{}title}\PY{p}{(}\PY{l+s+s2}{\PYZdq{}}\PY{l+s+s2}{Fluid Velocity (\PYZdl{}v\PYZus{}x (m/s)\PYZdl{})}\PY{l+s+s2}{\PYZdq{}}\PY{p}{)}
        \PY{n}{ax2}\PY{o}{.}\PY{n}{set\PYZus{}xlabel}\PY{p}{(}\PY{l+s+s2}{\PYZdq{}}\PY{l+s+s2}{\PYZdl{}x}\PY{l+s+s2}{\PYZbs{}}\PY{l+s+s2}{ (m)\PYZdl{}}\PY{l+s+s2}{\PYZdq{}}\PY{p}{)}
        \PY{n}{ax2}\PY{o}{.}\PY{n}{set\PYZus{}ylabel}\PY{p}{(}\PY{l+s+s2}{\PYZdq{}}\PY{l+s+s2}{\PYZdl{}y}\PY{l+s+s2}{\PYZbs{}}\PY{l+s+s2}{ (m)\PYZdl{}}\PY{l+s+s2}{\PYZdq{}}\PY{p}{)}
        \PY{c+c1}{\PYZsh{}plt.savefig(\PYZdq{}\PYZob{}:3.0f\PYZcb{}\PYZus{}basic\PYZus{}velocity.png\PYZdq{}.format(N))}
        
        
        \PY{n}{fig3}\PY{p}{,} \PY{n}{ax3} \PY{o}{=} \PY{n}{plt}\PY{o}{.}\PY{n}{subplots}\PY{p}{(}\PY{l+m+mi}{1}\PY{p}{,} \PY{l+m+mi}{1}\PY{p}{,} \PY{n}{figsize} \PY{o}{=} \PY{p}{(}\PY{l+m+mi}{10}\PY{p}{,} \PY{l+m+mi}{5}\PY{p}{)}\PY{p}{,}
                                 \PY{n}{dpi}\PY{o}{=}\PY{n}{res}\PY{p}{)}
        
        \PY{n}{ax3}\PY{o}{.}\PY{n}{imshow}\PY{p}{(}\PY{n}{p1}\PY{p}{,}\PY{n}{origin}\PY{o}{=}\PY{l+s+s1}{\PYZsq{}}\PY{l+s+s1}{lower}\PY{l+s+s1}{\PYZsq{}}\PY{p}{,}\PY{n}{cmap}\PY{o}{=}\PY{l+s+s1}{\PYZsq{}}\PY{l+s+s1}{Blues}\PY{l+s+s1}{\PYZsq{}}\PY{p}{)}
        \PY{n}{ax3}\PY{o}{.}\PY{n}{set\PYZus{}title}\PY{p}{(}\PY{l+s+s2}{\PYZdq{}}\PY{l+s+s2}{Pressure (\PYZdl{}Pa\PYZdl{})}\PY{l+s+s2}{\PYZdq{}}\PY{p}{)}
        \PY{n}{ax3}\PY{o}{.}\PY{n}{set\PYZus{}xlabel}\PY{p}{(}\PY{l+s+s2}{\PYZdq{}}\PY{l+s+s2}{\PYZdl{}x}\PY{l+s+s2}{\PYZbs{}}\PY{l+s+s2}{ (m)\PYZdl{}}\PY{l+s+s2}{\PYZdq{}}\PY{p}{)}
        \PY{n}{ax3}\PY{o}{.}\PY{n}{set\PYZus{}ylabel}\PY{p}{(}\PY{l+s+s2}{\PYZdq{}}\PY{l+s+s2}{\PYZdl{}y}\PY{l+s+s2}{\PYZbs{}}\PY{l+s+s2}{ (m)\PYZdl{}}\PY{l+s+s2}{\PYZdq{}}\PY{p}{)}
        \PY{c+c1}{\PYZsh{}plt.savefig(\PYZdq{}\PYZob{}:3.0f\PYZcb{}\PYZus{}basic\PYZus{}pressure.png\PYZdq{}.format(N))}
\end{Verbatim}

\begin{Verbatim}[commandchars=\\\{\}]
{\color{outcolor}Out[{\color{outcolor}3}]:} <matplotlib.text.Text at 0x11be6d668>
\end{Verbatim}
            
    \begin{center}
    \adjustimage{max size={0.9\linewidth}{0.9\paperheight}}{100_basic_quiver.png}
    \end{center}
    { \hspace*{\fill} \\}
    
    \begin{center}
    \adjustimage{max size={0.9\linewidth}{0.9\paperheight}}{output_3_2.png}
    \end{center}
    { \hspace*{\fill} \\}
    
    \begin{center}
    \adjustimage{max size={0.9\linewidth}{0.9\paperheight}}{output_3_3.png}
    \end{center}
    { \hspace*{\fill} \\}
    
    \subsection{Testing Results}\label{testing-results}

For the most basic case of channel flow, I can compare my results to the
results predicted by the analytically solved equation for the fluid's
velocity in the \(x\) direction. The equation for velcoity before
applying the relevant boundary conditions is

\[u(y) = -\frac{10}{\nu}\biggr(\frac{y^2}{2} + C_1y + C_2\biggr).\]

When we apply the no slip conditions at \(y=0,\tfrac L2,\) which force
fluid velocity to be zero, we constrain the equation such that
\(C_2 = 0\) and \(C_1 = -\frac L4.\) This gives us the equation for
velocity as a function of height in a channel as

\begin{equation}
u(y) = -\frac{10}{\nu}\biggr(\frac{y^2}{2} -\frac L4 y\biggr).
\end{equation}

    \begin{Verbatim}[commandchars=\\\{\}]
{\color{incolor}In [{\color{incolor}4}]:} \PY{c+c1}{\PYZsh{}compare to analytical solution}
        
        \PY{c+c1}{\PYZsh{}array with analytically determined velocity}
        \PY{n}{yrange} \PY{o}{=} \PY{n}{np}\PY{o}{.}\PY{n}{linspace}\PY{p}{(}\PY{l+m+mi}{0}\PY{p}{,}\PY{n}{L}\PY{o}{/}\PY{l+m+mi}{2}\PY{p}{,}\PY{n}{N}\PY{o}{+}\PY{l+m+mi}{1}\PY{p}{)}
        \PY{n}{perfect\PYZus{}u} \PY{o}{=} \PY{n}{solved\PYZus{}u}\PY{p}{(}\PY{n}{yrange}\PY{p}{)}
            
        \PY{n}{fig4}\PY{p}{,} \PY{n}{ax4} \PY{o}{=} \PY{n}{plt}\PY{o}{.}\PY{n}{subplots}\PY{p}{(}\PY{l+m+mi}{1}\PY{p}{,} \PY{l+m+mi}{1}\PY{p}{,} \PY{n}{figsize} \PY{o}{=} \PY{p}{(}\PY{l+m+mi}{10}\PY{p}{,} \PY{l+m+mi}{5}\PY{p}{)}\PY{p}{,}
                                 \PY{n}{dpi}\PY{o}{=}\PY{n}{res}\PY{p}{)}
        
        \PY{n}{ax4}\PY{o}{.}\PY{n}{plot}\PY{p}{(}\PY{n}{u1}\PY{p}{[}\PY{p}{:}\PY{p}{,}\PY{n+nb}{int}\PY{p}{(}\PY{n}{L}\PY{o}{/}\PY{l+m+mi}{2}\PY{p}{)}\PY{p}{]}\PY{p}{,}\PY{n}{label}\PY{o}{=}\PY{l+s+s1}{\PYZsq{}}\PY{l+s+s1}{Computational}\PY{l+s+s1}{\PYZsq{}}\PY{p}{)}
        \PY{n}{ax4}\PY{o}{.}\PY{n}{plot}\PY{p}{(}\PY{n}{perfect\PYZus{}u}\PY{p}{,}\PY{n}{label}\PY{o}{=}\PY{l+s+s1}{\PYZsq{}}\PY{l+s+s1}{Analytic}\PY{l+s+s1}{\PYZsq{}}\PY{p}{)}
        \PY{n}{ax4}\PY{o}{.}\PY{n}{set\PYZus{}title}\PY{p}{(}\PY{l+s+s2}{\PYZdq{}}\PY{l+s+s2}{Fluid Velocity for \PYZdl{}t\PYZus{}f=}\PY{l+s+si}{\PYZob{}:1.1f\PYZcb{}}\PY{l+s+s2}{\PYZdl{}s}\PY{l+s+s2}{\PYZdq{}}\PYZbs{}
                      \PY{o}{.}\PY{n}{format}\PY{p}{(}\PY{n}{tend}\PY{p}{)}\PY{p}{)}
        \PY{n}{ax4}\PY{o}{.}\PY{n}{set\PYZus{}xlabel}\PY{p}{(}\PY{l+s+s2}{\PYZdq{}}\PY{l+s+s2}{\PYZdl{}y}\PY{l+s+s2}{\PYZbs{}}\PY{l+s+s2}{ (m)\PYZdl{}}\PY{l+s+s2}{\PYZdq{}}\PY{p}{)}
        \PY{n}{ax4}\PY{o}{.}\PY{n}{set\PYZus{}ylabel}\PY{p}{(}\PY{l+s+s2}{\PYZdq{}}\PY{l+s+s2}{\PYZdl{}v\PYZus{}x}\PY{l+s+s2}{\PYZbs{}}\PY{l+s+s2}{ (m/s)\PYZdl{}}\PY{l+s+s2}{\PYZdq{}}\PY{p}{)}
        \PY{n}{ax4}\PY{o}{.}\PY{n}{legend}\PY{p}{(}\PY{p}{)}
        \PY{c+c1}{\PYZsh{}plt.savefig(\PYZdq{}100\PYZus{}basic\PYZus{}compare.png\PYZdq{})}
\end{Verbatim}

\begin{Verbatim}[commandchars=\\\{\}]
{\color{outcolor}Out[{\color{outcolor}4}]:} <matplotlib.legend.Legend at 0x11d3d0a20>
\end{Verbatim}
            
    \begin{center}
    \adjustimage{max size={0.9\linewidth}{0.9\paperheight}}{output_5_1.png}
    \end{center}
    { \hspace*{\fill} \\}
    
    \subsection{Ball in Channel}\label{ball-in-channel}

I'll now analyze the impact of placing a ball into a channel with this
flow. To ease the computational expense, I will model it as a circle in
a 2-d channel, which will give us a cross-section of the channel's flow.
The equations modelling the flow remain the same. The difference is that
we now add additional boundary conditions around the circle. These
boundary conditions are the same as for the upper/lower channel
boudnaries, i.e., the no slip condition and the unchanging pressure
condition. I developed the code so that the optional argument
"obstacle='ball'" triggers the additional boundary conditions to be
taken into account when solving the system.

    \begin{Verbatim}[commandchars=\\\{\}]
{\color{incolor}In [{\color{incolor}5}]:} \PY{n}{tend} \PY{o}{=} \PY{l+m+mf}{0.5}
        \PY{n}{N} \PY{o}{=} \PY{l+m+mi}{75}
        \PY{n}{L} \PY{o}{=} \PY{l+m+mi}{1}
        
        \PY{c+c1}{\PYZsh{}initiates quiver plot}
        \PY{n}{x} \PY{o}{=} \PY{n}{np}\PY{o}{.}\PY{n}{linspace}\PY{p}{(}\PY{l+m+mi}{0}\PY{p}{,} \PY{n}{L}\PY{p}{,} \PY{n}{N}\PY{p}{)}
        \PY{n}{y} \PY{o}{=} \PY{n}{np}\PY{o}{.}\PY{n}{linspace}\PY{p}{(}\PY{l+m+mi}{0}\PY{p}{,} \PY{n}{L}\PY{o}{/}\PY{l+m+mi}{2}\PY{p}{,} \PY{n}{N}\PY{p}{)}
        \PY{n}{X}\PY{p}{,} \PY{n}{Y} \PY{o}{=} \PY{n}{np}\PY{o}{.}\PY{n}{meshgrid}\PY{p}{(}\PY{n}{x}\PY{p}{,} \PY{n}{y}\PY{p}{)}
        
        \PY{n}{u2}\PY{p}{,}\PY{n}{v2}\PY{p}{,}\PY{n}{p2} \PY{o}{=} \PY{n}{solver}\PY{p}{(}\PY{n}{tend}\PY{p}{,}\PY{n}{N}\PY{p}{,}\PY{n}{L}\PY{p}{,}\PY{n}{obstacle}\PY{o}{=}\PY{l+s+s1}{\PYZsq{}}\PY{l+s+s1}{ball}\PY{l+s+s1}{\PYZsq{}}\PY{p}{)}
\end{Verbatim}

    \begin{Verbatim}[commandchars=\\\{\}]
{\color{incolor}In [{\color{incolor}6}]:} \PY{n}{fig5}\PY{p}{,} \PY{n}{ax5} \PY{o}{=} \PY{n}{plt}\PY{o}{.}\PY{n}{subplots}\PY{p}{(}\PY{l+m+mi}{1}\PY{p}{,} \PY{l+m+mi}{1}\PY{p}{,} \PY{n}{figsize} \PY{o}{=} \PY{p}{(}\PY{l+m+mi}{10}\PY{p}{,} \PY{l+m+mi}{5}\PY{p}{)}\PY{p}{,}
                                 \PY{n}{dpi}\PY{o}{=}\PY{n}{res}\PY{p}{)}
        
        \PY{n}{a} \PY{o}{=} \PY{l+m+mi}{2} \PY{c+c1}{\PYZsh{}frequency of arrows}
        \PY{n}{ax5}\PY{o}{.}\PY{n}{quiver}\PY{p}{(}\PY{n}{X}\PY{p}{[}\PY{p}{:}\PY{p}{:}\PY{n}{a}\PY{p}{,} \PY{p}{:}\PY{p}{:}\PY{n}{a}\PY{p}{]}\PY{p}{,} \PY{n}{Y}\PY{p}{[}\PY{p}{:}\PY{p}{:}\PY{n}{a}\PY{p}{,} \PY{p}{:}\PY{p}{:}\PY{n}{a}\PY{p}{]}\PY{p}{,}
                   \PY{n}{u2}\PY{p}{[}\PY{p}{:}\PY{p}{:}\PY{n}{a}\PY{p}{,} \PY{p}{:}\PY{p}{:}\PY{n}{a}\PY{p}{]}\PY{p}{,} \PY{n}{v2}\PY{p}{[}\PY{p}{:}\PY{p}{:}\PY{n}{a}\PY{p}{,} \PY{p}{:}\PY{p}{:}\PY{n}{a}\PY{p}{]}\PY{p}{)}
        \PY{n}{ax5}\PY{o}{.}\PY{n}{set\PYZus{}title}\PY{p}{(}\PY{l+s+s2}{\PYZdq{}}\PY{l+s+s2}{Fluid Velocity (\PYZdl{}v\PYZus{}x (m/s)\PYZdl{})}\PY{l+s+s2}{\PYZdq{}}\PY{p}{)}
        \PY{n}{ax5}\PY{o}{.}\PY{n}{set\PYZus{}xlabel}\PY{p}{(}\PY{l+s+s2}{\PYZdq{}}\PY{l+s+s2}{\PYZdl{}x}\PY{l+s+s2}{\PYZbs{}}\PY{l+s+s2}{ (m)\PYZdl{}}\PY{l+s+s2}{\PYZdq{}}\PY{p}{)}
        \PY{n}{ax5}\PY{o}{.}\PY{n}{set\PYZus{}ylabel}\PY{p}{(}\PY{l+s+s2}{\PYZdq{}}\PY{l+s+s2}{\PYZdl{}y}\PY{l+s+s2}{\PYZbs{}}\PY{l+s+s2}{ (m)\PYZdl{}}\PY{l+s+s2}{\PYZdq{}}\PY{p}{)}
        \PY{c+c1}{\PYZsh{}plt.savefig(\PYZdq{}\PYZob{}:2.0f\PYZcb{}\PYZus{}ball\PYZus{}quiver.png\PYZdq{}.format(N))}
        
        
        \PY{n}{fig6}\PY{p}{,} \PY{n}{ax6} \PY{o}{=} \PY{n}{plt}\PY{o}{.}\PY{n}{subplots}\PY{p}{(}\PY{l+m+mi}{1}\PY{p}{,} \PY{l+m+mi}{1}\PY{p}{,} \PY{n}{figsize} \PY{o}{=} \PY{p}{(}\PY{l+m+mi}{10}\PY{p}{,} \PY{l+m+mi}{5}\PY{p}{)}\PY{p}{,}
                                 \PY{n}{dpi}\PY{o}{=}\PY{n}{res}\PY{p}{)}
        
        \PY{n}{ax6}\PY{o}{.}\PY{n}{imshow}\PY{p}{(}\PY{n}{u2}\PY{p}{,}\PY{n}{origin}\PY{o}{=}\PY{l+s+s1}{\PYZsq{}}\PY{l+s+s1}{lower}\PY{l+s+s1}{\PYZsq{}}\PY{p}{,}\PY{n}{cmap}\PY{o}{=}\PY{l+s+s1}{\PYZsq{}}\PY{l+s+s1}{gray}\PY{l+s+s1}{\PYZsq{}}\PY{p}{)}
        \PY{n}{ax6}\PY{o}{.}\PY{n}{set\PYZus{}title}\PY{p}{(}\PY{l+s+s2}{\PYZdq{}}\PY{l+s+s2}{Fluid Velocity (\PYZdl{}v\PYZus{}x (m/s)\PYZdl{})}\PY{l+s+s2}{\PYZdq{}}\PY{p}{)}
        \PY{n}{ax6}\PY{o}{.}\PY{n}{set\PYZus{}xlabel}\PY{p}{(}\PY{l+s+s2}{\PYZdq{}}\PY{l+s+s2}{\PYZdl{}x}\PY{l+s+s2}{\PYZbs{}}\PY{l+s+s2}{ (m)\PYZdl{}}\PY{l+s+s2}{\PYZdq{}}\PY{p}{)}
        \PY{n}{ax6}\PY{o}{.}\PY{n}{set\PYZus{}ylabel}\PY{p}{(}\PY{l+s+s2}{\PYZdq{}}\PY{l+s+s2}{\PYZdl{}y}\PY{l+s+s2}{\PYZbs{}}\PY{l+s+s2}{ (m)\PYZdl{}}\PY{l+s+s2}{\PYZdq{}}\PY{p}{)}
        \PY{c+c1}{\PYZsh{}plt.savefig(\PYZdq{}\PYZob{}:2.0f\PYZcb{}\PYZus{}ball\PYZus{}velocity.png\PYZdq{}.format(N))}
\end{Verbatim}

\begin{Verbatim}[commandchars=\\\{\}]
{\color{outcolor}Out[{\color{outcolor}6}]:} <matplotlib.text.Text at 0x11c6a9d30>
\end{Verbatim}
            
    \begin{center}
    \adjustimage{max size={0.9\linewidth}{0.9\paperheight}}{output_8_1.png}
    \end{center}
    { \hspace*{\fill} \\}
    
    \begin{center}
    \adjustimage{max size={0.9\linewidth}{0.9\paperheight}}{output_8_2.png}
    \end{center}
    { \hspace*{\fill} \\}
    
    \subsection{Flow Around a Wing}\label{flow-around-a-wing}

We can also create a wing-shaped object in the channel, modelling its
effect on the flow of fuid. I've done this below, with the boundary
conditions adjusted to the new object's shape.

    \begin{Verbatim}[commandchars=\\\{\}]
{\color{incolor}In [{\color{incolor}7}]:} \PY{n}{tend} \PY{o}{=} \PY{l+m+mf}{0.5}
        \PY{n}{N} \PY{o}{=} \PY{l+m+mi}{75}
        \PY{n}{L} \PY{o}{=} \PY{l+m+mi}{1}
        
        \PY{c+c1}{\PYZsh{}initiates quiver plot}
        \PY{n}{x} \PY{o}{=} \PY{n}{np}\PY{o}{.}\PY{n}{linspace}\PY{p}{(}\PY{l+m+mi}{0}\PY{p}{,} \PY{n}{L}\PY{p}{,} \PY{n}{N}\PY{p}{)}
        \PY{n}{y} \PY{o}{=} \PY{n}{np}\PY{o}{.}\PY{n}{linspace}\PY{p}{(}\PY{l+m+mi}{0}\PY{p}{,} \PY{n}{L}\PY{o}{/}\PY{l+m+mi}{2}\PY{p}{,} \PY{n}{N}\PY{p}{)}
        \PY{n}{X}\PY{p}{,} \PY{n}{Y} \PY{o}{=} \PY{n}{np}\PY{o}{.}\PY{n}{meshgrid}\PY{p}{(}\PY{n}{x}\PY{p}{,} \PY{n}{y}\PY{p}{)}
        
        \PY{n}{u3}\PY{p}{,}\PY{n}{v3}\PY{p}{,}\PY{n}{p3} \PY{o}{=} \PY{n}{solver}\PY{p}{(}\PY{n}{tend}\PY{p}{,}\PY{n}{N}\PY{p}{,}\PY{n}{L}\PY{p}{,}\PY{n}{obstacle}\PY{o}{=}\PY{l+s+s1}{\PYZsq{}}\PY{l+s+s1}{wing}\PY{l+s+s1}{\PYZsq{}}\PY{p}{)}
\end{Verbatim}

    \begin{Verbatim}[commandchars=\\\{\}]
{\color{incolor}In [{\color{incolor}8}]:} \PY{n}{fig7}\PY{p}{,} \PY{n}{ax7} \PY{o}{=} \PY{n}{plt}\PY{o}{.}\PY{n}{subplots}\PY{p}{(}\PY{l+m+mi}{1}\PY{p}{,} \PY{l+m+mi}{1}\PY{p}{,} \PY{n}{figsize} \PY{o}{=} \PY{p}{(}\PY{l+m+mi}{10}\PY{p}{,} \PY{l+m+mi}{5}\PY{p}{)}\PY{p}{,}
                                 \PY{n}{dpi}\PY{o}{=}\PY{n}{res}\PY{p}{)}
        
        \PY{n}{a} \PY{o}{=} \PY{l+m+mi}{2} \PY{c+c1}{\PYZsh{}frequency of arrows}
        \PY{n}{ax7}\PY{o}{.}\PY{n}{quiver}\PY{p}{(}\PY{n}{X}\PY{p}{[}\PY{p}{:}\PY{p}{:}\PY{n}{a}\PY{p}{,} \PY{p}{:}\PY{p}{:}\PY{n}{a}\PY{p}{]}\PY{p}{,} \PY{n}{Y}\PY{p}{[}\PY{p}{:}\PY{p}{:}\PY{n}{a}\PY{p}{,} \PY{p}{:}\PY{p}{:}\PY{n}{a}\PY{p}{]}\PY{p}{,}
                   \PY{n}{u3}\PY{p}{[}\PY{p}{:}\PY{p}{:}\PY{n}{a}\PY{p}{,} \PY{p}{:}\PY{p}{:}\PY{n}{a}\PY{p}{]}\PY{p}{,} \PY{n}{v3}\PY{p}{[}\PY{p}{:}\PY{p}{:}\PY{n}{a}\PY{p}{,} \PY{p}{:}\PY{p}{:}\PY{n}{a}\PY{p}{]}\PY{p}{)}
        \PY{n}{ax7}\PY{o}{.}\PY{n}{set\PYZus{}title}\PY{p}{(}\PY{l+s+s2}{\PYZdq{}}\PY{l+s+s2}{Fluid Velocity (\PYZdl{}v\PYZus{}x (m/s)\PYZdl{})}\PY{l+s+s2}{\PYZdq{}}\PY{p}{)}
        \PY{n}{ax7}\PY{o}{.}\PY{n}{set\PYZus{}xlabel}\PY{p}{(}\PY{l+s+s2}{\PYZdq{}}\PY{l+s+s2}{\PYZdl{}x}\PY{l+s+s2}{\PYZbs{}}\PY{l+s+s2}{ (m)\PYZdl{}}\PY{l+s+s2}{\PYZdq{}}\PY{p}{)}
        \PY{n}{ax7}\PY{o}{.}\PY{n}{set\PYZus{}ylabel}\PY{p}{(}\PY{l+s+s2}{\PYZdq{}}\PY{l+s+s2}{\PYZdl{}y}\PY{l+s+s2}{\PYZbs{}}\PY{l+s+s2}{ (m)\PYZdl{}}\PY{l+s+s2}{\PYZdq{}}\PY{p}{)}
        \PY{c+c1}{\PYZsh{}plt.savefig(\PYZdq{}\PYZob{}:2.0f\PYZcb{}\PYZus{}wing\PYZus{}quiver.png\PYZdq{}.format(N))}
        
        
        \PY{n}{fig8}\PY{p}{,} \PY{n}{ax8} \PY{o}{=} \PY{n}{plt}\PY{o}{.}\PY{n}{subplots}\PY{p}{(}\PY{l+m+mi}{1}\PY{p}{,} \PY{l+m+mi}{1}\PY{p}{,} \PY{n}{figsize} \PY{o}{=} \PY{p}{(}\PY{l+m+mi}{10}\PY{p}{,} \PY{l+m+mi}{5}\PY{p}{)}\PY{p}{,}
                                 \PY{n}{dpi}\PY{o}{=}\PY{n}{res}\PY{p}{)}
        
        \PY{n}{ax8}\PY{o}{.}\PY{n}{imshow}\PY{p}{(}\PY{n}{u3}\PY{p}{,}\PY{n}{origin}\PY{o}{=}\PY{l+s+s1}{\PYZsq{}}\PY{l+s+s1}{lower}\PY{l+s+s1}{\PYZsq{}}\PY{p}{,}\PY{n}{cmap}\PY{o}{=}\PY{l+s+s1}{\PYZsq{}}\PY{l+s+s1}{gray}\PY{l+s+s1}{\PYZsq{}}\PY{p}{)}
        \PY{n}{ax8}\PY{o}{.}\PY{n}{set\PYZus{}title}\PY{p}{(}\PY{l+s+s2}{\PYZdq{}}\PY{l+s+s2}{Fluid Velocity (\PYZdl{}v\PYZus{}x (m/s)\PYZdl{})}\PY{l+s+s2}{\PYZdq{}}\PY{p}{)}
        \PY{n}{ax8}\PY{o}{.}\PY{n}{set\PYZus{}xlabel}\PY{p}{(}\PY{l+s+s2}{\PYZdq{}}\PY{l+s+s2}{\PYZdl{}x}\PY{l+s+s2}{\PYZbs{}}\PY{l+s+s2}{ (m)\PYZdl{}}\PY{l+s+s2}{\PYZdq{}}\PY{p}{)}
        \PY{n}{ax8}\PY{o}{.}\PY{n}{set\PYZus{}ylabel}\PY{p}{(}\PY{l+s+s2}{\PYZdq{}}\PY{l+s+s2}{\PYZdl{}y}\PY{l+s+s2}{\PYZbs{}}\PY{l+s+s2}{ (m)\PYZdl{}}\PY{l+s+s2}{\PYZdq{}}\PY{p}{)}
        \PY{c+c1}{\PYZsh{}plt.savefig(\PYZdq{}\PYZob{}:2.0f\PYZcb{}\PYZus{}wing\PYZus{}velocity.png\PYZdq{}.format(N))}
\end{Verbatim}

\begin{Verbatim}[commandchars=\\\{\}]
{\color{outcolor}Out[{\color{outcolor}8}]:} <matplotlib.text.Text at 0x11c131198>
\end{Verbatim}
            
    \begin{center}
    \adjustimage{max size={0.9\linewidth}{0.9\paperheight}}{output_11_1.png}
    \end{center}
    { \hspace*{\fill} \\}
    
    \begin{center}
    \adjustimage{max size={0.9\linewidth}{0.9\paperheight}}{output_11_2.png}
    \end{center}
    { \hspace*{\fill} \\}
    
    \begin{Verbatim}[commandchars=\\\{\}]
{\color{incolor}In [{\color{incolor} }]:} 
\end{Verbatim}


    % Add a bibliography block to the postdoc
    
    
    
    \end{document}
